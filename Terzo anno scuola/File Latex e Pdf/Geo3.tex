\documentclass[11pt, a4paper, twoside]{article}

% Pacchetti Base
\usepackage[italian]{babel}
\usepackage{amsfonts}
\usepackage{amssymb}
\usepackage{tikz}
\usetikzlibrary{cd}
\usepackage{extarrows}
\usepackage{mathrsfs}
\usepackage{amsthm}

% Estetica
\usepackage{tcolorbox}
\tcbuselibrary{theorems}
\usepackage{hyperref}
\usepackage{graphicx}
\graphicspath{ {./Immagini/} }
\usepackage{bm}
\usepackage{fouriernc}
\usepackage[T1]{fontenc}
\usepackage{tikzrput}
\usepackage[object=vectorian]{pgfornament}
\usepackage{xcolor}
\usepackage{fourier-orns}
\usepackage{fancyhdr}
\renewcommand{\headrule}{%
\vspace{-8pt}\hrulefill
\raisebox{-2.1pt}{\quad\decofourleft\decotwo\decofourright\quad}\hrulefill}
\usetikzlibrary{cd,decorations.pathmorphing,patterns}
\usepackage[T1]{fontenc}

% Pacchetti TCB
\tcbuselibrary{documentation}
\tcbuselibrary{breakable}

% Definizioni e Teoremi
\newtcbtheorem
	[number within = subsection]% init options
	{defn}% name
	{Definizione}% title
	{%
		colback=teal!5,
		colframe=teal!90!black!95,
		fonttitle=\bfseries,
	}% options
	{def}% prefix

\newtcbtheorem
	[use counter from = defn, number within = subsection]% init options
	{thm}% name
	{Teorema}% title
	{%
		colback=blue!5,
		colframe=blue!90!black!95,
		fonttitle=\bfseries,
	}% options
	{th}% prefix

\newtcbtheorem
	[use counter from = defn, number within = subsection]% init options
	{prop}% name
	{Proposizione}% title
	{%
		colback=red!5,
		colframe=red!90!black!95,
		fonttitle=\bfseries,
	}% options
	{pr}% prefix

\newtcbtheorem
	[use counter from = defn, number within = subsection]% init options
	{lemma}% name
	{Lemma}% title
	{%
		colback=red!5,
		colframe=red!90!black!95,
		fonttitle=\bfseries,
	}% options
	{le}% prefix

\newtcbtheorem
	[use counter from = defn, number within = subsection]% init options
	{cor}% name
	{Corollario}% title
	{%
		colback=red!5,
		colframe=red!90!black!95,
		fonttitle=\bfseries,
	}% options
	{co}% prefix

\newcommand{\vareps}{\varepsilon}
\newtheorem{es}{Esempio}
\theoremstyle{definition}
\newtheorem*{oss}{Osservazione}
\newtheorem{ese}{Esercizio}

\tcolorboxenvironment{ese}{
	colframe=black}
\newenvironment{sol}
	{\renewcommand\qedsymbol{$\blacksquare$}\begin{proof}[Soluzione]}
	{\end{proof}}

\tcolorboxenvironment{proof}{% `proof' from `amsthm'
	blanker,breakable,left=2.5mm,
	before skip=10pt,after skip=10pt,
	borderline west={0.5mm}{0pt}{black}}

% Formattazione Pagine
\fancyhf{}
\fancyhead[LE]{\nouppercase{\rightmark\hfill\leftmark}}
\fancyhead[RO]{\nouppercase{\leftmark\hfill\rightmark}}
\fancyfoot[LE,RO]{\hrulefill\raisebox{-2.1pt}{\quad\thepage\quad}\hrulefill}
\setlength{\headheight}{16pt}

% Margini
\topmargin=-0.45in
\evensidemargin=-0.15in
\oddsidemargin=-0.15in
\textwidth=6.5in
\textheight=9.0in
\headsep=0.25in

%Titolo
\newcommand{\titolo}[1]{
	\thispagestyle{empty}
	\vspace*{\fill}
	\tikzset{pgfornamentstyle/.style={draw = black, fill = teal!50}}
	\unitlength=1cm
	\begin{center}
	\begin{picture}(12,12)%
		\color{white}%
			\put(0,0){\framebox(12,12){%
			\rput[tl](-4,6){\pgfornament[width=8cm]{71}}%
			\rput[bl](-4,-6){\pgfornament[width=8cm,,symmetry=h]{71}}%
			\rput[tl](-6,6){\pgfornament[width=2cm]{63}}%
			\rput[tr](6,6){\pgfornament[width=2cm,,symmetry=v]{63}}%
			\rput[bl](-6,-6){\pgfornament[width=2cm,,symmetry=h]{63}}%
			\rput[br](6,-6){\pgfornament[width=2cm,,symmetry=c]{63}}%
			\rput[bl]{-90}(-6,4){\pgfornament[width=8cm]{46}}%
			\rput[bl]{90}(6,-4){\pgfornament[width=8cm]{46}}%
			\rput(0,0){\Huge \color{black} #1}%
			\rput[t](0,-0.5){\pgfornament[width=5cm]{75}}%
			\rput[b](0,0.5){\pgfornament[width=5cm]{69}}%
			\rput[tr]{-30}(-1,2.5){\pgfornament[width=2cm]{57}}%
			\rput[tl]{30}(1,2.5){\pgfornament[width=2cm,symmetry=v]{57}}}}%
	\end{picture}
	\end{center}
	\vspace*{\fill}
	\pagebreak
}

% Miscellanee
\author{Emanuele Fava}
\date{}
\pagestyle{fancy}

\usetikzlibrary{bending}
\usetikzlibrary{lindenmayersystems}

\pgfdeclarelindenmayersystem{cayley}{
	\rule{F -> F [ R [F] [+F] [-F] ]}
	\symbol{R}{
		\pgflsystemstep=0.5\pgflsystemstep
	}
}

\begin{document}

\titolo{Geometria 3}

\tableofcontents

\newpage

\section{Teoria dei Rivestimenti}

\subsection{Gruppo Fondamentale}

\begin{defn}{Gruppo Fondamentale}{}
	Sia $(X, \mathcal T)$ uno spazio topologico e sia $x_0 \in X$. Definiamo $\pi_1(X,x_0)$ il \textbf{Gruppo Fondamentale} come:
	\[ \pi_1(X,x_0)=\{ \text{Classi di omotopia a estremi fissati di cammini }\gamma:[0,1] \to X : \gamma(0) = \gamma(1) = x_0 \} \]
	Questo è un gruppo con l'operazione di giunzione tra cammini
\end{defn}

\begin{es}[$S^1$]
	Sappiamo che il gruppo fondamentale di $S^1$ è:
	\[ \pi_1(S^1,x_0) \cong \pi_1(S^1) \cong \mathbb Z \]
	Infatti, possiamo considerare il gruppo fondamentale della circonferenza in base al numero di giri che vengono fatti partendo da un punto.
	\begin{center}
		\begin{tikzpicture}
			\draw (0,-1) arc (-90:270:2cm and 1cm);
			\filldraw (0,-1) circle(1pt) node[above]{$x_0$} node[below]{$\rightarrow$};
		\end{tikzpicture}
	\end{center}
\end{es}

In questo caso possiamo omettere il punto base, e indicare astrattamente che tipo di gruppo è, in quanto $X = S^1$ è connesso per archi, quindi:
\[ \forall x_0,x_1 \in S^1 \qquad \pi_1(S^1, x_0) \cong \pi_1(S_1, x_1) \]
È sufficiente trovare un cammino che leghi $x_0$ e $x_1$. Bisogna fare tuttavia attenzione che non esiste un isomorfismo canonico, in quanto dipende dalla scelta di tale cammino $\gamma \in \Omega(X, x_0,x_1)$
\begin{center}
	\begin{tikzpicture}
		\draw (0,-1) arc (-90:270:2cm and 1cm);
		\filldraw (0,-1) circle(1pt) node[above]{$x_0$};
		\filldraw (2,0) circle(1pt) node[right]{$x_1$};
		\begin{scope}[decoration={
			markings,
			mark=at position 0.5 with {\arrow{>}}}]
			\draw[thick, postaction = {decorate}] (0,-1) arc(-90:0:2cm and 1cm) node[pos = 0.5, above]{$\gamma$};
		\end{scope}
	\end{tikzpicture}
\end{center}

\begin{es}[$S^1 \vee S^1$]
	Il bouquet di $2$ circonferenze (o wedge di $2$ circonferenze) ha gruppo fondamentale il prodotto libero di $\mathbb Z$ con sé stesso. Cioè:
	\[ \pi_1(S^1 \vee S^1) = \mathbb Z * \mathbb Z \]
	Con gruppo libero intendiamo una concatenazione di elementi da un parte e elementi dall'altra.
\end{es}

Esiste un teorema formidabile che permette di calcolare il gruppo fondamentale degli spazi topologici:

\begin{thm}{Teorema di Van Kampen}{}
	Sia $X = A\cup B$, dove $A,B$ sono degli aperti in $X$ connessi per archi la cui intersezione è connessa per archi e sia $x_0 \in A \cap B$. Allora:
	\[ \pi_1(X, x_0) \cong \frac{\pi_1(A, x_0) * \pi_1(B, x_0)}{\langle i_*([\gamma])*(j_*([\gamma]))^{-1} \sim id : [\gamma] \in \pi_1(A \cap B, x_0) \rangle} \]
	Dove abbiamo che:
	\begin{center}
		\begin{tikzcd}
			& A \arrow[hook, dr]\\
			A\cap B \arrow[hook, ur, "i"] \arrow[hook, dr, "j"'] && X\\
			& B \arrow[hook, ur]
		\end{tikzcd}
	\end{center}
\end{thm}

\begin{es}[$\mathbb R^n$]
	Sapendo che $\mathbb R^n$ è contraibile, cioè omotopicamente equivalente ad un punto abbiamo che:
	\[ \pi_1(\mathbb R^n) \cong \{1\} \]
\end{es}

\begin{es}[$S^n, n\geq 2$]
	Il gruppo fondamentale della sfera $n-$dimensionale, per $n\neq 1$ è $\pi_1(S^n) = \{1\}$ in quanto è semplicemente connesso:
	\begin{center}
		\begin{tikzpicture}
			\begin{scope}
				\clip (0,0) circle(1cm);
				\draw[dashed] (180:1) arc(180:0:1cm and 0.25cm) (90:1) arc(90:-90:0.25cm and 1cm);
				\fill[green, very nearly transparent] (195:1) arc(180:360:0.97 and 0.23) arc(-15:195:1);
				\draw[green] (195:1) arc (180:360:0.97 and 0.23);
				\draw[green, dashed] (195:1) arc(180:0:0.97 and 0.23);
				\fill[blue, very nearly transparent] (150:1) arc(150:390:1 and 0.25) arc (390:150:1);
				\draw[blue] (150:1) arc(150:390:1 and 0.25) arc (390:150:1);
				\draw[blue, dashed] (165:1) arc(180:0: 0.97 and 0.23);
			\end{scope}
			\draw (0,0) circle(1cm) (180:1) arc (180:360:1cm and 0.25cm) (90:1) arc (90:270:0.25cm and 1cm);
		\end{tikzpicture}
	\end{center}
\end{es}

Questo era un esempio in cui si vede che il gruppo fondamentale ci da informazioni limitate su che tipo di spazio topologico stiamo lavorando.

\begin{es}[$\mathbb P^2(\mathbb R)$]
	Sappiamo che il gruppo fondamentale dello spazio proiettivo reale è:
	\[ \pi_1(\mathbb P^2(\mathbb R)) = \mathbb Z/_2 \]

	\begin{center}
		\begin{tikzpicture}
			\draw (0,0) circle(1cm);
			\filldraw[green, very nearly transparent] (0,0) circle(1cm);
			\filldraw (0,1) circle(1pt) (0,-1) circle(1pt);
			\begin{scope}[decoration={
				markings,
				mark=at position 0.5 with {\arrow{>}}}]
				\draw[thick, postaction = {decorate}] (0,-1) arc(-90:90:1cm);
				\draw[thick, postaction = {decorate}] (0,1) arc(90:270:1cm);
			\end{scope}
		\end{tikzpicture}
	\end{center}
\end{es}

Prima di dare il prossimo teorema, ricordiamo delle definizioni:

\begin{defn}{Retrazione}{}
	Se $A \subseteq X$ $r:X \to A$ è una \textbf{Retrazione} se $r|_A = id_A$ ed è continua
\end{defn}

\begin{defn}{Retrazione per Deformazione}{}
	Sia $r:X \times [0,1] \to X$. Questa è una \textbf{Retrazione per Deformazione} se $r$ è continua tale che:
	\begin{itemize}
		\item $r|_{X \times \{0\}} = id_X$
		\item $Im(r|_{X \times \{1\}}) = A$
		\item $r(x,t) = x, \forall x \in A, \forall t \in [0,1]$
	\end{itemize}
\end{defn}

Osserviamo che la prima è istantanea, mentre la seconda è più graduale. Inoltre vale:
\[ r|_{X \times \{1\}} \text{ è una retrazione}\]
Quindi una retrazione per deformazione è un'omotopia tra identità e retrazione che tiene fissi tutti i punti di $A$, per ogni $t \in [0,1]$

\begin{thm}{di non retrazione}{}\label{nonretraz}
	Non esiste alcuna retrazione del tipo:
	\[ \gamma: D^2 \to S^1 = \partial D^2 \]
\end{thm}

Quello che il teorema ci dice è che non esiste nessuna funzione continua che porti tutti i punti del disco sul suo bordo

\begin{proof}
	Supponiamo tale retrazione esista, allora, con un diagramma, abbiamo che:
	\begin{center}
		\begin{tikzcd}
			S^1 \arrow[r, hook, "i"] & D^2 \arrow[l, bend left = 30, "r"]
		\end{tikzcd}
	\end{center}
	In particolare, questo ci dice che:
	\[r\circ i = id_{S^1}\]
	Ma allora, sfruttando la funtorialità di $\pi_1$, abbiamo che:
	\[ r_* \circ i_* = (id_{S^1})_* = id_{\pi_1(S^1, x)} \]
	Quindi il nostro diagramma diventa:
	\begin{center}
		\begin{tikzcd}
			{\pi_1(S^1, x_0)} \arrow[r, bend left = 30, "i_*"] \arrow[d, dotted, "{\cong}" description] & {\pi_1(D^2,x_0)} \arrow[l, bend left = 30, "r_*"] \arrow[d, dotted, "{\cong}" description]\\
			\mathbb Z & \{1\}
		\end{tikzcd}
	\end{center}
	Ma questo è assurdo perché $\pi_1(D^2, x_0)$ è banale, quindi $r_* \circ i_*$ deve mandare tutto in $1_{\pi_1(S^1,x_0)}$, quindi non può essere un'identità
\end{proof}

\begin{ese}
	Se $r:X \to A$ è una retrazione, con $x_0 \in A \subseteq X$, allora, posta $i:A \hookrightarrow X$ inclusione:
	\[ i_*:\pi_1(A,x_0) \to \pi_1(X,x_0) \text{ è iniettiva} \]
\end{ese}

\begin{thm}{del punto fisso di Brower}{}
	Ogni funzione continua $f:D^2 \to D^2$ ha un punto fisso
\end{thm}
\begin{proof}
	Supponiamo per assurdo che non ci siano punti fissi, cioè che:
	\[ f(x)\neq x \quad \forall x \in D^2 \]
	Definiamo allora una retrazione $r:D^2 \to S^1$ nel seguente modo:
	\begin{center}
		\begin{tikzpicture}
			\draw (0,0) circle(1cm);
			\filldraw (0,-0.5) circle(1pt) node[left]{$f(x_0)$} (0.5,-0.25) circle(1pt) node[above]{$x_0$} (1,0) circle(1pt) node[right]{$r(x)$};
			\filldraw (0,1) circle(1pt) node[above]{$x_1 = r(x_1)$} (-0.5,0.5) circle(1pt) node[below]{$f(x_1)$};
			\draw (0,-0.5) -- (1,0) (0,1) -- (-0.5,0.5);
		\end{tikzpicture}
	\end{center}
	Cioè consideriamo i punti in $D^2$ e consideriamo la retta che unisce il punto nel dominio con la sua immagine. Prolunghiamo poi tale retta finché non incontri $S^1$, a quel punto abbiamo trovato $r(x)$. Questa è la retrazione che stavamo cercando. Inoltre è continua, in quanto, intuitivamente, spostando di poco il punto spostiamo di poco la retta. Tuttavia questo è assurdo, in quanto, per il teorema precedente \hyperref{nonretraz}[1.1.5], avevamo che non esistevano retrazioni $\gamma:D^2 \to S^1$.\\
	Più precisamente, possiamo definire $r$ nel seguente modo:
	\[ r(x) = x + \lambda(x-f(x))\qquad \text{per }\lambda \geq 0, x \in D^2 \]
	A questo punto ci resta da trovare $\lambda$ e per farlo ci basta imporre $\|r(x)\| = 1$, cioè dobbiamo risolvere:
	\[ \|x + \lambda(x-f(x))\|^2 = 1 \]
	\textit{Possiamo mettere il valore assoluto per comodità.} A questo punto ci basta risolvere quest'equazione di secondo grado e trovare il termine positivo. Infatti, l'equazione diventa:
	\[ \|x\|^2 + 2\lambda \langle x, x-f(x) \rangle + \lambda^2 \| x- f(x)\|^2 = 0 \]
\end{proof}

% Da finire

\begin{thm}{di Invarianza della Dimensione}{}
	$\mathbb R^2$ non è omeomorfo a $\mathbb R^n$ se $n \neq 2$
\end{thm}
\begin{proof}
	Supponiamo per assurdo di avere un omeomorfismo $f:\mathbb R^2 \to \mathbb R^n$. Se rimuoviamo un punto $x$ da $\mathbb R^2$, allora abbiamo che $f$ si restringe su $\mathbb R^2 \setminus \{x\}$ ed è ancora un omeomorfismo, cioè:
	\[ \mathbb R^2 \setminus \{x\} \cong \mathbb R^n \setminus \{f(x)\} \]
	Tuttavia sappiamo che:
	\[ \pi_1(\mathbb R^2 \setminus \{x\}) \cong \pi_1(S^1) \cong \mathbb Z \]
	Mentre abbiamo che:
	\[ \pi_1(\mathbb R^n \setminus \{ f(x) \}) \cong \pi_1(S^{n-1}) \cong
		\begin{cases}
			\{1\} & \text{se }n\neq 2\\
			\mathbb Z & \text{se }n = 2
		\end{cases}
	\]
\end{proof}

\begin{oss}
	Questo teorema vale anche per dimensioni maggiori di $2$. Infatti si possono fari dimostrazioni simili usando gruppi di omotopia superiori $\pi_k$ oppure usando gruppi di omologia $H_k$ invece di $\pi_1$
\end{oss}

\begin{thm}{Fondamentale dell'Algebra}{}
	Ogni polinomio complesso $f(x) \in \mathbb C[x]$ di grado $\geq 1$ ha almeno una radice, in particolare se è di grado $n$ ha $n$ radici contate con le loro molteplicità
\end{thm}

\begin{proof}[Idea della Dimostrazione]
	Supponiamo per assurdo che esista un polinomio senza radici, cioè che $f(x) \neq 0, \forall x \in \mathbb C$. Immaginiamo di prendere di prendere un cammino chiuso $\gamma$ in $\mathbb C$ che abbracci l'origine e che sia "sufficientemente lontano" da $0$. Per $|x|$ grande abbiamo che possiamo approssimare $f(x) = x^n$, se $f(x)$ è monico. Quindi per $|x|$ grande, possiamo dire che $f \circ \gamma$ è un cammino che fa $n$ giri attorno all'origine in $\mathbb C^*$, \textit{in quanto l'argomento del numero complesso viene moltiplicato per $n$}.
	Tuttavia, questo è assurdo, in quanto abbiamo che:
	\[ [\gamma] \in \pi_1(\mathbb C) \qquad \mapsto \qquad [f \circ \gamma] \in \pi_1(\mathbb C^*) \]
	Ma questo è un omomorfismo di gruppi che manda l'identità in un elemento che non è l'elemento neutro di $\pi_1(\mathbb C^*)$. Questo è assurdo, quindi deve avere necessariamente avere una radice
\end{proof}

\subsection{Rivestimenti}

L'esempio di $\mathbb R$ con $S^1$ è un esempio di rivestimento. Guardiamola nel dettaglio:
\begin{center}
	\begin{tikzpicture}[domain = -3:3]
		\draw[decoration={aspect=0.3, segment length=7.5mm, amplitude=1cm,coil},decorate,opacity=0.9, thin, rotate = -90] (0,0) -- (4,0);
		\draw[thick, rotate = -90] (2,0) circle(0.25cm and 1cm);
	\end{tikzpicture}
\end{center}
Sia $\gamma$ un cammino chiuso in $S^1$, in particolare $\gamma \in \Omega(S^1,1,1)$ dove $1$ viene visto come elemento in $\mathbb C$. Allora, per un teorema visto in Geometria 2, possiamo sollevare in modo unico un cammino $\tilde \gamma$ che parte da $0 \in \mathbb R$. (\textit{È lo stesso principio della formula 1.}) In particolare, con un diagramma commutativo, abbiamo che:
\begin{center}
	\begin{tikzcd}
		& \mathbb R \arrow[d, "p"]\\
		{[0,1]} \arrow[ur, dashed, "\tilde \gamma"] \arrow[r, "\gamma"] & S^1
	\end{tikzcd}
\end{center}
Abbiamo quindi che $\tilde \gamma$ è un sollevamento se e solo se $\gamma = p \circ \gamma$, cioè se il diagramma commuta. Per definizione di sollevamento, abbiamo quindi che $\tilde \gamma(1) \in \mathbb Z$. Grazie a questa cosa, possiamo definire un isomorfismo tra $\pi_1(S^1,1)$ e $\mathbb Z$. (Infatti ci basta considerare un qualsiasi $[\gamma]$ elemento del gruppo fondamentale e considerare $\tilde \gamma(1)$)

Questa mappa è un sollevamento e quest'idea dei sollevamenti verrà applicata in generale ai rivestimenti.

\begin{defn}{Rivestimento}{}
	Un \textbf{Rivestimento} di uno spazio topologico $X$ è uno spazio $\tilde X$ con una mappa di "proiezione" $p:\tilde X \to X$ tale che ogni $x \in X$ ha un intorno $U$ (detto \textbf{Aperto Banalizzante} o \textbf{Aperto Ben Rivestito}) tale che $p^{-1}(U)$ è unione disgiunta di aperti di $\tilde X$, ciascuno mappato omeomorficamente su $U$ tramite $p$, cioè:
	\[ p^{-1} = \coprod_{i \in I} \tilde U_i, \tilde U_i \subseteq \tilde X \text{ aperto}\qquad \text{con }p|_{\tilde U_i}:\tilde U_i \to U \text{ omeomorfismi} \]
\end{defn}

\textit{In parole semplici, $\tilde X$ è uno spazio in cui la topologia locale coincide con quella di $X$, ma non quella globale}

\begin{oss}
	Dato $p:\tilde X \to X$ rivestimento, la cardinalità di $p^{-1}(x)$, per $x \in X$, che viene detta \textbf{Numero di Fogli sopra }$x$, è localmente costante. Quindi è costante se $X$ è connesso.
\end{oss}

\begin{defn}{Locale Costanza}{}
	Una funzione $f:X \to Y$ è detta \textbf{Localmente Costante} (tra $X$ spazio topologico e $Y$ insieme qualsiasi) se ogni $x \in X$ ha un intorno $U$ su cui $f$ è costante. Equivalentemente se $f$ è continua dando a $Y$ la topologia discreta.
\end{defn}

\begin{lemma}{}{}
	Se $f:X \to Y$ è localmente costante e $X$ è connesso, allora $f$ è costante.
\end{lemma}
\begin{proof}
	Sia $X$ connesso, allora, essendo $f$ continua, abbiamo che $f(X)$ è commesso rispetto alla topologia indotta da $Y$. Ma la topologia di $Y$ è quella discreta, quindi abbiamo necessariamente che $|f(X)| = 1$, quindi $f$ è costante.
\end{proof}

Nel caso del numero di fogli, abbiamo che $Y = \{\text{ cardinalità }\leq |X|\}$, oppure più semplicemente $\mathbb N \cup \{+\infty\}$. Tutto questo è diretta conseguenza della definizione di rivestimento.

\begin{es}
	Banalmente lo stesso insieme $X \amalg X$ è un rivestimento di $X$. Ma non solo con due:
	\[ \coprod_{n \in N}X \text{ è ancora un rivestimento di }X \]
	Questo perché gli stessi aperti di $X$ sono aperti banalizzanti
\end{es}

\begin{es}
	Consideriamo $p:\mathbb R \to S^1$ tramite $p(x) = e^{2 \pi ix}$. Per verificare che è un rivestimento, possiamo usare due aperti banalizzanti che ricoprono $S^1$
	\begin{center}
		\begin{tikzpicture}
			\draw[thick] (0,0) circle (1cm);
			\filldraw[green, very nearly transparent] (100:0.9) arc (100:-100:0.9) to[out = 170, in = 170] (-100:1.1) arc (-100:100:1.1) to[out = 190, in = 190] (100:0.9);
			\filldraw[blue, very nearly transparent] (80:0.9) arc (80:280:0.9) to[out = 10, in = 10] (-80:1.1) arc (280:80:1.1) to[out = -10, in = -10] (80:0.9);
		\end{tikzpicture}
	\end{center}
	In questo caso, le controimmagini $p^{-1}(U)_1$ e $p^{-1}(U_2)$ sono unioni disgiunte di segmenti aperti di $\mathbb  R$. In questo caso abbiamo che il numero di fogli è pari a $\aleph_0$
\end{es}

\begin{es}
	Anche la mappa $\mathbb C \xrightarrow{exp} \mathbb C^*$ è un rivestimento, infatti possiamo vederlo come:
	\begin{center}
		\begin{tikzcd}
			\mathbb C \arrow[d, dotted, "\cong" description] \arrow[r, "exp"] & {\mathbb C^*} \arrow[d, dotted, "\cong" description]\\
			{\mathbb R \times \mathbb R} & {\mathbb R^+ \times S^1}
		\end{tikzcd}
	\end{center}
	In particolare abbiamo che:
	\[ z \in \mathbb C, z = x + 2\pi i y \quad \mapsto \quad e^{x + 2\pi iy} = e^{x} e^{2 \pi i y} \]
	Quindi, sfruttando la proprietà universale del prodotto, può essere vista come un prodotto di due funzioni, una $\mathbb R \to \mathbb R^+$, che manda $x$ in $e^x$ che è un omeomorfismo, e una mappa $\mathbb R \to S^1$, che manda $y \in e^{i y}$ che è un rivestimento. Essendo questi due rivestimenti (in generale un omeomorfismo è un rivestimento), allora anche il prodotto stesso è un rivestimento
\end{es}

\begin{ese}
	Dati $p_1:\tilde X_1 \to X_1$ e $p_2:\tilde X_2 \to X_2$ rivestimenti, allora:
	\[ (p_1,p_2):\tilde X_1 \times \tilde X_2 \to X_1\times X_2 \text{ è un rivestimento} \]
\end{ese}

\begin{es}
	Una funzione $S^1 \to S^1$. Dato $n\geq 1$ naturale, $p(z) = z^n$. Per esempio con $n=3$ si ha che:
	\begin{center}
		\begin{tikzpicture}
			\draw[domain=0:1, smooth, variable = \t,white, very thick, double=black] plot({-sin(\t*(30))}, {1 + \t*(30/1080-1)}, {-cos(\t*(30))});
			\draw[domain=30:360,smooth,variable=\t,white,very thick,double=black] plot({-sin(\t)},\t/1080,{-cos(\t)});
			\draw[domain=360:720,smooth,variable=\t,white,very thick,double=black] plot({-sin(\t)},\t/1080,{-cos(\t)});
			\draw[domain=720:1080,smooth,variable=\t,white,very thick,double=black] plot({-sin(\t)},\t/1080,{-cos(\t)});
			\draw[domain=0:360,smooth,variable=\t,white,very thick,double=black] plot({-sin(\t)},-1,{-cos(\t)});
			\draw[->] (0,0,0) -- (0,-1,0) node[pos = 0.35, right]{$p$};
			\draw (1.75,0.5) node{$n=3$} (1.5,-1) node{$S^1$};
		\end{tikzpicture}
	\end{center}
\end{es}

\begin{oss}
	Tutti i rivestimenti connessi $\tilde X$ di $S^1$ sono o $\mathbb R$ o $S^1$. Per averli sconnessi ci basta prendere l'unione disgiunta di $\mathbb R$ oppure di $S^1$
\end{oss}

\begin{defn}{Numero di Fogli}{}
	Il \textbf{Numero di Fogli}, sopra ad un punto $x \in X$ di un rivestimento $p:\tilde X \to X$ è $|p^{-1}(x)|$
\end{defn}

Grazie alla connessione, abbiamo che il numero di fogli è costante. Per esempio nel seguente ricoprimento, si ha che il numero di fogli è costante ed è uguale a $3$
\begin{center}
	\begin{tikzpicture}
		\draw[domain=0:1, smooth, variable = \t,white, very thick, double=black] plot({-sin(\t*(30))}, {1 + \t*(30/1080-1)}, {-cos(\t*(30))});
		\draw[domain=30:360,smooth,variable=\t,white,very thick,double=black] plot({-sin(\t)},\t/1080,{-cos(\t)});
		\draw[domain=360:720,smooth,variable=\t,white,very thick,double=black] plot({-sin(\t)},\t/1080,{-cos(\t)});
		\draw[domain=720:1080,smooth,variable=\t,white,very thick,double=black] plot({-sin(\t)},\t/1080,{-cos(\t)});
		\draw[domain=0:360,smooth,variable=\t,white,very thick,double=black] plot({-sin(\t)},-1,{-cos(\t)});
		\draw[thin, dashed] (0,-1,1) -- (0,1,1);
		\filldraw (0,-1,1) circle(1pt) (0, 180/1080, 1) circle(1pt) (0,540/1080,1) circle(1pt) (0,900/1080,1) circle(1pt);
	\end{tikzpicture}
\end{center}
Se, invece, così non fosse, esisterebbe un punto che non avrebbe un intorno banalizzante, quindi verrebbe a mancare la definizione di rivestimento. Se invece lo spazio di partenza non è connesso, allora il numero di fogli varia a seconda del punto scelto:

\begin{es}[$X$ non connesso]
	Un esempio semplice di insieme non connesso è $X = S^1 \vee S^1$, che può avere un ricoprimento tale che una circonferenza abbia $3$ fogli mentre l'altra solo $2$.
	\begin{center}
		\begin{tikzpicture}
			\draw[domain=0:1, smooth, variable = \t,white, very thick, double=green] plot({-sin(\t*(30))}, {1 + \t*(30/1080-1)}, {-cos(\t*(30))});
			\draw[domain=30:360,smooth,variable=\t,white,very thick,double=green] plot({-sin(\t)},\t/1080,{-cos(\t)});
			\draw[domain=360:720,smooth,variable=\t,white,very thick,double=green] plot({-sin(\t)},\t/1080,{-cos(\t)});
			\draw[domain=720:1080,smooth,variable=\t,white,very thick,double=green] plot({-sin(\t)},\t/1080,{-cos(\t)});
			\draw[domain=0:360,smooth,variable=\t,white,very thick,double=green] plot({-sin(\t)},-1,{-cos(\t)});
			\draw[domain=0:1, smooth, variable = \t,white, very thick, double=blue] plot({-sin(\t*(30)) +2.5}, {2/3 + \t*(30/1080-2/3)}, {-cos(\t*(30))});
			\draw[domain=30:360,smooth,variable=\t,white,very thick,double=blue] plot({-sin(\t)+2.5},\t/1080,{-cos(\t)});
			\draw[domain=360:720,smooth,variable=\t,white,very thick,double=blue] plot({-sin(\t)+2.5},\t/1080,{-cos(\t)});
			\draw[domain=0:360,smooth,variable=\t,white,very thick,double=blue] plot({-sin(\t)+2.5},-1,{-cos(\t)});
		\end{tikzpicture}
	\end{center}
\end{es}

Un esempio interessante può essere dato dal rivestimento di $S^1 \vee S^1$

\begin{es}[$X=S^1 \vee S^1$]
	Vediamo un po' di rivestimenti di $X$. In particolare, se possiamo vedere $X$ come:
	\begin{center}
		\begin{tikzpicture}
			\begin{scope}[decoration={
				markings,
				mark=at position 0.5 with {\arrow{>}}}]
				\draw[thick, postaction = {decorate}, blue] (0,0) arc(0:-360:1cm) node[pos = 0.5, left]{$a$} node(a)[pos = 0.85]{};
				\draw[thick, postaction = {decorate}, green] (0,0) arc(-180:180:1cm) node[pos = 0.5, right]{$b$};
			\end{scope}
			\filldraw (0,0) circle(2pt);
			\filldraw[red, very nearly transparent] (0,0) circle(0.5cm);
			\filldraw[purple] (a) circle(2pt);
		\end{tikzpicture}
	\end{center}
	Un esempio di rivestimento può essere:
	\begin{center}
		\begin{tikzpicture}
			\begin{scope}[decoration={
				markings,
				mark=at position 0.5 with {\arrow{>}}}]
				\draw[thick, postaction = {decorate}, blue] (0,0) arc(0:-360:1cm) node[pos = 0.5, left]{$a$} node(a)[pos = 0.85]{};
				\draw[thick, postaction = {decorate}, green] (0,0) arc(180:0:1cm) node[pos = 0.5, above]{$b$};
				\draw[thick, postaction = {decorate}, green] (2,0) arc(0:-180:1cm) node[pos = 0.5, below]{$b$};
				\draw[thick, postaction = {decorate}, blue] (2,0) arc(180:-180:1cm) node[pos = 0.5, right]{$a$} node(b)[pos = 0.85]{};
			\end{scope}
			\filldraw (0,0) circle(2pt) (2,0) circle(2pt);
			\filldraw[red, very nearly transparent] (0,0) circle(0.5cm) (2,0) circle(0.5cm);
			\filldraw[purple] (a) circle(2pt);
			\filldraw[purple] (b) circle(2pt);
		\end{tikzpicture}
	\end{center}
	Possiamo vedere che questo è facilmente un rivestimento in quanto gli aperti banalizzanti sono principalmente di tre tipi:
	\begin{itemize}
		\item La stessa circonferenza {\color{blue} $a$} tolto il punto di contatto delle circonferenze (equivalentemente ogni suo aperto)
		\item La stessa circonferenza {\color{green} $b$} tolto il punto di contatto delle circonferenze (in modo analogo con ogni suo aperto)
		\item Un qualsiasi aperto simile a quello rosso in figura, purché non contenga completamente nessuna delle circonferenze
	\end{itemize}
	Nei primi due casi abbiamo che l'aperto ottenuto dalla controimmagine della mappa $p$ del rivestimento coincide esattamente ad un tratto di circonferenza (che sia di $a$ o che sia di $b$, purché siano le stesse). Nel terzo caso invece abbiamo invece che ogni aperto contiene $4$ tratti di circonferenza: un tratto entrante e uno uscente della circonferenza $a$ e uno entrante e uno uscente della circonferenza $b$. Se le frecce di $b$ fossero state invertite, non sarebbe più stato omeomorfismo, perché preso un qualunque aperto in $S^1\vee S^1$, contenente solamente $3$ archi di circonferenza, avrebbe avuto come controimmagine un aperto contenente $4$ archi di circonferenza. Non essendo omeomorfismo, non sarebbe stato neanche rivestimento. Tuttavia, in questo caso abbiamo che il numero di fogli è $2$.

	Questo non è l'unico esempio di rivestimento di $S^1 \vee S^1$, in quanto anche il seguente è un rivestimento di $X$:
	\begin{center}
		\begin{tikzpicture}
			\begin{scope}[decoration={
				markings,
				mark=at position 0.5 with {\arrow{>}}}]
				\draw[thick, postaction = {decorate}, blue] (-2,0) arc(180:0:1cm) node[pos = 0.5, above]{$a$};
				\draw[thick, postaction = {decorate}, blue] (0,0) arc(0:-180:1cm) node[pos = 0.5, below]{$a$};
				\draw[thick, postaction = {decorate}, green] (0,0) arc(180:0:1cm) node[pos = 0.5, above]{$b$};
				\draw[thick, postaction = {decorate}, green] (2,0) arc(0:-180:1cm) node[pos = 0.5, below]{$b$};
				\draw[thick, postaction = {decorate}, blue] (2,0) arc(180:0:1cm) node[pos = 0.5, above]{$a$};
				\draw[thick, postaction = {decorate}, blue] (4,0) arc(0:-180:1cm) node[pos = 0.5, below]{$a$};
				\draw[thick, postaction = {decorate}, green] (4,0) arc(180:0:1cm) node[pos = 0.5, above]{$b$};
				\draw[thick, postaction = {decorate}, green] (6,0) arc(0:-180:1cm) node[pos = 0.5, below]{$b$};
			\end{scope}
			\draw[thick, green] (-2,0) arc(0:60:1cm) (-2,0) arc(0:-60:1cm);
			\draw[thick, blue] (6,0) arc(180:120:1cm) (6,0) arc(180:240:1cm);
			\filldraw (0,0) circle(2pt) (2,0) circle(2pt) (-2,0) circle(2pt) (4,0) circle(2pt) (6,0) circle(2pt);
			\filldraw[red, very nearly transparent] (0,0) circle(0.5cm) (2,0) circle(0.5cm) (-2,0) circle(0.5cm) (4,0) circle(0.5cm) (6,0) circle(0.5cm);
			\draw (-3,0) node{$\cdots$} (7,0) node{$\cdots$};
		\end{tikzpicture}
	\end{center}
	Qui invece abbiamo che il numero di fogli è $+\infty$. Ovviamente, come nel caso precedente, si deve avere che le frecce devono essere orientate in modo diverso, altrimenti non si ha un omeomorfismo tra gli aperti banalizzanti.

	Un altro esempio di rivestimento a due fogli può essere:
	\begin{center}
		\begin{tikzpicture}
			\begin{scope}[decoration={
				markings,
				mark=at position 0.5 with {\arrow{>}}}]
				\draw[thick, postaction = {decorate}, blue] (0,1) arc(90:270:0.5cm and 1cm);
				\draw[thick, postaction = {decorate}, green] (0,1) arc(90:-90:1cm);
				\draw[thick, postaction = {decorate}, green] (0,-1) arc(-90:90:0.5cm and 1cm);
				\draw[thick, postaction = {decorate}, blue] (0,-1) arc(-90:-270:1cm);
			\end{scope}
			\filldraw (0,1) circle(2pt) (0,-1) circle(2pt);
			\filldraw[red, very nearly transparent] (0,1) circle(0.5cm) (0,-1) circle(0.5cm);
		\end{tikzpicture}
	\end{center}
	Un esempio di sollevamento a $3$ fogli può essere:
	\begin{center}
		\begin{tikzpicture}
			\begin{scope}[decoration={
				markings,
				mark=at position 0.5 with {\arrow{>}}}]
				\draw[thick, postaction = {decorate}, blue] (0,0) arc(0:-360:1cm) node[pos = 0.5, left]{$a$};
				\draw[thick, postaction = {decorate}, green] (0,0) arc(180:0:1cm) node[pos = 0.5, above]{$b$};
				\draw[thick, postaction = {decorate}, green] (2,0) arc(0:-180:1cm) node[pos = 0.5, below]{$b$};
				\draw[thick, postaction = {decorate}, blue] (2,0) arc(180:0:1cm) node[pos = 0.5, above]{$a$};
				\draw[thick, postaction = {decorate}, blue] (4,0) arc(0:-180:1cm) node[pos = 0.5, below]{$a$};
				\draw[thick, postaction = {decorate}, green] (4,0) arc(180:-180:1cm) node[pos = 0.5, right]{$b$};
			\end{scope}
			\filldraw (0,0) circle(2pt) (2,0) circle(2pt) (4,0) circle(2pt);
		\end{tikzpicture}
	\end{center}
	In questo caos possiamo notare che se prendiamo un cammino chiuso lungo $a$ in $S^1 \vee S^1$, allora nel suo sollevamento, tale cammino può essere aperto come può essere chiuso. Dipende dalla controimmagine del punto
\end{es}

\begin{ese}
	Sia $\tilde X$ un qualsiasi grafo orientato con archi etichettati con $a$ e $b$ tali che ogni punto abbia un arco $a$ entrante, un arco $a$ uscente, un arco $b$ entrante e un arco $b$ uscente, allora $p:\tilde X \to S^1\vee S^1$, definita da orientazione ed etichette, è un rivestimento. Inoltre tutti i rivestimenti di $S^1 \vee S^1$ hanno questa struttura
\end{ese}

\begin{es}[Rivestimento Semplicemente Connesso di $S^1 \vee S^1$]
	Esiste un rivestimento semplicemente connesso di $S^1 \vee S^1$:
	\begin{center}
		\begin{tikzpicture}
			\draw l-system [l-system={cayley, axiom=[F] [+F] [-F] [++F], step=1cm, order=4}];
		\end{tikzpicture}
	\end{center}
	Dove ogni arco orizzontale coincide con $a$, orientato verso destra, mentre ogni arco verticale coincide con $b$, orientato verso l'alto. In questo modo abbiamo che ogni punto ha un intorno con quattro archi, definito come sopra. Questa figura può essere vista o come grafo o come elemento in $\mathbb R^2$
\end{es}

\begin{defn}{Rivestimento Universale}{}
	Un rivestimento semplicemente connesso si chiama \textbf{Rivestimento Universale}
\end{defn}

\begin{oss}
	Il rivestimento universale di uno spazio topologico $X$ riveste qualsiasi altro rivestimento di $X$
\end{oss}

\begin{es}
	$\mathbb R$ è un rivestimento universale di $S^1$, così come $\mathbb C$ è un rivestimento universale di $\mathbb C^*$
\end{es}

\begin{oss}
	Se $p:\tilde X \to X$ è un qualsiasi rivestimento, allora abbiamo che $p_*:\pi_1(\tilde X, \tilde x_0) \to \pi_1(X,x_0)$ è iniettiva. Potremo quindi interpretare che $\pi_1(\tilde X, \tilde x_0)$ è un sotto gruppo di $\pi_1(X,x_0)$. Dimostreremo poi che c'è una corrispondenza fra rivestimenti e gruppi.
\end{oss}

\begin{es}
	Posto $X = S^1$, abbiamo allora che $\mathbb R \to S^1$ è un rivestimento che manda l'unico elemento di $\pi_1(\mathbb R)$, l'elemento neutro, nell'elemento neutro di $S^1$, mentre $S^1 \to S^1$, tale che $z \mapsto z^n$, manda $[1] \mapsto [n]$, cioè è il sottogruppo $n \mathbb Z$
\end{es}

\begin{oss}
	Dall'osservazione precedente, abbiamo che $\mathbb F_2= \mathbb Z * \mathbb Z = \pi_1(S^1 \vee S^1)$ contiene sottogruppi liberi di $m$ generatori, per $m \geq 2$.
\end{oss}

\begin{oss}
	Alcuni rivestimenti sembrano più simmetrici di altri. Dimostreremo poi che i loro gruppi fondamentali non sono altro che sottogruppi normali del gruppo fondamentale
\end{oss}

\end{document}
