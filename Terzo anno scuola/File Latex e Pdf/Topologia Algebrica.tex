\documentclass[11pt, a4paper, twoside]{article}

% Pacchetti Base
\usepackage[italian]{babel}
\usepackage{amsfonts}
\usepackage{amssymb}
\usepackage{tikz}
\usetikzlibrary{cd}
\usepackage{extarrows}
\usepackage{mathrsfs}
\usepackage{amsthm}

% Estetica
\usepackage{tcolorbox}
\tcbuselibrary{theorems}
\usepackage{hyperref}
\usepackage{graphicx}
\graphicspath{ {./Immagini/} }
\usepackage{bm}
\usepackage{fouriernc}
\usepackage[T1]{fontenc}
\usepackage{tikzrput}
\usepackage[object=vectorian]{pgfornament}
\usepackage{xcolor}
\usepackage{fourier-orns}
\usepackage{fancyhdr}
\renewcommand{\headrule}{%
\vspace{-8pt}\hrulefill
\raisebox{-2.1pt}{\quad\decofourleft\decotwo\decofourright\quad}\hrulefill}
\usetikzlibrary{cd,decorations.pathmorphing,patterns}
\usepackage[T1]{fontenc}

% Pacchetti TCB
\tcbuselibrary{documentation}
\tcbuselibrary{breakable}

% Definizioni e Teoremi
\newtcbtheorem
	[number within = subsection]% init options
	{defn}% name
	{Definizione}% title
	{%
		colback=teal!5,
		colframe=teal!90!black!95,
		fonttitle=\bfseries,
	}% options
	{def}% prefix

\newtcbtheorem
	[use counter from = defn, number within = subsection]% init options
	{thm}% name
	{Teorema}% title
	{%
		colback=blue!5,
		colframe=blue!90!black!95,
		fonttitle=\bfseries,
	}% options
	{th}% prefix

\newtcbtheorem
	[use counter from = defn, number within = subsection]% init options
	{prop}% name
	{Proposizione}% title
	{%
		colback=red!5,
		colframe=red!90!black!95,
		fonttitle=\bfseries,
	}% options
	{pr}% prefix

\newtcbtheorem
	[use counter from = defn, number within = subsection]% init options
	{lemma}% name
	{Lemma}% title
	{%
		colback=red!5,
		colframe=red!90!black!95,
		fonttitle=\bfseries,
	}% options
	{le}% prefix

\newtcbtheorem
	[use counter from = defn, number within = subsection]% init options
	{cor}% name
	{Corollario}% title
	{%
		colback=red!5,
		colframe=red!90!black!95,
		fonttitle=\bfseries,
	}% options
	{co}% prefix

\newcommand{\vareps}{\varepsilon}
\newtheorem{es}{Esempio}
\theoremstyle{definition}
\newtheorem*{oss}{Osservazione}
\newtheorem{ese}{Esercizio}

\tcolorboxenvironment{ese}{
	colframe=black}
\newenvironment{sol}
	{\renewcommand\qedsymbol{$\blacksquare$}\begin{proof}[Soluzione]}
	{\end{proof}}

\tcolorboxenvironment{proof}{% `proof' from `amsthm'
	blanker,breakable,left=2.5mm,
	before skip=10pt,after skip=10pt,
	borderline west={0.5mm}{0pt}{black}}

% Formattazione Pagine
\fancyhf{}
\fancyhead[LE]{\nouppercase{\rightmark\hfill\leftmark}}
\fancyhead[RO]{\nouppercase{\leftmark\hfill\rightmark}}
\fancyfoot[LE,RO]{\hrulefill\raisebox{-2.1pt}{\quad\thepage\quad}\hrulefill}
\setlength{\headheight}{16pt}

% Margini
\topmargin=-0.45in
\evensidemargin=-0.15in
\oddsidemargin=-0.15in
\textwidth=6.5in
\textheight=9.0in
\headsep=0.25in

%Titolo
\newcommand{\titolo}[1]{
	\thispagestyle{empty}
	\vspace*{\fill}
	\tikzset{pgfornamentstyle/.style={draw = black, fill = teal!50}}
	\unitlength=1cm
	\begin{center}
	\begin{picture}(12,12)%
		\color{white}%
			\put(0,0){\framebox(12,12){%
			\rput[tl](-4,6){\pgfornament[width=8cm]{71}}%
			\rput[bl](-4,-6){\pgfornament[width=8cm,,symmetry=h]{71}}%
			\rput[tl](-6,6){\pgfornament[width=2cm]{63}}%
			\rput[tr](6,6){\pgfornament[width=2cm,,symmetry=v]{63}}%
			\rput[bl](-6,-6){\pgfornament[width=2cm,,symmetry=h]{63}}%
			\rput[br](6,-6){\pgfornament[width=2cm,,symmetry=c]{63}}%
			\rput[bl]{-90}(-6,4){\pgfornament[width=8cm]{46}}%
			\rput[bl]{90}(6,-4){\pgfornament[width=8cm]{46}}%
			\rput(0,0){\Huge \color{black} #1}%
			\rput[t](0,-0.5){\pgfornament[width=5cm]{75}}%
			\rput[b](0,0.5){\pgfornament[width=5cm]{69}}%
			\rput[tr]{-30}(-1,2.5){\pgfornament[width=2cm]{57}}%
			\rput[tl]{30}(1,2.5){\pgfornament[width=2cm,symmetry=v]{57}}}}%
	\end{picture}
	\end{center}
	\vspace*{\fill}
	\pagebreak
}

% Miscellanee
\author{Emanuele Fava}
\date{}
\pagestyle{fancy}

\newcommand{\bigslant}[2]{{\raisebox{.2em}{$#1$}\left/\raisebox{-.2em}{$#2$}\right.}}
\let\amsamp=&
\usetikzlibrary{shapes.geometric}
\usetikzlibrary {intersections}
\usetikzlibrary{calc}
\usepackage{physics}
\usepackage{mathtools}
\begin{document}

\titolo{Topologia Algebrica}

\tableofcontents

\newpage

\section{Primi passi sulla teoria dell'Omologia}

\subsection{Omologia Singolare}

\begin{defn}{Inviluppo Convesso della Base Canonica}{}
	Sia $\mathbb{R}^{n+1}$ spazio vettoriale reale e $\{e_o,\dots,e_n\}$ la base canonica di $\mathbb{R}^{n+1}$ allora chiamiamo $\text{Inviluppo convesso}$ di $\{e_o,\dots,e_n\}$ l'insieme:
	\[
		\Delta^n:=\{(t_0,\dots,t_n)\,|\,t_i\geq 0\,\,\forall i,\quad\sum\limits_{i=0}^n t_i=1\}
	\]
\end{defn}

\begin{es}{}
	Caso $n=0$:	

	\begin{center}
		\begin{tikzpicture}
			\draw[thick, ->] (-1,0) -- (4.5,0) node[anchor=north west] {$\mathbb{R}$};
			\foreach \Point/\PointLabel in {(2,0)/1}
			\draw[fill=black] \Point circle (0.07) node[anchor=north] {$\PointLabel$};
			\draw (0,0) node[anchor=north] {0};
			\draw (0,0) circle (0.03);
			\draw (2,0)  node[anchor=south west] 	{$\Delta^0$};
		\end{tikzpicture}
	\end{center}
\end{es}

\begin{es}{}
	Caso $n=1$:	

	\begin{center}
		\begin{tikzpicture}
			\draw[thick, ->] (0,0) -- (3,0) node[anchor=north west] {$\mathbb{R}$};
			\draw[thick, ->] (0,0) -- (0,3) node[anchor=south east] {$\mathbb{R}$};
			\draw[fill=red] (2,0) circle (0.07) node[anchor=north] {$e_0$};
			\draw[fill=red] (0,2) circle (0.07) node[anchor=east] {$e_1$};
			\draw[red, -] (2,0) -- (0,2);
			\draw[red] (1,1) node[anchor=south west] {$\Delta^1$};
		\end{tikzpicture}
	\end{center}
\end{es}

\begin{es}{}
	Caso $n=2$(Considerare solo la faccia della piramide rossa, compresi i bordi):	

	\begin{center}
		\begin{tikzpicture}
			\draw[thick, ->] (0,0) -- (3,0) node[anchor=north west] {$\mathbb{R}$};
			\draw[thick, ->] (0,0) -- (0,3) node[anchor=south east] {$\mathbb{R}$};
			\draw[thick, ->] (0,0) -- (-1.4,-1.4) node[anchor=south east] {$\mathbb{R}$};
			\draw[fill=red] (2,0) circle (0.07) node[anchor=north] {$e_1$};
			\draw[fill=red] (0,2) circle (0.07) node[anchor=east] {$e_2$};
			\draw[fill=red] (-0.7,-0.7) circle (0.07) node[anchor=east] {$e_0$};
			\filldraw[red,nearly transparent] (2,0) -- (0,2) -- (-0.7,-0.7) -- (2,0);
			\draw[red] (1,1) node[anchor=south west] {$\Delta^2$};
			\draw[thick,-] (2,0) -- (0,2) -- (-0.7,-0.7) -- (2,0) ;
		\end{tikzpicture}
	\end{center}
\end{es}

\begin{oss}
	data questa definizione possiamo notare che fissato $n$ abbiamo $n+1$ funzioni affini naturali 
\end{oss}

\begin{defn}{$\partial_i$ facce}{}
	$\forall i \in\{0,\dots, n\}$ definisco l'i-esima faccia dell'invilippo come la funzione:
		\begin{align*}
			\partial_i :\, &\Delta^{n-1}\rightarrow \Delta^n\\
			              &(t_0,\dots,t_n)\mapsto (t_0,\dots,t_{i-1},0,t_{i+1},\dots,t_n)
		\end{align*}

\end{defn}

\begin{oss}
	Ogni sottoinsieme I$\subseteq\{0,\dots,k\}$ individua una faccia di $\Delta^{|\text{I}|-1}$
\end{oss}

\begin{es}{}
	$n=2$:	

	\begin{center}
		\begin{tikzpicture}
			\draw[thick, ->] (0,0) -- (3,0) node[anchor=north west] {$\mathbb{R}$};
			\draw[thick, ->] (0,0) -- (0,3) node[anchor=south east] {$\mathbb{R}$};
			\draw[thick, ->] (0,0) -- (-1.4,-1.4) node[anchor=south east] {$\mathbb{R}$};
			\draw[fill=red] (2,0) circle (0.07) node[anchor=north] {$e_1$};
			\draw[fill=red] (0,2) circle (0.07) node[anchor=east] {$e_2$};
			\draw[fill=red] (-0.7,-0.7) circle (0.07) node[anchor=east] {$e_0$};
			\filldraw[red,nearly transparent] (2,0) -- (0,2) -- (-0.7,-0.7) -- (2,0);
			\draw[red] (1,1) node[anchor=south west] {$\Delta^2$};
			\draw[thick,-] (2,0) -- (0,2) -- (-0.7,-0.7) -- (2,0) ;
			\draw[thick, ->] (2,0) -- (1,1);
			\draw[thick, ->] (-0.7,-0.7) -- (0.65,-0.35);
			\draw[thick, ->] (-0.7,-0.7) -- (-0.35,0.65);
		\end{tikzpicture}
	\end{center}
	Quindi notiamo in questo caso abbiamo che:
	\begin{center}
		\begin{align*}
			\partial_0(t_0,t_1)&=(0,t_0,t_1)  & &\partial_0(1,0)=e_1,\,\partial_0(0,1)=e_2 \\
			\partial_1(t_0,t_1)&=(t_0,0,t_1)  & &\partial_1(1,0)=e_0,\,\partial_1(1,0)=e_2 \\
			\partial_2(t_0,t_1)&=(t_0,t_1,0)  & &\partial_2(1,0)=e_0,\,\partial_2(1,0)=e_1 
		\end{align*}
	\end{center}
\end{es}

\begin{defn}{$n$-simplesso singolare}{}
	Sia X uno spazio topologico un \textbf{n-simplesso singolare} è una funzione continua $\sigma$ tale che:
	 \[
		\sigma : \Delta^n\rightarrow X
	\]
\end{defn}

\begin{es}
	Le facce $\partial_i$ sono $n-1$ Simplessi singolari di $\Delta^n$
\end{es}

\begin{defn}{Gruppo Abeliano libero generato}{}
	Sia E un insieme il \textbf{Gruppo Abeliano libero generato da E } si definisce come:
	\[
		\mathbb{Z}^E=\{\varphi:E\to \mathbb{Z}\, | \, \varphi(e)=0 \text{ tranne che per un numero finito di elementi di E}\}
	\] 
\end{defn}

\begin{defn}{$n$-catene singolari}{} \label{catene singolari}
	Il Gruppo abeliano libero generato dagli n-simplessi singolari, indicato con $C_n(X)$ si dice il Gruppo delle \textbf{n-catene singolari} di X
\end{defn}
 
\begin{oss}
	Sia $\sigma\in C_n(X)\Rightarrow\sigma=a_1\sigma_1+\dots+a_k\sigma_k$ con $a_1,\dots,a_k\in\mathbb{Z}$ e $\sigma_1,\dots,\sigma_k:\Delta^n\rightarrow X$
\end{oss}

\begin{oss}
	Sia $f:X\rightarrow Y$ funzione continua tra spazi topologici allora se $\sigma :\Delta^n\rightarrow X$ è un $n$-simplesso singolare di X allora vale che $f\circ\sigma :\Delta^n\rightarrow Y$ è continua quindi è un $n$-simplesso singolare di Y.

	Quindi è ben posto il morfismo di Gruppi $f_*$: 

	\begin{align*}
		f_* :\,  C_n(X)&\rightarrow C_n(Y)\\
			\sigma&\mapsto f\circ\sigma 
	\end{align*}

\end{oss}

\begin{oss}
	Siano $v_0,\dots,v_k\in\mathbb{R}^n$ con $n>k$ allora esiste un unica funzione affine (ovvero lineare sommata ad una traslazione) che manda elementi di $\Delta^k$ nell'inviluppo convesso di $v_0,\dots,v_k$ t.c:
	\begin{align*}
		\Delta^k&\rightarrow\text{Inviluppo convesso di }v_0,\dots,v_k\\
		(t_0,\dots,t_n)&\mapsto \sum\limits_{i=0}^k t_i v_i\\
		e_0&\mapsto v_0\\
		&\vdots\\
		e_k&\mapsto v_k
	\end{align*}
	La mappa è omeomorfismo $\iff v_1-v_0,\dots,v_k-v_0$ sono linearmenti indipendenti. 
	Chiamiamo tale mappa $<v_0,\dots,v_k>$.
\end{oss}
\begin{defn}{$\partial$}{}\label{partial}
	Definiamo $\partial:\,C_n(X)\rightarrow C_{n-1}(X)$ come:
	\[
		\partial\sigma=\sum\limits_{i=0}^{n}(-1)^i\sigma\circ\partial_i=\sum\limits_{i=0}^{n}(-1)^i\sigma\vert_{<e_0,\dots,\hat{e_i},\dots,e_n>}
	\]
\end{defn}

La notazione finale si può interpretare come: restringo la funzione $\sigma$ alle sue facce togliendo un vertice.

\begin{es}
	Sia $\sigma\in C_{2}(X)$ allora vale:
	\begin{align*}
		\partial\sigma&=\sigma\vert_{<e_1,e_2>}-\sigma\vert_{<e_0,e_2>}+\sigma\vert_{<e_0,e_1>}\\
		\partial\sigma\vert_{<e_1,e_2>}&=\sigma\vert_{<e_2>}-\sigma\vert_{<e_1>}\\
		\partial\sigma\vert_{<e_0,e_2>}&=\sigma\vert_{<e_2>}-\sigma\vert_{<e_0>}\\
		\partial\sigma\vert_{<e_0,e_1>}&=\sigma\vert_{<e_1>}-\sigma\vert_{<e_0>}\\
		\partial^2\sigma&=\sigma\vert_{<e_2>}-\sigma\vert_{<e_1>}-(\sigma\vert_{<e_2>}-\sigma\vert_{<e_0>})+\sigma\vert_{<e_1>}-\sigma\vert_{<e_0>}\\
		\partial^2\sigma&=0
	\end{align*}
\end{es} 

\begin{prop}{$\partial^2=0$}{}
	\[
		\partial^2=0 
	\]
\end{prop}
\begin{proof}
	Utilizzando la seconda notazione il teorema è molto immediato poichè se prendiamo $\partial\sigma=\sum\limits_{i=0}^{n}(-1)^i\sigma\vert_{<e_0,\dots,\hat{e_i},\dots,e_n>}$ calcolare nuovamente la funzione $\partial$ su tale quantità mi darà somme (con segni) di $\sigma\vert_{<e_0,\dots,\hat{e_i},\dots,\hat{e_j},\dots,e_n>}$ ed ognuno di questi simplessi appare 2 volte:
	\begin{enumerate}
		\item  In $\partial\sigma\vert_{<e_0,\dots,\hat{e_i},\dots>}\rightarrow$ con segno $(-1)^{i+j-1}$
		\item  In $\partial\sigma\vert_{<e_0,\dots,\hat{e_j},\dots>}\rightarrow$ con segno $(-1)^{i+j}$
	\end{enumerate}
	Allora avendo segno opposto la somma si annulla a 2 a 2 ogni volta ottenendo la Tesi.
\end{proof}

\begin{oss}
	Questo fatto equivale a dire:
	\[
		\text{Im}(\partial:\,C_{n+1}(\text{X})\rightarrow C_n(\text{X}))\subseteq \text{Ker}(\partial:\, C_{n}(\text{X})\rightarrow C_{n-1}(\text{X}))
	\]
\end{oss}

\begin{defn}{$n$-esimo Gruppo di Omologia Singolare}{}\label{Commuta}
	 Sia X spazio topologico, chiamiamo l'$\textbf{n-esimo Gruppo di Omologia Singolare}$ di X:
	\[
		\text{H}_n(\text{X})=\bigslant{\text{Ker}(\partial:\, C_{n}(\text{X})\rightarrow C_{n-1}(\text{X}))}{\text{Im}(\partial:\,C_{n+1}(\text{X})\rightarrow C_n(\text{X}))}
	\] 
\end{defn}

\begin{prop}{}
	Sia $f:X\rightarrow Y$ funzione continua tra spazi topologici e sia $f_*:C_n(\text{X})\rightarrow  C_n(\text{Y})$ il morfismo di gruppi indotto da $f$ allora vale $\partial f_*=f_*\partial$. Equivalentemente il seguente diagramma commuta:
	\begin{center}
		\begin{tikzcd} 
			C_n(\text{X}) \arrow[r, "\partial"] \arrow[d, "f_*"] 
				& C_{n-1}(\text{X}) \arrow[d, "f_*" ] \\ 
			C_n(\text{Y}) \arrow[r, "\partial"] 
				&  C_{n-1}(\text{Y})
		\end{tikzcd}
	\end{center}
\end{prop}
\begin{proof}
	Sia $\sigma\in C_n(\text{X})$ allora vale:
	\begin{align*}
		f_*(\partial\sigma)&=\sum\limits_{i=0}^n(-1)^i f_*(\sigma\circ\partial_i)=\sum\limits_{i=0}^n(-1)^i f\circ(\sigma\circ\partial_i)\\
					 &=\sum\limits_{i=0}^n(-1)^i (f\circ\sigma)\circ\partial_i=\sum\limits_{i=0}^n(-1)^i f_*(\sigma)\circ\partial_i\\
					 &=\partial(f_*(\sigma))
	\end{align*}
\end{proof}

\begin{defn}{Cicli e Bordi}{}
	\begin{itemize}
		\item $\sigma \in \text{Ker}(\partial:\, C_{n}(\text{X})\rightarrow C_{n-1}(\text{X}))$ è detto $n$-ciclo
		\item $\sigma\in\text{Im}(\partial:\,C_{n+1}(\text{X})\rightarrow C_n(\text{X}))$ è detto $n$-bordo
	\end{itemize}
\end{defn}

\begin{thm}{Morfismi tra $\text{H}_n(\text{X})$ e $\text{H}_n(\text{Y})$}{}
	Sia $f:X\rightarrow Y$ funzione continua tra spazi topologici, allora $f$ induce un morfismo tra $\text{H}_n(\text{X})$ e $\text{H}_n(\text{Y})$. Equivalentemente è ben definita:

	\begin{align*}
		\text{H}_n(f):\,\text{H}_n(\text{X}) &\rightarrow \text{H}_n(\text{Y}): \\ 
			              \text{[c]}&\mapsto \text{[}f_*(c)\text{]}
	\end{align*}

\end{thm}
\begin{proof}
	Utilizziamo la Proposizione \hyperref[Commuta]{1.1.9} sugli n-cicli e sugli n-bordi:
	\begin{align*}
		&\text{Sia }\sigma \in \text{Ker}(\partial:\, C_{n}(\text{X})\rightarrow C_{n-1}(\text{X}))\Rightarrow \partial f_*(\sigma)=f_*\partial(\sigma)=0\Rightarrow f_*(\sigma)\in \text{Ker}(\partial:\, C_{n}(\text{Y})\rightarrow C_{n-1}(\text{Y}))\\
		&\text{Se c}=\partial\omega\Rightarrow f_*c=f_*\partial\omega=\partial(f_*\omega) \text{ quindi ottengo un n-bordo }
	\end{align*}
	Otteniamo così la Tesi.
\end{proof}
\newpage
\begin{oss}
	Si può dimostrare che $\text{H}_n$ è un funtore e quindi se tra 2 spazi topologici ho un Omeomorfismo esso induce un isomorfismo tra i loro Gruppi di Omologia Singolare.\footnote{Trovate la dimostrazione a pagina 67 di ''An Introduction to Algebraic Topology'' di Joseph J. Rotman}
\end{oss}

\begin{ese}
	Calcolare il Gruppo di Omologia Singolare di $X=\{p\}$ spazio topologico con un punto.
\end{ese}
\begin{sol}
	Notiamo per prima cosa che fissato un $k\in\mathbb{N}$ allora i k-simplessi singolari sono tutti funzioni costanti e quindi vale $C_k(\text{X})\cong\mathbb{Z}$ inoltre vale che: 
	\[
		\partial\sigma_k=\left[\sum\limits_{i=0}^k(-1)^i\right]\sigma_{k-1}=
		 \begin{cases}
			0 & \text{se k è dispari} \\
			1 & \text{se k è pari}
		\end{cases}
	\]
	\begin{center}
		\begin{tikzcd}
			\cdots \arrow[r] & \mathbb{Z}\arrow[r, "1" ] & \mathbb{Z}\arrow[r, "0" ] & \mathbb{Z}\arrow[r, "1"]& \mathbb{Z}\arrow[r, "0" ] & \mathbb{Z}\arrow[r]& 0
		\end{tikzcd}
	\end{center}
	Caso $k>0$:
	\begin{enumerate}
		\item $k$ pari allora non ci sono $k$-cicli $\Rightarrow \text{H}_k(\{p\})=0$
		\item $k$ dispari allora $\text{Ker}\partial=\mathbb{Z}$ e $\text{Im}\partial=\mathbb{Z}\Rightarrow \text{H}_k(\{p\})=0$
	\end{enumerate} 
	Caso $k=0$:
	\[
		 \text{Ker}\partial=\mathbb{Z} \quad \text{Im}\partial=0\quad\Rightarrow \quad\text{H}_0(\{p\})=\mathbb{Z}
	\]
	Il gruppo di omologia singolare di un punto è:
	\begin{itemize}
		\item $\text{H}_0(\{p\})=\mathbb{Z}$
		\item $\text{H}_k(\{p\})=0\quad \forall k>0$
	\end{itemize}
\end{sol}

\begin{prop}{}
	Sia X$=\coprod\limits_{\alpha}\text{X}_{\alpha}$ spazio topologico scomposto lungo le sue componenti connesse per archi. Allora:
	\[
		\forall k \quad \text{H}_k(\text{X})\cong\bigoplus\limits_{\alpha}\text{H}_k(\text{X}_{\alpha})
	\]
\end{prop}
\begin{proof}
	Se $\sigma:\Delta^k\rightarrow \text{X}$ è un $k$-simplesso singolare essendo $\Delta^k$ connesso per archi, allora dalla continuità di $\sigma$ segue che $\sigma(\Delta^k)$ è connesso per archi e quindi è totalmente contenuto in una delle componenti connesse per archi di X. Questo mi implica che:
	\[
		\text{C}_k(\text{X})\cong\bigoplus\limits_{\alpha}\text{C}_k(\text{X}_{\alpha})
	\]
	Inoltre  dalla continuità di $\partial$:
	\[
		\sigma \in\text{C}_k(\text{X}_{\alpha})\quad \Rightarrow \quad \partial\sigma\in\text{C}_{k-1}(\text{X}_{\alpha})
	\]
	Questo implica che il nucleo e l'immagine di $\partial$ possono essere decomposti lungo le componenti connesse per archi, così facendo otteniamo la Tesi.
\end{proof}

\subsection{Omologia Simpliciale}

\begin{defn}{Complesso Omologico}{}
	Un $\textbf{Complesso Omologico}$, o complesso a catena, è una coppia, $(\text{C}_{\bullet},d_{\bullet})$, formata da:
	\begin{enumerate}
		\item Una successione di Gruppi Abeliani  $\text{C}_k$;
		\item Morfismi $\text{d}_k:\,\text{C}_k\rightarrow\text{C}_{k-1}$ tale che $\forall k\in\mathbb{N},\quad d_{k-1}\circ d_{k}=0$
		\begin{center}
			\begin{tikzcd}
				\cdots \arrow[r] & \text{C}_k\arrow[r, "d_k" ] & \text{C}_{k-1}\arrow[r, "d_{k-1}" ] & \text{C}_{k-2}\arrow[r, "d_{k-2}"]& \cdots\arrow[r] & \text{C}_0\arrow[r]& 0
			\end{tikzcd}
		\end{center}
	\end{enumerate}
\end{defn}

\begin{es}
	Le catene singolari \hyperref[catene singolari]{1.1.5} e la funzione $\partial$ \hyperref[partial]{1.1.6} costituiscono un complesso omologico.
\end{es}

\begin{defn}{Cicli di un Complesso Omologico}{}
	Sia $(\text{C}_{\bullet},d_{\bullet})$ un Complesso Omologico allora definiamo i $k$-cicli di un Complesso Omologico come:
	\[
		\text{Z}_k:=\text{Ker}(d_k)
	\]
\end{defn}

\begin{defn}{Bordi di un Complesso Omologico}{}
	Sia $(\text{C}_{\bullet},d_{\bullet})$ un Complesso Omologico allora definiamo i $k$-bordi di un Complesso Omologico come:
	\[
		\text{B}_k:=\text{Im}(d_{k+1})
	\]
\end{defn}

\begin{oss}
	$d_{k-1}\circ d_k=0\Rightarrow$ ogni $k$-bordo è un $k$-ciclo
\end{oss}

\begin{defn}{Gruppo di Omologia singolare di un Complesso Omologico}
	Il Gruppo di Omologia singolare di un Complesso Omologico $(\text{C}_{\bullet},d_{\bullet})$ è definito come:
	\[
		\text{H}_k(\text{C}_{\bullet}):= \bigslant{\text{Z}_k}{\text{B}_k}
	\]
\end{defn}

\begin{defn}{Morfismi di Complessi Omologici}{}
	$\text{Siano } (\text{C}_{\bullet},d_{\bullet}) \text{ e }  (\text{D}_{\bullet},d'_{\bullet}) \text{ Complessi Omologici allora un } \textbf{Morfismo di Complessi Omologici }$ $\varphi_{\bullet}:\,\text{C}_{\bullet}\rightarrow\text{D}_{\bullet}$ è una famiglia $\varphi_k:\,\text{C}_{k}\rightarrow\text{D}_{k}$ di morfismi tale che $\forall k$ $d'_{k}\circ\varphi_{k}=\varphi_{k-1}\circ d_{k}$ equivalentemente il seguente diagramma commuti per ogni k:
	\begin{center}
		\begin{tikzcd}
			\text{C}_k \arrow[r, "\varphi_{k}"] \arrow[d, "d_k"] 
				& \text{D}_k \arrow[d, "d'_k" ] \\ 
			\text{C}_{k-1}\arrow[r, "\varphi_{k-1}"] 
				&  \text{D}_{k-1}
		\end{tikzcd}
	\end{center}
\end{defn}

\begin{oss}
	$\varphi_k$ manda $k$-cicli in $k$-cicli e $k$-bordi in $k$-bordi quindi tramite il funtore $\text{H}_k$ definisce un morfismo tra gruppi:
	\[
		\text{H}_k(\varphi_k):\,\text{H}_k(\text{C}_{\bullet})\rightarrow\text{H}_k(\text{D}_{\bullet})
	\]
\end{oss}

\begin{es}
	Sia $f:\,\text{X}\rightarrow \text{Y}$ funzione continua tra spazi topologici allora essa definisce un morfismo tra complessi omologici:
	\[
		f_*:\,\text{C}_{\bullet}(\text{X})\rightarrow\text{C}_{\bullet}(\text{Y})
	\]
\end{es}

\begin{defn}{Complesso Simpliciale Finito}{}
	Un $\textbf{Complesso Simpliciale Finito}$ è il dato di:
	\begin{enumerate}
		\item Un insieme finito I non vuoto;
		\item Una famiglia $\Sigma$ di sottoinsiemi di I (eccetto il vuoto) con le proprietà:
		\begin{enumerate}
			\item Ogni elementi di I, considerato come sottoinsieme di cardinalità 1, appartiene a $\Sigma$;
			\item se $\sigma\in\Sigma$ e $\tau\subseteq\sigma\quad\Rightarrow\quad\tau\in\Sigma$
		\end{enumerate}
	\end{enumerate}
\end{defn}

\begin{es}
	Sia $\text{I}=\{0,1,2,3,4\}$ allora se $\{0,2,3\}\in\Sigma\,\Rightarrow\,\{0,2\},\,\{0,3\},\,\{2,3\}\in\Sigma$ quindi avremmo in questo caso il complesso simpliciale $(I,\Sigma)$ dato da: 
	\begin{itemize}
		\item  $\text{I}=\{0,1,2,3,4\}$;
		\item  $\Sigma=\{\{0\},\{1\},\{2\},\{3\},\{4\},\{0,2,3\},\{0,2\},\{0,3\},\{2,3\}\}$.
	\end{itemize} 
\end{es}

Fissando una biezione di I è possibile associale ad un complesso simpliciale un complesso omologico: 
\begin{defn}{$l$-simplessi}{}
	Dato $(I,\Sigma)$ Complesso Simpliciale Finito definiamo gli $l$-simplessi:
	\[
		\Sigma_{l}=\{\sigma\in\Sigma\,:\,|\sigma|=l+1\}
	\]
\end{defn}

\begin{defn}{Gruppo Abeliano libero generato da $\Sigma_{l}$}{}
	Chiamiamo $\text{C}_l^{\Sigma}$ il Gruppo abeliano libero generato da $\Sigma_{l}$, ed i suoi elementi della base sono: 
	\[
		\sigma=<i_0,\dots,i_l>
	\]
\end{defn}

\begin{es}\label{es 10}
	Prendiamo $\text{I}=\{a,b,c\}$ e consideriamo $\Sigma=\mathscr{P}(\text{I})\setminus\{\varnothing\}$ allora troviamo che gli elementi della base sono:
	\begin{itemize}
		\item Per $C_0^{\Sigma}$ sono $\{<a>,<b>,<c>\}$ quindi $C_0^{\Sigma}\cong \mathbb{Z}^3$
		\item Per $C_1^{\Sigma}$ sono $\{<a,b>,<b,c>,<a,c>\}$ quindi $C_1^{\Sigma}\cong \mathbb{Z}^3$
		\item Per $C_2^{\Sigma}$ sono $\{<a,b,c>\}$ quindi $C_2^{\Sigma}\cong \mathbb{Z}$
	\end{itemize}
\end{es}

\begin{oss}
	Notiamo che $C_l^{\Sigma}\cong \mathbb{Z}^{|\Sigma_l|}$
\end{oss}

\begin{defn}{$\partial^\Sigma$}{}
	Definiamo la funzione $\partial^\Sigma:\,\text{C}_l^{\Sigma}\rightarrow\text{C}_{l-1}^{\Sigma}$ sugli elementi della base come: 
	\[
		\forall\sigma=<i_0,\dots,i_l>\quad \Rightarrow\quad\partial^\Sigma\sigma:=\sum\limits_{k=0}^l (-1)^k <i_0,\dots,\hat{i}_k,\dots,i_l>
	\]
\end{defn}

\begin{oss}
	Utilizzando la stessa dimostrazione di prima troviamo $\partial^\Sigma\partial^\Sigma=0$
\end{oss}

\begin{ese}
	Calcolare l'omologia di $C_{\bullet}^\Sigma$ nel caso di:
	\begin{enumerate}
		\item $I=\{a,b,c\}\quad\Sigma=\mathscr{P}(\text{I})\setminus\{\varnothing\}$;
		\item $I=\{a,b,c\}\quad\Sigma=\mathscr{P}(\text{I})\setminus\{\{\varnothing\},\{a,b,c\}\}$;
	\end{enumerate}
\end{ese}
\begin{sol}
	Partiamo scrivendo la catena del primo aiutandoci con le osservazioni precedenti:
	\begin{center}
		\begin{tikzcd}
			0 \arrow[r,"\partial^\Sigma_3"] & \text{C}_2^\Sigma\arrow[r, "\partial^\Sigma_2" ] & \text{C}_{1}^\Sigma\arrow[r, "\partial^\Sigma_1" ] & \text{C}_{0}^\Sigma\arrow[r, "\partial^\Sigma_0"]& 0
		\end{tikzcd}
	\end{center}
	Che grazie all'esempio \hyperref[es 10]{10} diventa:
	\begin{center}
		\begin{tikzcd}
			0 \arrow[r,"\partial^\Sigma_3"] & \mathbb{Z}\arrow[r, "\partial^\Sigma_2" ] &\mathbb{Z}^3\arrow[r, "\partial^\Sigma_1" ] & \mathbb{Z}^3\arrow[r, "\partial^\Sigma_0"]& 0
		\end{tikzcd}
	\end{center}
	Notiamo innanzitutto che il $Ker(\partial^\Sigma_0)\cong\mathbb{Z}^3$ quindi ora cerchiamo il $Ker(\partial^\Sigma_1)$ per determinarne anche l'immagine. Scriviamo una combinazione lineare degli elementi della base di $\text{C}_{1}^\Sigma$con coefficienti in $\mathbb{Z}$ come generico elemento di tale gruppo e andiamo a vedere quando tale elemento, calcolato con $\partial^\Sigma_1$ è nullo:
	\begin{align*}
		&\partial^\Sigma_1(a_1(<a,b>)+a_2(<b,c>)+a_3(<a,c>))=0\iff \\ 
		&a_1(<b>-<a>)+a_2(<c>-<b>)+a_3(<c>-<a>)=0 \iff
		\begin{cases}
			a_1+a_3=0\\ 
			a_1-a_2=0\\
			a_2+a_3=0
		\end{cases}
	\end{align*} 
	Il sistema si risolve facilmente in quanto la matrice associata ha rango 2 questo mi da necessariamente che $Ker(\partial^\Sigma_1)\cong\mathbb{Z}$ e che $Im(\partial^\Sigma_1)\cong\mathbb{Z}^2$.

	Quindi intanto abbiamo trovato $\text{H}_0^\Sigma=\bigslant{\mathbb{Z}^3}{\mathbb{Z}^2}\cong\mathbb{Z}$.
	Ora cerchiamo invece il $Ker(\partial^\Sigma_2)$ e quindi la sua immagine.
	Ragioniamo come prima ovvero: 
	\[
		\partial^\Sigma_2(a_1(<a,b,c>))=0\iff a_1(<b,c>-<a,c>+<a,b>)=0 \iff a_1=0
	\]
	Quindi la funzione è iniettiva e di conseguenza questo mi implica che $\text{Im}(\partial^\Sigma_2)\cong\mathbb{Z}$ quindi abbiamo trovato che:
	\begin{itemize}
		\item $\text{H}_0^\Sigma=\bigslant{\mathbb{Z}^3}{\mathbb{Z}^2}\cong\mathbb{Z}$;
		\item $\text{H}_1^\Sigma=\bigslant{\mathbb{Z}}{\mathbb{Z}}\cong\{0\}$
		\item $\text{H}_2^\Sigma=\bigslant{\{0\}}{\{0\}}\cong\{0\}$.
	\end{itemize}
	Il secondo punto si risolve in modo analogo e da come risultati:
	\begin{itemize}
		\item $\text{H}_0^\Sigma\cong\mathbb{Z}$;
		\item $\text{H}_1^\Sigma\cong\mathbb{Z}$;
		\item $\text{H}_2^\Sigma\cong\{0\}$.
	\end{itemize}
\end{sol}

\begin{defn}{$\Sigma^{(l)}$}{}
	Sia $(\text{I},\Sigma)$ complesso simpliciale con $\text{I}=\{0,\dots,n\}$ insieme dei vertici definiamo $\Sigma^{(l)}$ come:
	\[
		\Sigma^{(l)}=\{\sigma\in\Sigma:\,|\sigma|\leq l+1\}
	\]
\end{defn}

\begin{defn}{$l$-scheletro}{}
	Sia $\text{K}=(\text{I},\Sigma)$ complesso simpliciale chiamiamo $l$-scheletro di K il complesso simpliciale:
	\[
		\text{K}^{(l)}=(\text{I},\Sigma^{(l)})\quad l\geq 0
	\]
\end{defn}

\begin{oss}
	Se il complesso omologico associato a K è:
	\begin{center}
		\begin{tikzcd}
			\cdots \arrow[r] & 0 \arrow[r] & \mathbb{Z}^{|\Sigma_l|}\arrow[r] &\mathbb{Z}^{|\Sigma_{l-1}|} \arrow[r] & \cdots
		\end{tikzcd}
	\end{center}
	allora il complesso omologico associato a $\text{K}^{(l)}$ è:
	\begin{center}
		\begin{tikzcd}
			\cdots \arrow[r] & 0 \arrow[r] & \Sigma^{|\Sigma_l|}\arrow[r] & \Sigma^{|\Sigma_{l-1}|} \arrow[r] & \cdots
		\end{tikzcd}
	\end{center}
	Inoltre questo è un esempio di $\textbf{sottocomplesso simpliciale}$.
\end{oss}
Cerchiamo di costruire Geometricamente un complesso simpliciale e di conseguenza un complesso omologico, partiamo con le funzioni:
\begin{defn}{$|\sigma|$}{}
	Sia $\text{K}=(I,\Sigma)\text{ Complesso Simpliciale } \text{I}=\{0,\dots,n\}$ e consideriamo $\mathbb{R}^{n+1}$ con base canonica $\{e_0,\dots,e_n\}$, allora se abbiamo $\sigma\in\Sigma$ un elemento della base $\sigma=<i_0,\dots,i_k>$ allora definisco $|\sigma|$ come l'unica funzione affine tale che: \footnote{ Attenzione: non è il valore assoluto per indicare la cardinalità, da ora la cardinalità la indicheremo con $\#$}
	\begin{center}
		\begin{align*}
			\mid\sigma\mid:\Delta^k&\rightarrow\mathbb{R}^n+1\\
			e_0&\mapsto e_{i_0}\\
			\vdots&\quad\,\,\,\vdots\\
			e_k&\mapsto e_{i_k}
		\end{align*}
	\end{center}
\end{defn}

\begin{defn}{Realizzazione Geometrica di un Complesso Simpliciale}{}
	Sia $\text{K}=(I,\Sigma)$ complesso simpliciale allora chiamiamo 
	
$|\text{K}|\,\,\textbf{Realizzazione Geometrica di un Complesso Simpliciale}$:
	\[
		|\text{K}|\stackrel{def}=\bigcup\limits_{\sigma\in\Sigma} \mid\sigma\mid(\Delta^{\#\sigma-1})
	\]
	
\end{defn}
\newpage
\begin{oss}
	Insiemisticamente si può dimostrare che vale:
	\[
		|\text{K}|=\bigsqcup\limits_{\sigma\in\Sigma}\mid\sigma\mid(\text{int}(\Delta^{\#\sigma-1}))
	\]
	da ciò inoltre possiamo comprendere che $|\text{K}|$ nella topologia indotta è compatto e $\text{T}2$.
\end{oss}

\begin{oss}
	se $\sigma\in\Sigma$ allora $\#\sigma=l+1$ e $\sigma=<i_0,\dots,i_k>$ quindi la funzione $|\sigma|\in\text{C}_l^{\Sigma}(|\text{K}|)$ è un simplesso della base. 
	In questo modo possiamo definire un morfismo tra complessi ( tra il complesso omologico di K e il complesso omologico di $|\text{K}|$):
	\begin{align*}
		\text{C}_{\bullet}^{\Sigma}(\text{K})&\rightarrow\text{C}_{\bullet}(|\text{K}|)\\
		\sigma\in\Sigma_l&\mapsto|\sigma|
	\end{align*}
	Inoltre possiamo vedere che se ho $\partial\sigma=\sum\limits_{k=0}^{l}(-1)^k<i_0,\dots,\hat{i}_k,\dots,i_l>$ allora vale $|\partial\sigma|=\partial|\sigma|$.
\end{oss}

\begin{defn}{Triangolare}{}
	Uno spazio topologico si dice $\textbf{Triangolare}$ se è omeomorfo alla realizzazione geometrica di un complesso simpliciale.
\end{defn}

\begin{defn}{Triangolazione}{}
	Una $\textbf{Triangolazione}$ è un omeomorfismo tra uno spazio topologico e una realizzazione geometrica.
\end{defn}
\newpage
\subsection{$\Delta$-Complessi}

\begin{defn}{$\Delta$-Complesso finito}{}\label{due}
	Un $\Delta$-Complesso finito è uno spazio topologico X tale che $\forall k\geq 0$ è dato un insieme $I_k$ finito di indici e funzioni continue(potrebbe confondere ma stiamo chiamando l'insieme delle funzioni continue e degli indici allo stesso modo), $\varphi_\alpha^{(k)}:\Delta^k\rightarrow X\quad \alpha$ tale che: 
	\begin{enumerate}
		\item Ogni $\varphi_\alpha^{(k)}$ è omeomorfismo ristretta alla parte interna del $k$-simplesso e 
			\[
				\text{X}=\bigsqcup\limits_{\alpha\in I_k}\varphi_\alpha^{(k)}\left(\text{int}\Delta^k\right);
			\]
		\item Per ogni $\varphi_\alpha^{(k)}$, per ogni $i=0,\dots,k$ allora: 
			\[
				\varphi_\alpha^{(k)}\circ\partial_i\in I_{k-1}\quad \text{(dove }I_{k-1}\text{ è l'insieme delle funzioni )} 
			\]
	\end{enumerate}
\end{defn}


\begin{es}{Toro.}
	Un esempio di spazio topologico che può essere visto come $\Delta$-Complesso è il toro nel seguente modo: 
	\begin{center}
		\begin{tikzpicture}
			\draw[thick,-] (-2,-1) -- (2,-1) -- (2,1) -- (-2,1) -- (-2,-1);
			\draw[thick,->] (-2,-1) -- (0,-1) node[anchor=north] {$\text{b}$};
			\draw[thick,->] (-2,1) -- (0,1) node[anchor=south] {$\text{b}$};
			\draw[thick,->] (-2,-1) -- (-2,0) node[anchor=east] {$\text{a}$};
			\draw[thick,->] (2,-1) -- (2,0) node[anchor=west] {$\text{a}$};
			\draw[thick,-] (-2,-1) -- (2,1);
			\draw[thick,->] (-2,-1) -- (0,0) node[anchor=south] {$\text{c}$};
			\draw[fill=black] (2,-1) circle(0.03) node[anchor=west] {p$\quad \varphi_{B}(1)$};
			\draw[fill=black] (2,1) circle(0.03) node[anchor=west] {$\varphi_{A}(2)=\varphi_{B}(2)$};
			\draw[fill=black] (-2,1) circle(0.03) node[anchor=east] {$\varphi_{A}(1)$};
			\draw[fill=black] (-2,-1) circle(0.03) node[anchor=east] {$\varphi_{A}(0)=\varphi_{B}(0)$};
			\draw (-1,0.5) node {$\Large\text{A}$};
			\draw (1,-0.5) node {$\Large\text{B}$};
			\draw[fill=black] (-5,0) circle (0.07) node[anchor=north] {1};
			\draw[fill=black] (-7,2) circle (0.07) node[anchor=south] {2};
			\draw[fill=black] (-7.7,-0.7) circle (0.07) node[anchor=north] {0};
			\draw[thick,-] (-5,0) -- (-7,2) -- (-7.7,-0.7) -- (-5,0) ;
			\draw[thick, ->] (-5,0) -- (-6,1);
			\draw[thick, ->] (-7.7,-0.7) -- (-6.35,-0.35);
			\draw[thick, ->] (-7.7,-0.7) -- (-7.35,0.65);
			\draw (-6,1.5) node[anchor=south] {$\varphi_A$};
			\draw [->] (-6,1.5) to [out=30] (-1,1.2);
			\draw (-6.35,-0.7) node[anchor=north] {$\varphi_B$};
			\draw[->] (-6.35,-0.7) to [out=-30,in=210] (1,-1.2);
		\end{tikzpicture}
	\end{center} 
	Quindi possiamo usare un abuso di notazione per indicare sia indici (o le facce) che le funzioni stesse sulla quale poi calcoliamo la nostra $\partial$: 
	\begin{itemize}
		\item $I_2=\{A,B\}$ è l'insieme che contiene le funzioni che mandano (mantenendo l'orientamento delle frecce) l'inviluppo convesso sopra nei rispettivi 2 ''triangoli'' A e B;
		\item $I_1=\{a,b,c\}$ sono le funzioni che mandano i lati del inviluppo nei ''lati'' a,b,c;
		\item $I_0=\{p\}$ la funzione che manda i vertici dell'inviluppo nel ''vertice'' p (ho un unico vertice e non 4 in quanto tutti e 4 i vertici sono in relazione tra loro).
	\end{itemize} 
	Grazie all'abuso di notazioni e al fatto che questi li possiamo vedere, come elementi che generano dei gruppi liberi abeliani, allora possiamo calcolare il gruppo di omologia del complesso omologico descritto sopra (del toro).
\end{es}

\begin{oss}
	$\varphi_{\alpha}^{(k)}$ sono k-simplessi singolari e quindi defininedo $\text{C}_k^{\Delta}(X)\stackrel{def}=\text{Span}_{\mathbb{Z}}\{\varphi_{\alpha}^{(k)}\}\subseteq\text{C}_k(X)$ e considerando \hyperref[due]{(2)} vale $\partial\left(\text{C}_k^{\Delta}(X)\right)\subseteq\text{C}_{k-1}^{\Delta}(X)$ allora possiamo definire il seguente complesso omologico:
	\begin{center}
		\begin{tikzcd}
			 \arrow[r,"\partial"] & \text{C}_k^{\Delta}(X)\arrow[r, "\partial" ] &\text{C}_{k-1}^{\Delta}(X)\arrow[r, "\partial" ] & \dots\arrow[r, "\partial"]& \text{C}_0^{\Delta}(X)\arrow[r, "\partial"]& 0
		\end{tikzcd}
	\end{center}
\end{oss}

\begin{ese}
	Calcolare i gruppi di omologia del Toro.
\end{ese}

\begin{sol}
	Prendiamo $I_2=\{A,B\}$ e consideriamo il gruppo abeliano libero generato da $I_2$ allora essendo finito è isomorfo a $\mathbb{Z}^2$ con base $\{A,B\}$ se consideriamo anche $I_1$ e $I_0$ otteniamo il seguente complesso omologico:
	\begin{center}
		\begin{tikzcd}
			0 \arrow[r,"\partial"] & \underset{\text{A,B}}{\mathbb{Z}^2}\arrow[r, "\partial" ] &\underset{\text{a,b,c}}{\mathbb{Z}^3}\arrow[r, "\partial" ] & \underset{\text{p}}{\mathbb{Z}}\arrow[r, "\partial"]& 0
		\end{tikzcd}
	\end{center}
	Quindi, aiutandoci con l'algebra lineare, possiamo scrivere le matrici associate alle nostre funzioni $\partial$ e quindi calcolare nucleo e immagine. Partendo da sinistra, la prima, è la funzione che manda 0 in 0 e quindi possiamo già dire che $\text{H}_{2}^{\Delta}(X)=\text{Ker}(\partial)$, dove il nucleo è della seconda funzione da sinistra. 
	Calcoliamo la matrice associata della seconda funzione della catena, ovvero, prendiamo la base $\{A,B\}$ e calcoliamo la funzione su tale base (stiamo sempre usando l'abuso di notazione di prima):
	\begin{itemize}
		\item $\partial A=\varphi_A(<1,2>)-\varphi_A(<0,2>)+\varphi_A(<0,1>)=b-c+a=a+b-c$
		\item $\partial B=\varphi_B(<1,2>)-\varphi_B(<0,2>)+\varphi_B(<0,1>)=b+a-c=a+b-c$
	\end{itemize}
	La matrice associata è quindi: $\text{M}(\partial)=\begin{pmatrix} 1&1\\1&1\\-1&-1\end{pmatrix}$ che ha rango 1 di conseguenza Im$(\partial)\cong\mathbb{Z}$ e di conseguenza Ker$(\partial)\cong\mathbb{Z}<A-B>$ ovvero generato da A-B.
	Ora calcoliamo la matrice associata alla terza funzione della catena sapendo che $\partial a =\partial b =\partial c= p-p=0$ e quindi: $M(\partial)=\begin{pmatrix}0&0&0\end{pmatrix}$ di conseguenza Ker$(\partial)\cong\mathbb{Z}^3$
	Quindi abbiamo $\text{H}_{1}^{\Delta}(X)\cong\bigslant{\mathbb{Z}^3}{\mathbb{Z}}\cong{\mathbb{Z}^2}$ con generatori a,b. 
	Per ultimo abbiamo banalmente $\text{H}_{0}^{\Delta}(X)\cong\mathbb{Z}$ con generatore p. 

	Ricapitolando:
	\begin{center}
		\begin{tikzcd}[ampersand replacement=\&]
			0 \arrow[r,"\partial"] \& \underset{\text{A,B }}{\mathbb{Z}^2}\arrow{r}{\partial}[swap]{\begin{pmatrix}1 &1\\1 &1\\-1 &-1\end{pmatrix}} \&\underset{\text{ a,b,c }}{\mathbb{Z}^3}\arrow{r}{\partial }[swap]{(0,0,0)} \& \underset{\text{ p}}{\mathbb{Z}}\arrow[r, "\partial"]\& 0
		\end{tikzcd}
	\end{center}
	\begin{itemize}
		\item $\text{H}_{2}^{\Delta}(X)\cong\mathbb{Z}<A-B>$;
		\item $\text{H}_{1}^{\Delta}(X)\cong\mathbb{Z}^2<a,b>$;
		\item $\text{H}_{0}^{\Delta}(X)\cong\mathbb{Z}<p>$.
	\end{itemize}
\end{sol}

\newpage

\begin{ese}\label{Bottiglia di Klein}
	Calcolare i gruppi di omologia della Bottiglia di Klein.
\end{ese}
\begin{sol}
	Partiamo con il disegnare la bottiglia ed elencare gli insiemi $I_k$:
	\begin{center}
		\begin{tikzpicture}
			\draw[thick,-] (-2,-1) -- (2,-1) -- (2,1) -- (-2,1) -- (-2,-1);
			\draw[thick,->] (-2,-1) -- (0,-1) node[anchor=north] {$\text{b}$};
			\draw[thick,->] (-2,1) -- (0,1) node[anchor=south] {$\text{b}$};
			\draw[thick,->>] (-2,-1) -- (-2,0) node[anchor=east] {$\text{a}$};
			\draw[thick,->>]  (2,1) -- (2,0) node[anchor=west] {$\text{a}$};
			\draw[thick,-] (-2,-1) -- (2,1);
			\draw[thick,->] (-2,-1) -- (0,0) node[anchor=south] {$\text{c}$};
			\draw[fill=black] (2,-1) circle(0.03) node[anchor=west] {p$\quad \varphi_{B}(2)$};
			\draw[fill=black] (2,1) circle(0.03) node[anchor=west] {$\varphi_{A}(2)=\varphi_{B}(1)$};
			\draw[fill=black] (-2,1) circle(0.03) node[anchor=east] {$\varphi_{A}(1)$};
			\draw[fill=black] (-2,-1) circle(0.03) node[anchor=east] {$\varphi_{A}(0)=\varphi_{B}(0)$};
			\draw (-1,0.5) node {$\Large\text{A}$};
			\draw (1,-0.5) node {$\Large\text{B}$};
			\draw[red] (0.2,1.7) node[anchor=west]{$\Delta^2$};
			\draw[red,->] (0.2,1.7) to [out=150, in=50](-1,0.8);
			\draw[red,->] (0.7,1.5) to [out=-30, in=70](1,-0.2);
			\draw[green] (-2.5,1.7) node[anchor=east]{$\Delta^1$};
			\draw[green,->] (-3.1,1.6) to [out=210, in=150] (-2.4,0);
			\draw[green,->] (-2.65,1.75) to [out=20, in=160] (0,1.5);
			\draw[green,->] (-2.7,1.55) to (0,0.4);
			\draw[orange] (3,0) node[anchor=west]{$\Delta^0$};
			\draw[orange,->] (3.1,-0.1) to[out=160, in=70] (2,-1);
		\end{tikzpicture}
	\end{center}
	\begin{itemize}
		\item $I_2=\{A,B\}$;
		\item $I_1=\{a,b,c\}$;
		\item $I_0=\{p\}$;
	\end{itemize}
	Quindi il complesso omologico associato è ''simile'' a quello del toro\footnote{ATTENZIONE!! anche se i gruppi liberi sono isomorfi le funzioni $\partial$ sono diverse!!}:
	\begin{center}
		\begin{tikzcd}
			0 \arrow[r,"\partial"] & \underset{\text{A,B}}{\mathbb{Z}^2}\arrow[r, "\partial" ] &\underset{\text{a,b,c}}{\mathbb{Z}^3}\arrow[r, "\partial" ] & \underset{\text{p}}{\mathbb{Z}}\arrow[r, "\partial"]& 0
		\end{tikzcd}
	\end{center}
	Seguendo i passi dell'esercizio precedente calcoliamo le matrici:
	\begin{itemize}
		\item $\partial A=\varphi_A(<1,2>)-\varphi_A(<0,2>)+\varphi_A(<0,1>)=b-c+a=a+b-c$
		\item $\partial B=\varphi_B(<1,2>)-\varphi_B(<0,2>)+\varphi_B(<0,1>)=a-b+c$
		\item $\partial a =\partial b =\partial c= p-p=0$
	\end{itemize}
	Quindi la matrice della seconda $\partial$ è $\text{M}(\partial)=\begin{pmatrix}1&1\\1&-1\\-1&1\end{pmatrix}$ e invece per la terza funzione è come prima la funzione nulla:
	\begin{center}
		\begin{tikzcd}[ampersand replacement=\&]
			0 \arrow[r,"\partial"] \& \underset{\text{A,B }}{\mathbb{Z}^2}\arrow{r}{\partial}[swap]{\begin{pmatrix}1 &1\\1 &-1\\-1 &1\end{pmatrix}} \&\underset{\text{ a,b,c }}{\mathbb{Z}^3}\arrow{r}{\partial }[swap]{(0,0,0)} \& \underset{\text{ p}}{\mathbb{Z}}\arrow[r, "\partial"]\& 0
		\end{tikzcd}
	\end{center}
	Intanto sappiamo che $\text{H}_2^{\Delta}=\text{Ker}\begin{pmatrix}1 &1\\1 &-1\\-1 &1\end{pmatrix}=0$ quindi è il gruppo banale. 

	Ora per calcolare  $\text{H}_1^{\Delta}=\bigslant{\mathbb{Z}^3}{\text{Im}\begin{pmatrix}1 &1\\1 &-1\\-1 &1\end{pmatrix}}\cong\bigslant{<a,b,c>}{<a+b-c,a-b+c>}$ cerchiamo di capire quali sono le classi di equivalenza usando le relazioni:
	\begin{itemize}
		\item $[a]=[b-c]=[c-b]\rightarrow [2a]=[0]$
		\item $[2b]=[2c]$
	\end{itemize}
	Sviluppiamo un po' di conti e calcoliamo la classe di un generico elemento di $<a,b,c>$ siano $x,y,z\in\mathbb{Z}$ calcoliamo $[ax+by+cz]$:
	\begin{center}
		\begin{align*}
			[ax+by+cz]&=[b-c]x+[b]y+[c]z=[b](x+y)+[c](y-x)\\&=[b](x+y \text{ mod } 2)+[c]m=[b]n+[c]m \quad n\in\mathbb{Z}_2\,\,m\in\mathbb{Z}
		\end{align*}
	\end{center}
	Quindi abbiamo dimostrato che vale $\text{H}_1^{\Delta}\cong\mathbb{Z}\bigoplus\mathbb{Z}_2$. Invece il gruppo $\text{H}_0^{\Delta}$ è come nel caso del Toro. 

	Ricapitolando:
	\begin{itemize}
		\item $\text{H}_{2}^{\Delta}(X)\cong\{0\}$;
		\item $\text{H}_{1}^{\Delta}(X)\cong\mathbb{Z}\bigoplus\mathbb{Z}_2$;
		\item $\text{H}_{0}^{\Delta}(X)\cong\mathbb{Z}<p>$.
	\end{itemize}
	Volendo possiamo considerare invece di gruppi abeliani liberi con cofficienti in $\mathbb{Z}$(i procedimenti sono analoghi) considerando gruppi abeliani liberi con coefficienti in $\mathbb{Z}_2$ otteniamo:
	\begin{itemize}
		\item $\text{H}_{2}^{\Delta}(X)\cong\mathbb{Z}_2$;
		\item $\text{H}_{1}^{\Delta}(X)\cong\mathbb{Z}_2\bigoplus\mathbb{Z}_2$;
		\item $\text{H}_{0}^{\Delta}(X)\cong\mathbb{Z}_2$.
	\end{itemize}
\end{sol}

\begin{ese}
	Calcolare i gruppi di omologia del Piano Proiettivo: $\mathbb{P}^2(\mathbb{R})$. 
\end{ese}
\begin{sol}
	Partiamo con il disegno e poi andiamo a calcolare le matrici: 
	\begin{center}
		\begin{tikzpicture}
			\draw[thick,-] (-2,-1) -- (2,-1) -- (2,1) -- (-2,1) -- (-2,-1);
			\draw[thick,->>] (-2,-1) -- (0,-1) node[anchor=north] {$\text{b}$};
			\draw[thick,->>] (2,1) -- (0,1) node[anchor=south] {$\text{b}$};
			\draw[thick,->] (-2,-1) -- (-2,0) node[anchor=east] {$\text{a}$};
			\draw[thick,->]  (2,1) -- (2,0) node[anchor=west] {$\text{a}$};
			\draw[thick,-] (-2,-1) -- (2,1);
			\draw[thick,->] (-2,-1) -- (0,0) node[anchor=south] {$\text{c}$};
			\draw[fill=black] (2,-1) circle(0.03) node[anchor=west] {$\varphi_{B}(2)$};
			\draw[fill=black] (2,1) circle(0.03) node[anchor=west] {$\varphi_{A}(1)=\varphi_{B}(1)$};
			\draw[fill=black] (-2,1) circle(0.03) node[anchor=east] {$\varphi_{A}(2)$};
			\draw (-2,1) node[anchor=south] {q};
			\draw[fill=black] (-2,-1) circle(0.03) node[anchor=east] {$\varphi_{A}(0)=\varphi_{B}(0)$};
			\draw (-2,-1) node[anchor=north] {p};
			\draw (-1,0.5) node {$\Large\text{A}$};
			\draw (1,-0.5) node {$\Large\text{B}$};
		\end{tikzpicture}
	\end{center}
	\begin{itemize}
		\item $I_2=\{A,B\}$;
		\item $I_1=\{a,b,c\}$;
		\item $I_0=\{p,q\}$;
		\item $\partial A=\varphi_A(<1,2>)-\varphi_A(<0,2>)+\varphi_A(<0,1>)=c+b-a=-a+b+c$;
		\item $\partial B=\varphi_B(<1,2>)-\varphi_B(<0,2>)+\varphi_B(<0,1>)=c+a-b=a-b+c$;
		\item $\partial a =q-p=-p+q$;
		\item $\partial b =q-p=-p+q$;
		\item $\partial c =p-p=0$;
	\end{itemize}
	Quindi il complesso omologico che otteniamo è: 
	\begin{center}
		\begin{tikzcd}[ampersand replacement=\&]
			0 \arrow[r,"\partial"] \& \underset{\text{A,B }}{\mathbb{Z}^2}\arrow{r}{\partial}[swap]{\left(\begin{smallmatrix}-1 &1\\1 &-1\\1 &1\end{smallmatrix}\right)} \&\underset{\text{ a,b,c }}{\mathbb{Z}^3}\arrow{r}{\partial }[swap]{\left(\begin{smallmatrix}-1&-1&0\\1&1&0\end{smallmatrix}\right)} \& \underset{\text{ p,q}}{\mathbb{Z}^2}\arrow[r, "\partial"]\& 0
		\end{tikzcd}
	\end{center}
	$\text{H}_2^{\Delta}=\text{Ker}\begin{pmatrix}-1 &1\\1 &-1\\1 &1\end{pmatrix}=0$ quindi è il gruppo banale (il rango per colonne è 2) vediamo di trovare dei generatori dell'immagine di tale funzione in particolare vediamo che l'immagine è un sottogruppo del $\text{Ker}\begin{pmatrix}-1 &-1& 0\\1 &1&0\end{pmatrix}=<a-b,c>$ riprendendo le equazioni di prima possiamo vedere che se chiamiamo $x:=a-b$ e $y:=c$ allora otteniamo: 
	\begin{itemize}
		\item $\partial A=y-x$ 
		\item $\partial B=x+y$ 
	\end{itemize} 
	Che equivale a ridefinire la matrice della funzione $\partial:\mathbb{Z}^2\rightarrow\mathbb{Z}^2\cong\text{Ker}(\partial)$ nel seguente modo(facilitandoci i conti):
	\[
		M(\partial)=\begin{pmatrix}-1&1\\1&1\end{pmatrix}
	\]
	Da ciò possiamo vedere che $\text{H}_1^{\Delta}=\bigslant{\text{Ker}(\partial)}{\text{Span}\left\{\begin{pmatrix}-1\\1\end{pmatrix},\begin{pmatrix}1\\1\end{pmatrix}\right\}}\cong\bigslant{<x,y>}{<x-y,x+y>}\cong \mathbb{Z}_2$
	Infine $\text{H}_0^{\Delta}=\bigslant{<p,q>}{<p-q>}\cong\mathbb{Z}$.
	
	Ricapitolando:
	\begin{itemize}
		\item $\text{H}_{2}^{\Delta}(X)\cong\{0\}$;
		\item $\text{H}_{1}^{\Delta}(X)\cong\mathbb{Z}_2$;
		\item $\text{H}_{0}^{\Delta}(X)\cong\mathbb{Z}<p-q>$.
	\end{itemize}
\end{sol}
\begin{ese}
	Calcolare il gruppo di omologia del seguente $\Delta-$complesso:
\begin{center}
	\begin{tikzpicture}
		\draw[thick,-] (-1,2.4142135) -- (-2.4142135,1) -- (-2.4142135,-1)--(-1,-2.4142135) -- (1,-2.4142135)--(2.4142135,-1) -- (2.4142135,1) -- (1,2.4142135) -- (-1,2.4142135);
		\draw[thick,-] (1,-2.4142135)--(-2.4142135,-1);
		\draw[thick,->] (1,-2.4142135)--(-0.7071,-1.7071) node[anchor=south] {e};
		\draw[thick,-] (1,-2.4142135)--(-2.4142135,1);
		\draw[thick,->] (1,-2.4142135)--(-0.7071,-0.7071) node[anchor=south] {f};
		\draw[thick,-] (1,-2.4142135)-- (-1,2.4142135);
		\draw[thick,->] (1,-2.4142135)-- (0,0) node[anchor=south west] {g};
		\draw[thick,-] (1,-2.4142135)-- (1,2.4142135);
		\draw[thick,->] (1,-2.4142135)-- (1,0) node[anchor=south west] {h};
		\draw[thick,-] (1,-2.4142135)-- (2.4142135,1);
		\draw[thick,->] (1,-2.4142135)-- (1.7071,-0.7071) node[anchor=west] {i};
		\draw[thick,->] (1,-2.4142135)--(0,-2.4142135) node[anchor=north] {a};
		\draw[thick,->] (-1,-2.4142135)--(-1.7071,-1.7071) node[anchor=north east ] {b};
		\draw[thick,->] (-2.4142135,1)--(-2.4142135,0) node[anchor=east ] {a};
		\draw[thick,->] (-1,2.4142135)--(-1.7071,1.7071) node[anchor=south east ] {b};
		\draw[thick,->] (-1,2.4142135)--(0,2.4142135) node[anchor=south] {d};
		\draw[thick,->] (1,2.4142135)--(1.7071,1.7071) node[anchor=south west] {c};
		\draw[thick,->] (2.4142135,-1)--(2.4142135,0) node[anchor=west] {d};
		\draw[thick,->] (1,-2.4142135)--(1.7071,-1.7071) node[anchor= north west] {c};
		\draw[fill=black] (1,-2.4142135) circle(0.05) node[anchor=north] {p};
		\draw[red] (-0.8,-2) node {$\Large \text{A}$};
		\draw[red] (-1.2,-0.9) node {$\Large \text{B}$};
		\draw[red] (-0.5,0) node {$\Large \text{C}$};
		\draw[red] (0.6,0.1) node {$\Large \text{D}$};
		\draw[red] (1.4,-0.3) node {$\Large \text{E}$};
		\draw[red] (2.07,-1.07) node {$\Large \text{F}$};
	\end{tikzpicture}
\end{center}
\end{ese}
\begin{sol}
	La catena omologica è:
	\begin{center}
		\begin{tikzcd}
			0 \arrow[r,"\partial"] & \underset{\substack{\text{A,B,C} \\ \text{D,E,F}}}{\mathbb{Z}^6} \arrow[r,"\partial"] & \underset{\substack{\text{a,b,c} \\ \text{d,e,f}\\ \text{g,h,i}}}{\mathbb{Z}^9}\arrow[r,"\partial"] &\underset{p}{\mathbb{Z}}\arrow[r,"\partial"] & 0
		\end{tikzcd}
	\end{center}
	\begin{itemize}
		\item $I_2=\{A,B,C,D,E,F\}$;
		\item $I_1=\{a,b,c,d,e,f,g,h,i\}$;
		\item $I_0=\{p\}$;
		\item $\partial A=b-a+e=-a+b-e$;
		\item $\partial B=a-e+f$;
		\item $\partial C =b-f+g$;
		\item $\partial E =c-i+h=c+h-i$;
		\item $\partial F =d-i+c=c+d-i$;
		\item $\partial a=\partial b=\partial c=\partial d=\partial e=\partial f=\partial g=\partial h=\partial i=0$
	\end{itemize}
	Quindi abbiamo la matrice $\text{M}(\partial)=\left(\begin{smallmatrix}1&1&0&0&0&0\\ 1&0&1&0&0&0\\ 0&0&0&0&1&1\\ 0&0&0&1&0&1\\ -1&-1&0&0&0&0\\ 0&1&-1&0&0&0\\ 0&0&1&1&0&0 \\ 0&0&0&-1&1&0\\0&0&0&0&-1&-1\end{smallmatrix}\right)$ cerchiamo il nucleo e il rango. 
	Sappiamo che il suo rango (per colonne) può essere al massimo 6 inoltre notiamo che è presente una sottomatrice $\left(\begin{smallmatrix}0&0&0&0&1\\ 0&0&0&1&0\\-1&-1&0&0&0\\0&1&-1&0&0\\0&0&1&1&0\end{smallmatrix}\right)$ invertibile quindi il rango è almeno 5. 

	Facendo un po' operazioni sulle colonne si trova che (1)+(4)+(5)=(2)+(3)+(6) equivalentemente un generatore del Ker$(\partial)$ è A-B+D-C+E-F quindi il rango di  $\text{M}(\partial)$ è 5. La sesta colonna è combinazione lineare delle prime 5 quindi Im$(\partial)=<a+b-e,a-e+f,b-f+g,c+h-i,c+d-i>$.

	Quindi $H_2^{\Delta}\cong \text{Ker}(\partial)=\mathbb{Z}<A-B+D-C+E-F >$

	$H_1^{\Delta}\cong \bigslant{<a,b,c,d,e,f,g,h,i>}{<a+b-e,a-e+f,b-f+g,c+h-i,c+d-i>}$ 
	
	Cerchiamo $[a_1a+a_2b+a_3c+a_4d+a_5e+a_6f+a_7g+a_8h+a_9i]$:

	$\begin{cases}
		&[a]=[e-b] \\ 
		&[a]=[e-f] \\ 
		&[g]=[b-f] \\ 
		&[d]=[h-g] \\
		&[c]=[i-h]
	\end{cases} \Rightarrow
	\begin{cases}
		&[a]=[e-b] \\
		&[b]=[f] \\
		&[g]=[b-f]=[0] \\	
		&[d]=[h-g]=[h]\\
		&[c]=[i-h]		
	\end{cases}$

	Di conseguenza la classe di equivalenza diventa:
	\begin{align}
		[a_1a+a_2b+a_3c+a_4d+a_5e+a_6f+a_7g+a_8h+a_9i]&=a_1[a]+a_2[b]+a_3[c]+a_4[d]+a_5[e]+\\
											     &+a_6[f]+a_7[g]+a_8[h]+a_9[i]\\
											     &=a_1[e-b]+a_2[b]+a_3[i-h]+a_4[h]+\\
											     &+a_5[e]+a_6[f]+a_8[h]+a_9[i] \\
											     &=(a_2-a_1+a_6)[b]+(a_1+a_5)[e]+\\
											     &+(a_4+a_8-a_3)[h]+(a_3+a_9)[i]
	\end{align}
	Quindi Abbiamo dimostrato che $\text{H}_1^\Delta\cong\mathbb{Z}^4$ e Banalmente $\text{H}_0^\Delta\cong\mathbb{Z}$. 
\end{sol}

\newpage

\section{Gruppi Abeliani}
\subsection{Gruppi Abeliani Liberi}

\begin{defn}{Gruppo Abeliano finitamente generato}{}
	Sia G un Gruppo abeliano allora G è finitamente generato se esistono: $g_1,\dots,g_r\in \text{G}$ tale che ogni $g\in\text{G}$ si scrive come:
	\[
		g=a_1g_1+\dots+a_rg_r\quad\text{con }a_1,\dots,a_r\in\mathbb{Z}
	\]
	Equivalentemente esiste un omomorfismo(di gruppi) suriettivo $\varphi:\,\mathbb{Z}^r\rightarrow \text{G}$ 
\end{defn}

\begin{defn}{Gruppo abeliano libero}{}
	Un gruppo abeliano si dice libero se esistono:$g_1,\dots,g_r\in \text{G}$ tale che ogni $g$ si scrive in $\color{red}{\text{modo unico}}$ come:  
	\[
		g=a_1g_1+\dots+a_rg_r\quad\text{con }a_1,\dots,a_r\in\mathbb{Z}
	\]
	Equivalentemente esiste un omomorfismo(di gruppi) suriettivo e iniettivo $\varphi:\,\mathbb{Z}^r\rightarrow \text{G}$ ovvero $\text{G}\cong\mathbb{Z}^r$.
\end{defn}
	Chiamiamo Rango di G (gruppo abeliano libero) $r$, ovvero il numero di elementi che scrivono ogni elemento di G in modo unico e l'insieme di tali elementi (analogamente agli spazi vettoriali) si definisce base di G.\footnote{l'esistenza e l'unicità vengono dimostrate nel libro ''An introduction to Algebraic Topology'' di Joseph J. Rotman a pagina 60}. 

\begin{oss}
	Dal primo teorema di isomorfismo otteniamo che un gruppo abeliano finitamente generato è un quoziente di $\mathbb{Z}^r$:
	\[
		\text{G}\cong\bigslant{\mathbb{Z}^r}{\text{Ker}(\varphi)}
	\]
	Questo implica per esempio che $\mathbb{Q}$ non è finitamente generato in quanto non è isomorfo a $\mathbb{Z}$ ma sono solo in biezione (numerabile).
\end{oss}

\begin{es}
	$\mathbb{Z}$ è un gruppo abeliano libero con base $\{1\}$.
\end{es}
\label{sottogruppi}
\begin{thm}{}
	pOgni sottogruppo di un gruppo abeliano libero G (finitamente generato) è ancora un gruppo abeliano libero (finitamente generato e con rango $\leq$ rango di G)  
\end{thm}
\begin{proof}
	Dimostriamolo per induzione sul r=rango di G. Per semplificare le notazioni scegliamo una base di G e identifichiamo esso con $\mathbb{Z}^r$ in quanto isomorfi.

	Caso r=1: $\text{G}\cong\mathbb{Z}$ allora i sottogruppi non vuoti di G sono tutti del tipo $n\mathbb{Z}$ per un qualche $n$ e sappiamo che $n\mathbb{Z}\cong\mathbb{Z}$ (basta prendere $\varphi:\mathbb{Z}\rightarrow n\mathbb{Z}$ t.c. $\varphi(z)=nz$)

	Prendiamo che per ogni le ipotesi valgono per rango $\leq$ r-1. Sia H sottogruppo di $\text{G}\cong \mathbb{Z}^r$ quindi possiamo vederlo come sottogruppo di $\mathbb{Z}^r$ definiamo $\pi: \mathbb{Z}^r\rightarrow\mathbb{Z}$ tale che $\pi\left(\begin{smallmatrix}a_1\\\vdots\\a_r\end{smallmatrix}\right)=a_r$, consideriamo l'insieme Ker$\left(\eval{\pi}_{\text{H}}\right)$=Ker$(\pi)\cap\text{H}=\left\{\left(\begin{smallmatrix}a_1\\ \vdots \\a_{r-1}\\ 0\end{smallmatrix}\right)\in\text{H}\right\}$ questo è un sottogruppo di $\mathbb{Z}^{r-1}$ e quindi per ipotesi induttiva è libero di rango s $\leq$ r-1, ovvero ammette una base $\{\gamma_1,\dots,\gamma_s\}$.

	Consideriamo Im$\left(\eval{\pi}_{\text{H}}\right)$ sottogruppo di $\mathbb{Z}$ allora avremo 2 casi:
	\begin{enumerate}
		\item Im$\left(\eval{\pi}_{\text{H}}\right)$=$\{0\}\Rightarrow \text{H}\subseteq \text{Ker}\pi\cong\mathbb{Z}^{r-1}$
		\item Im$\left(\eval{\pi}_{\text{H}}\right)=n_0\mathbb{Z}$
	\end{enumerate}
	Nel primo caso per ipotesi induttiva H è gruppo libero. 

	Nel secondo caso prendiamo $\gamma_{s+1}\in\text{H}$ tale che $\pi(\gamma_{s+1})=n_0$ mostriamo che $\{\gamma_1,\dots,\gamma_s,\gamma_{s+1}\}$ è una base di H: 
	\begin{align*}
		&\text{sia }\gamma\in\text{H}\Rightarrow \\ 
		&\pi(\gamma)=kn_0 \Rightarrow \pi(\gamma-k\gamma_{s+1})=0 \Rightarrow \\ 
		& \gamma-k\gamma_{s+1}\in\text{Ker}\cap\text{H}\Rightarrow\\ 
		&\gamma-k\gamma_{s+1}=a_1\gamma_1+\dots+a_s\gamma_s\quad \text{con }a_i\in\mathbb{Z}\text{ unici}\\ 
		&\gamma=a_1\gamma_1+\dots+a_s\gamma_s+k\gamma_{s+1}.
	\end{align*}
	La scrittura è unica banalmente e dall'ipotesi s $\leq$ r-1 otteniamo s+1 $\leq$ r.
\end{proof}

\begin{cor}{}
	pSia H sottogruppo di $\mathbb{Z}^n$ allora H è libero e H$\cong\mathbb{Z}^r$ con r$\leq$ n.
\end{cor}
\begin{oss}
	Il rango di un gruppo abeliano libero(finitamente generato) è unico ovvero prese 2 basi esse hanno lo stesso numero di elementi. Questo implica che presa una matrice di cambiamento di base essa è invertibile e il determinante vale $\pm$1.
\end{oss}
\label{Teorema 2.1.5}
\begin{thm}{}
	pSia G un gruppo libero di rango r, sia H un sottogruppo di G di rango s allora esiste una base $\{\gamma_1,\dots,\gamma_r\}$ di G e degli interi $t_1,\dots,t_s\in\mathbb{Z}$ tali che $\{t_1\gamma_1,\dots,t_s\gamma_s\}$ è una base di H 
\end{thm}

Prima di dimostrare il teorema dimostriamo il seguente corollario del teorema molto utile 
\begin{cor}{}
	pSia G gruppo abeliano finitamente generato allora:
	\[
		\text{G}\cong\mathbb{Z}^{r-s} \oplus \left(\bigoplus\limits_{i=1}^s\bigslant{\mathbb{Z}}{t_i\mathbb{Z}}\right)
	\]
\end{cor}

\begin{es}
	Un esempio di gruppo abeliano finitamente generato di questo tipo è il gruppo di omologia $\text{H}_1^{\Delta}$ del complesso omologico associato alla bottiglia di Klein nell \hyperref[Bottiglia di Klein]{Esercizio 4}.
\end{es}

\newpage

\begin{proof}{Corollario.}
	Dalla definizione di gruppo abeliano finitamente generato abbiamo che G è il quoziente di un gruppo abeliano libero F ovvero $G\cong \bigslant{\text{F}}{\text{H}}$ inoltre dal teorema \hyperref[sottogruppi]{2.1.3} sappiamo che H è un gruppo libero a sua volta. Sia $\{t_1\gamma_1,\dots,t_s\gamma_s\}$ base di H con $\{\gamma_1,\dots,\gamma_r\}$ base di F (teorema precedente) allora: $\text{G}=\bigslant{\text{F}}{\text{H}\cong \bigslant{<\gamma_1,\dots,\gamma_r>}{<t_1\gamma_1,\dots,t_s\gamma_s>}}$. 

	Calcoliamo le classi di equivalenza di un generico elemento di F sapendo che valgono:
	$\begin{cases}
		&[t_1\gamma_1]=[0]\Rightarrow [a\gamma_1]=(a \text{ mod } t_1)[\gamma_1] \quad \forall a\in \mathbb{Z}\\ 
		&\vdots\\
		&[t_s\gamma_s]=[0]\Rightarrow [a\gamma_s]=(a  \text{ mod } t_s)[\gamma_s] \quad \forall a\in \mathbb{Z}
	\end{cases}$
	
	Quindi sia $a_1\gamma_1+\cdots+a_r\gamma_r$ con $a_1,\dots,a_r\in\mathbb{Z}$ allora vale:
	\[
		[a_1\gamma_1+\cdots+a_r\gamma_r]=(a_1  \text{ mod } t_1)[\gamma_1]+\cdots+(a_s  \text{ mod } t_s)[\gamma_s] +a_{s+1}[\gamma_{s+1}]+\cdots+a_r[\gamma_r] 
	\]
	Allora abbiamo dimostrato che:
	\[
		\text{G}\cong\bigslant{\text{F}}{\text{H}}\cong \mathbb{Z}^{r-s} \oplus \left(\bigoplus\limits_{i=1}^s\bigslant{\mathbb{Z}}{t_i\mathbb{Z}}\right)
	\]
\end{proof}

\begin{oss}
	La parte $\bigoplus\limits_{i=1}^s\bigslant{\mathbb{Z}}{t_i\mathbb{Z}}$ è detta torsione e r-s è il rango di G gruppo finitamente generato. Usando il teorema cinese del resto la torsione può essere scritto come $\bigoplus\bigslant{\mathbb{Z}}{p_i^{r_i}\mathbb{Z}}$ 
\end{oss}

\begin{proof}\hyperref[Teorema 2.1.5]{Teorema 2.1.5}
	: identifichiamo G con $\mathbb{Z}^r$ scegliendo una base $\{e_1,\dots,e_r\}$ ora consideriamo la matrice A che ha come colonne una base di H (ovvero la matrice della funzione identità che ha come base di partenza la base di H e in arrivo la base di G):
	\[
		A=\begin{pmatrix}a_{1,1}&\cdots&a_{1,s}\\ \vdots& &\vdots\\a_{r,1}&\cdots& a_{r,s}\end{pmatrix}
	\]

	Quello che vogliamo fare è applicare ad essa una matrice del cambiamento di base per ottenere una base di G tale che: 
	\[
		A=\begin{pmatrix}t_1&0&\cdots&0\\ 
					  0&t_2 &\cdots&0\\ 
					  \vdots&\vdots&\vdots& \vdots\\ 
		     			  0& 0&0 & t_s 
		     \end{pmatrix}
	\]

	Sappiamo dall'algebra lineare che fare un cambiamento di base equivale al poter compiere le seguenti operazioni sulla matrice A:
	\begin{enumerate}
		\item Posso permutare gli elementi della base $\{e_1,\dots,e_r\}$ e quindi sulla matrice ottenere una permutazione delle righe;
		\item Posso sommare a un elemento $e_i$ un multiplo di $e_j$ con $i\neq j$ che sulla matrice equivale a sommare a una riga il multiplo di un altra;
		\item Posso permutare gli elementi della base di H ovvero (base in partenza) cambiare l'ordine delle colonne;
		\item Posso sommare a un elemento della base di H un multiplo di un altro elemento che equivale a sommare a una colonna un multiplo di un altra. 
	\end{enumerate}

	Se A=0 allora abbiamo già fatto in quanto H è sottogruppo banale. 
	
	Sia A$\neq$0 allora prendiamo l'elemento più piccolo, in valore assoluto, di A: usando (1) e (3) questo elemento lo possiamo supporre come $a_{1,1}$. 

	Distinguiamo 2 casi possibili:

	Caso 1, tutti gli elementi della prima riga e della prima colonna sono divisibili per $a_{1,1}$ e quindi usando (2) e (4) otteniamo:
	\[
		A=\begin{pmatrix}a_{1,1}&0&\cdots&0\\ 0& & &\\ \vdots& &\Large\text{A'}&\\0& & & \end{pmatrix}
	\]

	Caso 2, ho un elemento $a_{1,k}$ (o $a_{t,1}$) non divisibile per $a_{1,1}$ allora usiamo la divisione con resto per ottenere $a_{1,k}=qa_{1,1}+r$ con $r\neq 0$ e $|r|<|a_{1,1}|$ quindi posso sostituire la prima colonna con $C_k -qC_1$ (stessa cosa potevo farla con le righe eventualmente) e poi portare $r$ alla posizione 1,1.
	 
	Fatto ciò posso rifare il tutto e se rientro di nuovo nel secondo caso usiamo di nuovo la divisione euclidea. La divisione euclidea in $\mathbb{Z}$ è un algoritmo ben definito ed ha una fine ovvero tra due numeri ho sempre un massimo comun divisore, quindi dopo un numero finito di passi arriverò per forza al Caso 1. 
	Otteniamo alla fine necessariamente, tramite un cambio base, una matrice della forma:
	\[
		A=\begin{pmatrix}t_{1}&0&\cdots&0\\ 0& & &\\ \vdots& &\Large\text{A'}&\\0& & & \end{pmatrix}
	\]
	Quindi possiamo fare lo stesso procedimento su A' fino a quando non completiamo la matrice. In particolare potevamo anche dimostrarlo formalmente usando l'induzione sulla dimensione della matrice e applicando il processo iterativo. 
\end{proof}

\newpage

\section{Omologia e Omotopia}
\subsection{Invarianza Omotopica}
\begin{defn}{Complessi simpliciali di ricoprimenti aperti}{}
	Sia X spazio topologico, $U_0,\dots,U_n$ un ricoprimento aperto $\mathfrak{N}$ definiamo il complesso simpliciale N($\mathfrak{N}$):
	\begin{itemize}
		\item $I=\{0,\dots,n\}$
		\item $<i_0,\dots,i_k>\in \text{N}(\mathfrak{N})$ se $U_{i_0}\cap\dots\cap U_{i_k}\neq \varnothing$.
	\end{itemize}
\end{defn}

\end{document}
