\documentclass[11pt, a4paper, twoside]{article}

% Pacchetti Base
\usepackage[italian]{babel}
\usepackage{amsfonts}
\usepackage{amssymb}
\usepackage{tikz}
\usetikzlibrary{cd}
\usepackage{extarrows}
\usepackage{mathrsfs}
\usepackage{amsthm}

% Estetica
\usepackage{tcolorbox}
\tcbuselibrary{theorems}
\usepackage{hyperref}
\usepackage{graphicx}
\graphicspath{ {./Immagini/} }
\usepackage{bm}
\usepackage{fouriernc}
\usepackage[T1]{fontenc}
\usepackage{tikzrput}
\usepackage[object=vectorian]{pgfornament}
\usepackage{xcolor}
\usepackage{fourier-orns}
\usepackage{fancyhdr}
\renewcommand{\headrule}{%
\vspace{-8pt}\hrulefill
\raisebox{-2.1pt}{\quad\decofourleft\decotwo\decofourright\quad}\hrulefill}
\usetikzlibrary{cd,decorations.pathmorphing,patterns}
\usepackage[T1]{fontenc}

% Pacchetti TCB
\tcbuselibrary{documentation}
\tcbuselibrary{breakable}

% Definizioni e Teoremi
\newtcbtheorem
	[number within = subsection]% init options
	{defn}% name
	{Definizione}% title
	{%
		colback=teal!5,
		colframe=teal!90!black!95,
		fonttitle=\bfseries,
	}% options
	{def}% prefix

\newtcbtheorem
	[use counter from = defn, number within = subsection]% init options
	{thm}% name
	{Teorema}% title
	{%
		colback=blue!5,
		colframe=blue!90!black!95,
		fonttitle=\bfseries,
	}% options
	{th}% prefix

\newtcbtheorem
	[use counter from = defn, number within = subsection]% init options
	{prop}% name
	{Proposizione}% title
	{%
		colback=red!5,
		colframe=red!90!black!95,
		fonttitle=\bfseries,
	}% options
	{pr}% prefix

\newtcbtheorem
	[use counter from = defn, number within = subsection]% init options
	{lemma}% name
	{Lemma}% title
	{%
		colback=red!5,
		colframe=red!90!black!95,
		fonttitle=\bfseries,
	}% options
	{le}% prefix

\newtcbtheorem
	[use counter from = defn, number within = subsection]% init options
	{cor}% name
	{Corollario}% title
	{%
		colback=red!5,
		colframe=red!90!black!95,
		fonttitle=\bfseries,
	}% options
	{co}% prefix

\newcommand{\vareps}{\varepsilon}
\newtheorem{es}{Esempio}
\theoremstyle{definition}
\newtheorem*{oss}{Osservazione}
\newtheorem{ese}{Esercizio}

\tcolorboxenvironment{ese}{
	colframe=black}
\newenvironment{sol}
	{\renewcommand\qedsymbol{$\blacksquare$}\begin{proof}[Soluzione]}
	{\end{proof}}

\tcolorboxenvironment{proof}{% `proof' from `amsthm'
	blanker,breakable,left=2.5mm,
	before skip=10pt,after skip=10pt,
	borderline west={0.5mm}{0pt}{black}}

% Formattazione Pagine
\fancyhf{}
\fancyhead[LE]{\nouppercase{\rightmark\hfill\leftmark}}
\fancyhead[RO]{\nouppercase{\leftmark\hfill\rightmark}}
\fancyfoot[LE,RO]{\hrulefill\raisebox{-2.1pt}{\quad\thepage\quad}\hrulefill}
\setlength{\headheight}{16pt}

% Margini
\topmargin=-0.45in
\evensidemargin=-0.15in
\oddsidemargin=-0.15in
\textwidth=6.5in
\textheight=9.0in
\headsep=0.25in

%Titolo
\newcommand{\titolo}[1]{
	\thispagestyle{empty}
	\vspace*{\fill}
	\tikzset{pgfornamentstyle/.style={draw = black, fill = teal!50}}
	\unitlength=1cm
	\begin{center}
	\begin{picture}(12,12)%
		\color{white}%
			\put(0,0){\framebox(12,12){%
			\rput[tl](-4,6){\pgfornament[width=8cm]{71}}%
			\rput[bl](-4,-6){\pgfornament[width=8cm,,symmetry=h]{71}}%
			\rput[tl](-6,6){\pgfornament[width=2cm]{63}}%
			\rput[tr](6,6){\pgfornament[width=2cm,,symmetry=v]{63}}%
			\rput[bl](-6,-6){\pgfornament[width=2cm,,symmetry=h]{63}}%
			\rput[br](6,-6){\pgfornament[width=2cm,,symmetry=c]{63}}%
			\rput[bl]{-90}(-6,4){\pgfornament[width=8cm]{46}}%
			\rput[bl]{90}(6,-4){\pgfornament[width=8cm]{46}}%
			\rput(0,0){\Huge \color{black} #1}%
			\rput[t](0,-0.5){\pgfornament[width=5cm]{75}}%
			\rput[b](0,0.5){\pgfornament[width=5cm]{69}}%
			\rput[tr]{-30}(-1,2.5){\pgfornament[width=2cm]{57}}%
			\rput[tl]{30}(1,2.5){\pgfornament[width=2cm,symmetry=v]{57}}}}%
	\end{picture}
	\end{center}
	\vspace*{\fill}
	\pagebreak
}

% Miscellanee
\author{Emanuele Fava}
\date{}
\pagestyle{fancy}


\begin{document}

\titolo{Fisica Matematica 3}

\tableofcontents
\newpage

\section{Elementi Base}

\subsection{Definizioni Principali}

\begin{defn}{Punto secondo Euclide}{}
	Un punto è un ente che non ha parti ovvero non ha dimensioni
\end{defn}

\begin{defn}{Punto Materiale}{}
	Un $\textbf{Punto Materiale}$ è un punto Euclideo alla quale viene associato un valore di $\mathbb{R}^+$ chiamato Massa del punto materiale
\end{defn}

\begin{defn}{Massa del Punto Materiale}{}
	Si fissa un punto materiale campione $P_0$ al quale viene associato un \textbf{numero positivo} $m_0>0$ (unità di misura). Ad un generico punto $P$ si associa un numero positivo $m(P)>0$ (\textit{$P$ può anche essere non specificato se non c'è possibilità di errore}), che dipende dal numero campione fissato e dalla natura della materia di cui è costituito.
	\[ P \mapsto m(P) \]
\end{defn}

\begin{defn}{Densità del Corpo nel Punto $P$}{}
	Si definisce \textbf{Densità del Corpo nel Punto $P$} la quantità:
	\[\rho(P):= \frac{dm}{dv}\]
\end{defn}

\begin{defn}{Corpo Rigido}{}
	Un \textbf{Corpo Rigido} $\mathscr C$ è un sistema meccanico (un insieme di punti materiali) costituito da infiniti punto tali che la loro distanza reciproca si mantiene costante nel tempo, cioè:
	\[\forall P, Q \in \mathscr C, \|P-Q\| = \text{ costante}, \forall t\]
\end{defn}

Possiamo immaginarlo in questo modo
\begin{center}
	\begin{tikzpicture}
		\shade[ball color = gray!40, opacity = 0.4] (0,-0) ellipse (2cm and 1.5cm);
		\draw (0,0) circle (2cm and 1.5cm);
		\filldraw (0.6,0.4) circle (1pt) node[above]{$R_0$} (1.4,0) circle (1pt) node[right]{$Q_0$} (0.8,-0.4) circle (1pt) node[below]{$S_0$} (0.4,0) circle(1pt) node[left]{$P_0$};
		\draw (0,2) node{$t_0$} (-1.7, 1.4)node{$\mathscr C$} (2.5,0)node{$\Rightarrow$};
		\draw (0.6,0.4) -- (1.4,0) -- (0.8,-0.4) -- cycle;
		\draw[dashed] (0.4,0) -- (0.6,0.4) (0.4,0) -- (1.4,0) (0.4,0) -- (0.8,-0.4);
	\end{tikzpicture}
	\begin{tikzpicture}
		\shade[ball color = gray!40, opacity = 0.4, rotate = 90] (0,-0) ellipse (2cm and 1.5cm);
		\draw[rotate = 90] (0,0) circle (2cm and 1.5cm);
		\filldraw[rotate = 90] (0.6,0.4) circle (1pt) node[left]{$R(t)$} (1.4,0) circle (1pt) node[right]{$Q(t)$} (0.8,-0.4) circle (1pt) node[right]{$S(t)$} (0.4,0) circle(1pt) node[below]{$P(t)$};
		\draw[rotate = 90] (1.7,-1.4) node{$t$} (1.7, 1.4)node{$\mathscr C$};
		\draw[rotate = 90] (0.6,0.4) -- (1.4,0) -- (0.8,-0.4) -- cycle;
		\draw[dashed, rotate = 90] (0.4,0) -- (0.6,0.4) (0.4,0) -- (1.4,0) (0.4,0) -- (0.8,-0.4);
	\end{tikzpicture}
\end{center}



\end{document}
