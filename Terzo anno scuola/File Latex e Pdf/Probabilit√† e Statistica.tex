\documentclass[11pt, a4paper, twoside]{article}

% Pacchetti Base
\usepackage[italian]{babel}
\usepackage{amsfonts}
\usepackage{amssymb}
\usepackage{tikz}
\usetikzlibrary{cd}
\usepackage{extarrows}
\usepackage{mathrsfs}
\usepackage{amsthm}

% Estetica
\usepackage{tcolorbox}
\tcbuselibrary{theorems}
\usepackage{hyperref}
\usepackage{graphicx}
\graphicspath{ {./Immagini/} }
\usepackage{bm}
\usepackage{fouriernc}
\usepackage[T1]{fontenc}
\usepackage{tikzrput}
\usepackage[object=vectorian]{pgfornament}
\usepackage{xcolor}
\usepackage{fourier-orns}
\usepackage{fancyhdr}
\renewcommand{\headrule}{%
\vspace{-8pt}\hrulefill
\raisebox{-2.1pt}{\quad\decofourleft\decotwo\decofourright\quad}\hrulefill}
\usetikzlibrary{cd,decorations.pathmorphing,patterns}
\usepackage[T1]{fontenc}

% Pacchetti TCB
\tcbuselibrary{documentation}
\tcbuselibrary{breakable}

% Definizioni e Teoremi
\newtcbtheorem
	[number within = subsection]% init options
	{defn}% name
	{Definizione}% title
	{%
		colback=teal!5,
		colframe=teal!90!black!95,
		fonttitle=\bfseries,
	}% options
	{def}% prefix

\newtcbtheorem
	[use counter from = defn, number within = subsection]% init options
	{thm}% name
	{Teorema}% title
	{%
		colback=blue!5,
		colframe=blue!90!black!95,
		fonttitle=\bfseries,
	}% options
	{th}% prefix

\newtcbtheorem
	[use counter from = defn, number within = subsection]% init options
	{prop}% name
	{Proposizione}% title
	{%
		colback=red!5,
		colframe=red!90!black!95,
		fonttitle=\bfseries,
	}% options
	{pr}% prefix

\newtcbtheorem
	[use counter from = defn, number within = subsection]% init options
	{lemma}% name
	{Lemma}% title
	{%
		colback=red!5,
		colframe=red!90!black!95,
		fonttitle=\bfseries,
	}% options
	{le}% prefix

\newtcbtheorem
	[use counter from = defn, number within = subsection]% init options
	{cor}% name
	{Corollario}% title
	{%
		colback=red!5,
		colframe=red!90!black!95,
		fonttitle=\bfseries,
	}% options
	{co}% prefix

\newcommand{\vareps}{\varepsilon}
\newtheorem{es}{Esempio}
\theoremstyle{definition}
\newtheorem*{oss}{Osservazione}
\newtheorem{ese}{Esercizio}

\tcolorboxenvironment{ese}{
	colframe=black}
\newenvironment{sol}
	{\renewcommand\qedsymbol{$\blacksquare$}\begin{proof}[Soluzione]}
	{\end{proof}}

\tcolorboxenvironment{proof}{% `proof' from `amsthm'
	blanker,breakable,left=2.5mm,
	before skip=10pt,after skip=10pt,
	borderline west={0.5mm}{0pt}{black}}

% Formattazione Pagine
\fancyhf{}
\fancyhead[LE]{\nouppercase{\rightmark\hfill\leftmark}}
\fancyhead[RO]{\nouppercase{\leftmark\hfill\rightmark}}
\fancyfoot[LE,RO]{\hrulefill\raisebox{-2.1pt}{\quad\thepage\quad}\hrulefill}
\setlength{\headheight}{16pt}

% Margini
\topmargin=-0.45in
\evensidemargin=-0.15in
\oddsidemargin=-0.15in
\textwidth=6.5in
\textheight=9.0in
\headsep=0.25in

%Titolo
\newcommand{\titolo}[1]{
	\thispagestyle{empty}
	\vspace*{\fill}
	\tikzset{pgfornamentstyle/.style={draw = black, fill = teal!50}}
	\unitlength=1cm
	\begin{center}
	\begin{picture}(12,12)%
		\color{white}%
			\put(0,0){\framebox(12,12){%
			\rput[tl](-4,6){\pgfornament[width=8cm]{71}}%
			\rput[bl](-4,-6){\pgfornament[width=8cm,,symmetry=h]{71}}%
			\rput[tl](-6,6){\pgfornament[width=2cm]{63}}%
			\rput[tr](6,6){\pgfornament[width=2cm,,symmetry=v]{63}}%
			\rput[bl](-6,-6){\pgfornament[width=2cm,,symmetry=h]{63}}%
			\rput[br](6,-6){\pgfornament[width=2cm,,symmetry=c]{63}}%
			\rput[bl]{-90}(-6,4){\pgfornament[width=8cm]{46}}%
			\rput[bl]{90}(6,-4){\pgfornament[width=8cm]{46}}%
			\rput(0,0){\Huge \color{black} #1}%
			\rput[t](0,-0.5){\pgfornament[width=5cm]{75}}%
			\rput[b](0,0.5){\pgfornament[width=5cm]{69}}%
			\rput[tr]{-30}(-1,2.5){\pgfornament[width=2cm]{57}}%
			\rput[tl]{30}(1,2.5){\pgfornament[width=2cm,symmetry=v]{57}}}}%
	\end{picture}
	\end{center}
	\vspace*{\fill}
	\pagebreak
}

% Miscellanee
\author{Emanuele Fava}
\date{}
\pagestyle{fancy}


\begin{document}

\titolo{Statistica}

\tableofcontents

\newpage

\section{Spazio di Probabilità}

Per poter studiare Probabilità e statistica possiamo immaginare di fare un parallelismo con l'analisi:

\begin{center}
	\begin{tabular}{|l|l|}
		\hline
		\textbf{Probabilità e Statistica} & \textbf{Analisi Matematica}\\
		\hline
		{Spazio di Probabilità} & {Numeri reali}\\
		\hline
		{Variabile Aleatoria} & {Variabile reale}\\
		\hline
		{Convergenza per successioni di variabili aleatorie} & {Successioni}\\
		\hline
		{($\mathbb M$) Processo Stocastico} & {Funzioni}\\
		\hline
		{($\mathbb M$) Calcolo Stocastico} & {Calcolo differenziale e integrale}\\
		\hline
		{($\mathbb M$) Equazioni Differenziali Stocastiche } & {Equazioni Differenziali}\\
		\hline
	\end{tabular}
\end{center}

\textit{Il $(\mathbb M)$ sta indicare che non sono argomenti trattati in questo corso, ma saranno approfonditi alla magistrale}

\subsection{Cenni di Teoria della Misura}

\begin{defn}{Spazio Misurabile}{}
	Si definisce \textbf{Spazio Misurabile} una coppia $(\Omega, \mathcal F)$, dove $\Omega$ è un insieme non vuoto e $\mathcal F$ è una $\sigma-$algebra, cioè una famiglia non vuota di sottoinsiemi di $\Omega$ che soddisfa due proprietà:
	\begin{enumerate}
		\item $A \in \mathcal F \Rightarrow A^c \in \mathcal F$, cioè $\mathcal F$ è chiuso rispetto al complementare
		\item Se $(A_n)_{n \in \mathbb N}$ è una successione in $\mathcal F$, allora
			\[ \bigcup_{n \in \mathbb N}A_n \in \mathcal F \]
			cioè $\mathcal F$ è chiusa rispetto all'unione numerabile
	\end{enumerate}
\end{defn}

Il prefisso $"\sigma-"$ davanti a queste parole sta ad indicare che si tratta di una proprietà \underline{numerabile}. Quindi le due proprietà sopra elencate possono essere scritte come "$\mathcal F$ è $\sigma-\cup-$chiuso"

\begin{oss}
	Sottolineiamo che $\Omega$ può essere un insieme qualunque (un esempio, forse il più facile da tenere a mente è quello della Teoria della Misura di Lebesgue, un esempio può essere $(\Omega, \mathcal F) = (\mathbb R^n, \mathcal M(\mathbb R^n))$) Quindi possiamo prendere insiemi qualunque, che siano numerici, come $\mathbb N, \mathbb C$, vettoriali $\mathbb R^n$, oppure qualcosa di totalmente diverso, come l'insieme delle funzioni continue $C(\mathbb R)$, l'insieme dei polinomi $\mathbb R[x]$ addirittura l'insieme delle sedie in una stanza.
\end{oss}

\begin{oss}[$\sigma-$algebre banali]
	Esistono due tipi di $\sigma-$algebre particolari, dette \textbf{Banali} in quanto sono definite come:
	\[ \mathcal F = \{\varnothing, \Omega\} \qquad \text e \qquad \mathcal F = \mathcal P(\Omega) \]
	In particolare abbiamo che valgono:
	\[ \{\varnothing, \Omega\} \subseteq \mathcal F \subseteq \mathcal P(\Omega) \qquad \forall \mathcal F \text{ }\sigma- \text{algebra}\]
\end{oss}

\begin{prop}{}{}
	Se $\mathcal F$ è una $\sigma-$algebra, allora è $\cup-$chiusa (cioè l'unione finita è chiusa)
\end{prop}

\begin{proof}
	Siano $A_1,...,A_n \in \mathcal F$ e siano:
	\[
		\bigcup_{k=1}^n A_n = \bigcup_{k=1}^{+\infty} \overline A_k \qquad \text{dove } \overline A_k =
		\begin{cases}
			A_k & k\leq n\\
			A_n & k>n
		\end{cases}
	\]
	Tuttavia, sappiamo che $\mathcal F$ è una $\sigma-$algebra, quindi il secondo elemento sta in $\mathcal F$. Tuttavia, essendo il secondo uguale al primo abbiamo che anche il primo sta in $\mathcal F$. Per l'arbitrarietà degli $A_k$, segue che $\mathcal F$ è $\cup-$chiuso
\end{proof}
\begin{oss}
	Per, definizione di $\sigma-$algebra, $\mathcal F$ è non vuoto. Sia $A\in\mathcal F$ vale che:
	\[
		A^c  \in \mathcal F \quad\Rightarrow\quad A\cup A^c=\Omega\in \mathcal F \text{ e }\varnothing=\Omega^c\in \mathcal F
	\]
questo implica che $\varnothing$ e $\Omega$ appartengono a ogni $\mathcal F$ $\sigma$-algebra. 
\end{oss}
\begin{oss}
	Prendiamo $(A_n)_{n \in \mathbb N}$ successione in $\mathcal F$ allora dalla definizione vale che:
	\[
		\bigcap\limits_{n\in\mathbb{N}} A_n=\left(\bigcup\limits_{n\in\mathbb{N}}A^c\right)^c\in\mathcal F
	\]
\end{oss}
\begin{defn}{Misura su $(\Omega,\mathcal F)$}{}
	Definiamo una \textbf{Misura} su uno spazio misurabile come una funzione $\mu:\mathcal F\rightarrow[0,+\infty]$ tale che:
	\begin{enumerate}
		\item $\mu(\varnothing)=0$
		\item $\mu$ è $\sigma$-additiva ovvero:
		\[
			\text{Se }(A_n) \text{ è una successione in }\mathcal F \text{ i cui elementi sono disgiunti } \implies\mu(\bigcup\limits_{n\geq1}A_n)=\sum\limits_{n\geq1}\mu(A_n)
		\]
	\end{enumerate}
\end{defn}
\begin{oss}[la misura è additiva su somme finite]
	 Usiamo la Notazione $\biguplus$ per parlare di unione disgiunta. Siano $A_1,\dots,A_n$ disgiunti e siano $A_k=\varnothing$ $\forall k\geq n+1$, dalla definizione di misura segue banalmente che $\mu(A_k)=0$ $\forall k\geq n+1$ allora dal fatto poichè $\mu$ è $\sigma$-additiva vale che:
	\[
		\mu\left(\biguplus\limits_{k=1}^n A_k\right)=\mu\left(\biguplus\limits_{k=1}^{+\infty} A_k\right)=\sum\limits_{k=1}^{+\infty}\mu(A_k)=\sum\limits_{k=1}^n\mu(A_k)
	\]
\end{oss}
\begin{defn}{Spazio di Probabilità}{}
	Definiamo uno \textbf{Spazio di Probabilità} come una tripla $(\Omega,\mathcal F, P)$, ovvero uno spazio misurabile $(\Omega,\mathcal F)$ con misura $P$ tale che:
	\[
		P(\Omega)= 1
	\] 
\end{defn}
\begin{defn}{Spazio Campionario}{}
	Manu è bello e qui ci scrive quello che vuole :3
\end{defn}
\begin{defn}{Spazio Discreto}{}
	Uno spazio si dice $\textbf{Discreto}$ se $\Omega$ è finito o numerabile in questo caso prendiamo:
	\[
		\mathcal F =\mathcal P(\Omega)\quad \text{e scriviamo}\quad (\Omega,\mathcal F, P)=(\Omega, P)
	\]
\end{defn}
\begin{es}[Lancio di un dado]
	In questo caso abbiamo: $\Omega=\{1,\dots,6\}$, $\mathcal F=$ famiglia degli eventi dove $A\in\mathcal F$ è un evento ovvero ''un affermazione relativa all'esito dell'esperimento'', $P=$ Misura di probabilità è la funzione $P:\mathcal F\rightarrow[0,1]$ che manda A nella probabilià che l'esito sia positivo.   

Per esempio sia $A= \{1,3,5\}\subseteq \Omega$ ovvero le possibili facce che lanciando il Dado diano un esito positivo, allora $P(A)=\frac{1}{2}$. Notiamo inoltre che lanciando il dado uscirà sempre almeno una faccia questo equivale a dire che $P(\varnothing)=0$ e $P(\Omega)=1$.
\end{es}
\begin{oss}
	Possiamo fare un parallelismo tra gli insiemi misurabili e la probabilità aiutandoci con la ''terminologia'':
	\begin{center}
		\begin{tabular}{|l|l|}
			\hline
			\textbf{Analisi e Insiemistica} & \textbf{Probabilità e statistica}\\
			\hline
			{$A\cup B$} & {Evento $A$ ''oppure'' Evento $B$}\\
			\hline
			{$A\cap B$} & {Evento $A$ ''e'' Evento $B$}\\
			\hline
			{$A^c$} & {''Non Evento A''}\\
			\hline
			{$\mu(\mathbb{Q})=0$} & {$P(A)=0\Rightarrow A$ è ''non misurabile''}\\
			\hline
			{$\mu(\mathbb{R}\setminus\mathbb{Q})=+\infty$} & {$P(A)=1\Rightarrow A$ è ''quasi certo''}\\
			\hline
		\end{tabular}
	\end{center}
\end{oss}
\end{document}

