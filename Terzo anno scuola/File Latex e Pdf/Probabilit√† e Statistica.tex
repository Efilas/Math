\documentclass[11pt, a4paper, twoside]{article}

% Pacchetti Base
\usepackage[italian]{babel}
\usepackage{amsfonts}
\usepackage{amssymb}
\usepackage{tikz}
\usetikzlibrary{cd}
\usepackage{extarrows}
\usepackage{mathrsfs}
\usepackage{amsthm}

% Estetica
\usepackage{tcolorbox}
\tcbuselibrary{theorems}
\usepackage{hyperref}
\usepackage{graphicx}
\graphicspath{ {./Immagini/} }
\usepackage{bm}
\usepackage{fouriernc}
\usepackage[T1]{fontenc}
\usepackage{tikzrput}
\usepackage[object=vectorian]{pgfornament}
\usepackage{xcolor}
\usepackage{fourier-orns}
\usepackage{fancyhdr}
\renewcommand{\headrule}{%
\vspace{-8pt}\hrulefill
\raisebox{-2.1pt}{\quad\decofourleft\decotwo\decofourright\quad}\hrulefill}
\usetikzlibrary{cd,decorations.pathmorphing,patterns}
\usepackage[T1]{fontenc}

% Pacchetti TCB
\tcbuselibrary{documentation}
\tcbuselibrary{breakable}

% Definizioni e Teoremi
\newtcbtheorem
	[number within = subsection]% init options
	{defn}% name
	{Definizione}% title
	{%
		colback=teal!5,
		colframe=teal!90!black!95,
		fonttitle=\bfseries,
	}% options
	{def}% prefix

\newtcbtheorem
	[use counter from = defn, number within = subsection]% init options
	{thm}% name
	{Teorema}% title
	{%
		colback=blue!5,
		colframe=blue!90!black!95,
		fonttitle=\bfseries,
	}% options
	{th}% prefix

\newtcbtheorem
	[use counter from = defn, number within = subsection]% init options
	{prop}% name
	{Proposizione}% title
	{%
		colback=red!5,
		colframe=red!90!black!95,
		fonttitle=\bfseries,
	}% options
	{pr}% prefix

\newtcbtheorem
	[use counter from = defn, number within = subsection]% init options
	{lemma}% name
	{Lemma}% title
	{%
		colback=red!5,
		colframe=red!90!black!95,
		fonttitle=\bfseries,
	}% options
	{le}% prefix

\newtcbtheorem
	[use counter from = defn, number within = subsection]% init options
	{cor}% name
	{Corollario}% title
	{%
		colback=red!5,
		colframe=red!90!black!95,
		fonttitle=\bfseries,
	}% options
	{co}% prefix

\newcommand{\vareps}{\varepsilon}
\newtheorem{es}{Esempio}
\theoremstyle{definition}
\newtheorem*{oss}{Osservazione}
\newtheorem{ese}{Esercizio}

\tcolorboxenvironment{ese}{
	colframe=black}
\newenvironment{sol}
	{\renewcommand\qedsymbol{$\blacksquare$}\begin{proof}[Soluzione]}
	{\end{proof}}

\tcolorboxenvironment{proof}{% `proof' from `amsthm'
	blanker,breakable,left=2.5mm,
	before skip=10pt,after skip=10pt,
	borderline west={0.5mm}{0pt}{black}}

% Formattazione Pagine
\fancyhf{}
\fancyhead[LE]{\nouppercase{\rightmark\hfill\leftmark}}
\fancyhead[RO]{\nouppercase{\leftmark\hfill\rightmark}}
\fancyfoot[LE,RO]{\hrulefill\raisebox{-2.1pt}{\quad\thepage\quad}\hrulefill}
\setlength{\headheight}{16pt}

% Margini
\topmargin=-0.45in
\evensidemargin=-0.15in
\oddsidemargin=-0.15in
\textwidth=6.5in
\textheight=9.0in
\headsep=0.25in

%Titolo
\newcommand{\titolo}[1]{
	\thispagestyle{empty}
	\vspace*{\fill}
	\tikzset{pgfornamentstyle/.style={draw = black, fill = teal!50}}
	\unitlength=1cm
	\begin{center}
	\begin{picture}(12,12)%
		\color{white}%
			\put(0,0){\framebox(12,12){%
			\rput[tl](-4,6){\pgfornament[width=8cm]{71}}%
			\rput[bl](-4,-6){\pgfornament[width=8cm,,symmetry=h]{71}}%
			\rput[tl](-6,6){\pgfornament[width=2cm]{63}}%
			\rput[tr](6,6){\pgfornament[width=2cm,,symmetry=v]{63}}%
			\rput[bl](-6,-6){\pgfornament[width=2cm,,symmetry=h]{63}}%
			\rput[br](6,-6){\pgfornament[width=2cm,,symmetry=c]{63}}%
			\rput[bl]{-90}(-6,4){\pgfornament[width=8cm]{46}}%
			\rput[bl]{90}(6,-4){\pgfornament[width=8cm]{46}}%
			\rput(0,0){\Huge \color{black} #1}%
			\rput[t](0,-0.5){\pgfornament[width=5cm]{75}}%
			\rput[b](0,0.5){\pgfornament[width=5cm]{69}}%
			\rput[tr]{-30}(-1,2.5){\pgfornament[width=2cm]{57}}%
			\rput[tl]{30}(1,2.5){\pgfornament[width=2cm,symmetry=v]{57}}}}%
	\end{picture}
	\end{center}
	\vspace*{\fill}
	\pagebreak
}

% Miscellanee
\author{Emanuele Fava}
\date{}
\pagestyle{fancy}


\begin{document}

\titolo{Statistica}

\tableofcontents

\newpage

\section{Spazio di Probabilità}

Per poter studiare Probabilità e statistica possiamo immaginare di fare un parallelismo con l'analisi:

\begin{center}
	\begin{tabular}{|l|l|}
		\hline
		\textbf{Probabilità e Statistica} & \textbf{Analisi Matematica}\\
		\hline
		{Spazio di Probabilità} & {Numeri reali}\\
		\hline
		{Variabile Aleatoria} & {Variabile reale}\\
		\hline
		{Convergenza per successioni di variabili aleatorie} & {Successioni}\\
		\hline
		{($\mathbb M$) Processo Stocastico} & {Funzioni}\\
		\hline
		{($\mathbb M$) Calcolo Stocastico} & {Calcolo differenziale e integrale}\\
		\hline
		{($\mathbb M$) Equazioni Differenziali Stocastiche } & {Equazioni Differenziali}\\
		\hline
	\end{tabular}
\end{center}

\textit{Il $(\mathbb M)$ sta indicare che non sono argomenti trattati in questo corso, ma saranno approfonditi alla magistrale}

\subsection{Cenni di Teoria della Misura}

\begin{defn}{Spazio Misurabile}{}
	Si definisce \textbf{Spazio Misurabile} una coppia $(\Omega, \mathcal F)$, dove $\Omega$ è un insieme non vuoto e $\mathcal F$ è una $\sigma-$algebra, cioè una famiglia non vuota di sottoinsiemi di $\Omega$ che soddisfa due proprietà:
	\begin{enumerate}
		\item $A \in \mathcal F \Rightarrow A^c \in \mathcal F$, cioè $\mathcal F$ è chiuso rispetto al complementare
		\item Se $(A_n)_{n \in \mathbb N}$ è una successione in $\mathcal F$, allora
			\[ \bigcup_{n \in \mathbb N}A_n \in \mathcal F \]
			cioè $\mathcal F$ è chiusa rispetto all'unione numerabile
	\end{enumerate}
\end{defn}

Il prefisso $"\sigma-"$ davanti a queste parole sta ad indicare che si tratta di una proprietà \underline{numerabile}. Quindi le due proprietà sopra elencate possono essere scritte come "$\mathcal F$ è $\sigma-\cup-$chiuso"

\begin{oss}
	Sottolineiamo che $\Omega$ può essere un insieme qualunque (un esempio, forse il più facile da tenere a mente è quello della Teoria della Misura di Lebesgue, un esempio può essere $(\Omega, \mathcal F) = (\mathbb R^n, \mathcal M(\mathbb R^n))$) Quindi possiamo prendere insiemi qualunque, che siano numerici, come $\mathbb N, \mathbb C$, vettoriali $\mathbb R^n$, oppure qualcosa di totalmente diverso, come l'insieme delle funzioni continue $C(\mathbb R)$, l'insieme dei polinomi $\mathbb R[x]$ addirittura l'insieme delle sedie in una stanza.
\end{oss}

\begin{oss}[$\sigma-$algebre banali]
	Esistono due tipi di $\sigma-$algebre particolari, dette \textbf{Banali} in quanto sono definite come:
	\[ \mathcal F = \{\varnothing, \Omega\} \qquad \text e \qquad \mathcal F = \mathcal P(\Omega) \]
	In particolare abbiamo che valgono:
	\[ \{\varnothing, \Omega\} \subseteq \mathcal F \subseteq \mathcal P(\Omega) \qquad \forall \mathcal F \text{ }\sigma- \text{algebra}\]
\end{oss}

\begin{prop}{}{}
	Se $\mathcal F$ è una $\sigma-$algebra, allora è $\cup-$chiusa (cioè l'unione finita è chiusa)
\end{prop}

\begin{proof}
	Siano $A_1,...,A_n \in \mathcal F$ e siano:
	\[
		\bigcup_{k=1}^n A_n = \bigcup_{k=1}^{+\infty} \overline A_k \qquad \text{dove } \overline A_k =
		\begin{cases}
			A_k & k\leq n\\
			A_n & k>n
		\end{cases}
	\]
	Tuttavia, sappiamo che $\mathcal F$ è una $\sigma-$algebra, quindi il secondo elemento sta in $\mathcal F$. Tuttavia, essendo il secondo uguale al primo abbiamo che anche il primo sta in $\mathcal F$. Per l'arbitrarietà degli $A_k$, segue che $\mathcal F$ è $\cup-$chiuso
\end{proof}
\begin{oss}
	Per, definizione di $\sigma-$algebra, $\mathcal F$ è non vuoto. Sia $A\in\mathcal F$ vale che:
	\[
		A^c  \in \mathcal F \quad\Rightarrow\quad A\cup A^c=\Omega\in \mathcal F \quad \text{e} \quad \varnothing=\Omega^c\in \mathcal F
	\]
questo implica che $\varnothing$ e $\Omega$ appartengono a ogni $\mathcal F$ $\sigma$-algebra.
\end{oss}
\begin{oss}
	Prendiamo $(A_n)_{n \in \mathbb N}$ successione in $\mathcal F$ allora dalla definizione vale che:
	\[
		\bigcap\limits_{n\in\mathbb{N}} A_n=\left(\bigcup\limits_{n\in\mathbb{N}}A^c\right)^c\in\mathcal F
	\]
\end{oss}
\begin{defn}{Misura su $(\Omega,\mathcal F)$}{}
	Definiamo una \textbf{Misura} su uno spazio misurabile come una funzione $\mu:\mathcal F\rightarrow[0,+\infty]$ tale che:
	\begin{enumerate}
		\item $\mu(\varnothing)=0$
		\item $\mu$ è $\sigma$-additiva ovvero:
		\[
			\text{Se }(A_n) \text{ è una successione in }\mathcal F \text{ i cui elementi sono disgiunti } \implies\mu\left(\bigcup\limits_{n\geq1}A_n\right)=\sum\limits_{n\geq1}\mu\left(A_n\right)
		\]
	\end{enumerate}
\end{defn}
\begin{oss}[La misura è additiva su somme finite]
	Usiamo la Notazione $\biguplus$ per parlare di unione disgiunta. Siano $A_1,\dots,A_n$ disgiunti e siano $A_k=\varnothing$ $\forall k\geq n+1$, dalla definizione di misura segue banalmente che $\mu(A_k)=0$ $\forall k\geq n+1$ allora dal fatto poichè $\mu$ è $\sigma$-additiva vale che:
	\[
		\mu\left(\biguplus\limits_{k=1}^n A_k\right)=\mu\left(\biguplus\limits_{k=1}^{+\infty} A_k\right)=\sum\limits_{k=1}^{+\infty}\mu(A_k)=\sum\limits_{k=1}^n\mu(A_k)
	\]
\end{oss}
\begin{defn}{Spazio di Probabilità}{}
	Definiamo uno \textbf{Spazio di Probabilità} come una tripla $(\Omega,\mathcal F, P)$, cioè uno spazio misurabile $(\Omega,\mathcal F)$ con misura $P$ tale che:
	\[
		P(\Omega)= 1
	\]
\end{defn}

Diamo adesso dei nomi agli strumenti che stiamo utilizzando, in particolare utilizzando un esempio, quello del dado a $6$ facce:

\begin{center}
	\begin{tabular}{|l|l|l|}
		\hline
		\textbf{Elemento} & \textbf{Nome} & \textbf{Esempio}\\
		\hline
		{$\Omega$} & \textbf{Spazio Campionario} & {$\Omega = \{1,2,3,4,5,6\}$}\\
		\hline
		{$\omega \in \Omega$} & \textbf{Esito} & {$\omega = 1$}\\
		\hline
		{$\mathcal F$} & \textbf{Famiglia degli Eventi} & {$\mathcal F = \mathcal P(\Omega)$}\\
		\hline\textbf{}
		{$A \in \mathcal F$} & \textbf{Evento} & {$A = \{1,3,5\}$}\\
		\hline
		{$P$} & \textbf{Misura di Probabilità} & {$P(A)$}\\
		\hline
	\end{tabular}
\end{center}


\begin{defn}{Spazio Discreto}{}
	Uno spazio si dice $\textbf{Discreto}$ se $\Omega$ è finito o numerabile in questo caso prendiamo:
	\[
		\mathcal F =\mathcal P(\Omega)\quad \text{e scriviamo}\quad (\Omega,\mathcal F, P)=(\Omega, P)
	\]
\end{defn}
\begin{es}[Lancio di un dado]
	In questo caso abbiamo: $\Omega=\{1,\dots,6\}$, $\mathcal F=$ famiglia degli eventi dove $A\in\mathcal F$ è un evento ovvero ''un affermazione relativa all'esito dell'esperimento'', $P=$ Misura di probabilità è la funzione $P:\mathcal F\rightarrow[0,1]$ che manda A nella probabilià che l'esito sia positivo.

Per esempio sia $A= \{1,3,5\}\subseteq \Omega$ ovvero le possibili facce che lanciando il Dado diano un esito positivo, allora $P(A)=\frac{1}{2}$. Notiamo inoltre che lanciando il dado uscirà sempre almeno una faccia questo equivale a dire che $P(\varnothing)=0$ e $P(\Omega)=1$.
\end{es}
\begin{oss}
	Possiamo fare un parallelismo tra gli insiemi misurabili e la probabilità aiutandoci con la ''terminologia'':
	\begin{center}
		\begin{tabular}{|l|l|}
			\hline
			\textbf{Analisi e Insiemistica} & \textbf{Probabilità e statistica}\\
			\hline
			{$A\cup B$} & {Evento $A$ ''oppure'' Evento $B$}\\
			\hline
			{$A\cap B$} & {Evento $A$ ''e'' Evento $B$}\\
			\hline
			{$A^c$} & {''Non Evento A''}\\
			\hline
			{$\mu(\mathbb{Q})=0$} & {$P(A)=0\Rightarrow A$ è ''non misurabile''}\\
			\hline
			{$\mu(\mathbb{R}\setminus\mathbb{Q})=+\infty$} & {$P(A)=1\Rightarrow A$ è ''quasi certo''}\\
			\hline
		\end{tabular}
	\end{center}
\end{oss}

\begin{es}[Corridore e le due Gare]\label{Corridore}
	Abbiamo un corridore che partecipa a 2 gare dove, ottimisticamente parlando, vorrebbe vincere le due gare:
	\begin{itemize}
		\item $A$ = Vince la prima gara
		\item $B$ = Vince la seconda gara
	\end{itemize}
	Supponiamo di avere come dati:
	\begin{itemize}
		\item $P(A)$ = $30\%$ la probabilità di vincere di la prima gara
		\item $P(B)$ = $40\%$ la probabilità di vincere la seconda gara
		\item $P(A\cup B)$ = $50\%$ la probabilità di vincere la prima o la seconda gara
	\end{itemize}

	\begin{center}
		\begin{tikzpicture}
			\draw (-1,1) arc(-270:90:2cm and 1 cm) node[above]{$A$};
			\draw (1,1) arc(-270:90:2cm and 1cm) node[above]{$B$};
			\filldraw[green, very nearly transparent] (-1,1) arc(-270:90:2cm and 1 cm);
			\filldraw[blue, very nearly transparent] (1,1) arc(-270:90:2cm and 1cm);
			\draw (0,0) node{$A\cap B$};
		\end{tikzpicture}
	\end{center}

	Allora dall'additività della misura possiamo ricavare:
		\[
			P(A\cup B)=P(A\uplus (B\setminus A))= P(A)+P(B\setminus A)=P(A)+P(B\setminus A)+P(A\cap B)- P(A\cap B)
		\]
	e poiché $P(B\setminus A)+P(A\cap B)=P(B)$ otteniamo:
		\[
			P(A\cup B)=P(A)+P(B)- P(A\cap B)\quad \Rightarrow\quad P(A\cap B)=P(A)+P(B)-P(A\cup B)
		\]
	Andando a sostituire otteniamo:
		\[
			P(A\cup B)=P(A)+P(B)-P(A\cup B)=0.3+0.4-0.5=0.2=20\%
		\]
\end{es}

\begin{oss}
	L'esempio precedente funziona perché siamo in uno spazio discreto. Prendiamo $(\Omega,P)$ discreto ovvero $\#\Omega\leq\aleph_0$ quindi considerando $\omega$ esito e $\{\omega\}$ evento elementare sappiamo che vale:
	\[
		P(A)=P\left(\biguplus\limits_{\omega\in A} \{\omega\}\right)=\sum\limits_{\omega\in A}P(\{\omega\})
	\]
	Quindi data $P$ misura di probabilità posso definire la seguente funzione:
	\begin{align*}
		p:\quad&\Omega\rightarrow[0,1]\\
		&\omega\mapsto P(\{\omega\})=p(\omega)
	\end{align*}
	Dove sappiamo che valgono:
	\begin{itemize}
		\item $1=P(\Omega)=\sum\limits_{w\in\Omega} p(\omega)$
		\item $p(\omega)\geq 0$
	\end{itemize}
	Concludendo possiamo dire che se $(\Omega,P)$ è discreto, posso studiare la probabilità sui singoli eventi elementari.
\end{oss}

È importante osservare che avere la probabilità dei singoli esiti o eventi elementari risulta fondamentale nel calcolo di vari eventi. In particolare se lo spazio campionario di partenza è molto grande oppure numerabile.

\begin{es}
	Se abbiamo uno spazio campionario $\Omega$ tale che $\# \Omega = 100$, allora abbiamo che $\# \mathcal P(\Omega) = 2^{100}$.
	Risulta quindi estremamente più comodo conoscere le probabilità dei singoli eventi elementari che di tutti gli eventi possibili.
	Per questo motivo, in questo caso, è opportuno ricordare $p$ definita sugli esiti che $P$ sugli eventi.

	Viceversa, se $\Omega$ è discreto, cioè è tale che $\Omega = \{\omega_1,\omega_2,...\}$, e conosciamo le singole probabilità di ogni esito, allora possiamo considerare una successione $(p_n)n$, definite come $p_n = p(\omega_n)$, e possiamo calcolare $P(A)$, per $A \in \mathcal F$, come:
	\[ P(A) = \sum_{\omega_n \in A}p_n \]
	Quindi $p$ mi definisce una misura di probabilità $P$.
\end{es}

Tutto quello che abbiamo fatto è fattibile in quanto abbiamo che $\Omega$ è discreto, se così non fosse non sarebbe possibile.

\begin{es}[Probabilità Uniforme]
	Se abbiamo che $\# \Omega = N$, allora possiamo prendere $p_n = \frac 1N$ per $n \in \{1,2,...,N\}$, allora segue che la probabilità di ogni evento è:
	\[ P(A) = \frac{\# A}{\# \Omega} \]
	Cioè andiamo a guardare le cardinalità degli insiemi. In sintesi questo sarebbe "Casi Favorevoli su Casi Totali"
\end{es}

\begin{es}[Continuo del Corridore, Esempio \ref{Corridore}] %Serve per i collegamenti interni
	Possiamo scrivere $\Omega$ come:
	\[ \Omega = \{ p_1 = vv, p_2 = vp, p_3 = pv, p_4 = pp \} \]
	Allora, riprendendo gli eventi $A$ e $B$, questi possono essere scritti come:
	\[ A = \{ vv, vp \} = \{p_1,p_2\} = 30\% \qquad B = \{ vv, pv \} = \{p_1,p_3\} = 40\% \]
	Da cui segue che:
	\[ A\cup B = \{vv,vp,pv\} = \{p_1,p_2,p_3\} = 50\% \qquad A \cap B = \{ vv \}=\mathbf{??}\% \]
	Abbiamo allora che $P$ è univocamente determinata se conosciamo $p_1,p_2,p_3,p_4$. Tuttavia, per come abbiamo definito p, abbiamo che:
	\[ p_1+p_2+p_3+p_4 = 1 \qquad \Rightarrow \qquad p_4 = 1-p_1-p_2-p_3 \]
	Otteniamo quindi un sistema lineare di $4$ operazioni a $4$ incognite:
	\[
		\begin{cases}
			p_1 + p_2 = 0,3\\
			p_1 + p_3 = 0,4\\
			p_1 + p_2 + p_3 = 0,5\\
			p_1 + p_2 + p_3 + p_4 = 1
		\end{cases}
	\]
	Andandolo a risolvere otteniamo lo stesso risultato ottenuto nell'esempio precedente
\end{es}

Questo è anche un esempio di probabilità non uniforme

\begin{oss}
	La probabilità uniforme non è altro che un esempio di probabilità, il caso più semplice e a volte anche quello meno interessante, in quanto basta solamente guardare la cardinalità degli insiemi degli eventi e può essere usata solamente per $\Omega$ finiti. Se infatti $\Omega$ fosse numerabile, avremmo che:
	\[ 1 = P(\Omega) = \sum_{\omega \in \Omega}P(\{\omega\}) \]
	Poiché tutti gli elementi sono uguali, questa serie diverge oppure è uguale a $0$.
\end{oss}

\subsection{Proprietà Generali}

\begin{prop}{Monotonia}{}
	Vale la proprietà di monotonia per la Misura di Probabilità $P$, cioè:
	\[ \forall A,B \in \mathcal F: \quad A \subseteq B \quad \Rightarrow \quad P(A)\leq P(B) \]
\end{prop}
\begin{proof} %Unico caso in cui non separerei gli ambienti
	Graficamente abbiamo che:
	\begin{center}
		\begin{tikzpicture}
			\draw (2,0) arc(0:360:2cm and 1cm) node[right]{$B$};
			\draw (1,0) arc(0:360:1cm and 0.5cm) node[right]{$A$};
		\end{tikzpicture}
	\end{center}
	Allora possiamo considerare:
	\[ P(B) = P(A \uplus (B\setminus A)) = P(A) + \underbrace{P(B\setminus A)}_{\geq 0} \geq P(A) \]
\end{proof}

\begin{oss}
	Se $B = \Omega$, allora abbiamo che:
	$1= P(\Omega) = P(A) + P(A^c) \quad \Rightarrow \quad P(A^c) = 1-P(A)$
\end{oss}

\begin{es}[Lancio $8$ dadi]
	Sia $\Omega$ l'insieme dei possibili risultati lanciando $8$ dadi e sia $A= $ "esca almeno un $6$". In questo caso (così come quelli futuri in cui ci sarà "almeno"), è più facile calcolare $P(A^c)$, dove $A^c =$ "Non esce neanche un $6$". Scrivendo tutto per bene in formule abbiamo che:
	\[ \Omega = \{(\omega_1,...,\omega_8): \omega_i \in \{1,...,6\}\} \qquad \Rightarrow \# \Omega = 6^8\]
	Allora abbiamo che:
	\[ P(A) = 1-P(A^c) = 1 - \frac{\# A^c}{\# \Omega} = 1 - \frac{5^8}{6^8} \]
\end{es}

\end{document}
% Cerca di mantenere l'indentazione e di spaziare tra i vari ambienti
% Non sembra, ma aiuta di un sacco la lettura del codice e fidati che ti troverai meglio anche tu
% Ogni tanto se vuoi lascia commenti che li leggo

