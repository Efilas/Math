\documentclass[11pt,a4paper,twoside]{article}
% Pacchetti Base
\usepackage[italian]{babel}
\usepackage{amsfonts}
\usepackage{amssymb}
\usepackage{tikz}
\usetikzlibrary{cd}
\usepackage{extarrows}
\usepackage{mathrsfs}
\usepackage{amsthm}

% Estetica
\usepackage{tcolorbox}
\tcbuselibrary{theorems}
\usepackage{hyperref}
\usepackage{graphicx}
\graphicspath{ {./Immagini/} }
\usepackage{bm}
\usepackage{fouriernc}
\usepackage[T1]{fontenc}
\usepackage{tikzrput}
\usepackage[object=vectorian]{pgfornament}
\usepackage{xcolor}
\usepackage{fourier-orns}
\usepackage{fancyhdr}
\renewcommand{\headrule}{%
\vspace{-8pt}\hrulefill
\raisebox{-2.1pt}{\quad\decofourleft\decotwo\decofourright\quad}\hrulefill}
\usetikzlibrary{cd,decorations.pathmorphing,patterns}
\usepackage[T1]{fontenc}

% Pacchetti TCB
\tcbuselibrary{documentation}
\tcbuselibrary{breakable}

% Definizioni e Teoremi
\newtcbtheorem
	[number within = subsection]% init options
	{defn}% name
	{Definizione}% title
	{%
		colback=teal!5,
		colframe=teal!90!black!95,
		fonttitle=\bfseries,
	}% options
	{def}% prefix

\newtcbtheorem
	[use counter from = defn, number within = subsection]% init options
	{thm}% name
	{Teorema}% title
	{%
		colback=blue!5,
		colframe=blue!90!black!95,
		fonttitle=\bfseries,
	}% options
	{th}% prefix

\newtcbtheorem
	[use counter from = defn, number within = subsection]% init options
	{prop}% name
	{Proposizione}% title
	{%
		colback=red!5,
		colframe=red!90!black!95,
		fonttitle=\bfseries,
	}% options
	{pr}% prefix

\newtcbtheorem
	[use counter from = defn, number within = subsection]% init options
	{lemma}% name
	{Lemma}% title
	{%
		colback=red!5,
		colframe=red!90!black!95,
		fonttitle=\bfseries,
	}% options
	{le}% prefix

\newtcbtheorem
	[use counter from = defn, number within = subsection]% init options
	{cor}% name
	{Corollario}% title
	{%
		colback=red!5,
		colframe=red!90!black!95,
		fonttitle=\bfseries,
	}% options
	{co}% prefix

\newcommand{\vareps}{\varepsilon}
\newtheorem{es}{Esempio}
\theoremstyle{definition}
\newtheorem*{oss}{Osservazione}
\newtheorem{ese}{Esercizio}

\tcolorboxenvironment{ese}{
	colframe=black}
\newenvironment{sol}
	{\renewcommand\qedsymbol{$\blacksquare$}\begin{proof}[Soluzione]}
	{\end{proof}}

\tcolorboxenvironment{proof}{% `proof' from `amsthm'
	blanker,breakable,left=2.5mm,
	before skip=10pt,after skip=10pt,
	borderline west={0.5mm}{0pt}{black}}

% Formattazione Pagine
\fancyhf{}
\fancyhead[LE]{\nouppercase{\rightmark\hfill\leftmark}}
\fancyhead[RO]{\nouppercase{\leftmark\hfill\rightmark}}
\fancyfoot[LE,RO]{\hrulefill\raisebox{-2.1pt}{\quad\thepage\quad}\hrulefill}
\setlength{\headheight}{16pt}

% Margini
\topmargin=-0.45in
\evensidemargin=-0.15in
\oddsidemargin=-0.15in
\textwidth=6.5in
\textheight=9.0in
\headsep=0.25in

%Titolo
\newcommand{\titolo}[1]{
	\thispagestyle{empty}
	\vspace*{\fill}
	\tikzset{pgfornamentstyle/.style={draw = black, fill = teal!50}}
	\unitlength=1cm
	\begin{center}
	\begin{picture}(12,12)%
		\color{white}%
			\put(0,0){\framebox(12,12){%
			\rput[tl](-4,6){\pgfornament[width=8cm]{71}}%
			\rput[bl](-4,-6){\pgfornament[width=8cm,,symmetry=h]{71}}%
			\rput[tl](-6,6){\pgfornament[width=2cm]{63}}%
			\rput[tr](6,6){\pgfornament[width=2cm,,symmetry=v]{63}}%
			\rput[bl](-6,-6){\pgfornament[width=2cm,,symmetry=h]{63}}%
			\rput[br](6,-6){\pgfornament[width=2cm,,symmetry=c]{63}}%
			\rput[bl]{-90}(-6,4){\pgfornament[width=8cm]{46}}%
			\rput[bl]{90}(6,-4){\pgfornament[width=8cm]{46}}%
			\rput(0,0){\Huge \color{black} #1}%
			\rput[t](0,-0.5){\pgfornament[width=5cm]{75}}%
			\rput[b](0,0.5){\pgfornament[width=5cm]{69}}%
			\rput[tr]{-30}(-1,2.5){\pgfornament[width=2cm]{57}}%
			\rput[tl]{30}(1,2.5){\pgfornament[width=2cm,symmetry=v]{57}}}}%
	\end{picture}
	\end{center}
	\vspace*{\fill}
	\pagebreak
}

% Miscellanee
\author{Emanuele Fava}
\date{}
\pagestyle{fancy}


\begin{document}

\section*{Lista Teoremi}

\subsection*{Capitolo 1}

\begin{itemize}
	\item \textbf{Vettore} e \textbf{Versore}
	\item \textbf{Prodotto per scalare, Somma di vettori, Prodotto scalare, Vettori Ortogonali, Vettori Paralleli, Prodotto Vettoriale} e \textbf{Prodotto Misto}
	\item \textbf{Vettori Variabili} e \textbf{Limiti di vettori}
	\item \textbf{(P)} Esiste il limite del vettore se e solo se esistono i limiti delle sue componenti. \textit{Si sfrutta la norma del vettore}
	\item \textbf{Vettore continuo} e \textbf{derivata di un vettore}, con e senza sistema di riferimento
	\item Rappresentazione intrinseca e direttori della tangente
	\item Regole di Derivazione di Vettori e vettore a modulo costante (\textit{Cosa Implica?})
	\item \textbf{Direzione della Normale Principale, Piano Osculatore, Cerchio Osculatore} e \textbf{Raggio di Curvatura}
	\item \textbf{Vettore alla Normale Principale}
	\item \textit{Qual è il significato di $\rho_C$ piccolo? Come posso scrivere la Curvatura? E nel caso di $\gamma = f(x)$ piana?}
	\item \textbf{(!)} Prima applicazione delle derivate di vettori e versori $\vec h, \vec n$
\end{itemize}

\subsection*{Capitolo 2}

\begin{itemize}
	\item \textbf{Sistema Meccanico} e \textbf{Sistema in Moto}
	\item \textbf{Equazione Vettoriale del Moto} e \textbf{Equazione delle Traiettoria}. \textbf{Legge Oraria del Moto} e \textbf{Diagramma Orario}
	\item \textbf{Velocità di un Punto} e i conseguenti \textbf{Moti Diretto} o \textbf{Retrogrado}. \textit{Che succede quando sono nulli e per quanto tempo?}
	\item \textbf{Accelerazione} e le sue \textbf{Componenti}. \textit{Da dove saltano fuori e che succede se si annullano?}
	\item \textbf{Moto Uniforme} e \textbf{Moto Uniformemente Vario}
	\item \textbf{Moto Smorzato} (\textit{Quali tipi?}) e \textbf{Moto Periodico} (\textit{Quale è importante e come si chiamano i vari fattori?})
	\item \textbf{Moto Circolare} e \textbf{Velocità Angolare}. \textit{Come possono essere scritti i vari vettori?}
	\item \textbf{(!) Moto Piano} e \textbf{Coordinate Polari} (\textit{Come si chiamano le strutture delle coordinate polari?})
	\item \textit{Come possiamo rappresentare il punto e che vettori possiamo prendere? Come diventano velocità e accelerazione e come si chiamano le loro componenti?}
	\item \textbf{(T)} Similitudine tra Polari e Intrinseche
	\item \textbf{(!) Velocità Areolare} (\textit{Come possiamo calcolarla?})
	\item \textbf{(P)} $a_\theta = \frac{2}{\rho} \frac{dS'}{dt}$ (\textit{Calcolo di un limite})
	\item \textbf{(C)} \textit{Che succede se $a_\theta=0$?}
	\item \textbf{(T)} Teorema di Binet (\textit{Considerazioni prima})
	\item \textit{Come si passa da un sistema di riferimento ad un altro?}
	\item \textbf{(P)} Valgono $S' = \frac{\dot yx - \dot xy}{2}$ e $\dot \theta = \frac{2S'}{x^2 + y^2}$ (\textit{Passaggio in polari})
\end{itemize}

\subsection*{Capitolo 3}

\begin{itemize}
	\item \textbf{Corpo Rigido}. (\textit{Come possiamo determinare le varie configurazioni del corpo rigido})
	\item \textbf{(P)} Se $\vec a \cdot \vec b = 0$, allora esistono infiniti $\vec c$ tali che $\vec b = \vec c \times \vec a$ (\textit{Costruiscine manualmente uno})
	\item \textbf{(!) (T)} Teorema di Poisson (\textit{Per l'esistenza si crea, per l'unicità per assurdo})
	\item \textit{A che serve il teorema di Poisson?}
	\item \textbf{Formula Fondamentale della Cinematica del Corpo Rigido}
	\item \textit{Da che dipende il vettore di Poisson?}
	\item \textbf{Moto di Traslazione}
	\item \textbf{(P)} Un corpo rigido è traslatorio se e solo se $\vec \omega = 0$ (\textit{dalla formula fondamentale della Cinematica del Corpo Rigido})
	\item \textbf{Moto Rotatorio} (\textit{Come si chiamano le sue componenti?})
	\item \textbf{(P)} CNS tale che il moto di un corpo rigido sia di rotazione è che $O_1:\vec (P) = \vec \omega \times (P-O_1)$ (\textit{$\vec \omega$ è parallelo a $P-O_1$})
	\item \textbf{(P)} Il vettore di Poisson può essere scritto come $\vec \omega = \dot \theta \mathbf k$ (\textit{Analizzare i casi possibili. Come si chiama $\theta$?})
	\item \textit{Come si chiama in questo caso il vettore di Poisson?}
	\item \textbf{Stato Cinetico} e \textbf{Moto di Rototraslazione}
	\item \textit{È ammesso fare la composizione di moti?}
	\item \textbf{(P)} \textit{Come è influente l'Invariante $\mathcal I$?}
\end{itemize}

\subsection*{Capitolo 4}

\begin{itemize}
	\item \textit{Cosa succede quando abbiamo due sistemi di riferimento? E come si chiamano?}
	\item \textbf{Velocità di Trascinamento}
	\item \textbf{(T)} Teorema di Composizione delle Velocità (\textit{Quanto fatto prima})
	\item \textbf{Accelerazione di Trascinamento} e \textbf{di Coriolis}
	\item \textbf{(T)} Teorema di Coriolis (\textit{Quanto fatto prima})
	\item \textit{Come cambia il moto al variare dell'accelerazione di Coriolis?}
	\item \textbf{Formula Generale che lega un cambiamento di Variabili} e \textbf{Sistemi di Riferimento Equivalenti}
	\item \textbf{(P)} $(O)\approx (O_1) \Leftrightarrow \vec a(P) = \vec a_1(P), \forall P \in\mathbb R^3$ (\textit{Sfruttando il teorema di Coriolis e velocità e accelerazioni di trascinamento})
	\item \textbf{Sistema di Riferimento Inerziale} (\textit{Come si passa da un sistema all'altro?})
\end{itemize}

\subsection*{Capitolo 5}

\begin{itemize}
	\item \textbf{Massa del Punto Materiale} e \textbf{di un Corpo}
	\item \textbf{Densità del Corpo in un punto \textit{P}}
	\item Massa Tramite Integrali in varie dimensioni
	\item \textbf{Configurazione, Gradi di Libertà} e \textbf{Parametri Lagrangiani}
	\item \textit{Quanti e quali sono i parametri lagrangiani in un corpo rigido?}
	\item \textbf{Corpo Libero} e \textbf{Vincolato}
	\item \textbf{Spostamento Consentito} e \textbf{Proibito} (\textit{Come si suddividono?})
	\item \textbf{Configurazione Interna} e \textbf{di Confine}
	\item \textbf{Tipologie di Vincoli}
	\item \textit{Come possiamo scrivere gli spostamenti rispetto ai parametri lagrangiani?}
\end{itemize}

\subsection*{Capitolo 6}
\begin{itemize}
	\item \textbf{Forza} sia rispetto alla fisica sia rispetto alla matematica
	\item \textbf{Forza Costante} e \textbf{Forza Nulla}
	\item \textbf{(T)} Postulato delle Reazioni Vincolari
	\item \textbf{Vincolo Liscio}
	\item \textbf{Forze Interne} ed \textbf{Esterne}
	\item \textbf{Linea d'Azione} e \textbf{Principio di Azione e Reazione}
	\item \textbf{Momento di una Forza rispetto ad un Polo}
	\item \textbf{Sistema di Forze}
	\item \textbf{Vettore Risultante delle Forze} e \textbf{Vettore Risultante dei Momenti}
	\item \textbf{(P)} Il momento risultante di un sistema di forze è invariante rispetto al polo se e solo se $\vec R=0$ (\textit{Basta applicare le definizioni})
	\item \textbf{Sistemi di Forze Equivalenti}
	\item \textbf{(P)} Se $\vec R = \vec R'$ e esiste $O$ tale che $\vec \Omega(O)= \vec \Omega'(O)$, allora vale per tutti i Poli (\textit{Basta applicare le definizioni})
	\item \textit{Quali sono i sistemi fondamentali delle forze?}
	\item \textbf{(T)} Che cosa si può dire di un generico sistema di forze?
	\item \textbf{(!) Forza Peso} e \textbf{Baricentro}
	\item \textit{Come si calcolano i centri di massa nelle varie dimensioni? Qual è un caso particolare?}
\end{itemize}

\subsection*{Capitolo 7}

\begin{itemize}
	\item \textbf{Lavoro Fondamentale di una Forza} (\textit{Come varia se lo spostamento è reale o virtuale?})
	\item Ricava le Forze Generalizzate di Lagrange
	\item \textit{Perché si chiamano generalizzate?}
	\item \textbf{Forza Posizionale}
	\item \textit{Come possiamo trovare una condizione affinché $dL$ sia un differenziale esatto?}
	\item \textbf{(P)} Quale è CNS affinché si abbia una forma differenziale esatta?
	\item \textbf{Forza Conservativa} (\textit{Che cosa implica?})
	\item \textbf{(P)} Come varia il lavoro di una forza conservativa in una curva qualsiasi?
	\item \textbf{(P)} Come varia il lavoro di una forza conservativa in una curva chiusa?
	\item \textit{Quali sono esempi di Forze Conservative e qual è il loro potenziale?}
	\item \textbf{Sistema di Forze Conservativo}
	\item \textbf{(P)} Se le forze sono conservative, allora lo è anche il sistema e vale $U(q) = \sum_{s=1}^{N} U_s(q)$ (\textit{Basta applicare le definizioni})
\end{itemize}

\subsection*{Capitolo 8}

\begin{itemize}
	\item \textbf{Quantità di Moto} (\textit{Quando è valida?})
	\item \textit{A cosa è equivalente un sistema in cui agiscono $N$ forze su un punto?}
	\item \textit{Come si suddividono le forze in base alla quantità di moto?}
	\item \textit{Come possono essere viste rispetto alle "concezioni di forze"?}
	\item \textbf{Legge Fondamentale della Dinamica}
	\item \textbf{Soluzione di un'Equazione Differenziale} e \textbf{Equazione Fondamentale del Secondo Ordine in Forma Normale}
	\item \textbf{(T)} Teorema di Cauchy
	\item \textit{Come diventa il nuovo sistema di equazioni differenziali? Come cambia se il punto è libero oppure no?}
	\item \textbf{Configurazione di Equilibrio}
	\item \textbf{(P)} $P_0$ è condizione di equilibrio $\Leftrightarrow$ $\vec R(t,P_0,0)=0, \forall t\geq t_0$ (\textit{Sfruttando definizioni})
	\item \textbf{(P)} \textit{Come varia per le Forze Posizionali}
	\item \textit{Come diventa la configurazione di equilibrio se il punto è libero o meno?}
	\item \textit{Che cosa si può dedurre sfruttando il principio di azione e reazione?}
	\item \textbf{Configurazione di Equilibrio per un Sistema Meccanico}
	\item \textbf{(P)} Quale è la CNS affinché una configurazione $q$ sia di equilibrio?
	\item \textbf{(T)} Prima Equazione Cardinale della Statica
	\item \textbf{(T)} Seconda Equazione Cardinale della Statica
	\item \textbf{(T)} Quale è una condizione necessaria affinché un corpo qualunque sia in equilibrio?
	\item \textbf{(P)} Come cambia per un corpo rigido?
	\item \textbf{(?)} Esempio della Leva
	\item \textit{Quale è la limitazione del sistema delle leggi cardinali della statica?}
	\item \textbf{(P)} Principio dei Lavori Virtuali per le Reazioni Vincolai e per le Forze Attive
	\item \textbf{Vincolo Bilaterale}
	\item \textbf{(P)} Nel caso di vincoli bilaterali, quale è una condizione necessaria e sufficiente affinché una condizione sia di equilibrio per un generico sistema di forze?
	\item \textbf{(P)} Come diventa nel caso di vincoli bilaterali e forze conservative?
	\item \textbf{Configurazione Stabile di Equilibrio} e \textbf{Configurazione Instabile di Equilibrio}
\end{itemize}

\subsection*{Capitolo 9}

\begin{itemize}
	\item \textit{Come possiamo scrivere la legge di Newton come equazione differenziale?}
	\item \textbf{Quantità di Moto, Momento della Quantità di Moto} e \textbf{Energia Cinetica}
	\item \textbf{(!) (T)} Teorema delle Forze Vive (\textit{Smagheggi con i differenziali})
	\item \textbf{(T)} Come possiamo scrivere nel caso di vincoli fissi?
	\item \textbf{Energia Potenziale} e \textbf{Energia Meccanica Totale}
	\item \textbf{(!) (T)} Teorema di Conservazione dell'Energia
	\item \textbf{Integrale Primo del Moto}
	\item \textit{A che cosa ci serve sapere che esiste? Quale è un caso in cui un'equazione differenziale del secondo ordine è equivalente ad una del primo ordine?}
	\item \textit{Partendo dall'integrale primo, come si può arrivare alla formula di Newton?}
	\item Discussione alla Weierstrass con \textbf{Molteplicità di uno Zero}, \textbf{Criterio di Integrabilità per Integrali Generalizzati}, \textbf{Punto di Inversione del Moto} e \textbf{Periodo in un moto Periodico}
	\item \textbf{Valori Ammissibili per $E$}
\end{itemize}

\subsection*{Capitolo 10}

\begin{itemize}
	\item \textit{Che cosa sappiamo dire per la sola forza elastica?}
	\item \textbf{Equazione Differenziale Omogenea del Secondo Grado} e \textbf{Equazione Associata all'Equazione Differenziale}
	\item \textit{Come possiamo trovare una soluzione partendo dall'equazione associata?}
	\item \textit{Alla fine che tipo di moto è quello generato dalla forza elastica?}
	\item \textit{Come diventa il tutto se aggiungiamo la forza non conservativa della resistenza del moto?}
	\item \textit{Come cambia il tutto se aggiungiamo un'altra forza sinusoidale?}
	\item \textbf{Risonanza}
	\item \textit{Cosa possiamo dire per il pendolo matematico? Come varia se prendiamo le piccole o le grandi oscillazioni?}
\end{itemize}

\subsection*{Capitolo 11}

\begin{itemize}
	\item \textit{Come possiamo passare dal Teorema di Coriolis al caso delle forze? E che nome prendono?}
	\item \textbf{Peso}
	\item \textbf{(!)} \textit{Come si può passare dalla forza gravitazionale alla forza peso}
	\item \textit{In che cosa consiste il problema dei due corpi?}
	\item \textbf{Massa Ridotta}
	\item \textbf{Seconda Legge di Keplero} (\textit{Da dove si ricava?})
	\item \textit{Che cosa ricaviamo utilizzando la formula di Binet?}
	\item \textbf{Prima Legge di Keplero} e \textbf{Eccentricità di una Curva}
	\item \textbf{Terza Legge di Keplero} (\textit{Si ricava dalla seconda})
\end{itemize}

\subsection*{Capitolo 12}

\begin{itemize}
	\item \textbf{Quantità del Moto del Corpo}
	\item \textbf{(P)} $\vec Q = M\vec v_G$ (\textit{Segue dal calcolo del baricentro})
	\item \textbf{Momento della Quantità di Moto rispetto ad un polo} e \textbf{Energia Cinetica del Corpo}
	\item \textbf{(!) (T)} Teorema di König (\textit{Segue dal teorema di composizione delle velocità con il sistema baricentrico})
	\item \textit{Che significato ha il teorema di König?}
	\item \textit{Come si può calcolare l'energia cinetica nel caso di un moto traslatorio, di un moto rotatorio e di un moto rototraslatorio?}
	\item \textbf{Momento di Inerzia}
	\item \textit{Come si può calcolare il momento di Inerzia?}
	\item \textbf{(T)} Teorema di Huygens (\textit{Si prende un sistema di riferimento passante per il baricentro - traslato di una dimensione - e si fanno i conti espliciti})
\end{itemize}

\subsection*{Capitolo 13}

\begin{itemize}
	\item \textbf{(T)} Prima Equazione della Dinamica (\textit{Partendo dalla Quantità di Moto di un Sistema})
	\item \textbf{(T)} Equazione del Moto del Baricentro
	\item \textbf{(T)} Equazione del Momento della Quantità di Moto (\textit{Partendo dal Momento della Quantità di Moto})
	\item \textbf{(T)} Seconda Equazione Cardinale della Dinamica
	\item \textbf{Funzione Omogenea}
	\item \textbf{(T)} Teorema di Eulero (\textit{Partendo da $f(\lambda)= \lambda^m f(x)$})
	\item \textbf{(!)} \textit{Come possiamo scrivere l'energia di un sistema meccanico in funzione dei parametri lagrangiani? Come varia nel caso di vincoli fissi?}
	\item \textbf{(P)} $\det \mathscr T \neq 0$ (\textit{Per assurdo})
	\item \textbf{(T)} Teorema delle Forze Vive per sistemi soggetti a vincoli fissi (\textit{Smaghegghi con i differenziali})
	\item \textbf{Energia Potenziale} e \textbf{Energia Meccanica}
	\item \textbf{Forze Perdute}
	\item \textbf{(!) Principio di D'Alembert} \textit{A che cosa serve e come può essere riformulato?}
	\item \textit{Come può essere adattato il Principio dei Lavori Virtuali per le Forze Attive ad un generico caso}
	\item \textbf{(T)} Principio dei Lavori Virtuali per le Reazioni Vincolari
	\item \textit{Come può essere apportato al caso delle Forze Attive?}
	\item \textbf{Equazione Simbolica della Dinamica}
	\item \textit{Sfruttando il fatto che stiamo lavorando con spostamenti virtuali e i parametri lagrangiani cosa otteniamo?}
	\item \textit{Come si ricavano le equazioni di Lagrange?}
	\item \textbf{Equazioni di Lagrange}
	\item \textbf{(!)} \textit{Come possono essere scritte in forma normale?}
	\item \textbf{Equazioni di Lagrange in Forma Normale}
	\item \textbf{(!) (T)} Teorema dell'Energia Cinetica con le Equazioni di Lagrange \textit{Si moltiplica tutto per $\dot q_i$ e si sfrutta Eulero}
	\item \textit{Come si arriva alla Lagrangiana del Sistema?}
	\item \textbf{Funzione Lagrangiana di un Sistema} e \textbf{Funzione Hamiltoniana}
	\item \textbf{(T)} Vale $\frac{d\mathcal H}{dt} + \frac{\partial \mathcal L}{\partial t}=0$ (\textit{Si fa come nel teorema delle Forze Vive})
\end{itemize}

\end{document}
