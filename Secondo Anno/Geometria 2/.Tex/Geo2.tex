\documentclass[11pt,a4paper,twoside]{article}
\usepackage[italian]{babel}
\usepackage{amsthm}
\usepackage{tcolorbox}
\tcbuselibrary{theorems}
\usepackage{amsfonts}
\usepackage{hyperref}
\usepackage{graphicx}
\graphicspath{ {./Immagini/} }
\usepackage{amssymb}
\usepackage{tikz}
\usetikzlibrary{cd,decorations.pathmorphing,patterns}
\usepackage{bm}
\usepackage[T1]{fontenc}
\usepackage{tikzrput}
\usepackage[object=vectorian]{pgfornament}
\usepackage{xcolor}
\usepackage{extarrows}

\usepackage{fourier-orns}
\usepackage{fancyhdr}
\renewcommand{\headrule}{%
\vspace{-8pt}\hrulefill
\raisebox{-2.1pt}{\quad\decofourleft\decotwo\decofourright\quad}\hrulefill}

\usetikzlibrary{decorations.markings}

\newtcbtheorem
	[number within = subsection]% init options
	{defn}% name
	{Definizione}% title
	{%
		colback=teal!5,
		colframe=teal!90!black!95,
		fonttitle=\bfseries,
	}% options
	{def}% prefix


\newtcbtheorem
	[use counter from = defn, number within = subsection]% init options
	{thm}% name
	{Teorema}% title
	{%
		colback=blue!5,
		colframe=blue!90!black!95,
		fonttitle=\bfseries,
	}% options
	{th}% prefix


\newtcbtheorem
	[use counter from = defn, number within = subsection]% init options
	{prop}% name
	{Proposizione}% title
	{%
		colback=red!5,
		colframe=red!90!black!95,
		fonttitle=\bfseries,
	}% options
	{pr}% prefix


\newtcbtheorem
	[use counter from = defn, number within = subsection]% init options
	{lemma}% name
	{Lemma}% title
	{%
		colback=red!5,
		colframe=red!90!black!95,
		fonttitle=\bfseries,
	}% options
	{le}% prefix

\newtcbtheorem
	[use counter from = defn, number within = subsection]% init options
	{cor}% name
	{Corollario}% title
	{%
		colback=red!5,
		colframe=red!90!black!95,
		fonttitle=\bfseries,
	}% options
	{co}% prefix

\newcommand{\vareps}{\varepsilon}

\newtheorem{es}{Esempio}

\theoremstyle{definition}
\newtheorem*{oss}{Osservazione}
\newtheorem*{ese}{Esercizio}

\tcolorboxenvironment{ese}{
	colframe=black}

\newenvironment{sol}
	{\renewcommand\qedsymbol{$\blacksquare$}\begin{proof}[Soluzione]}
	{\end{proof}}

% Margins
\topmargin=-0.45in
\evensidemargin=0in
\oddsidemargin=0in
\textwidth=6.5in
\textheight=9.0in
\headsep=0.25in

\title{Geometria 2}
\author{Emanuele Fava}
\date{}
\pagestyle{headings}

\begin{document}

\thispagestyle{empty}
\topskip 0pt
\vspace*{\fill}

\tikzset{pgfornamentstyle/.style={draw = black, fill = teal!50}}
\unitlength=1cm

\begin{center}
\begin{picture}(10,10)%
	\color{white}%
		\put(0,0){\framebox(10,10){%
		\rput[tl](-3,5){\pgfornament[width=6cm]{71}}%
		\rput[bl](-3,-5){\pgfornament[width=6cm,,symmetry=h]{71}}%
		\rput[tl](-5,5){\pgfornament[width=2cm]{63}}%
		\rput[tr](5,5){\pgfornament[width=2cm,,symmetry=v]{63}}%
		\rput[bl](-5,-5){\pgfornament[width=2cm,,symmetry=h]{63}}%
		\rput[br](5,-5){\pgfornament[width=2cm,,symmetry=c]{63}}%
		\rput[bl]{-90}(-5,3){\pgfornament[width=6cm]{46}}%
		\rput[bl]{90}(5,-3){\pgfornament[width=6cm]{46}}%
		\rput(0,0){\Huge \color{black} Geometria 2}%
		\rput[t](0,-0.5){\pgfornament[width=5cm]{75}}%
		\rput[b](0,0.5){\pgfornament[width=5cm]{69}}%
		\rput[tr]{-30}(-1,2.5){\pgfornament[width=2cm]{57}}%
		\rput[tl]{30}(1,2.5){\pgfornament[width=2cm,symmetry=v]{57}}}}%
\end{picture}
\end{center}

\vspace*{\fill}

\newpage

\pagestyle{fancy}
\fancyhf{}
\fancyhead[LE]{\nouppercase{\rightmark\hfill\leftmark}}
\fancyhead[RO]{\nouppercase{\leftmark\hfill\rightmark}}
\fancyfoot[LE,RO]{\hrulefill\raisebox{-2.1pt}{\quad\thepage\quad}\hrulefill}
\setlength{\headheight}{16pt}

% Optional TOC
\tableofcontents
\newpage
\newpage
%--Paper--

\section{Topologia: Introduzione}
\subsection{Definizioni Base}

\begin{defn}{Spazio Topologico}{}
	Uno spazio topologico è una coppia $(X, \mathcal T)$ dove $X$ è un insieme e $\mathcal T \subseteq \mathcal P(x)$ tale che:
	\begin{enumerate}
		\item $\varnothing, X \in \mathcal T$
		\item $\forall \{U_i\}_{i \in \mathbb N}$ famiglia di elementi di $\mathcal T$,
			\[ \bigcup_{i \in I} U_i \in \mathcal T \]
		\item $\forall \{U_i\}_{i \in \mathbb N}$ famiglia finita di elementi di $\mathcal T$,
			\[ \bigcap_{i \in I} U_i \in \mathcal T \]
	\end{enumerate}
	Se valgono queste $3$, $\mathcal T$ si dice \textbf{Topologia} su $X$ e gli elementi di $\mathcal T$ sono detti aperti.
\end{defn}

\begin{es}
	La coppia $(\mathbb R, \mathcal T_\mathcal E)$ è detta \textbf{Topologia Euclidea}, gli elementi di $\mathcal T_E$ sono unioni arbitrarie di intervalli aperti, infatti:
	\begin{enumerate}
		\item $\varnothing \in \mathcal T_\mathcal E$
		\item $\displaystyle{\bigcup_{n\in \mathbb R^+} ]-n, n[ = \mathbb R}$
	\end{enumerate}
	Le altre proprietà sono banalmente verificate.\\
	Ricordiamo che stiamo palando di \underline{unioni arbitrarie} di intervalli aperti come elementi di $\mathcal T_\mathcal E$
\end{es}

\begin{defn}{Funzione Continua}{}
	Dati due spazi topologici $(X, \mathcal T)$ e $(Y, \mathcal S)$ una funzione:
	\[ f \colon (X, \mathcal T) \rightarrow (Y, \mathcal S) \]
	si dice \textbf{Continua} se \[ \forall V \in \mathcal S, f^{-1}(V) \in \mathcal T\]
	Cioè per ogni insieme aperto appartenente a $\mathcal S$, la sua controimmagine appartiene a $\mathcal T$
\end{defn}

Non è detto che valga il concetto di limite, per questo la definizione in questo modo. D'altronde:

\begin{es}
	Ogni funzione analiticamente continua tra numeri reali è topologicamente continua nella topologia euclidea, cioè:
	\[
		f \colon \mathbb R \rightarrow \mathbb R \; \text{Analiticamente continua} \quad \Rightarrow \quad f:(\mathbb R, \mathcal T_\mathcal E) \rightarrow (\mathbb R, \mathcal T_\mathcal E) \; \text{Topologicamente continua}
	\]
\end{es}

\begin{oss}
	La continuità dipende \textbf{fortemente} dalla Topologia scelta. Ecco perché viene specificata
\end{oss}

Come abbiamo già visto con gruppi e anelli, il lavoro su questi oggetti si basa sulle funzioni. In questo caso, tra Spazi Topologici si lavora con funzioni continue.

\begin{defn}{Spazi Topologici Omeomorfi}{}
	Due spazi Topologici $(X, \mathcal T)$ e $(Y, \mathcal S)$ si dicono \textbf{Omeomorfi} se:
	\[
		\exists f \colon (X, \mathcal T) \rightarrow (Y, \mathcal S) \; \text{continua e con inversa continua}
	\]
	Dunque
	\[
		\exists f^{-1} \colon (Y, \mathcal S) \rightarrow (X, \mathcal T) \;\text{continua}
	\]
	In tal caso si indica con:
	\[ (X, \mathcal T) \cong (Y, \mathcal S) \]
	In particolare, come nei gruppi, i due hanno la stessa cardinalità.
\end{defn}

\begin{es}
	\[ (]-1,1[, \mathcal T_\mathcal E) \cong (\mathbb R, \mathcal T_\mathcal E)\]
	Un esempio di omeomorfismo può essere:
	\[ t \xrightarrow{f} \tan\left(\frac \pi 2 t\right) \]
	Infatti essa è definita su tutto il dominio, è uniformemente continua, è biunivoca e la sua inversa
	\[ \frac 2 \pi \arctan(x) \xleftarrow{f^{-1}} x \; \text{è continua}\]
	Dunque sono omeomorfi
\end{es}

\begin{es}
	Andiamo a vedere direttamente con le figure.
	\begin{center}
		\begin{tikzpicture}
			\draw (-1,0) circle [radius=0.66];
			\draw (0.34,-0.66) rectangle (1.60,0.66);
			\node (uguale) at (0,0){$\cong$};
	\end{tikzpicture}
	\end{center}

	Troviamo un omomorfismo per verificare che sia vero.\\
	Rudimentalmente, tracciamo una semiretta che parte dall'origine.

	\begin{center}
		\begin{tikzpicture}
			\draw (0,0) circle [radius=0.66];
			\draw (-0.6,-0.6) rectangle (0.6,0.6);
		\draw (0,0) -- (2,1);
		\end{tikzpicture}

	\end{center}
	È la forma che lega i punti sulla circonferenza e sul quadrato.\\
	Formalmente è lasciato come esercizio. L'inversa è esattamente l'opposto del procedimento precedente
\end{es}

\newpage

\begin{es}
	\[(\mathbb R, \mathcal T_\mathcal E) \cong \text{ Circonferenza tranne il punto nord}\]
	(Per intenderci, il punto nord è il punto situato nel radiante $\theta = \pi/2$)\\
	Possiamo tracciare il segmento che unisce il punto Nord ai vari punti della circonferenza, per poi prolungarla sulla retta orizzontale passante per il centro della circonferenza. In questo modo abbiamo trovato una funzione (da formalizzare) che lega l'insieme dei numeri reali alla circonferenza escluso tale punto. \\
	Questo tracoletto prende il nome di \textbf{Proiezione Stereografica}
	\begin{center}
		\begin{tikzpicture}
			\draw (0,0) circle [radius=0.66];
			\draw (-3,0) -- (3,0);
			\node (Nord) at (0,1){$N$};
			\draw (0,0.66) -- (1.5,0);
		\end{tikzpicture}
	\end{center}
\end{es}

\begin{es}
	\[
		(]0,1[\cup ]2,3[, \mathcal T_\mathcal E) \not \cong (\mathbb R; \mathcal T_\mathcal E)
	\]
	Il primo non è connesso, il secondo sì. Vedremo più avanti cosa significa, in particolare. Il concetto è lo stesso di analisi matematica 2. In particolare il primo non è connesso per archi:
	\[
		\forall x_1,x_2 \in ]0,1[\cup]2,3[, \not \exists \text{ arco che collega i punti}
	\]
	C'è evidentemente un "buco" che si deve percorrere obbligatoriamente.
\end{es}

Si osservi che esiste sempre un cammino in $(\mathbb R, \mathcal T)$, per esempio infatti:
\[
	f\colon[0,1] \rightarrow \mathbb R \qquad t \xrightarrow f tx + (1-t)x
\]
In particolare un camino continuo è una funzione da $[0,1]$ a $(X, \mathcal T)$ continua

\begin{es}
	$(\mathbb R, \mathcal T_\mathcal E)$ e $(\mathbb R^2, \mathcal T_ \mathcal E)$ non sono omeomorfi.\\
	Infatti se per assurdo esistesse $f$ omeomorfismo tra i due, cioè:
	\[ f\colon (\mathbb R, \mathcal T_\mathcal E) \rightarrow (\mathbb R^2, \mathcal T_\mathcal E)\]
	allora
	\[ f{\big |}_{\mathbb R\setminus\{0\}}: \mathbb R \setminus \{0\} \rightarrow \mathbb R^2\setminus \{0, f(0)\}\] è una funzione che è definita tra un insieme non connesso (il primo) e uno connesso (il secondo)
\end{es}

\begin{es}
	$(\mathbb R^2, \mathcal T_\mathcal E)$ e $(\mathbb R^3, \mathcal T_\mathcal E)$ non sono omeomorfi. La tecnica di prima non funziona, in quanto entrambi rimarrebbero connessi. Se togliamo una circonferenza invece? Non ancora. Se provassimo con una sfera? Non sappiamo quali sono le loro immagini. Ancora non sappiamo lavorarci
\end{es}

Questa sarà una questione che ci porteremo avanti, quando definiremo per bene il concetto di connessione.

Una volta trovata l'intuizione, ecco alcuni esempi di Topologie applicabili sostanzialmente a ciascun insieme.

\subsection{Varie Topologie}

\begin{defn}{Topologia Banale}{}
	Sia $X$ un insieme. Si definisce \textbf{Topologia Grossolana} o \textbf{Banale} su $X$ la Topologia definita come:
	\[ \mathcal T_{GR} = \{\varnothing, X\}\]
\end{defn}

\begin{defn}{Topologia Discreta}{}
	Sia $X$ un insieme. Si definisce \textbf{Topologia Discreta} su $X$ l'insieme:
	\[ \mathcal T_D = \mathcal P(X) = \{\text{L'insieme delle parti di }X\}\]
	Si ovviamente che $\varnothing \in \mathcal T_D$
\end{defn}

\begin{oss}
	Notiamo che c'è l'uguaglianza:
	\[ \mathcal T_D = \mathcal T_{GR} \]
	se e solo se $X$ ha un solo elemento, oppure $X = \varnothing$. Di conseguenza su questi due insiemi esista una sola topologia. La dimostrazione di questo fatto è ovvia
\end{oss}

\begin{defn}{Topologia Cofinita}{}
	Sia $X$ un insieme. Si definisce \textbf{Topologia Cofinita} su $X$ e si indica con $\mathcal T_{Cof}$ se:
	\[ A \in \mathcal T_{Cof} \quad \Leftrightarrow \quad A = X \vee A = \varnothing \vee X\setminus A \text{ è finito} \]
\end{defn}

Analizziamo questa definizione.\\
Abbiamo ovviamente che $X \in \mathcal T_{Cof}$ e $\varnothing \in \mathcal T_{Cof}$
Sia dunque $\{U_i\}_{i \in I}$ una famiglia di elementi di $\mathcal T_{Cof}$ e valutiamo se
\[
	\bigcup_{i \in I}U_i \in \mathcal T_{Cof}
\]
Dunque consideriamo:
\[
	X \setminus \left( \bigcup_{i \in I} U_i\right) = \bigcap_{i \in I}\left(X \setminus U_i\right)
\]
In particolare abbiamo quindi che:
\[ X\setminus U_i = X \qquad \vee \qquad X\setminus U_i \text{ è finito} \]
Dalla seconda otteniamo che anche l'intersezione è finita, in quanto l'intersezione di ogni insieme è contenuto in $X\setminus U_i$. Quindi anche la seconda proprietà dello spazio topologico è verificata.\\
Cosa sappiamo dire della terza? Sia $\{U_i\}_{i \in I}$ una famiglia finita di insiemi (cioè $I$ è finito). Valutiamo quindi l'intersezione di tutti quegli insiemi. Abbiamo che:
\[ X \setminus \left(\bigcap_{i \in I}U_i\right) = \bigcup_{i \in I}(X\setminus U_i) \]
Se $X \setminus U_i =X$ allora è banalmente verificato. Se invece $X\setminus U_i$ è finito, allora l'insieme finito di insiemi finiti è ancora finito, quindi anche la terza proprietà di uno Spazio Topologico sono verificate.

\begin{defn}{Topologia della Semi Continuità}{}
	Sia $X = \mathbb R$. Si definisce \textbf{Topologia della Semi Continuità Superiore} gli aperti del tipo $]-\infty, a[$ con $a \in \mathbb R$, $\varnothing$ o $\mathbb R$.\\
	Nel caso della \textbf{Topologia della Semi Continuità Inferiore} gli aperti sono del tipo $]b, +\infty[$ con $b \in \mathbb R$, $\varnothing$ oppure $\mathbb R$.
\end{defn}

Valutiamo se effettivamente sono topologie (per il caso superiore).\\
Per definizione abbiamo che $\varnothing, \mathbb R \in \mathcal T$, quindi la prima proprietà è verificata.\\
Se un elemento è $\mathbb R$, allora è banale. Se invece non lo è, possiamo considerare $\{U_i\}_{i \in I}$ famiglia di insiemi aperti definiti come \[ U_i := ]-\infty, a_i[\]
Allora otteniamo che:
\[ \bigcup_{i \in I}(U_i) = \bigcup_{i \in I}(]-\infty, a_i[) = ]-\infty, \sup_{i \in I}a_i[  \in \mathcal T\]
Quindi la seconda proprietà è verificata.\\
Se un elemento è l'insieme vuoto, allora è banale. Se invece non è così, possiamo considerare:
\[ U_i = ]-\infty, a_i[ \qquad \text{Con }I \text{ finito}\]
Allora, se andiamo a prendere l'intersezione abbiamo che:
\[ \bigcap_{i \in I}U_i = ]-\infty,  \min_{i \in I}a_i[ \in \mathcal T\]
\textit{Si prende il minimo e non l'inferiore, in quanto si ha l'intersezione \underline{finita} di insiemi}\\
E quindi è verificata anche la terza proprietà.\\
Se invece $\sup a_i = -\infty$, allora tutti gli $a_i = -\infty$


\subsection{Basi}
\begin{defn}{Base}{}
	Una \textbf{base} $\mathcal B$ di una Topologia $\mathcal T$ è un sottoinsieme $\mathcal B \subseteq\mathcal T$ tale che ogni elemento si scriva come \textit{unione} di elementi della base $\mathcal B$
\end{defn}

\begin{es}
	Una base $\mathcal B$ della Topologia discreta è:
	\[ \mathcal B_{\mathcal T_D} = \{\text{Singoletti (Elementi singoli)}\} = \{\{x\}: x \in X\}\]
	Infatti: \[A \subseteq X \qquad A = \bigcup_{x \in A}\{x\}\]
\end{es}

\begin{es}
	$\mathcal T$ è una base della topologia $\mathcal T$
\end{es}

\textbf{Notazione}: Indichiamo il disco unitario di dimensione $n$ con:
\[ D^n := \{x \in \mathbb R^n : \|x\|\leq 1\} \]
Indichiamo invece la sfera unitaria di dimensione $n$ come:
\[ S^n := \{x \in \mathbb R^{n+1} : \|x\| = 1\}\]
\textit{La palla aperta centrata in $x$ e di raggio $r$ viene indicata come in analisi}

\begin{oss}
	$\varnothing$ è scrivibile come unione di nessun elemento di $\mathcal B$, cioè andiamo a considerare la famiglia "vuota" di sottoinsiemi di $\mathcal B$.
\end{oss}

\begin{thm}{Teorema della Base}{}\label{Base}
	Siano $X$ un insieme e $\mathcal B \subseteq \mathcal P(X)$. Allora esiste una topologia $\mathcal T$ su $X$ di cui $\mathcal B$ è una base se e solo se:
	\begin{enumerate}
		\item $X = \bigcup_{B \in \mathcal B}B$
		\item Per ogni $A, B \in \mathcal B$ e per ogni $p \in A \cap B$, esiste $p \in C_p \in \mathcal B$ tale che \[C_p \subseteq A \cap B\]
	\end{enumerate}
\end{thm}

\begin{proof}
	\fbox{$\Rightarrow$} Sia $X \in \mathcal T$ elemento della topologia $\mathcal T$. Allora esiste una famiglia di insiemi $\{B_i\}_{i \in I}\subseteq \mathcal B$ tale che:
	\[ X = \bigcup_{i \in I}B_i \qquad \Rightarrow \qquad X = \bigcup_{B \in \mathcal B}B\]
	Inoltre, se $A,B \in \mathcal B\subseteq \mathcal T$. Allora anche $A \cap B \in \mathcal T$. Esiste quindi una famiglia di insiemi $\{C_i\}_{i\in I}$ tali che $C_i \in \mathcal B$ tali che: \[A \cap B = \bigcup_{i \in I}C_i\]
	Ma allora per ogni $p \in A \cap B$ si ha che:
	\[\exists C_p \in \mathcal B: p \in C_p \subseteq A \cap B\]
	\fbox{$\Leftarrow$} Dato $\mathcal B$, sia $\mathcal T \subseteq \mathcal P(X)$ data da tutte le unioni degli elementi di $\mathcal B$. Abbiamo che $X \in \mathcal T$ (per il primo punto) e che $\varnothing \in \mathcal T$ (come unione vuota). Sia poi $\{U_i\}_{i \in I}$ famiglia di insiemi in $\mathcal T$. Allora:
	\[\forall i, \exists J_i: U_i = \bigcup_{j \in J_i} B_{j,i}, B_{j,i} \in \mathcal B\]
	Ma allora abbiamo che:
	\[\bigcup_{i \in I} U_i = \bigcup_{i \in I} \left( \bigcup_{j \in J_i}B_{j,i} \right)\]
	Dimostriamo ora che $\forall A_1, A_2 \in \mathcal T$ si ha che $A_1 \cap A_2 \in \mathcal T$.\\
	Per quanto appena fatto abbiamo che:
	\[ A_1 = \bigcup_{i \in I}B_{1,i} \qquad A_2 = \bigcup_{j \in J}B_{2,j}\]
	Da cui segue subito che:
	\[A_1 \cap A_2 = \bigcup_{i \in I, j \in J}B_{1,i}\cap B_{2,j}\]
	Per il punto $2$, abbiamo che:
	\[B_{1,i} \cap B_{2,j} = \bigcup_p C_p \text{ con }p \in B_{1,i}\cap B_{2,j}\text{ e } \bigcup_p C_p \in \mathcal T\]
\end{proof}

\begin{es}
	Sia $X = \mathbb R^n$ e sia $d_\mathcal E$ metrica euclidea. Definiamo $B_\varepsilon(x):=\{y \in \mathbb R^n: d_\varepsilon(x,y)<\varepsilon\}$. Definiamo poi $\mathcal B_\mathcal E:=\{B_\varepsilon(x):x \in \mathbb R^n, \varepsilon \in \mathbb R^+\}$. Chiamiamo poi $\mathcal T_\mathcal E$ la \textbf{topologia euclidea} di $\mathbb R^n$.\\
	Mostriamo che effettivamente è una topologia.\\
	Notiamo che:
	\[ \mathbb R^n = \bigcup_{x \in \mathbb R^n}B_1 (x) = \bigcup_{x \in \mathbb R^n, \varepsilon \in \mathbb R^+} B_\varepsilon(x) \]
	Quindi è verificato il primo punto del teorema.\\
	Consideriamo poi:

	\begin{center}
		\begin{tikzpicture}
			\draw (0,0) circle (0.8);
			\filldraw (0,0) circle(1pt) node[align = center, above]{$x$};
			\draw (0.7,-0.7) circle (0.8);
			\filldraw (0.7, -0.7) circle (1pt) node[align = center, below]{$y$};
			\filldraw (0.35, -0.35) circle (1pt) node[align = right, above]{$e$};
			\draw (1.2,0.6) node{$B_\varepsilon(x)$};
			\draw (2.1, -0.7) node{$B_{\varepsilon'}(y)$};
		\end{tikzpicture}
	\end{center}

	\textit{Se l'intersezione fosse banale non dimostreremmo nulla}\\
	Prendiamo un elemento $e$ all'interno dell'intersezione, segue dunque che:
	\[ d(x, e)<\varepsilon \qquad d(y, e)< \varepsilon' \]
	Consideriamo allora $\varepsilon''$ come: \[\varepsilon'' = \min\left\{ {\varepsilon - d(x,e)}, {\varepsilon' - d(y,e)} \right\} = \min\left\{ {\varepsilon - \delta}, {\varepsilon' - \delta'} \right\}\qquad \text{con } \delta=d(x,e)\; \delta'=d(y,e) \]
	Notiamo se $w \in B_{\varepsilon''}(e)$ allora si ha che:
	\[ d(x,w) \leq d(x,e) + d(e,w) = \delta + {\varepsilon -\delta}\quad \Rightarrow \quad w \in B_\varepsilon(x)\]
\end{es}

Quest'esempio può essere esteso a qualsiasi spazio metrico e ottenere la seguente definizione.

\begin{defn}{Topologia Associata alla metrica $d$}{}
	Sia $(X,d)$ uno spazio metrico. Possiamo allora definire $\mathcal B_d = \{B_\varepsilon(x): x \in X, \varepsilon \in \mathbb R^+\}$. Essa è una base della \textbf{Topologia associata alla metrica $d$} $\mathcal T_d$
\end{defn}

\begin{es}
	\textbf{Topologia di Sorgenfrey}. Essa ha come base $\mathcal B= \{[b,a[ : b,a \in \mathbb R\}$, cioè è una base costituita da intervalli semiaperti. Inoltre, sapendo che: \[]a,b[ = \bigcup_{c<a}[c,b[\] Questa topologia si dice più \textbf{fine} della topologia euclidea.
\end{es}

\begin{defn}{Funzione aperta}{}
	Sia $f: (X, \mathcal T) \to (Y, \mathcal S)$. Si dice che $f$ è \textbf{aperta} se \[\forall U \in \mathcal T, f(U) \in \mathcal S \]
\end{defn}

Abbiamo parlato di funzioni continue e aperte. Sia \[f \colon (X, \mathcal T) \to (Y, \mathcal S)\]
Osserviamo che se $\mathcal S$ è grossolana, $f$ è continua. $f^{-1}$ è composto solo da $X$ e $\varnothing$ che sono aperti, quindi $f$ è continua.

Se $\mathcal T$ è discreta allora $f$ è continua. Mentre se $\mathcal S$ è discreta, allora $f$ è aperta. Inoltre se $\mathcal S$ è grossolana, allora $f$ è continua.
Infine, se $\mathcal T$ è grossolana e $f$ è suriettiva, allora $f$ è aperta.

\begin{es}
	Consideriamo l'applicazione:
	\[id\colon (X, \mathcal T_D) \to (X, \mathcal T_{GR})\]
	Cosa possiamo dire su $id$? È continua, ma l'inversa non è continua.
\end{es}

\begin{defn}{Finezza di una Topologia}{}
	Siano $X$ insieme e siano $\mathcal T, \mathcal S$ due Topologie su $X$. Diciamo che $\mathcal T$ è \textbf{più fine} di $\mathcal S$ se:
	\[\forall A \in \mathcal S, A \in \mathcal T\]
	Viceversa, diremo che $\mathcal T$ è \textbf{meno fine} di $\mathcal S$ se:
	\[\forall B \in \mathcal T, B \in \mathcal S\]
	Se le due Topologie non sono confrontabili, allora non vale nessuna delle due condizioni.
\end{defn}

\begin{oss}
	$\mathcal T$ è più fine di $\mathcal S$ se $id \colon (X, \mathcal T) \to (X,\mathcal S)$ è continua.\\
	Se $\mathcal T$ è meno fine di $\mathcal S$, allora $id\colon (X, \mathcal T) \to (X, \mathcal S)$ è aperta.
\end{oss}

\begin{oss}
	Una base \textit{non} può generare due topologie distinte. È possibile sempre, però, da una Topologia, trovarne una più o meno fine.
\end{oss}

\begin{es}
	Le topologie $\mathcal T_{GR}$ e $\mathcal T_D$ sono rispettivamente le topologie meno fine e più fine di tutte.
\end{es}

\begin{es}
	Sia $X = \mathbb R$ e consideriamo $\mathcal T_\mathcal E, \mathcal T_{Cof}, \mathcal T_{SCS}$ (della Semi Continuità Cuperiore) e $\mathcal T_\mathcal S$ (di Sorgenfrey). Confrontandole si ottiene che $\mathcal S$ è più fine di $\mathcal T_\mathcal E$, $\mathcal T_{SCS}$ è meno fine di $\mathcal T_\mathcal E$ e $\mathcal T_{Cof}$ è meno fine di $\mathcal T_\mathcal E$. $\mathcal T_{Cof}$ e $\mathcal T_{SCS}$ non sono confrontabili.
\end{es}

\begin{prop}{}{}
	Siano $(X, \mathcal T)$, $(Y, \mathcal S)$ e $(Z, \mathcal U)$ degli spazi topologici. Allora se:
	\begin{itemize}
		\item $f \colon X \to Y$ è aperta (o continua)
		\item $g \colon Y \to Z$ è aperta (o continua)
	\end{itemize}
	Allora $g \circ f$ è aperta (o continua)
\end{prop}

\begin{proof}
	Sia $A \in \mathcal U$. Allora abbiamo che:
	\[(g \circ f)^{-1}(A) = f^{-1}(g^{-1}(A)) \in\mathcal T\]
	\textit{Con ${}^{-1}$ si intende la preimmagine}. Allora $g\circ f$ è continua. Per gli aperti è analogo.
\end{proof}

\subsection{Topologia con Insiemi Chiusi e Intorni}

\begin{defn}{Spazio Topologico Rispetto ai Chiusi}{}
	Una coppia $(X, \overline{\mathcal T})$ è detta \textbf{Spazio Topologico rispetto ai Chiusi} se $X$ è un insieme e valgono:
	\begin{enumerate}
		\item $\varnothing, X \in \overline {\mathcal T}$
		\item L'intersezione arbitraria di chiusi è ancora chiusa (quindi in $\overline {\mathcal T}$)
		\item Unioni finite di chiusi è ancora chiusa, quindi ancora in $\overline{\mathcal T}$
	\end{enumerate}
	I suoi elementi vengono detti \textbf{Chiusi} e i loro complementari sono detti aperti.
\end{defn}

\begin{defn}{Insiemi Claperti}{}
	Un insieme \textbf{Claperto} è un insieme chiuso e aperto contemporaneamente
\end{defn}

\begin{defn}{Funzione Continua}{}
	Una funzione $f: (X, \mathcal T)\to (Y, \mathcal S)$ è detta \textbf{chiusa} se:
	\[ \forall U \text{ chiuso di }X, f(U) \text{ è chiuso in }Y\]
\end{defn}

\begin{defn}{Chiusura, Interno e Frontiera di un insieme}{}
	Sia $Y\subseteq (X, \mathcal T)$. Definiamo \textbf{chiusura} di $Y$:
	\[\overline Y := \bigcap_{Y \subseteq F}F \quad F \text{ insiemi chiusi}\]
	Definiamo \textbf{interno} di un insieme $Y$:
	\[\text{\textit{\r Y}}:= \bigcup_{A i\subseteq Y} A\quad A \text{ insiemi aperti}\]
	Definiamo \textbf{chiusura} di un insieme $Y$:
	\[\partial Y := \overline Y \setminus \mathring Y\]
\end{defn}

\begin{oss}
	Vale l'inclusione: $\bigcup \overline A \subseteq \overline{\bigcup A}$
\end{oss}

\begin{es}
	Con i numeri reali abbiamo che:
	\[\bigcup_{n>0} \left[\frac 1{n+1}, \frac 1{n}\right] \subseteq \overline{\bigcup_{n>0}\left[ \frac 1{n+1}, \frac 1n \right]} = [0,1] \]
	Oppure abbiamo che:
	\[ \bigcup_{n>0}\left\{ \frac 1n \right\} \subseteq \overline{\bigcup_{n>0} \left\{ \frac 1n \right\}} = \{0\} \cup \bigcup_{n>0}\left\{ \frac 1n \right\} \]
\end{es}

\begin{es}
	In $(X, \mathcal T_{GR})$, se $Y \neq \varnothing$, allora $Y = X$. Da cui segue che $\mathring Y \neq \varnothing$ se $Y\neq X$ oppure $\mathring Y = X$ se $Y = X$. Inoltre $\partial Y = X$ se $Y= \varnothing$
\end{es}

\begin{es}
	In $(X, \mathcal T_D)$ si ha che $\overline Y = Y$, $\mathring Y = Y$ e $\partial Y = \varnothing$
\end{es}

\begin{es}
	$(X, \mathcal T_{Cof})$, se $|Y| = +\infty$, allora $\overline Y = X$
\end{es}

\begin{defn}{Insieme Denso}{}
	Sia $Y \subseteq (X, \mathcal T)$. Si dice che $Y$ è \textbf{denso} se $\overline Y = X$
\end{defn}

\begin{es}
	$\mathbb Z \subseteq (\mathbb R, T_{Cof})$ e $\mathbb Q \subseteq (\mathbb R, \mathcal T_E)$ sono densi
\end{es}

\begin{defn}{Intorno}{}
	Sia $(X, \mathcal T)$ spazio topologico e sia $p \in X$. Si dice che $W$ è un \textbf{intorno} di $p$ se:
	\[\exists U \in \mathcal T: p \in U \subseteq W\]
\end{defn}

\begin{lemma}{}{}
	Sia $W$ intorno di $p$ e sia $V \subseteq X$. Allora $W \cup V$ è un intorno di $p$. Inoltre se $W_i$ sono un insieme di intorni di $p$ (con $i \in \{1,...,n\}$), allora:
	\[ \bigcap_{1,...,n}W_i \text{ è un intorno}\]
\end{lemma}

\begin{proof}
	Per definizione di intorno, esiste $U \in \mathcal T$ tale che $p \in U \subseteq W$. Ma allora è vero che:
	\[ p \in U \subseteq W \cup V \text{ è ancora un intorno di p}\]
	Siano poi $U_i \in \mathcal T$ tali che $p \in U_i \subseteq W_i$. Allora:
	\[ p \in \bigcap_{i = 1}^n U_i \subseteq \bigcap_{i = 1}^n W_i\]
	Sappiamo anche che $\cap U_i$ è ancora un aperto in quanto è intersezione finita di aperti, quindi $\cap W_i$ è effettivamente un intorno.
\end{proof}

\begin{oss}
	Se $p \in V$ è tale che $p \in \mathring V$, allora $V$ è un intorno di $p$. Inoltre, se $U \subseteq \mathcal T$, allora $U$ è intorno di tutti i suoi punti.
\end{oss}

\begin{defn}{Sistema Fondamentale di Intorni}{}
	Sia $x \in (X, \mathcal T)$. Definisco $I(x)$ come l'insieme degli intorni di $x$, cioè:
	\[ I(x) = \{ U \subseteq X: U \text{ è un intorno di }x \} \]
	Sia $\mathcal I \subseteq I(x)$. $\mathcal I$ si definisce \textbf{Sistema Fondamentale di Intorni} se:
	\[\forall J \in I(x), \exists K \in \mathcal I: K\subseteq J\]
\end{defn}

\begin{es}
	In $(\mathbb R, \mathcal T_E)$ si ha che:
	\[ \{]-\varepsilon, \varepsilon[: \varepsilon \in \mathbb R^+\} \qquad \text e \qquad \left\{\left] -\frac 1n, \frac 1n \right[: n \in \mathbb N \right\}\]
	Sono due sistemi fondamentali di intorni per $p = 0$
\end{es}

\begin{es}
	Dati $x \in (X, \mathcal T)$, possiamo definire:
	\[\mathcal I_\mathcal T = \{U \subseteq X: U\text{ è aperto}, U \in I(x)\} = I(x) \cap \mathcal T\]
	Questo è un sistema fondamentale di intorni.
\end{es}

\begin{es}
	Sia $x \in(X, \mathcal T)$ uno spazio topologico e sia $\mathcal B$ una sua base, allora possiamo definire:
	\[\mathcal I_\mathcal B = \{U \subseteq X: U \in \mathcal B, U \text{ è un intorno di x}\} = \mathcal B \cap I(x)\]
	È un sistema fondamentale di intorni, infatti se $p \in U$, per la definizione di base $\mathcal B$, $U$ è scrivibile come unione di elementi di $\mathcal B$, che quindi contiene $p$
\end{es}

\subsection{Funzioni e Topologie Indotte}

\begin{defn}{Funzione Continua in un Punto}{}
	Sia $f \colon (X, \mathcal T) \to (Y, \mathcal S)$ una funzione. Si dice che $f$ è continua in $x \in X$ se:
	\[\forall V \text{ intorno di }f(x), \exists U \text{ intorno di }x: f(U)\subseteq V\]
\end{defn}

\begin{oss}
	È sufficiente verificare la condizione su dei sistemi fondamentali di intorni.
\end{oss}

\begin{oss}
	Se si ha che $(X, \mathcal T) = (Y, \mathcal S) = (\mathbb R, \mathcal T_E)$ e gli intorni sono della forma $]f(x)-\varepsilon, f(x)+\varepsilon[$ e $]x-\delta, x + \delta[$, allora $f$ è continua in $x$ se e solo se:
	\[ \forall \varepsilon > 0, \exists \delta >0: f{\big (}]x-\delta, x+\delta[{\big )} \subseteq ]f(x)-\varepsilon, f(x)+\varepsilon[\]
	Cioè:
	\[\forall y: |x-y|<\delta \Rightarrow |f(x)-f(y)|<\varepsilon\]
	In particolare vale la seguente proposizione
\end{oss}

\begin{prop}{}{}
	Vale la doppia implicazione:
	\[ f\colon(X, \mathcal T) \to (Y, \mathcal S) \text{ è continua }\Leftrightarrow \forall x \in X, f \text{ è continua in }x \]
\end{prop}

\begin{proof}
	Facciamo la dimostrazione sfruttando gli insiemi aperti.\\
	\fbox{$\Rightarrow$} Sia $A$ un insieme aperto con $f(x) \in A \subseteq Y$. Allora si ha che $x \in f^{-1}(A)$. $f$ è continua, quindi $f^{-1}(A)$ è un aperto che contiene $x$, dunque è un intorno di $x$.\\Quindi si ha che $f(f^{-1}(A)) \subseteq A$ (${}^{-1}$ \textit{è usato come controimmagine}).\\
	Quindi $f$ è continua in $x$, e per l'arbitrarietà di $x\in X$, segue che $f$ è continua in $x,\forall x \in X$

	\fbox{$\Leftarrow$} Sia $A$ un aperto di $Y$. Dimostriamo che $f^{-1}(A) \in \mathcal T$. Sia $x \in f^{-1}(A)$. Sappiamo che $f$ è continua in $x$, allora esiste un aperto $U_x$ che contiene $x$ e $f(U_x)\subseteq A$. Allora posso scrivere:
	\[ f^{-1}(A) = \bigcup_{x \in f^{-1}(A)}U_x \]
	in quanto se $f(U_x) \subseteq A \Rightarrow U_x \subseteq f^{-1}(A)$. \\
	$U_x$ è un aperto, dunque la loro unione è ancora aperta, quindi $f$ è continua.
\end{proof}

\begin{defn}{Topologia Indotta}{}
	Sia $(X, \mathcal T)$ spazio topologico e sia $Y \subseteq X$. Si definisce \textbf{topologia indotta} $\mathcal T_Y$ da $\mathcal T$ su $Y$ l'\underline{unica} topologia tale che:
	\[\forall f:(Z, \mathcal S) \to (Y, \mathcal T_Y), f\text{ è continua}\Leftrightarrow i\circ f \text{ è continua}\] con $i:Y \to X$ inclusione, cioè vale:

	\begin{center}
		\begin{tikzpicture}
			\draw (0,0) node(y){$Y$} (2,0) node(x){$X$} (0,-1.5) node(z){$Z$};
			\draw[->] (y) to node[pos=0.5, above]{$i$} (x);
			\draw[->] (z) to node[pos = 0.5, left]{$f$} (y);
			\draw[->] (z) to node[pos=0.5, below, right]{$i \circ f$} (x);
		\end{tikzpicture}
	\end{center}
\end{defn}

\begin{defn}{$\mathcal T_Y$}{}
	\[ \mathcal T_Y = \{Y \cap A: A \in \mathcal T\} \]
\end{defn}

\begin{prop}{}{}
	Le due definizioni sono equivalenti.
\end{prop}

\begin{proof}
	Sia $A \in \mathcal T$, allora per la funzione di inclusione si ha che $i^{-1}(A) = A \cap Y$, quindi necessariamente $i^{-1}(A)$ deve essere un aperto di $\mathcal T_Y$, cioè $A \cap Y$ è un aperto di $\mathcal T_Y$. In particolare $\mathcal T_Y$ è più fine di $\mathcal T'_{Y}$, dove:
	\[\mathcal T'_{Y} = \{A \cap Y: A \in \mathcal T\}\]
	Per la prima definizione si ha che:

	\begin{center}
		\begin{tikzpicture}
			\draw (0,0) node(y){$(Y, \mathcal T_Y)$} (3,0) node(x){$(X, \mathcal T)$} (0,-1.5) node(z){$(Y, \mathcal T'_{Y})$};
			\draw[->] (y) to node[pos=0.5, above]{$i$} (x);
			\draw[->] (z) to node[pos = 0.5, left]{$id$} (y);
			\draw[->] (z) to node[pos=0.5, below, right]{$i$} (x);
		\end{tikzpicture}
	\end{center}
	Dunque coincidono
\end{proof}

\begin{es}
	Sia $(\mathbb R, \mathcal T_E)$ e sia $\mathbb Z \subseteq \mathbb R$, la Topologia indotta è discreta
\end{es}

\begin{es}
	Sia $(\mathbb R, \mathcal T_E)$ e sia $[0, +\infty[ \subseteq \mathbb R$. Allora ha una base:
	\[ \{ ]x,y[:x<y, y \in \mathbb R \cup \{+ \infty\} \} \cup \{ [0, y[: y \in \mathbb R \cup \{+ \infty\}\} \]
\end{es}

\begin{oss}
	Osserviamo che in generale una se $\mathcal B \subseteq \mathcal T$ è una base, allora $\mathcal B_Y = \{B \cap Y: B \in \mathcal B\}$. In $[0, +\infty[$ con la Topologia indotta, un intorno di $0$ è un intorno destro nel senso di Analisi 1
\end{oss}

\begin{lemma}{}{}\label{LComodo}
	Sia $Z \subseteq Y \subseteq (X, \mathcal T)$ con $(X, \mathcal T)$ spazio topologico. Sia $(Y, \mathcal T_Y)$ la topologia indotta da $X$ su $Y$. Allora:
	\begin{enumerate}
		\item Se $Y$ è aperto in $X$, allora $Z$ è aperto in $(Y, \mathcal T_Y)$ se e solo se $Z$ è aperto in $(X, \mathcal T_X)$
		\item Se $Y$ è chiuso in $X$, allora $Z$ è chiuso in $(Y, \mathcal T_Y)$ se e solo se $Z$ è chiuso in $(X, \mathcal T_X)$
	\end{enumerate}
\end{lemma}

\begin{es}
	sia $Z \subseteq [0,1] \subseteq (\mathbb R, \mathcal T_\mathcal E)$. $Z$ è chiuso in $[0,1]$ con topologia indotta se e solo se $Z$ è chiuso in $(\mathcal R, \mathcal T_\mathcal E)$
\end{es}

\begin{proof}
	Facciamo il caso con $Y$ aperto:\\
	\fbox{$\Rightarrow$} Abbiamo che $Z$ è aperto se e solo se $\exists A$ aperto di $X:Z = A \cap Y$. Ma questa è un'intersezione di aperti, quindi è ancora aperto.\\
	\fbox{$\Leftarrow$} Se $Z$ è un aperto di $X$, allora $Z = Y\cap Z$, ma dalla definizione di Topologia indotta, si ha che $Z \in \mathcal T_Y$.\\
	Per i chiusi è la stessa identica cosa, basta sostituire "aperto" con "chiuso"
\end{proof}

\begin{prop}{}{}
	Sia $f \colon (X, \mathcal T_X) \to (Y, \mathcal T_Y)$ e sia $f(X)$ munito della Topologia indotta da $\mathcal T_{f(X)}$ da $\mathcal T_Y$. Se $f$ è iniettiva e aperta/chiusa allora $f$ induce un omeomorfismo tra $X$ e $f(X)$
\end{prop}

\begin{proof}
	Diamo la dimostrazione per il caso $f$ aperta.\\
	Sia $\bar f = f{\big |}_{Im(f)}$ ($Im(f)$ è il codominio). Per ipotesi $(\star)$ abbiamo che $\bar f$ è continua, iniettiva e suriettiva. Vogliamo dimostrare che $f(A)$ è un aperto della topologia indotta.\\
	Per ipotesi abbiamo che $f(A) \in \mathcal T_Y$ e $f(A) = f(A) \cap Im(f)$, quindi (come nella dimostrazione precedente) abbiamo che è un aperto della Topologia indotta ($f(A) \in \mathcal T_{f(X)}$).
	Con i chiusi è la stessa identica cosa.

	$\star$ : \textit{Per dimostrare che $\bar f$ è continua, abbiamo che A $\in \mathcal T_Y$, quindi \[f^{-1}(A) = f^{-1}(A\cap Im(f)) = \bar f^{-1}(A \cap Im(f))\] Quindi $\bar f$ è continua.}
\end{proof}

\begin{prop}{}{}
	Siano $A \subseteq Y \subseteq (X, \mathcal T_X)$ con $\mathcal T_Y$ topologia indotta su $Y$ da $\mathcal T_X$, allora la chiusura di $A$ in $Y$ è uguale alla chiusura di $A$ in $X$ intersecata con $Y$, cioè:
	\[\overline A^Y = \overline A^X \cap Y\]
\end{prop}

\begin{es}
	Consideriamo $(0,1) \subseteq (0,2) \subseteq (\mathbb R, \mathcal T_\mathcal E)$ e $(0,2)$ con la topologia indotta dalla topologia euclidea. Chi è $\overline{(0,1)}^{(0,2)}?$
	\[\overline{(0,1)}^{(0,2)} = (0,1] = [0,1] \cap (0,2)\]
	Come si può dimostrare senza la proposizione? Sfruttando il lemma \ref{LComodo}.\\
	Dalla definizione si aperto, abbiamo che $(0,1)$ è non chiuso in $(0,2)$ se e solo se $(0,2)\setminus (0,1)$ è non aperto. Ma è vero che $(0,2)\setminus (0,1) = [1,2)$ e vogliamo dimostrare che non è aperto in $(0,2)$.\\
	Sfruttando il lemma, siccome $(0,2)$ è un aperto di $\mathbb R$, si ha che $[1,2)$ è aperto in $(0,2)$ se e solo se lo è in $(\mathbb R, \mathcal T_\mathcal E)$.\\
	Se $[1,2)$ fosse aperto, sarebbe intersezione di elementi della base $\mathcal B = \{B_\varepsilon(x)\}_{x \in \mathbb R, \varepsilon \in \mathbb R^+}$, cioè è intersezione di aperti del tipo $(x -\varepsilon, x + \varepsilon)$ ma il solo punto $1$ non è in nessun intervallo aperto di questa forma.
	Quindi $[1,2)$ non è aperto, quindi $(0,1)$ è non chiuso.
\end{es}

Adesso la dimostrazione della proposizione $1.5.8$
\begin{proof}
	Vediamo come questi insiemi possono essere scritti:
	\[\overline A^X = \bigcap_{F \in \mathcal C}F \quad \text{con }\mathcal C = \{F \text{ chiuso}: A\subseteq F \text{ in }X\}\]
	Invece l'altro insieme:
	\[\overline A^Y = \bigcap_{G \in \mathcal C'}G\]
	Con $\mathcal C'$ definito come:
	\begin{align*}
		\mathcal C' &= \{G \text{ chiuso}:A \subseteq F \text{ in }X\} = \{F \cap Y: F \text{ chiuso di }X, A \subseteq F \cap Y\} =\\
		&= \{F \cap Y: F \text{ chiuso di }X, A\subseteq F\}
	\end{align*}
	Sapendo questa cosa, si ottiene che:
	\[ \overline A^Y = \bigcap_{G \in \mathcal C'} = \bigcap_{F \in \mathcal C}(F \cap Y) = \left( \bigcap_{F \in \mathcal C}\right) \cap Y = \overline A^X \cap Y\]
\end{proof}

\subsection{Topologia del Prodotto}

Prima di dare l'effettiva definizione, cerchiamo di capire che cosa si intenda effettivamente con Topologia prodotto. Siano quindi $(X, \mathcal T_X)$ e $(Y, \mathcal T_Y)$ due spazi topologici.\\
Allora si ha che:

\begin{center}
	\begin{tikzpicture}
		\draw (0,1.5) node(p){$(X \times Y, \mathcal T_?)$} (-2,0) node(x){$(X, \mathcal T_X)$} (2,0) node(y){$(Y, \mathcal T_Y)$};
		\draw[->] (p) -- node[pos = 0.5, left]{$p_X\;$} (x);
		\draw[->] (p) -- node[pos = 0.5, right]{$\;p_Y$} (y);
	\end{tikzpicture}
\end{center}
Dove $p_X$ e $p_Y$ sono rispettivamente le proiezioni su $X$ e su $Y$

Se fisso $y \in Y$ e prendo $p_X{\big|}_{X \cap \{y\}} \colon X \cap \{y\} \to X$ ho che tale funzione è continua, biettiva e con inversa continua. In $\mathcal T_?$ dobbiamo mettere tutti gli aperti in modo tali che tali proiezioni siano di questo tipo.\\
Bisogna metter il "numero minimo" di aperti.

\begin{defn}{Topologia Prodotto (1)}{}
	La \textbf{Topologia Prodotto} $\mathcal T_{X \times Y}$ su $X \times Y$ è la topologia meno fine tra quelle che rendono $p_X$ e $p_Y$ continue
\end{defn}

\begin{defn}{Topologia Prodotto (2)}{}
	La \textbf{Topologia Prodotto} su $X \times Y$ è la topologia con \underline{Base} (Chiamata Base Canonica) definita come:
	\[\mathcal B_C = \{U \times V: U \in \mathcal T_X,V \in \mathcal T_Y \}\]
\end{defn}

\begin{es}
	In $\mathbb R^2$ il complementare di una parabola è l'esempio di facile di Topologia Prodotto
\end{es}

In un certo senso vanno dimostrate le definizioni date. Iniziamo a dimostrare che è effettivamente una base:

\begin{proof}
	Per fare questa dimostrazione, mostriamo che vale il teorema della base:\\
	\fbox{1} Notiamo che $X \times Y$ è uguale a $X \times Y$ con $X \in \mathcal T_X$ e $Y \in \mathcal T_Y$, quindi $X \times Y \in \mathcal B_C$, da cui \[\bigcup_{A \in \mathcal B_C}A = X \times Y\]
	\fbox{2} Siano $U \times V, U' \times V' \in \mathcal B_C$, come è fatto $(U\times V) \cap (U' \times V')$?\\
	Sappiamo che la prima coordinata è in $U \cap U'$, mentre la seconda coordinata in $V \cap V'$, da cui segue che:
	\[ (U \times V) \cap (U' \times V') = (U \cap U') \times (V \cap V') \in \mathcal B_C \]
	Quindi $\forall p \in (U \times V)\cap (U' \times V')$ ho che:
	\[p \in (U \cap U') \times (V \times V') = (U \times V) \cap (U' \times V')\]
	Quindi $\mathcal B_C$ è chiusa per intersezioni
\end{proof}

\begin{prop}{}{}
	Le due definizioni coincidono
\end{prop}

\begin{oss}
	In questo modo dimostro anche la proprietà descritta nella prima definizione.
\end{oss}

\begin{proof}
	Mostriamo che la Topologia della seconda definizione è la meno fine che renda $p_X$ e $p_Y$ continue.\\
	Sia $U \in \mathcal T_X$ e $V \in \mathcal T_Y$. Come sono fatte $p^{-1}_X(U)$ e $p^{-1}_Y(V)$?
	\[p^{-1}_X(U) = U \times Y \in \mathcal B_C \qquad p^{-1}_Y(V) = X \times v \in \mathcal B_C\]
	Quindi $p_X$ e $p_Y$ sono continue rispetto alla seconda definizione.\\
	Sia $\mathcal S$ Topologia di $X \times Y$ che renda $p_X$ e $p_Y$ continue. Allora
	\[ \forall U \in \mathcal T_X, \forall V \in \mathcal T_Y, p^{-1}_X(U) \in \mathcal S\; \text e \; p^{-1}_Y(V) \in \mathcal S \Rightarrow p^{-1}_X(U)\cap p^{-1}_Y(V) \in \mathcal S\]
	Non solo, ma vale anche che $p_X^{-1}(U) \cap p_Y^{-1}(V) = U \times V$.\\
	Quindi abbiamo mostrato che $\mathcal B_C \subseteq \mathcal S$\\
	Abbiamo inoltre che:
	\[\forall A \in \mathcal T_{X \times Y}, A = \bigcup_{i \in I, U_i \times V_i \in \mathcal S} U_i \times V_i \quad \Rightarrow \quad A \in \mathcal S\]
	Quindi $\mathcal S$ è più fine di $\mathcal T_{X \times Y}$
\end{proof}

\begin{cor}{Proprietà Universale della Topologia Prodotto}{}
	Siano $(X, \mathcal T_X)$, $(Y, \mathcal T_Y)$ e $(Z, \mathcal T_Z)$ spazi topologici e sia $\mathcal T_{X \times Y}$ Topologia Prodotto su $X \times Y$. Sia poi $f \colon Z \to X \times Y$.\\
	Allora $f$ è continua se e solo se $p_X \circ f$ e $p_Y \circ f$ sono continue.\\
	In tal caso vale:
	\begin{center}
		\begin{tikzpicture}
			\draw (0,1.5) node(p){$(X \times Y, \mathcal T_{X \times Y})$} (-2,0) node(x){$(X, \mathcal T_X)$} (2,0) node(y){$(Y, \mathcal T_Y)$} (0,3) node(z){$(Z, \mathcal T_Z)$};
			\draw[->] (p) -- node[pos = 0.5, left]{$p_X\;$} (x);
			\draw[->] (p) -- node[pos = 0.5, right]{$\;p_Y$} (y);
			\draw[->] (z) -- node[pos = 0.5, left]{$f$} (p);
			\draw[->] (z) to[bend right] node[pos = 0.5, left]{$p_X \circ f$} (x);
			\draw[->] (z) to[bend left] node[pos=0.5, right]{$\; p_Y \circ f$} (y);
		\end{tikzpicture}
	\end{center}
\end{cor}

\begin{proof}
	\fbox{$\Rightarrow$} È banale, in quanto è combinazione di funzioni continue.

	\fbox{$\Leftarrow$} Sfruttando l'esercizio $2.15$ del foglio di esercizi, mi basta dimostrare che:
	\[\forall U \times V \in \mathcal B_C, f^{-1}(U \times V) \in \mathcal T_Z\]
	Andando nel dettaglio abbiamo che:
	\begin{align*}
		f^{-1}(U \times V) &= \{z \in Z: f(z) \in U \times V\} = \{z \in Z: (p_X \circ f)(z) \in U, (p_Y \circ f)(z)\in V\}\\
		&= \{z \in Z: (p_X \circ f)(z) \in U\} \cap\{z \in Z: (p_Y \circ f)(z) \in V\}\\
		&= (p_X \circ f)^{-1}(U) \cap (p_Y \circ f)^{-1}(V) \in \mathcal T_Z
	\end{align*}
	In particolare l'appartenenza a $\mathcal T_Z$ segue dal fatto che entrambi i membri appartengono già a $\mathcal T_Z$ in quanto $p_X$ e $p_Y$ sono continue per ipotesi.
	Quindi $f$ è continua.
\end{proof}

\begin{cor}{}{}
	Dati $X, Y, Z$ come nel corollario precedente, se le funzioni $g \colon Z \to (X, \mathcal T_X)$ e $h \colon Z \to (Y, \mathcal T_Y)$ sono continue, allora:
	\[ gh \colon Z \to (X \times Y, \mathcal T_{X \times Y}) \; \text{è continua}\]
\end{cor}

\begin{proof}
	Basta considerare $g$ come $p_X \circ gh$ e $h$ come $p_Y \circ gh$ e poi applicare il teorema precedente.
\end{proof}

\begin{es}
	Siano $(X, \mathcal T_D)$ e $(Y, \mathcal T_D)$ spazi topologici. Allora:
	\[\mathcal T_{X \times Y}\text{ ha base } \mathcal B_C = \{U \times V: U \subseteq X, V \subseteq Y\}\]
	Ma, data la Topologia discreta, si ha che: $\forall (x,y) \in X \times Y$ ho che:
	\[ \{(x,y)\} = \{x\} \times \{y\} \in \mathcal B_C\]
	Quindi $\mathcal B_C$ è la topologia discreta
\end{es}

\begin{es}
	Siano $(X, \mathcal T_{GR})$ e $(Y, \mathcal T_{GR})$, allora:
	\[\mathcal B_C = \{\varnothing, X\times Y\}\]
	Quindi la Base Canonica della Topologia Prodotto è la Topologia Grossolana
\end{es}

\begin{es}
	Siano $(X, \mathcal T_{Cof})$ e $(Y, \mathcal T_{Cof})$ con $|X| = |Y| = + \infty$ (così da non ricadere nel caso precedente). Come è fatta la Topologia Prodotto?\\
	Sappiamo che la Base Canonica è definita come $\mathcal B_C = \{U \times V: U \text{ aperto in }X, V \text{ aperto in }Y\}$. I chiusi li otteniamo come intersezioni di $F \times G$ con $F$ chiusi di $X$ e $G$ chiusi di $Y$. Possiamo immaginare di rappresentarlo come:

	\begin{center}
		\begin{tikzpicture}
			\draw (1,1) -- (1,3) -- (4,3) -- (4,1) -- (1,1) (0,1) -- (0,3) (1,0) -- (4,0);
			\draw (0, 0.75) node{$X$} (0.75, 0) node{$Y$} (5, 3) node{$X \times Y$};
			\draw[->] (0.75,2) -- node[pos = 0.5, above]{$p_X$} (0.25,2);
			\draw[->] (2.5,0.75) -- node[pos = 0.5, right]{$p_Y$} (2.5, 0.25);
			\filldraw (3, 2) circle (1pt) (3.5, 1.5) circle (1pt) (1.5, 2.5) circle (1pt) (2, 1.75) circle (1pt);
			\draw[green] (2.25,1) -- (2.25,3) (3.25,1) -- (3.25,3);
			\draw[blue] (1,1.25) -- (4, 1.25) (1,2.25) -- (4, 2.25);
			\filldraw[blue] (0, 1.25) circle (1pt) (0, 2.25) circle (1pt);
			\filldraw[green, draw = black] (2.25, 0) circle (1pt) (3.25, 0) circle (1pt);
		\end{tikzpicture}
	\end{center}

	La Topologia Prodotto è \underline{più fine} della Topologia Cofinita e i chiusi sono:
	\begin{itemize}
		\item Punti (chiuso, chiuso)
		\item Preimmagini di $p_X^{-1}(punto)$ o "rette" orizzontali (chiuso, $Y$)
		\item Preimmagini di $p_Y^{-1}(punto)$ o "rette" verticali ($X$, chiuso)
		\item $\varnothing$ e $X \times Y$
	\end{itemize}
\end{es}

\begin{es}\label{es32}
	$(\mathbb R^2, \mathcal T_\mathcal E) = (\mathbb R^2, \mathcal T_{Prod})$ rispetto a $(\mathbb R, \mathcal T_\mathcal E)$ e $(\mathbb R, \mathcal T_\mathcal E)$.\\
	Mostriamo che gli elementi di una base del primo sono aperti per la seconda topologia e viceversa.\\
	Sappiamo che $(\mathbb R^2, \mathcal T_\mathcal E)$ ha come base $\{B_\varepsilon(x, y)\}_{(x,y) \in \mathbb R^2, \varepsilon \in \mathbb R^+}$, mentre $(\mathbb R^2, \mathcal T_{Prod})$ ha come base:
	\[ \{B_\varepsilon(X) \times B_{\varepsilon'}(Y)\}_{x,y\in \mathbb R, \varepsilon' \in \mathbb R} = (x- \varepsilon, x+ \varepsilon) \times (y-\varepsilon', y+ \varepsilon')\]
\end{es}

\begin{prop}{}{}
	Siano $(X, \mathcal T_X)$ e $(Y, \mathcal T_Y)$ due spazi topologici e sia $(X \times Y, \mathcal T)$ lo spazio topologico prodotto. Siano poi $p_X \colon X \times Y \to X$ e $p_Y \colon X \times Y \to Y$, allora:
	\begin{enumerate}
		\item $p_X$ e $p_Y$ sono aperte
		\item $\forall x \in X, \forall y \in Y, p_X \colon X \times \{y\} \to X$ e $p_Y \colon \{x\}\times Y \to Y$ sono omeomorfismi, dove $X \times \{y\}$ e $\{x\} \times Y$ hanno la topologia indotta.
	\end{enumerate}
\end{prop}

\begin{proof}
	\fbox{1} Uso l'esercizio $2.15$ e lo utilizzo sulla base canonica $\mathcal B_C = \{ U \times V: U \in \mathcal T_X, V \in \mathcal T_Y\}$. Si ha quindi che:
	\[ p_X(U \times V) = U \in \mathcal T_X\quad \text e \quad p_Y(U \times V) = V \in \mathcal T_Y \]
	Da cui segue che $p_X$ e $p_Y$ sono aperte.

	\fbox{2} \textit{Lo facciamo per la funzione $p_X$, in quanto per la funzione $p_Y$ è la stessa cosa.}\\
	Consideriamo:
	\[ p_X{\big |}_{X \times \{y\}} \colon X \times \{y\} \to X \]
	Abbiamo che $p_X$ è banalmente suriettiva, in quanto per ogni elemento $x \in X$ esiste $(x, y) \in X \times \{y\}$ tale che $p_X(x,y) = x$, da cui segue direttamente anche l'iniettività.
	Per studiare la continuità e l'apertura della funzione, andiamo a studiare questo diagramma commutativo:

	\begin{center}
		\begin{tikzpicture}
			\draw (0,0) node(a){$X \times \{y\}$} (0, 1.5)node(b){$X \times Y$} (2, 1.5)node(c){$X$};
			\draw[->] (a) --node[pos = 0.5, left]{$i$} (b);
			\draw[->] (b) --node[pos = 0.5, above]{$p_X$} (c);
			\draw[->] (a) --node[pos = 0.4, right]{$p_X \circ i = p_X{\big |}_{X \times \{y\}}$} (c);
		\end{tikzpicture}
	\end{center}

	Allora segue che:
	\[ (p_X \circ i)^{-1}(U) = p_X^{-1}(U) \cap (X \times \{y\})\]
	Essendo $p_X^{-1}(U)$ un aperto in $X \cap Y$, quest'insieme è aperto nella topologia indotta, quindi per l'arbitrarietà dell'aperto, la funzione è continua.\\
	Mostriamo che è aperta su una base $\mathcal B = \{ (U \times V) \cap (X \times \{y\}): U \times V \in \mathcal B_C\}$. Quest'insieme è esattamente uguale a $\{U \times \{y\} : U \in \mathcal T_X\}$. Allora segue che:
	\[p_X{\big |}_{X \times \{y\}} (U \times \{y\}) = U \in \mathcal T_X\]
	Quindi è aperta. Da tutto questo segue che è un omeomorfismo.
\end{proof}

Dimostriamo ora l'esempio \ref{es32}, però generalizziamolo al caso $\mathbb R^n = \mathbb R^p \times \mathbb R^{n-p}$
\begin{proof}
	Dimostriamo che un elemento di una base dell'una è aperto nell'altra e viceversa.\\
	\textit{Per pura semplicità di notazione poniamo con $\xi_x$ le prime p componenti del vettore $x$ in $\mathbb R^n$, cioè $\xi_x \in \mathbb R^p$, e sia $y_x$ le sue ultime $n-p$ componenti. Cioè siano $\xi_x \in \mathbb R^p$ e $y_x \in \mathbb R^{n-p}$ come le abbiamo viste in analisi.}\\
	Siano $\mathcal B_1 = \{B_{\varepsilon_1}(x) : \varepsilon_1 \in \mathbb R^+, x \in \mathbb R^n\}$ base per $(\mathbb R^n , \mathcal T_\mathcal E)$ e sia $\mathcal B_2 = \{B_{\varepsilon_2}(\xi) \times B_{\varepsilon_3}(y) : \xi \in \mathbb R^p, y \in \mathbb R^{n-p}, \varepsilon_2, \varepsilon_3 \in \mathbb R^p\}$.
	Quello che vogliamo mostrare è che:

	\begin{center}
		\begin{tikzpicture}
			\draw (0,0) circle (1.5cm) (0, 0) -- (1, 0) -- (1,1) -- (0,1) -- (0,0);
			\draw (2.5, 1.5) -- (2.5, -1.5) -- (5.5, -1.5) -- (5.5,1.5) -- (2.5,1.5) (4.5, 0.5) circle (0.5cm);
			\filldraw (0.5, 0.5) circle (1pt) node[right]{$z$} (4.5, 0.5) circle (1pt) node[right]{$z$};
		\end{tikzpicture}
	\end{center}

	Sia $z \in B_\varepsilon(x)$, allora si ha che:
	\[d(z, x) = \delta - \varepsilon \quad \Rightarrow \quad B_{\frac{\varepsilon - \delta}{2}}(\xi_z) \times B_{\frac{\varepsilon - \delta}2}(y_z) \subseteq B_\varepsilon(x)\]
	Visto che ogni punto $w  B_{\frac{\varepsilon - \delta}{2}}(\xi_z) \times B_{\frac{\varepsilon - \delta}2}(y_z)$ ha distanza da $z$ pari a $d(z, w)<\varepsilon-\delta$, segue che:
	\[ d(w, x) < d(w, z) + d(z, x) < \varepsilon \]

	Sia poi $z \in B_{\varepsilon_1}(\xi_x) \times B_{\varepsilon_2}(y_x)$, allora si ha che:
	\[d(\xi_z, \xi_x)=\delta_1 < \varepsilon_1 \qquad \text e \qquad d(y_x, y_z)= \delta_2 < \varepsilon_2\]
	Poniamo allora $\eta = \sqrt{(\varepsilon_1-\delta_1)^2 + (\varepsilon_2 - \delta)^2}$.
	Allora segue che:
	\[B_{\frac \eta 2}(z) \subseteq B_{\varepsilon_1}(\xi_x) \times B_{\varepsilon_2}(y_x)\]
\end{proof}

La cosa appena dimostrata però non è valido con i chiusi.

\begin{es}
	Sia $p_1 \colon (\mathbb R^2, \mathcal T_\mathcal E) \to (\mathbb R, \mathcal T_\mathcal E)$ definita come $p_1(xy) = x$ non è chiusa. Infatti, posto $F = \{(x, y) \in \mathbb R^2: xy = 1\}$ si ha che:
	\[ p_1(F) = \mathbb R\setminus \{0\} \text{ che non è chiuso}\]
\end{es}

\newpage

\section{Teoria delle Categorie}
\subsection{Prime definizioni}

\begin{defn}{Categoria e Morfismo}{}
	Una \textbf{Categoria} $\mathcal C$ è il dato di:
	\begin{enumerate}
		\item Una collezione di oggetti $\bm{Ob}(\mathcal C)$
		\item $\forall A, B \in \bm{Ob}(\mathcal C)$, è dato un insieme $\bm{Mor}_\mathcal C(A, B)$ chiamato insieme dei \textbf{Morfismi} in $\mathcal C$ tra $A$ e $B$
		\item $\forall A, B, C \in \bm{Ob}(\mathcal C)$ è data una composizione, cioè:
			\begin{align*}
				\bm{Mor}_\mathcal C(A, B) \times \bm{Mor}_\mathcal C(B, C) & \to \bm{Mor}_\mathcal C(A, C)\\
				(f, g) & \mapsto g \circ f
			\end{align*}
	\end{enumerate}
	Tali che:
	\begin{itemize}
		\item[(a)] se $A \neq C$ oppure $B \neq D$, allora $\bm{Mor}_\mathcal C(A, B) \cap \bm{Mor}_\mathcal C(C, D) = \varnothing$
		\item[(b)] $\forall A \in \bm{Ob}(\mathcal C), \exists 1_A \in \bm{Mor}_\mathcal C(A, A) : \forall B \in \bm{Ob}(\mathcal C), \forall f \in \bm{Mor}_\mathcal C(A, B)$ e $\forall g \in \bm{Mor}_\mathcal C(B, A), f \circ 1_A = f \text e 1_A \circ g = g$
		\item[(c)] La composizione è associativa.
	\end{itemize}
\end{defn}

\begin{defn}{$\bm{Set}$}{}
	$\bm{Set}$ è la categoria degli insiemi con $\bm{Ob}(\bm{Set})$ sono gli insiemi e $\forall A, B \in \bm{Ob}(\bm{Set})$ si ha che $\bm{Mor}_{\bm{Set}}(A, B) = \{f \colon A \to B \text{ funzioni}\}$
\end{defn}

Si è detto che una categoria è una "collezione" di oggetti e non un "insieme" di oggetti, proprio per evitare di parlare di insiemi di insiemi. Così facendo si evita di cadere nel paradosso di Russel.

\begin{defn}{$\bm{Grp}$, $\bm{Ring}$, $\bm{Vec}$}{}
	$\bm{Grp}$ è la categoria dei Gruppi, $\bm{Ring}$ è la categoria degli anelli e $\bm{Vect}_{\mathbb K}$ è la categoria dei $\mathbb K$-spazi vettoriali
\end{defn}

Vediamo qualche esempio sulle categorie.

\begin{es}
	La categoria vuota è una categoria
\end{es}

\begin{es}
	Esistono categorie con un solo oggetto, cioè $\bm{Ob}(A)$ con $\bm{Mor}(A, A)$ dove al suo interno ho un'operazione, la composizione che è associativa e ha un elemento neutro.\\
	In particolare, per ogni gruppo $G$ si costruisce una categoria con un solo oggetto $A$ e $\bm{Mor}(A, A)=G$, dove $\forall g,h \in G$ definiamo $g \circ h = h \cdot g$
\end{es}

\begin{es}
	Sia $B$ un insieme parzialmente ordinato. Voglio definire una categoria su tale insieme. Definisco $\mathcal B$ una categoria associata a $B$ tale che $\bm{Ob}(\mathcal B) = B$ e $\forall a, b \in B$ si abbia:
	\[ \bm{Mor}_\mathcal B(a,b) = \begin{cases} \varnothing & \text{se }a \not \leq b\\ i_{a \to b} & \text{se }a \leq b \end{cases} \]

	Andiamo a definire la composizione:\\
	Siano $a,b \in B$, se si ha $a \leq b$ e $b \leq c$ allora "definisco" (virgolettato perché non ho altre possibilità):
	\begin{align*}
		\bm{Mor}(a,b) \times \bm{Mor}(b,c) & \to \bm{Mor}(a,c)\\
		(i_{a \to b}, i_{b \to c}) & \mapsto (i_{a \to c})
	\end{align*}
	Si vede poi facilmente che $1_a = i_{a \to a}$ e si vede anche che la composizione è associativa.
\end{es}

Volendo possiamo rappresentare una categoria con un grafo:

\begin{es}
	Sia $B = \{a,b,c\}$ tali che $a\leq b$ e $c \leq b$. Allora otteniamo che:
	\begin{center}
		\begin{tikzcd}[column sep = tiny]
			& B \arrow[loop left, "i_{b \to b}"]\\
			A \arrow[loop left, "i_{a \to a}"] \arrow[ur, "i_{a \to b}"] && C \arrow[loop right, "i_{c \to c}"] \arrow[ul, "i_{c \to b}"']
		\end{tikzcd}
	\end{center}
\end{es}

Volendo possiamo anche fare il contrario, cioè partire da un grado e associare una categoria:
\begin{es}
	Sia per esempio il grafo definito come:
	\begin{center}
		\begin{tikzcd}
			A \arrow[r, "f"] & B \arrow[r, "g"] & C
		\end{tikzcd}
	\end{center}
	Allora la categoria ad essa associata è:
	\begin{center}
		\begin{tikzcd}
			A \arrow[r, "f"] \arrow[loop below, "1_A"] \arrow[rr, bend left = 50, "f \circ g"] &
			B \arrow[r, "g"] \arrow[loop below, "1_B"] &
			C \arrow[loop below, "1C"]
		\end{tikzcd}
	\end{center}
\end{es}

\begin{es}
	Dato $A$ insieme, considero il suo insieme delle parti $\mathcal P(A)$ con l'operazione di inclusione $\subseteq$. Chiamiamo $\bm{Set}_A$ la categoria associata a $(\mathcal P(A), \subseteq)$. In essa ho morfismi tra $B$ e $C$ se e solo se $B \subseteq C$. Questo prende il nome di \textbf{Morfismo di Inclusione} $i_{B \to C}:B \to C$
\end{es}

\begin{defn}{$\bm{Top}$}{}
	$\mathbf{Top}$ è la categoria i cui oggetti sono gli spazi topologici e i morfismi sono le funzioni continue.
\end{defn}

\begin{defn}{Categoria associata ad una topologia}{}
	Sia $(X, \mathcal T)$ uno spazio topologico, definisco $\bm{Top}_X$ la \textbf{Categoria associata a} $(X, \mathcal T)$
\end{defn}

\begin{es}
	La categoria associata a $(X, \mathcal T)$ è:
	\begin{center}
		\begin{tikzcd}
			\varnothing \arrow[loop left, "i_{\varnothing \to \varnothing}"] \arrow[r, "i_{\varnothing \to X}"] & X \arrow[loop right, "i_{X \to X}"]
		\end{tikzcd}
	\end{center}
\end{es}

\begin{defn}{$\bm{Set}_{pt}$}{}
	Si definisce $\bm{Set}_{pt}$ la \textbf{Categoria degli insiemi puntati}. Ha come oggetti le coppie $X, x_0$ con $x_0 \in X$ e i suoi morfismi sono del tipo:
	\[\bm{Mor}((X, x_0), (Y, y_0)) = \left.\begin{cases} f: X \to Y\\ f(x_0) = y_0 \end{cases}\right\}\]
\end{defn}

\begin{oss}
	Notiamo che in questa categoria non c'è $\varnothing$, perché un oggetto deve avere almeno un punto
\end{oss}

\begin{defn}{Categoria Concreta}{}
	Una categoria è detta \textbf{Concreta} se i suoi oggetti sono insiemi (con proprietà aggiuntive) e i suoi morfismi sono funzioni (particolari)
\end{defn}

\begin{defn}{Isomorfismo}{}
	Data $\mathcal C$ categoria e $A, B \in \bm{Ob}(\mathcal C)$, $f \in \bm{Mor}_\mathcal C(A, B)$ è detto \textbf{Isomorfismo} se $\exists g \in \bm{Mor}_\mathcal C(B, A)$ tale che:
	\[ f \circ g \text { e } g \circ f \text{ siano le identità nel posto giusto} \]
\end{defn}

\begin{defn}{Diagramma}{}
	Un \textbf{Diagramma} all'interno di una categoria $\mathcal C$ è il dato di un grafo orientato dove ogni punto è etichettato usando gli oggetti di $\mathcal C$ e ogni freccia è etichettata con un morfismo tra i due oggetti.
	Un diagramma si dice \textbf{Commutativo} se il risultato di una composizione di morfismi sul diagramma dipende solo dai vertici di partenza e di arrivo.
\end{defn}

Vediamo degli esempi in cui dei grafi sono commutativi.

\begin{es}
	Studiamo questo grafo:
	\begin{center}
		\begin{tikzcd}
			A \arrow[r, "f"] & B \arrow[loop right, "g"]
		\end{tikzcd}
	\end{center}
	Questo diagramma è commutativo se e solo se $\forall k \in \mathbb N, g^k \circ f = f$\\
	Vediamo quest'altro grafo:
	\begin{center}
		\begin{tikzcd}
			A \arrow[r, bend left=30, "f"] \arrow[r, bend right = 30, "g"] & B
		\end{tikzcd}
	\end{center}
	In questo caso il grafo è commutativo se e solo se $f = g$
\end{es}

Vediamo un caso un pelo più complesso.

\begin{es}
	Vediamo questo grafo:
	\begin{center}
		\begin{tikzcd}
			& Z \arrow[ddl, bend right = 30, "f_A"'] \arrow [d, "f_P"] \arrow[ddr, bend left = 30, "f_B"]\\
			& P \arrow[dl, "p_A"] \arrow[dr, "p_B"']\\
			A && B
		\end{tikzcd}
	\end{center}
	Affinché questo grafo sia commutativo, devono essere verificate contemporaneamente:
	\[ f_B = p_B \circ f_P \qquad \text e \qquad f_A = p_A \circ f_P \]
\end{es}

\subsection{Monomorfismo e Epimorfismo}

Abbiamo visto con l'analisi e con l'algebra lineare che le funzioni possono essere iniettive e suriettive, vediamo come possiamo rappresentare questi concetti con la teoria delle categorie.

Vediamo prima con le funzioni iniettiva. Supponiamo di avere $f$ iniettiva e di avere il seguente grafo.
\begin{center}
	\begin{tikzcd}
		Z \arrow[r, bend left = 30, "g_1"] \arrow[r, bend right = 30, "g_2"] & X \arrow[r, "f"] &Y
	\end{tikzcd}
\end{center}
Essendo $f$ iniettiva possiamo dire che se $f \circ g_1 = f \circ g_2$, allora $g_1 = g_2$.

Se invece abbiamo $h$ suriettiva e abbiamo il seguente grafo abbiamo che:
\begin{center}
	\begin{tikzcd}
		Y \arrow[r, "h"] &Z \arrow[r, bend right = 30, "g_2"] \arrow[r, bend left = 30, "g_1"] &X
	\end{tikzcd}
\end{center}
Allora possiamo dire che se $g_1 \circ h = g_2 \circ h$, allora si ha che $g_1 = g_2$.

Diamo adesso le definizioni corrispondenti al concetto di iniettività e di suriettività.

\begin{defn}{Monomorfismo}{}
	Siano $X, Y \in \bm{Ob}(\mathcal C)$ e sia $f \in \bm{Mor}_\mathcal C(X, Y)$. $f$ è detto \textbf{Monomorfismo} se:
	\[\forall Z \in \bm{Ob}(\mathcal C), \forall g_1, g_2 \in \bm{Mor}_\mathcal C(Y, Z), f\circ g_1 = f \circ g_2 \Rightarrow g_1 = g_2\]
	Cioè vale la cancellazione a sinistra
\end{defn}

\begin{defn}{Epimorfismo}{}
	Siano $Y, Z \in \bm{Ob}(\mathcal C)$ e sia $h \in \bm{Mor}_\mathcal C (Y,Z)$. $h$ è detto \textbf{Epimorfismo} se:
	\[\forall X \in \bm{Ob}(\mathcal C), \forall g_1, g_2 \in \bm{Mor}_\mathcal C(Z, X), g_1 \circ h = g_2 \circ h \Rightarrow g_2 = g_2 \]
	Cioè vale la cancellazione a destra
\end{defn}

\begin{es}
	Un monomorfismo in $\bm{Set}$ è una funzione iniettiva e un epimorfismo in $\bm{Set}$ è una funzione suriettiva.
\end{es}

\textbf{Attenzione:} Questa è una generalizzazione che non sempre è vera per le Categorie Concrete.

\begin{es}
	Sia $\bm{Ring}$ la categoria degli anelli e sia $i \colon \mathbb Z \to \mathbb Q$ la funzione di inclusione.
	Questo è un epimorfismo.\\
	Infatti sia $R$ un anello e siano $g_1,g_2 \colon \mathbb Q \to R$ omomorfismo di anelli, allora:
	\begin{center}
		\begin{tikzcd}
			\mathbb Z \arrow[r, "i"] & \mathbb Q \arrow[r, bend left = 30, "g_1"] \arrow[r, bend right = 30, "g_2"] & R
		\end{tikzcd}
	\end{center}
	Abbiamo che $g_1 \circ i = g_2 \circ i$.\\
	Infatti, $\forall n \in \mathbb Z, g_1(n) = g_2(n)$
	Inoltre se $\frac mn \in \mathbb Q$, allora si ha che:
	\[ g_1 \left( \frac mn \right) = g_1(m)\cdot g_1(n)^{-1} = g_2(m) \cdot g_2(n)^{-1} = g_2 \left( \frac mn \right) \]
	Da cui segue che $g_1 = g_2$.\\
	Qui abbiamo utilizzato solo il fatto che $\mathbb Q = Q(\mathbb Z)$ e che $g$ è un omomorfismo di anelli.
\end{es}

\begin{defn}{Categoria Opposta}{}
	Data $\mathcal C$ categoria, si definisce $\mathcal C$ \textbf{Categoria Opposta} $\mathcal C^{op}$ come la categoria tale che:
	\begin{enumerate}
		\item $\bm{Ob}(\mathcal C) = \bm{Ob}(\mathcal C^{op})$
		\item $\forall A, B \in \bm{Ob}(\mathcal C)$ si ha che $\bm{Mor}_{\mathcal C^{op}}(A, B) = \bm{Ob}_\mathcal C(A, B)$
		\item $\forall A, B, C \in \bm{Ob}(\mathcal C)$ definisco la composizione come:
			\[ \bm{Mor}_{\mathcal C^{op}}(A,B) \times \bm{Mor}_{\mathcal C^{op}}(B, C) \to \bm{Mor}_{\mathcal C^{op}}(A, C) \]
			Come \[ g^{op} \circ f^{op} = (f \circ g)^{op}\]
	\end{enumerate}
\end{defn}

Utilizzando i grafi abbiamo che, se vale il seguente grafo in $\mathcal C$:
\begin{center}
	\begin{tikzcd}
		C \arrow[r, "g"] \arrow [rr, bend right = 30, "f \circ g"'] & B \arrow[r, "g"] & A
	\end{tikzcd}
\end{center}
Il grafo nella categoria opposta $\mathcal C^{op}$ corrispondente sarà:
\begin{center}
	\begin{tikzcd}
		C & B \arrow[l, "g^{op}"'] & A \arrow[l, "f^{op}"'] \arrow[ll, bend left = 30, "(f \circ g)^{op} = g^{op} \circ f^{op}"]
	\end{tikzcd}
\end{center}

Il dualismo negli spazi vettoriali è un esempio di categoria opposta e la funzione $\imath: V \to V^*$ prende il nome di funtore che lega le due categorie.

\begin{oss}
	Se $f \in \bm{Mor}_{\mathcal C}(A,B)$ è un monomorfismo, allora $f^{op} \in \bm{Mor}_{\mathcal C^{op}}(B, A)$ è un epimorfismo
\end{oss}

\subsection{Prodotto e Coprodotto}

\begin{defn}{Prodotto di Categorie}{}
	Un \textbf{Prodotto} $(P, p_A, p_B)$ tra due oggetti $A, B$ di una categoria $\mathcal C$ è il dato di $P \in \bm{Ob}(\mathcal C)$ con $p_A = \bm{Mor}_{\mathcal C}(P, A)$ e $p_B = \bm{Mor}_\mathcal C(P, B)$ tale che:
	\[\forall Z \in \bm{Ob}(\mathcal C), \forall f_A \in \bm{Mor}_\mathcal C(P, A), \forall f_B \in \bm{Mor}_\mathcal C(P,B), \exists ! f_P \colon Z \to P, f_P \in \bm{Mor}_\mathcal C(A, P):\]
	\begin{center}
		\begin{tikzcd}
			& Z \arrow[ddl, bend right = 30, "f_A"'] \arrow [d, "f_P"] \arrow[ddr, bend left = 30, "f_B"]\\
			& P \arrow[dl, "p_A"] \arrow[dr, "p_B"']\\
			A && B
		\end{tikzcd}
	\end{center}
	Sia commutativo, cioè $f_A = p_A \circ f_P$ e $f_B = p_B \circ f_P$
\end{defn}

\begin{es}
	In $\bm{Set}$ abbiamo che:
	\begin{center}
		\begin{tikzcd}
			& A \times B \arrow[dl, "p_A"'] \arrow[dr, "p_B"] && B \times A \arrow[dl, "p'_B"'] \arrow[dr, "p'_A"]\\ A&&B&&A
		\end{tikzcd}
	\end{center}
	Definite in modo che $p_A(a,b) = a$, $p_B(a,b) = b$ e $p'_B(b,a) = b$, $p_A'(b,a) = a$\\
	Inoltre $\forall Z \in \bm{Set}$, $\forall f_A:Z \to A$ e $f_B:Z \to B$ definiamo:
	\[ f_P : Z\to A \times B,\; z \mapsto (f_A(z), f_B(z)) \qquad \text e \qquad f_P: Z \to B\times A, \; z \mapsto (f_B(z), f_A(z)) \]
\end{es}

\begin{defn}{Coprodotto}{}
	Allo stesso modo definisco il \textbf{Coprodotto} $(Q, i_A, i_B)$ di due oggetti $A, B \in\bm{Ob}(\mathcal C)$ tale che:
	\[\forall Z \in \bm{Ob}(\mathcal C), \forall f_A: A \to Z, \forall f_B : B \to Z, \exists ! i_Q:\]
	\begin{center}
		\begin{tikzcd}
			& Z\\
			& Q \arrow[u, "i_Q"]\\
			A \arrow[uur, bend left= 30, "f_A"] \arrow[ur, "i_A"'] && B \arrow[uul, bend right = 30, "f_B"'] \arrow[ul, "i_B"]
		\end{tikzcd}
	\end{center}
\end{defn}

\begin{es}
	Nella categoria $\bm{Set}$, il coprodotto è l'unione disgiunta $A \amalg B$, con:
	\[ A \amalg B = \{(c, C):C \in \{A, B\}, c \in C\} \]
	In particolare, se esiste $a \in A \cap B$, si ha che ci sono $(a, A)$ e $(b, A)$\\
	Allora abbiamo:
	\[ i_A : A \to A \amalg B,\; i_A(a) = (a, A)\qquad \text e \qquad i_B:B \to A \amalg B,\; i_B(b) = (b, B)\]
	Allora possiamo definire $i_Q(a, A) = f_A(a)$ e $i_Q(b, B) = f_B(b)$ e siamo apposto.
\end{es}
Nel caso $A = B$, sostanzialmente si indicizzano gli insiemi.

Non sempre si riesce a trovare un prodotto / coprodotto (per esempio a diversa caratteristica), oppure:
\begin{center}
	\begin{tikzcd}
		& C \arrow[loop above] \arrow[dl] \arrow[dr]\\
		A \arrow[loop left] && B \arrow[loop right]\\
		& D \arrow[loop below] \arrow[ul] \arrow[ur]
	\end{tikzcd}
\end{center}
In questo caso non ho prodotti tra $A$ e $B$ e la sua opposta non avrà coprodotti.

\begin{prop}{}{}
	Un prodotto (analogamente un coprodotto) tra due oggetti $A$ e $B$, se esiste, è unico a meno di isomorfismo.
\end{prop}

\begin{proof}
	Siano $(P, p_A, p_B)$ e $(P', p'_A, p'_B)$ due prodotti e sia il grafo seguente la rappresentazione di parte della categoria:
	\begin{center}
		\begin{tikzcd}
			& P \arrow[dl, "p_A"'] \arrow[d, green, "f_{P'}"] \arrow[dr, "p_B"]\\
			A & P' \arrow[l, "p'_A"] \arrow[d, green, "f_P"] \arrow[r, "p'_B"']& B\\
			& P\arrow[ul, "p_A"] \arrow[ur, "p_B"']
		\end{tikzcd}
	\end{center}
	Abbiamo che $P'$ è un prodotto, quindi esiste un'unica funzione $f_{P'} : P \to P'$ tale che:
	\[ p_A = p_A' \circ f_{P'}\qquad \text e \qquad p_B = p_B' \circ f_{P'} \]
	Ma è anche vero che $P$ è un prodotto, quindi esiste un'unica funzione $f_P:P' \to P$ tale che:
	\[ p_A' = p_A \circ f_P \qquad \text e \qquad p_B' = p_B \circ f_P\]
	Questi due sono i candidati per l'isomorfismo.\\
	Siccome $P$ è un prodotto, esiste una funzione $h:P \to P'$ tale che:
	\[ p_A = p_A' \circ h \qquad \text e \qquad p_B = p_B' \circ h \]
	Ma per l'unicità si ha che $h = id$:
	\[ p_A = p_A' \circ f_{P'} = p_A \circ f_P \circ f_{P'}\qquad \text e \qquad p_B = p_B \circ f_P \circ f_{P'} \]
	Per unicità della funzione si ha che $f_P \circ f_{P'} = id_P$\\
	Scambiando $P$ e $P'$ si ottiene l'altra composizione $f_{P'} \circ f_P = id_{P'}$\\
	Con il coprodotto è la stessa cosa, invertendo le frecce.
\end{proof}

\begin{ese}[Coprodotto e Prodotto in $\bm{Top}$]
	Dati due spazi topologici $(X, \mathcal T_X)$ e $(Y, \mathcal T_Y)$, definiamo:
	\[ (X \times Y, \mathcal T_X \times \mathcal T_Y) \]
	Vediamo che per il prodotto categorico possiamo prendere quello già definito.\\
	Infatti, sia $(Z, \mathcal T_Z)$ e siano $f_X:Z \to X$ e  $f_Y:Z \to Y$ continue e sia:
	\[ f: Z \to X \times Y\qquad z \mapsto (f_X(z), f_Y(z)) \]
	Per la proprietà universale del prodotto topologico si ha che:
	\[ f\text{ è continua}\Leftrightarrow \text{lo sono }p_X \circ f = f_X \text{ e }p_Y \circ f = f_Y \]
	Quindi la definizione di prodotto categorico è ancora soddisfatta.\\
	Per il coprodotto consideriamo:
	\[ (X \amalg Y, \mathcal T_{X \amalg Y}) \]
	Dove si ha che $A \in \mathcal T_{X \amalg Y}$ se e solo se $A|_X \in \mathcal T_X$ e $A|_Y \in \mathcal T_Y$. Notiamo che qui, rispetto a prima, ci sono più insiemi.\\
	Qui, sia $X$, sia $Y$ sono sia aperti sia chiusi in $X \amalg Y$
\end{ese}

\begin{ese}[Coprodotto tra Gruppi]
	Dati $G, H$ gruppi, cerchiamo $G \star H$ gruppo e due omomorfismi $i_G: G \to G\star H$ e $i_H: H \to G \star H$ tali che per ogni gruppo $Z$ con $f_G:G \to Z$ e $f_H:H \to Z$ ho un'unica $i_{G \star H}:G \star H \to Z$ tale che valga il seguente diagramma commutativo.
	\begin{center}
		\begin{tikzcd}
			& Z\\
			& G \star H \arrow[u, "i_{G \star H}"]\\
			G \arrow[uur, bend left= 30, "f_G"] \arrow[ur, "i_G"'] && H \arrow[uul, bend right = 30, "f_H"'] \arrow[ul, "i_H"]
		\end{tikzcd}
	\end{center}
	Cioè valgono:
	\[ f_G = i_{G\star H} \circ i_G \qquad \text e \qquad f_H = i_{G \star H} \circ i_H\]
\end{ese}
Il gruppo che stiamo cercando è proprio il \textbf{Prodotto Libero} dei gruppi $G$ e $H$. Supponiamo che $G$ e $H$ siano gruppi distinti.

\begin{defn}{Prodotto Libero tra gruppi}{}
	Si definisce \textbf{Prodotto Libero} $G \star H$ di $G$ e $H$ l'insieme delle parole/stringhe di $G \cup H$ di elementi tali che:
	\begin{enumerate}
		\item C'è la parola vuota $\vareps$
		\item È costituito da parole \textbf{Ridotte}, cioè se valgono:
			\begin{enumerate}
				\item Ogni lettera è diversa da $e_G$ e da $e_H$
				\item Due lettere adiacenti appartengono a gruppi diversi
			\end{enumerate}
	\end{enumerate}
\end{defn}

Ogni parola può essere semplificata ad una parola del gruppo secondo le seguenti regole:
\begin{enumerate}
	\item Se c'è una lettere uguale a $e_G$ oppure ad $e_H$ la togliamo
	\item Se ci sono due lettere adiacenti in $G$ (o in $H$) le sostituiamo con il loro prodotto.
		\[\forall g_1, g_2 \in G, \quad \cdots hg_1g_2h \cdots \quad \Rightarrow \quad \cdots h(g_1 \cdot g_2)h \cdots \]
		Dove $(g_1\cdot g_2)$ è il prodotto degli elementi $g_1$ e $g_2$
\end{enumerate}

\begin{thm}{}{}
	Il risultato della semplificazione non dipende dall'ordine delle regole che usiamo
\end{thm}

\begin{oss}
	Per convenzione abbiamo che $\vareps$ è una parola ridotta
\end{oss}

Per essere un gruppo, dobbiamo definire anche un'operazione binaria su $G \star H$. Definiamo tale operazione come: se $\alpha, \beta \in G \star H$, definiamo $\alpha\beta$ come la semplificazione fino alla parola ridotta della concatenazione di $\alpha$ e $\beta$.

\begin{prop}{Proprietà dell'operazione su $G\star H$}{}
	Quest'operazione gode delle seguenti proprietà:
	\begin{enumerate}
		\item È associativa
		\item C'è l'elemento neutro $\vareps$
		\item L'inverso di $\alpha = \alpha_1\alpha_2 \cdots \alpha_n$ è $\alpha^{-1} = \alpha_n^{-1}\cdots \alpha_2^{-1}\alpha_1^{-1}$
	\end{enumerate}
\end{prop}

\begin{proof}
	Dimostriamo giusto la terza proprietà:
	\[ \alpha_1\cdots \alpha_{n-1}\alpha_n\alpha_n^{-1}\alpha_{n-1}^{-1}\cdots \alpha_1^{-1} =  \alpha_1\cdots \alpha_{n-1}\vareps\alpha_{n-1}^{-1}\cdots \alpha_1^{-1} = \alpha_1\cdots \alpha_{n-1}\alpha_{n-1}^{-1}\cdots \alpha_1^{-1} \]
	Andando avanti con le semplificazioni si ottiene $\vareps$
\end{proof}

\begin{es}
	Vediamo il prodotto libero di $2$ gruppi ciclici di ordine $2$. Siano $G = \{ e_G, a\}$ e $H=\{e_H, b\}$tali gruppi. Come sono fatte le parole ridotte?
	\[G \star H = \{ \vareps, a, b, ab, ba, aba, bab,... \}\]
	Notiamo che questo è un gruppo infinito numerabile (cioè con un numero finito e numerabile di elementi).\\
	Vediamo un esempio di concatenazione in quest'insieme:
	\[ (abab)\cdot(bab) = aba\underbrace{bb}_\vareps ab = ab\underbrace{aa}_\vareps b = a\underbrace{bb}_\vareps = a \]
\end{es}

\begin{ese}
	Provare a costruire un omomorfismo $G \star H \to \mathbb Z$
\end{ese}

\begin{es}
	Vediamo come è definito il prodotto libero di due gruppi ciclici infiniti $G \cong H \cong \mathbb Z$ con $G = \{a^k : k \in \mathbb Z\}$ e $H = \{b^k : k \in \mathbb Z\}$. Gli elementi di $G \star H$ sono:
	\[ G\star H = \{\vareps, a^k, b^k, a^k b^\ell, b^k a\ell : b, \ell \in \mathbb Z \setminus\{0\}\}\]
	Un esempio di operazione in questo gruppo è:
	\[ (a^2b^{-2}a^3)\cdot (a^-3)b^4a = a^2b^{-2}a^3a^{-3}b^4a = a^2b^{-2}b^4a = a^2b^2a \in G \star H \]
\end{es}

\begin{oss}
	Se $G = \{e_G\}$ allora si ha che $G \star H \cong H$, infatti esiste un isomorfismo che manda $e_H$ nell'elemento $\vareps$, mentre manda gli elementi non nulli $h \in H$ nelle rispettive parole $h \in G \star H$
\end{oss}

Volendo si può costruire il prodotto libero di una famiglia arbitraria di gruppi.

\begin{defn}{Gruppo Libero di $n$ generatori}{}
	Sia $n \in \mathbb N^+$. Il \textbf{Gruppo Libero di $n$ generatori} è:
	\[ \mathbb F_n = \underbrace{\mathbb Z \star \cdots \star \mathbb Z}_{n \text{ volte}}\]
	Scelte delle lettere $a_1,...,a_n$, gli elementi di $\mathbb F_n$ sono le parole ridotte nell'alfabeto:
	\[\bigcup_{i =1}^n\{a_i^k: k\in \mathbb Z\}\]
	Gli elementi $a_i$ prendono il nome di \textbf{generatori liberi} di $\mathbb F_n$
\end{defn}

\begin{thm}{}{}
	Siano $G$ e $H$ gruppi e sia $G \star H$ il loro prodotto libero. Siano le funzioni $\phi_G : G \to G \times H$ e $\phi_H:H \to G\star H$ tali che mandano i rispettivi elementi neutri e gli elementi non nulli nelle parole di $G \star H$ costituite solo da quell'elemento, cioè:
	\[ \begin{matrix}\phi_G(e_G) = \vareps\\ \phi_G(g) = g \in G \star H\end{matrix} \qquad \begin{matrix} \phi_H(e_H) = \vareps\\ \phi_H(h) = h\in G \star H\end{matrix}\]
	Queste due funzioni sono omomorfismi iniettivi.
\end{thm}

Questo teorema può essere facilmente dimostrato sfruttando le definizioni di omomorfismi di gruppi e considerando il prodotto di due elementi.

\begin{lemma}{}{}
	Se $G$ e $H$ sono dei gruppi, allora la terna $(G \star H, \phi_G, \phi_H)$ è il coprodotto di $G \star H$ nella categoria dei gruppi
\end{lemma}

\begin{proof}
	Dalla definizione di coprodotto, si ha che per ogni gruppo $K$ e per ogni coppia di morfismi $\psi_G: G \to K$, $\psi_H :H \to K$ esiste un unico morfismo $f: G \star H \to K$ tale che:
	\[ \psi_G= f \circ \phi_G\qquad \text e \qquad \psi_H = f \circ \phi_G \]
	Con un grafo si avrebbe che:
	\begin{center}
		\begin{tikzcd}
			& Z\\
			& G \star H \arrow[u, "f"]\\
			G \arrow[uur, bend left= 30, "\psi_G"] \arrow[ur, "\phi_G"'] && H \arrow[uul, bend right = 30, "\psi_H"'] \arrow[ul, "\phi_H"]
		\end{tikzcd}
	\end{center}
	Siano $K, \psi_G, \psi_H$ dati. Mostriamo prima l'unicità e poi l'esistenza.

	\textit{Unicità}: Supponiamo esista tale $f$. Allora si ha che:
	\[\forall g \in G \setminus\{e_G\}, (f \circ \phi_G)(g) = f(\phi_G(g)) = f(\text{parola con una lettera }g) = \psi_G(g)\]
	Allo stesso modo, $\forall h \in H$ si ha che $f(\text{parola con unica lettera }h) = \psi_H(h)$.\\
	Allora $f$ è univocamente determinata sulle parole ridotte di lunghezza $1$ di $H$.\\
	Poiché poi $f$ è un omomorfismo di gruppi e queste parole di lunghezza $1$ generano $G \star H$, allora $f$ è univocamente determinata.
	\textit{Un esempio può essere:}
	\[ f(g_1h_1g_2h_2g_3) = \psi_G(g_1)\psi_H(h_1)\psi_G(g_2)\psi_H(h_2)\psi_G(g_3) \]

	\textit{Esistenza}: l'analogo dell'esempio soprastante per ogni parola in $G \star H$ è un omomorfismo di gruppi che va da $G \star H$ a $K$ e che verifica:
	\[ f \circ \phi_G = \psi_G\qquad \text e \qquad f \circ \phi_H = \psi_H \]
\end{proof}

\begin{prop}{}{}
	Sia $n \in \mathbb N^+$ e sia $\mathbb F_n$ gruppo libero di generatori $x_1,...,x_n$ e sia $G$ un gruppo, allora:
	\[Hom_{\bm{Grp}}(\mathbb F_n, G) \cong G \times \cdots \times G = G^n\qquad f \mapsto (f(x_1),...,f(x_n))\]
\end{prop}

\begin{prop}{}{}
	Siano $V, W$ due $\mathbb K$ spazi vettoriali e sia $\{v_1,...,v_n\}$ una base di $V$, Allora:
	\[\begin{matrix} Hom_\mathbb K(V, W) \to V \times \cdots \times V\\ f \mapsto (f(v_1),...,f(v_n))\end{matrix}\qquad \text{è biettiva}\]
\end{prop}

\newpage

\section{Identificazioni e Topologia Quoziente}

\subsection{Identificazioni e Immersioni}

\begin{defn}{Identificazione}{}
	Sia $f: (X, \mathcal T_X) \to (Y, \mathcal T_Y)$ una funzione continua e suriettiva. $f$ è detta \textbf{identificazione} se:
	\[ A \subseteq Y \quad \Leftrightarrow \quad f^{-1}(A) \text{ è aperto in }X\]
\end{defn}

\begin{oss}
	Dagli esercizi (2.34) segue che $\mathcal T_Y$ è la topologia più fine che rende $f$ continua. Se $f$ è anche aperta o chiusa, allora è detta \textbf{identificazione aperta} o \textbf{identificazione chiusa}.
\end{oss}

\begin{defn}{Immersione}{}
	Sia $f:(X,\mathcal T_X) \to (Y, \mathcal T_Y) $ continua e iniettiva. $f$ è detta \textbf{Immersione} se:
	\[ B \subseteq Y \text{ aperto}\quad \Leftrightarrow \quad f^{-1}(B)\text{ aperto in }X \]
\end{defn}

\begin{oss}
	Possiamo dare la stessa definizione anche con i chiusi in quanto si ha che:
	\[ f^{-1}(Y \setminus A) = f^{-1}(Y) \setminus f^{-1}(A) \]
\end{oss}

\begin{defn}{Insieme $f$-saturo}{}
	Sia $B\subseteq X$. $B$ è detto $f$-saturo se vale:
	\[\forall (x,b) \in X \times B\text{ tali che }f(x) = f(b), \text{ si ha che }x \in B\]
	Cioè $B = f^{-1}(f(B))$
\end{defn}

\begin{lemma}{}{}\label{ident}
	Sia $f: (X, \mathcal T_X) \to (Y, \mathcal T_Y)$ una funzione continua e suriettiva. $f$ è un'identificazione se e solo se $B$ è aperto in $Y$ e $\exists A \subseteq X$ $f$-saturo con $f(A)=B$
\end{lemma}

\begin{proof}
	\fbox{$\Rightarrow$} $f$ è suriettiva, quindi $\forall A \subseteq Y$, $f^{-1}(f(A))=A$. Inoltre se $A$ è aperto, $f^{-1}(A)$ è aperto per continuità di $f$ e $f^{-1}(A)$ è $f$-saturo per definizione.

	\fbox{$\Leftarrow$} Sia $B \subseteq Y$ tale che $f^{-1}(B)$ è aperto. Allora $f^{-1}(B)$ è aperto e $f$-saturo e, per ipotesi, ho che $f(f^{-1}(B))=B$ è aperto.
\end{proof}

\begin{lemma}{}{}
	Sia $f:(X, \mathcal T_X) \to (Y, \mathcal T_Y)$ una funzione continua, suriettiva e aperta/chiusa, allora $f$ è un'identificazione aperta/chiusa.
\end{lemma}

\begin{proof}
	Sia $A \subseteq Y$ tale che $f^{-1}(A)$ è aperto/chiuso (\textit{Volendo si può anche aggiungere che sia $f$-saturo, ma non è rilevante}). $f$ è aperta/chiusa, allora $f(f^{-1}(A))$ è aperto/chiuso, quindi $f$ è un'identificazione.
\end{proof}

\begin{es}\label{49}
	Siano $(X, \mathcal T_X)$ e $(Y,\mathcal T_Y)$ spazi topologici e sia $X \times Y$ con topologia prodotto. Allora $p_X:X\times Y \to X$ e $p_Y:X \times Y \to Y$ sono delle identificazioni, in quanto sono continue, sueriettive e aperte.
\end{es}

\begin{lemma}{Proprietà Universale delle Identificazioni}{}
	Sia $f:X \to Y$ una identificazione e $g:X \to Z$ continua, allora esiste $h:Y \to Z$ continua tale che $g = h \circ f$ se e solo se $\forall y \in Y$, $g$ è costante su $f^{-1}(y)$ (cioè è costante sulle fibre di $f$). O equivalentemente $g = h\circ f$ se e solo se $g$ è costante nelle classi di equivalenza sulla relazione $\sim_f$ (con $x \sim_f x' \Leftrightarrow f(x) = f(x')$)\\
	\textit{Chiaramente sono tutti spazi topologici con le rispettive topologie}
\end{lemma}

Il grafo corrispondente a questo lemma è:

\begin{center}
	\begin{tikzcd}
		X \arrow[r, "g"] \arrow[d, "f"'] & Z\\
		Y \arrow[ur, dashed, "h"']
	\end{tikzcd}
\end{center}

\begin{proof}
	\fbox{$\Rightarrow$} Per ogni $x \in X$ abbiamo che $g(x) = h(f(x))$, allora $\forall x' \in f^{-1}(f(x))$ (fibra di $f$) abbiamo che:
	\[ g(x') = h(f(x')) = h(f(x)) =g(x) \]
	Da cui segue che $g$ è costante sulle fibre di $f$

	\fbox{$\Leftarrow$} Definiamo, sfruttando la suriettività di $f$, $h$ nel seguente modo:
	\[ h(y) = g(x) \qquad \text{con }x \in f^{-1}(y)\]
	Mostriamo che questa è la funzione cercata. Cominciamo con il mostrare che $h$ è continua.\\
	Sia $U \subseteq Z$ aperto. Vogliamo mostrare che $h^{-1}(U)$ è aperto. Sappiamo che $G$ è continua, allora $h^{-1}(U)$ è aperto in $X$. Ma abbiamo che $g^{-1}(U) = f^{-1}(h^{-1}(U))$. Ma abbiamo anche che $f$ è un'identificazione, quindi $h^{-1}(U)$ è aperto se e solo se $f^{-1}(h^{-1}(U))$ è aperto. Essendo il secondo un aperto di $X$, allora si ha che $h^{-1}(U)$ è un aperto di $Y$, quindi $h$ è continua
\end{proof}

\begin{es}
	Riprendendo l'esempio delle proiezioni \ref{49}, aggiungendo $g:X \times Y \to Z$ si ha che $\exists h:X \to Z$ se e solo se $g$ non dipende dalla seconda coordinata.
	\begin{center}
		\begin{tikzcd}
			X\times Y \arrow[r, "g"] \arrow[d, "f"'] & Z\\
			X \arrow[ur, dashed, "h"']
		\end{tikzcd}
	\end{center}
\end{es}

\subsection{Quoziente}

\begin{defn}{Spazio Topologico Quoziente}{}
	Sia $(X, \mathcal T_X)$ uno spazio topologico e sia $\sim$ una relazione di equivalenza su $X$ e sia $\pi:X \to X/_\sim$ proiezione sul quoziente. La \textbf{Topologia Quoziente} su $X/_\sim$ è la topologia più fine che renda $\pi$ continua, cioè:
	\[ B \subseteq X/_\sim \text{ aperto} \quad \Leftrightarrow \quad \pi^{-1}(B)\subseteq X\text{ aperto} \]
\end{defn}

\begin{oss}
	Da quanto detto, è automatico che $\pi$ è un'identificazione
\end{oss}

\begin{defn}{Insieme di Rappresentanti}{}
	Sia $X$ un insieme e $\sim$ una relazione di equivalenza. $A \subseteq X$ è detto \textbf{Insieme di Rappresentanti} se, equivalentemente:
	\begin{enumerate}
		\item $\pi|_A:A \to X/_\sim$ è invertibile
		\item $\forall x \in X, \exists !a \in A : x \sim a$
		\item $\exists g: X/_\sim \to A : \pi|_A \circ g = id|_{X/_\sim}$
	\end{enumerate}
\end{defn}

\begin{prop}{}{}
	Sia $\pi:X \to X/_\sim$ proiezione sul quoziente e sia $\bar f:X/_\sim \to Y$ con $(Y, \mathcal T_Y)$ spazio topologico e $X/_\sim$ con topologia quoziente. Allora:
	\[ \bar f \text{ è continua }\Leftrightarrow f = \bar f \circ \pi \text{ è continua}\]
\end{prop}

Con un diagramma commutativo abbiamo che:

\begin{center}
	\begin{tikzcd}
		X \arrow[r, "f"] \arrow[d, "\pi"'] & Y\\
		X/_\sim \arrow[ur, "\bar f"']
	\end{tikzcd}
\end{center}
\textit{La differenza da prima è che prima dovevamo costruirci la funzione, ora invece la abbiamo, dobbiamo mostrare che valgono le implicazioni}

\begin{proof}
	\fbox{$\Rightarrow$} Sappiamo che $f = \bar f \circ \pi$ è composizione di funzioni continue, quindi è ancora continua.

	\fbox{$\Leftarrow$} Sappiamo che $f$ è continua e costante sulle fibre di $\pi$, quindi la per la proprietà universale delle identificazioni, segue che esiste $f':X/_\sim \to Y$ continua tale che $f' \circ \pi$.
	Con un diagramma commutativo abbiamo che:
	\begin{center}
		\begin{tikzcd}
			X \arrow[r, "f"] \arrow[d, "\pi"'] & Y\\
			X/_\sim \arrow[ur, "\bar f"'] \arrow[ur, bend right = 50, "f"']
		\end{tikzcd}
	\end{center}
	Ma sappiamo anche che:
	\[ f' \circ \pi = f = \bar f \circ \pi \]
	Ma abbiamo che $\pi$, essendo un'identificazione, è un epimorfismo, cioè è suriettiva, quindi posso cancellare a destra, quindi ottengo che: $f' = \bar f$
\end{proof}

\begin{prop}{}{}
	Sia $f:X \to Y$ continua e $\pi: X \to X/_\sim$ con topologia quoziente. Allora esiste $\bar f:X/_\sim \to Y$ continua tale che $f = \bar f \circ \pi$ se e solo se $f$ è costante sulle classi di $\sim$-equivalenza
\end{prop}

\begin{es}
	Sia $(X, \mathcal T_{GR})$, $\sim$ relazione di equivalenza, $\pi:X \to X/_\sim$ proiezione al quoziente e $X/_\sim$ con topologia grossolana. Allora $B \in X/_\sim$ è aperto se e solo se $\pi^{-1}(B)$ è aperto in $X$, cioè se e solo se $B = \varnothing$ oppure $B = X/_\sim$
\end{es}

\begin{es}
	Con $(X, \mathcal T_D)$ è la stessa cosa. $B$ è aperto in $X/_\sim$ se e solo se $\pi^{-1}(B)$ è aperto in $X$. La topologia quoziente è proprio quella discreta.
\end{es}

\begin{defn}{Insieme $\sim$-saturo}{}
	Dato $(X, \mathcal T_X)$ spazio topologico e $\sim$ relazione di equivalenza con quoziente $X/_\sim$, un insieme $A \subseteq X$ è detto $\mathbf \sim$\textbf{-saturo} se è $\pi$-saturo (cioè se $a \in A$ e $b \sim a$, allora $b \in A$)
\end{defn}

\begin{prop}{}{}
	Dato $(X, \mathcal T_X)$ spazio topologico e $\pi:X \to X/_\sim$ con $X/_\sim$ con spazio topologico quoziente, ho che gli aperti/chiusi di $X/_\sim$ sono:
	\[ \{ \pi(B): B \subseteq X\text{ è aperto/chiuso} \} \]
\end{prop}

\begin{proof}
	Immediato, segue dal lemma \ref{ident}.5, in quanto si ha che $\pi$ è un'identificazione.
\end{proof}

\subsection{Contrazioni}

\begin{defn}{Contrazione}{}
	Sia $A\subseteq X$ con $(X, \mathcal T_X)$ spazio topologico. Si definisce \textbf{Contrazione} di $A$ dentro $X$ il quoziente rispetto alla relazione $\sim_A$ definita come:
	\[ x \sim_A y \quad \Leftrightarrow \quad (x = y \vee x,y \in A)\]
	In tal caso si denota $\pi_A:X \to X/_A$
\end{defn}

\begin{es}
	Sia $\varnothing \subseteq (X, \mathcal T_X)$ e siano $\pi_\varnothing:X \to X/_\varnothing$ e $\pi_X:X \to X/_X$
	\begin{itemize}
		\item $X/_X$ è un singoletto e ha un'unica topologia possibile.
		\item Con la seconda funzione ho che $\pi_\varnothing (x) = [x]$, quindi è iniettiva e suriettiva. Inoltre:
			\[ A \subseteq X/_\varnothing \text{ aperto con }A = \{[x]:x \in A'\} \Leftrightarrow A'\text{ aperto}\]
	\end{itemize}
\end{es}

\begin{es}[Importante]
	Sia $([0,1], \mathcal T_\mathcal E)$ spazio topologico e voglio collassare $\{0,1\}$. Intuitivamente si ha che questo rappresenta una circonferenza, però voglio mostrare formalmente che:
	\[ [0,1]/_{\{0,1\}} \cong S^1 = \{z \in \mathbb C: |z|=1\} \cong \left\{\begin{pmatrix}x\\ y\end{pmatrix} \in \mathbb R^2: x^2 + y^2 = 1\right\} \]
	Cerchiamo una funzione $f:[0,1] \to S^1$ che sia continua, iniettiva, suriettiva e costante su $0$ e $1$:
	\[ t \mapsto e^{i 2 \pi t}\qquad \vee \qquad t \mapsto \begin{pmatrix} \cos(2 \pi t)\\ \sin(2 \pi t) \end{pmatrix}\]
	Esiste quindi $\bar f:[0,1]\to S^1$ con $f = \bar f \circ \pi$ tale che $\bar f$ sia continua, suriettiva e iniettiva. Vedremo che $\bar f$ è anche chiusa, quindi avremo che $[0,1]/_{\{0,1\}}\cong S^1$
\end{es}

\begin{defn}{Saturazione}{}
	Dato $(X, \mathcal T_X)$ spazio topologico, $\sim$ relazione di equivalenza e $\pi:(X, \mathcal T_X) \to (X/_\sim, \mathcal T_\pi)$ proiezione sul quoziente e $A\subseteq X$, la \textbf{Saturazione} di $A$ (rispetto a $\sim$ o a $\pi$) è il più piccolo insieme saturo ($\pi$-saturo o $\sim$-saturo) contenente $A$, cioè:
	\[ \pi^{-1}(\pi(A)) \]
\end{defn}

\begin{prop}{}{}
	Siano $X, \sim, X/_\sim$ come prima, allora:
	\begin{enumerate}
		\item $\pi$ è aperta/chiusa se e solo se la saturazione di aperti/chiusi è a sua volta aperta/chiusa
		\item Gli aperti/chiusi di $X/_\sim$ sono della forma:
			\[ \{\pi(A) : A \subseteq X \text{ è aperto/chiuso saturo}\} \]
	\end{enumerate}
\end{prop}

\begin{proof}
	Il secondo punto lo abbiamo già dimostrato per le identificazioni. Mostriamo il primo punto.

	\fbox{$\Rightarrow$} Supponiamo $\pi$ aperta e sia $A\subseteq X$ aperto. Allora abbiamo che $\pi(A)$ è un aperto di $X/_\sim$, ma per la definizione di topologia quoziente abbiamo che $\pi^{-1}(\pi(A))$ è ancora aperto (dalla continuità di $\pi$). Quindi la saturazione di aperti è aperta.

	\fbox{$\Leftarrow$} Sia $A \subseteq X$ aperto e so per ipotesi che $\pi^{-1}(\pi(A))$ è aperto. Allora so che $\pi(A)$ è ancora aperto sempre per la definizione di topologia quoziente.

	Per i chiusi è la stessa identica cosa
\end{proof}

\begin{cor}{}{}
	Sia $A \subseteq (X, \mathcal T_X)$ e sia $\pi_A:X \to X/_A$ contrazione di $A$. Se $A$ è aperto/chiuso, allora $\pi$ è aperta/chiusa. Inoltre gli aperti/chiusi di $X/_A$ sono immagini di:
	\[ \{\text{aperti/chiusi contenuti in }X\setminus A\} \cup \{\text{aperti/chiusi contenenti }A\}\]
\end{cor}

\begin{proof}
	Sia $A$ aperto/chiuso e sia $B\subseteq X$ aperto/chiuso a sua volta. Voglio mostrare che:
	\[ \pi_A^{-1}(\pi_A(B)) \text{ è anch'esso aperto/chiuso}\]
	\textit{Però come è fatto?}\\
	Abbiamo che può essere di due forme:
	\[
		\pi^{-1}_A(\pi_A(B)) = \begin{cases}
			B & \text{se }A\cap B = \varnothing\\
			A \cup B & \text{altrimenti}
		\end{cases}
	\]
	Notiamo che in entrambi i casi otteniamo degli insiemi aperti/chiusi, quindi $\pi_A$ è aperta/chiusa.\\
	Questo ci mostra come sono fatti gli aperti/chiusi saturi
\end{proof}

\begin{es}
	Consideriamo $[0,1]/_{\{0,1\}}$. In questo caso abbiamo che $\pi_{\{0,1\}}$ è chiusa. Ma è anche aperta?\\
	No, infatti mi basta prendere l'insieme $[0,\frac 12)$. Questo ha saturazione $[0, \frac 12) \cup \{1\}$ che non è aperto, quindi $\pi_{\{0,1\}}$ non è aperta
\end{es}

\begin{es}
	Sia $(\mathbb R, \mathcal T_\mathcal E)$ e contraggo $[0,1)$. Allora abbiamo che $\pi_{[0,1)}$ non è né aperta né chiusa. Infatti mi basta prendere al suo interno $[0,\frac 12]$ e $(0, \frac 12)$.
	Il primo ha saturazione $[0,1)$ che non è chiuso, mentre il secondo ha saturazione $[0,1)$ che non è aperto.
\end{es}

\begin{es}
	Consideriamo adesso $(\mathbb R, \mathcal T_\mathcal E)$ e contraggo $\mathbb Q$, cioè considero:
	\[ \pi_\mathbb Q: (\mathbb R,\mathcal T_\mathcal E) \to \mathbb R/_\mathbb Q \]
	Nel quoziente, tutti gli aperti non vuoti contengono $[\mathbb Q]$ (la classe dei numeri razionali).\\
	L'unico chiuso contenente $[\mathbb Q]$ è $\mathbb R/_\mathbb Q$. Altri chiusi che possiamo prendere in $\mathbb R/_\mathbb Q$ sono per esempio:
	\[ \left\{ a + \frac 1n : n \in \mathbb N \right\} \cup \{a\}\quad \text{con }a \in \mathbb R\setminus \mathbb Q\]
	Oppure possiamo prendere:
	\[ \{ a + n: n \in \mathbb Z, a \in \mathbb R\setminus \mathbb Q \} \]
	Questo è un caso più unico che raro in cui un punto $[\mathbb Q]$ è denso in $\mathbb R/_\mathbb Q$
\end{es}

Prima di dare il prossimo esempio, diamo delle definizioni che risulteranno più che utili.

\begin{defn}{Sfera $n$-dimensionale e Disco $n$-dimensionale}{}
	Si definiscono rispettivamente \textbf{Sfera $n$-dimensionale} e \textbf{Disco $n$-dimensionale} gli insiemi:
	\[ S^n = \{x \in \mathbb R^{n+1} : \|x\| = 1\} \qquad D^n = \{x \in \mathbb R^n : \|n\|\leq 1\}\]
	Sono entrambi elementi indotti dalla topologia euclidea. Inoltre vale:
	\[ S^{n-1}\subseteq D^n \]
\end{defn}

\begin{es}\label{sndn}
	Sia $D^n \supset S^{n-1}$ e definiamo il quoziente come $\pi:D^n \to D^n/_{S^{n-1}}$. Come è fatto il quoziente?.\[ D^n/_{S^{n-1}} \cong S^n\]
	Notiamo che con $n = 1$ abbiamo che:
	\[ D^1 = [-1,1]\quad S^0 = {-1,1}\qquad D^1/_{S^0} \cong S^1 \]
	L'idea che abbiamo è quella di "gonfiare" il disco, in modo che la circonferenza che delimita il disco diventi la "bocca di un palloncino" che poi andiamo a chiudere:
	\begin{center}
		\begin{tikzpicture}
			\draw (-4,0) circle (1 and 0.5) (4,0) circle (1) (4, 0.75) circle (0.5 and 0.25) (0,0) circle (1 and 0.5);
			\draw (1,0) arc(360:180:1cm);
		\end{tikzpicture}
	\end{center}
	Per questione di estrema comodità, ridefiniamo $S^n$ come:
	\[S^n = \{(x,y) \in \mathbb R^n \times \mathbb R: \|x\|^2 + y^2 = 1\}\]
	Definisco $f$ come:
	\[ f:D^n \to S^n \qquad f(x) = \left( 2x \sqrt{1 -\|x\|^2}, 2\|x\|^2 -1 \right) \]
	Notiamo allora che $f(S^{n-1}) = \{(0,\cdots,0,1)\}$

	Abbiamo banalmente che $f$ è continua e ben definita.

	Mostriamo che $f$ è suriettiva.\\
	Fissiamo $(z,y) \in S^n$. Allora dalla seconda coordinata ho che:
	\[ 2\|x\|^2 -1= y\qquad \Rightarrow \qquad \|x\|^2 = \frac{y+1}2 \]
	Dalla prima coordinata ho invece che:
	\[2x\sqrt{1-\|x\|^2} = z \qquad \Rightarrow \qquad x = \frac{z}{2 \displaystyle{\sqrt{1 - \left( \frac{y+1}2 \right)}}}\]
	Da cui segue immediatamente che $f(x) = (z,y)$, quindi $f$ è suriettiva.

	Mostriamo che $f$ è iniettiva su $D^n\setminus S^{n-1}$.\\
	Dalla prima coordinata ho che:
	\[ f(x_1) = f(x_2) \qquad \Rightarrow \qquad \|x^1\| = \|x^2\|\]
	Dalla seconda ottengo invece che:
	\[ x_1 = x_2 \]
	Quindi è iniettiva su tale restrizione.

	Abbiamo quindi che:
	\begin{center}
		\begin{tikzcd}
			D^n \arrow[r, "f"] \arrow[d, "\pi"] & S^{n}\\
			D^n/_{S^{n-1}} \arrow[ur, dashed, "\bar f"']
		\end{tikzcd}
	\end{center}
	Cioè $f$ è costante sulle classi di equivalenza. Esiste quindi un'unica $\bar f$ continua tale che $f = \bar f \circ \pi$ che sia suriettiva e iniettiva. Vedremo poi anche che $\bar f$ è chiusa. Ma per il momento va bene così
\end{es}

\begin{es}[La retta con origine doppia]
	Sia $X\subseteq \mathbb R^2$ definito come:
	\[ X = \{(x,y) \in \mathbb R^2; y = \pm 1\} \]
	Definisco su $X$ la relazione di equivalenza $\sim$ definita come:
	\[ (x,y) \sim (x',y') \quad \Leftrightarrow \quad (x,y) = (x',y') \vee x\neq 0, x' = x \]
	Quello che abbiamo sostanzialmente è che:
	\begin{center}
		\begin{tikzpicture}
			\draw (-8,0) -- (8,0) (-8,1)--(8,1);
			\filldraw (0,0) circle (1pt) node[below]{$(0,-1)$} (0,1) circle(1pt) node[above]{$(0,1)$} (2,1) circle (1pt) (2,0) circle (1pt);
			\draw[<->, dashed] (2,0) -- node[pos = 0.5, right]{Sono in relazione di equivalenza} (2,1);
		\end{tikzpicture}
	\end{center}
	Quello che stiamo facendo è incollare le due rette in tutti i punti, tranne che nell'origine.
	In maniera intuitiva, il nostro quoziente è:
	\begin{center}
		\begin{tikzpicture}
			\draw (-8,0.5) -- (-0.05,0.5) (0.05,0.5) -- (8,0.5);
			\filldraw (0,0) circle (1pt) node[below]{$(0,-1)$} (0,1) circle(1pt) node[above]{$(0,1)$};
		\end{tikzpicture}
	\end{center}
	Vediamo che i punti sono chiusi, cioè:
	\[\forall (x,y)\in X, \pi^{-1}(\pi(x,y)) \text{ è chiuso}\]
	Inoltre, utilizzando il corollario precedente abbiamo che:
	\[ \pi^{-1}(\pi(x,y)) = \begin{cases} (x,y) & \text{se }x=0\\ \{ (x,y),(x,-y) \} & \text{se }x \neq 0 \end{cases}\]
	Quindi abbiamo la proprietà che i singoletti sono chiusi.\\
	Vogliamo poi vedere che dati punti distinti, essi possono (o meno) essere contenuti in aperti disgiunti.
	In particolare, voglio mostrare che questo non vale per $O_1$ e $O_2$.\\
	Siano $A_1 \ni O_1$ e $A_2 \ni O_2$ aperti, allora abbiamo che esistono $B_1$ e $B_2$ aperti saturi tali che:
	\[ \pi(B_i)=A_i \qquad (0,1)\in B_1\quad (0,-1)\in B_2\]
	Allora esistono $\vareps_1$ e $\vareps_2$ positivi tali che:
	\[ (-\vareps_1,\vareps_1) \times\{1\} \subseteq B_1\qquad (-\vareps_2,\vareps_2) \times\{-1\} \subseteq B_2 \]
	Sappiamo tuttavia che $B_1$ e $B_2$ sono saturi, per cui:
	\[ (-\vareps_1,\vareps_1) \times\{-1\} \subseteq B_1\qquad (-\vareps_2,\vareps_2) \times\{1\} \subseteq B_2 \]
	Quindi abbiamo che $B_1 \cap B_2 \neq \varnothing$, da cui segue che $A_1 \cap A_2 \neq \varnothing$
\end{es}

\subsection{Quozienti con Azioni di Gruppo}

\begin{defn}{Quoziente Gruppi}{}
	Sia $(X, \mathcal T_X)$ uno spazio topologico e sia $G$ un gruppo tale che $G \subseteq \bm{Omeo}(X)$. Sia $\sim$ la relazione di equivalenza definita da:
	\[x \sim y \quad \Leftrightarrow \quad \exists g \in G :g(x)=y\]
\end{defn}

\begin{defn}{Quoziente con Azioni di Gruppo}{}
	Sia $p:G \times X \to X$ un'azione di gruppo, \textit{cioè $\forall g \in G$, la funzione $x \mapsto p(x,g)$ che chiamiamo $g(x)$ è un omeomorfismo e $g(h(x))=(g \circ h)(x)$)}. Posso definire $\sim$ su $X$ con:
	\[ x \sim y \qquad \Rightarrow \qquad \exists g \in G: y = p(g,x) = g(x) \]
	Anche qui il quoziente viene indicato con $X/_G$
\end{defn}

L'azione di un gruppo su uno spazio topologico può essere indicato anche come:
\[ G \curvearrowright (X, \mathcal T) \]

\begin{prop}{}{}
	Sia $\pi:X \to X/_G$ un quoziente rispetto all'azione di gruppo, allora $\pi$ è aperta. Inoltre, se $|G|< + \infty$, $\pi$ è anche chiusa.
\end{prop}

\begin{proof}
	Sia $A\subseteq X$, allora si ha che:
	\[ \pi^{-1}(\pi(A)) = \bigcup_{g \in G}g(A) \]
	Infatti:
	\[ \bigcup_{g \in G}g(A) = \{y \in X: \exists x \in A, g \in G:y = g(x)\} = \{y \in X:\exists x \in A, y \sim x\} = \pi^{-1}(\pi(A))\]
	Inoltre, se $A$ è aperto, allora $g(A)$ è aperto, quindi $\pi^{-1}(\pi(A))$ è unione di aperti, quindi $\pi$ è aperta.\\
	In maniera analoga si dimostra che per $|G|<+ \infty$ e $A$ chiuso
\end{proof}

\begin{es}
	Sia $(\mathbb R, \mathcal T_\mathcal E)$ e definisco l'azione $\mathbb Z \curvearrowright \mathbb R$ tramite:
	\[ p:\mathbb Z \times \mathbb R \to \mathbb R \qquad (n, x)\mapsto x + n \]
	Consideriamo ora $\mathbb R/_\mathbb Z$ (\textit{Questo è un chiarissimo esempio in cui la notazione fa altamente schifo} $\sim$ Mongardi). Vediamo ora come agisce il gruppo.\\
	Notiamo che $[0] = \mathbb Z$ e, più in generale, si ha che $[x] = x + \mathbb Z$\\
	Consideriamo un chiuso in cui sono sono presenti almeno una volta tutte le classi, cioè $[0,1]$. Nel suo interno compaiono tutte le classi (ad eccezione degli estremi). Sembra quasi che questo quoziente sia isomorfo a $S^1$. Mostriamo che è proprio così.\\
	Consideriamo $f:\mathbb R/ \to S^1$, definita come $t \mapsto e^{2\pi i t}$ (se la vediamo come circonferenza nel piano complesso). Notiamo allora che:
	\[ \forall n \in \mathbb N, f(t) = f(t+n) \]
	Notiamo che $f$ è continua e suriettiva, inoltre:
	\[\exists \bar f : \mathbb R/_\mathbb Z \to S^1 \; \text{tale che}:\]
	\begin{center}
		\begin{tikzcd}
			\mathbb R \arrow[r, "f"] \arrow[d, "\pi"] & S^1\\
			\mathbb R/_\mathbb Z \arrow[ur, "\bar f"']
		\end{tikzcd}
	\end{center}
	In questo modo abbiamo anche che $\bar f$ è iniettiva.\\
	Cerchiamo un compatto che suerietti su $\mathbb R/_\mathbb Z$ per poter applicare il teorema \ref{housdorff}.9.\\
	Utilizziamo la seguente proposizione che dimostreremo più avanti: "$f$ continua e $X$ compatto, allora $f(X)$ compatto". Scegliamo $[0,1]$ e come funzione continua prendiamo $\pi|_{[0,1]}$.
	\[ \pi|_{[0,1]}:[0,1] \to \mathbb R/_\mathbb Z \text{ è suriettiva}\]
	Sapendo che $\mathbb R/_\mathbb Z$ è compatto, allora per il teorema \ref{housdorff}.9 è chiusa, quindi è un omomorfismo.
\end{es}

\newpage

\section{Proprietà degli Spazi Topologici}

\subsection{Assiomi Di Numerabilità e di Separabilità}

\begin{defn}{Assiomi di Numerabilità}{}
	Uno spazio topologico $(X, \mathcal T_X)$ soddisfa gli \textbf{Assiomi di Separabilità} $N1$ e $N2$ se:
	\begin{itemize}
		\item[$N1$:] Ogni punto di $X$ ha un sistema fondamentale di intorni di cardinalità al più numerabile (\textit{Questa è una condizione locale})
		\item[$N2$:] Esiste una base $\mathcal B$ di $\mathcal T$ di cardinalità al più numerabile (\textit{Questa è una condizione globale})
	\end{itemize}
\end{defn}

\begin{defn}{Separabile}{}
	Uno spazio topologico $(X, \mathcal T)$ è \textbf{Separabile} se esiste un denso al più numerabile
\end{defn}

\begin{es}
	Lo spazio topologico $(X, \mathcal T_{GR})$ è $N1, N2$ ed è separabile. La verifica è pressoché immediata
\end{es}

\begin{es}
	Lo spazio metrico $(X, d)$ con $\mathcal T_d$ distanza indotta dalla metrica $d$ soddisfa $N1$ con
	\[\mathcal I = \{B_{\frac 1n}(x)\}_{n>0} \]
	Sistema fondamentale di intorni formato da palle centrate in $x$ e di raggio $\frac 1n$ con $n \in \mathbb N^*$
\end{es}

\begin{es}
	Lo spazio topologico euclideo $(\mathbb R, \mathcal T_\mathcal E)$ soddisfa $N1, N2$ ed è separabile. Soddisfa $N1$ in quanto basta prendere $\overline {\mathbb Q} = \mathbb R$. Soddisfa $N2$ in quanto posso prendere la base:
	\[ \mathcal B = \{B_{\frac 1n}(x) : x \in \mathbb Q, n \in \mathbb N^*\} \]
	In maniera del tutto analoga, anche $(\mathbb R^n, \mathcal T_\mathcal E)$ soddisfa $N1, N2$ ed è separabile
\end{es}

\textit{Alcuni degli assiomi che stanno per essere definiti sono già stati visti nel corso di Analisi 2}

\begin{defn}{Assiomi di Separabilità}{}
	Uno spazio topologico $(X, \mathcal T)$ soddisfa gli \textbf{Assiomi di Separabilità} $T1,T2,T3,T4$ se:
	\begin{itemize}
		\item[$T1$:] $\forall x,y \in X,x \neq y,\exists U \in I(x), \exists V \in I(y)$ intorni di $x$ e di $y$ rispettivamente tali che $y \not \in U$ e $x \not \in V$
		\item[$T2$:] $\forall x, y \in X, x\neq y, \exists U \in I(x), \exists V \in I(y):U \cap V = \varnothing$
		\item[$T3$:] $\forall F$ chiuso, $\forall x \not \in F, \exists U,V$ aperti tali che $x \in V, F \subseteq U, U \cap V = \varnothing$
		\item[$T4$:] $\forall F,G$ chiusi \underline{disgiunti}, $\exists U,V$ aperti tali che $F \subseteq U, G \subseteq V$ e $U \cap V = \varnothing$
	\end{itemize}
\end{defn}

\begin{es}
	Lo spazio topologico $(X, \mathcal T_{GR})$ soddisfa $T3$ e $T4$. Infatti $X$ e $\varnothing$ sono chiusi disgiunti ma $X \subseteq X$ e $\varnothing \subseteq \varnothing$ e $X \cap \varnothing = \varnothing$, quindi $T4$ è soddisfatta. Per $T3$, se $F = X$, allora non esiste un $x$ che non appartenga a $F$, se invece $F = \varnothing$, allora è banalmente verificata. $T2$ invece non è soddisfatta, in quanto non esistono due introni che non siano disgiunti e che non siano banali
\end{es}

\begin{es}
	Lo spazio topologico $(X, \mathcal T_D)$ soddisfa tutti gli assiomi di separabilità. Infatti:
	\begin{itemize}
		\item Per $T1$ e $T2$ basta prendere $x \neq y$ e considerare $x \in \{x\}$ e $y \in \{y\}$
		\item Per $T3$ basta considerare $F$ chiuso con $x \notin F$, allora $F\subseteq F$ e $x \subseteq \{x\}$ sono degli aperti disgiunti
		\item Per $T4$ basta prendere $F,G$ chiusi disgiunti, allora $F \subseteq F$ e $G \subseteq G$ sono aperti disgiunti
	\end{itemize}
\end{es}

\begin{oss}
	Bisogna fare attenzione al fatto che $T3\not \Rightarrow T2$, infatti un singoletto può non essere chiuso
\end{oss}

\begin{es}
	Lo spazio topologico $(X, \mathcal T_{Cof})$ è $T1$. Infatti, prendiamo $x \neq y$, allora posso prendere:
	\[U = X \setminus \{y\}\qquad \text e\qquad V = X \setminus \{x\}\]
	Notiamo che nel primo fa parte $x$ mentre nel secondo fa parte $y$. Questi sono degli aperti tali che:
	\[x \in U, y \not \in U\qquad y \in V, x \not \in V\]
\end{es}

\begin{oss}\label{conngros}
	Se consideriamo lo spazio topologico $(X, \mathcal T_{Cof})$ con $|X|=+\infty$, questo \underline{non è mai} $T2$. Infatti, se prendiamo $U,V$ aperti non vuoti in $\mathcal T_{Cof}$, allora
	\[ U \cap V \neq \varnothing \]
	Se per assurdo infatti avremmo che $V \cap U = \varnothing$, ne seguirebbe che $U \subseteq X \setminus V$. Ma $V$ è un aperto non vuoto, quindi nella topologia cofinita $X \setminus V$ è finito. Ma anche $U$ è un aperto non vuoto, quindi anche $X \setminus U$ è finito, quindi, visto che $|X| = +\infty$, segue che $U$ ha cardinalità infinita, ma questo è un assurdo perché avremmo che un insieme infinito sta dentro un insieme finito
\end{oss}

\begin{es}\label{dT2}
	Sia $(X, d)$ spazio metrico con topologia $\mathcal T_d$ indotta dalla metrica. Allora $(X, d)$ è $T2$.
	Infatti dati $x,y \in X$ tali che $d(x,y) = \delta$, posti $U = B_\frac \delta 2(x)$ e $V = B_\frac \delta 2(y)$, si ha che $V \cap U = \varnothing$
\end{es}

\begin{oss}
	Gli spazi metrici sono $T1,T2,T3,T4$
\end{oss}

\subsection{Connessione e Compattezza}

\begin{defn}{Connessione per Archi}{}
	Sia $(X, \mathcal T)$ uno spazio topologico, diciamo che $X$ è \textbf{Connesso per Archi} (e lo indichiamo con $CPA$) se:
	\begin{itemize}
		\item $X \neq \varnothing$
		\item $\forall x,y \in X, \exists \gamma:([0,1], d) \to (X, \mathcal T)$ tale che $\gamma(0)=x$ e $\gamma(1)=y$ che sia continua
	\end{itemize}
\end{defn}

\begin{es}
	Sia $X$ non vuoto. Allora possiamo dire che $(X, \mathcal T_{GR})$ è connesso per archi. Infatti tutte le funzioni $\gamma:([0,1],d) \to (X, \mathcal T_{GR})$ sono continue, quindi posso definire $\gamma(0)=x$ e $\gamma(y)=1$
\end{es}

\begin{es}
	$(\mathbb R, \mathcal T_\mathcal E)$ è connesso per archi. Allo stesso modo è connesso anche $(\mathbb R^n, \mathcal T_\mathcal E)$. In realtà, $\mathbb R^n$ continua con ogni topologia meno fine di quella euclidea. Infatti la funzione $\gamma:[0,1]\to \mathbb R$ continua ad essere continua se metto una topologia meno di quella euclidea sul codominio
\end{es}

\begin{es}
	Sia $X \subseteq (\mathbb R, \mathcal T_\mathcal E)$ con $X$ convesso, cioè $\forall x,y \in X, [a,b] \subseteq X$. Questo è connesso per archi rispetto alla topologia indotta da quella euclidea. Basta infatti prendere $\gamma(t) = x+t(y-x)$
\end{es}

\begin{defn}{Connessione}{}
	Uno spazio topologico $(X, \mathcal T)$ è detto \textbf{Connesso} se $\forall U,V$ aperti non vuoti tali che $V \cup U = X$ si ha che l'intersezione non nulla
\end{defn}

Per convenzione, lo spazio vuoto è non connesso

\begin{es}
	$(X, \mathcal T_{GR})$ con $X$ non vuoto è uno spazio topologico connesso.
\end{es}

\begin{es}
	Lo spazio Topologico $(X, \mathcal T_{D})$ con $|X|>1$, questo non è più connesso. Infatti possiamo prendere $x \in X$, allora $\{x\}$ e $X\setminus \{x\}$ sono aperti non vuoti la cui unione da $X$ ma la loro intersezione è vuota
\end{es}

\begin{thm}{}{}
	Lo spazio topologico $([0,1], \mathcal T_\mathcal E)$ è connesso.
\end{thm}

\begin{es}
	Sia $(\mathbb Q, \mathcal T)$ con $\mathcal T$ topologia indotta da quella euclidea, non è connesso. Intuitivamente, se abbiamo $z$ irrazionale, allora esistono $q_1, q_2 \in \mathbb Q$ tali che:
	\[q_1 <z<q_2\]
	In particolare, i due aperti da prendere sono $] -\infty, 0]\cap \mathbb Q$ e $[0, \infty[\cap \mathbb Q$
\end{es}

\begin{defn}{Ricoprimento e Ricoprimento Aperto}{}
	Dato $(X, \mathcal T)$ spazio topologico, una famiglia di insiemi $\{U_i\}_{i \in I}$ è detta \textbf{Ricoprimento} di $X$ se:
	\[ X = \bigcup_{i \in I}U_i \]
	Se gli $U_i$ sono aperti, allora si dice che è un ricoprimento \textbf{aperto}
\end{defn}

\begin{defn}{Compattezza}{}
	Sia $(X, \mathcal T)$ spazio topologico, si dice che è \textbf{Compatto} se per ogni ricoprimento aperto di $X$, esiste un sottoricoprimento finito che ricopre $X$
\end{defn}

\begin{es}
	Lo spazio topologico $(\mathbb R, \mathcal T_\mathcal E)$ non è compatto, infatti $\{[-n,n]\}_{n \in \mathbb B}$ è un ricoprimento aperto, ma non esiste un sottoricoprimento finito che contenga $\mathbb R$
\end{es}

\begin{thm}{}{}
	Lo spazio topologico $([0,1], \mathcal T_\mathcal E)$ è compatto
\end{thm}

\begin{thm}{}{}
	Un qualunque spazio topologico $(X, \mathcal T_{GR})$ con topologia grossolana è compatto
\end{thm}

\begin{es}
	Lo spazio topologico $(X, \mathcal T_D)$ con $|X|= + \infty$ non è compatto, in quanto l'unione dei singoletti non connette un sottoricoprimento finito che ricopra $X$
\end{es}

\begin{es}
	Lo spazio topologico $(X, \mathcal T_{Cof})$ è compatto.
\end{es}

\begin{thm}{di Heine Borel}{}
	Sia $X \subseteq \mathbb R^n$ con topologia indotta da topologia euclidea. Allora $(X, \mathcal T_X)$ è compatto se e solo se è chiuso e limitato in $\mathbb R^n$
\end{thm}

\begin{thm}{del compatto Hausdorff}{}\label{housdorff}
	Sia $f:(X, \mathcal T_X) \to (Y, \mathcal T_Y)$ continua con $X$ compatto e $Y$ $T2$, allora $f$ è chiusa.
\end{thm}

\begin{oss}
	Un quoziente di uno spazio compatto è ancora compatto
\end{oss}

\subsection{Proprietà Locali}

\begin{defn}{Spazio Topologico Localmente Connesso}{}
	Uno spazio topologico $(X, \mathcal T_X)$ è \textbf{localmente connesso} se per ogni punto di $X$, esiste un sistema fondamentale di intorni connessi.
\end{defn}

\begin{defn}{Spazio Topologico Localmente Connesso per Archi}{}
	$(X, \mathcal T)$ si dice \textbf{Localmente Connesso per Archi} se per ogni suo punti esiste un sistema fondamentale di intorni connessi per archi
\end{defn}

\begin{es}
	Consideriamo il seguente spazio topologico:
	\begin{center}
		\begin{tikzpicture}
			\draw (-5,0) -- (5,0) (-5,1) -- (5,1);
			\draw plot [smooth cycle] coordinates {(0,0) (1,1) (3,1) (1,0) (2,-1)};
		\end{tikzpicture}
	\end{center}
	Questi non sono connessi, ma sono localmente connessi.
\end{es}

\begin{defn}{Spazio Topologico Localmente Compatto}{}
	Uno Spazio Topologico si dice \textbf{Localmente Compatto} se per ogni suo punto, esiste un unico intorno di compatti di $X$
\end{defn}

\begin{es}
	$(\mathbb R^n, \mathcal T_\mathcal E)$ è localmente connesso, localmente connesso per archi e localmente compatto
\end{es}

\begin{defn}{Spazio Topologico Localmente Euclideo}{}
	Uno spazio topologico $(X, \mathcal T)$ è \textbf{Localmente Euclideo} se:
	\[\forall x \in X, \exists U_x \in \mathcal T_X \text{ aperto}, \exists f:U \to (\mathbb R^n, \mathcal T_\mathcal E)\text{ continua, aperta e iniettiva}\]
\end{defn}

\begin{es}
	Lo spazio topologico $(X, \mathcal T_\mathcal E)$ è localmente euclideo
\end{es}

\begin{es}
	$(S^n, \mathcal T_\mathcal E)$ è localmente euclideo. Infatti, se prendo $U_0$ come $S^n \setminus \{(0,...,0,1)\}$, ho che $U_0 \cong \mathbb R^n$ tramite proiezione stereografica per esempio
\end{es}

Se ci guardiamo un attimo, noi siamo su $S^2$ (molto a grandi linee). Eppure tutto ci sembra piano. Concettualmente è questo il significato di localmente euclideo.

\begin{oss}
	Questa è la stessa dinamica dell'atlante geografico.
\end{oss}

\subsection{Risultati sugli assiomi di Numerabilità}

\begin{lemma}{}{}
	Uno spazio topologico $(X, \mathcal T)$ che soddisfi $N2$ è anche separabile
\end{lemma}

\begin{proof}
	Sia $\mathcal B = \{U_n\}_{n \in \mathbb N}$ una base numerabile di $\mathcal T$. Per ogni $n$, scegliamo un punto $x_n \in U_n$ (ciò è possibile assumendo l'assioma della scelta). Sia poi $E = \{x_n : n \in \mathbb N\}$. Dimostriamo che $\overline E = X$\\
	Infatti, se fosse stato diverso, sarebbe esistito un $U = X \setminus \overline E$ aperto non vuoto completamente disgiunto da $E$, cioè $U \cap E = \varnothing$, ma $U$ è aperto, quindi è unione di elementi della base $\mathcal B$, quindi esiste $n_0 \in \mathbb N$ tale che:
	\[ U_{n_0} \subseteq U \qquad \Rightarrow \qquad x_{n_0}\in U \]
	Ma questo è assurdo, quindi $\overline E = X$
\end{proof}

\begin{lemma}{}{}
	Ogni spazio metrico separabile soddisfa $N2$
\end{lemma}

\begin{proof}
	Sia $(X, d)$ uno spazio metrico con $\mathcal T_d$ topologia indotta dalla metrica. Sia poi $E \subseteq X$ denso numerabile. Mostriamo che:
	\[ \mathcal B = \{ B_{\frac 1{2^n}}(e) : e \in E, n \in \mathbb N \} \text{ è una base di }\mathcal T_d \]
	Sia $U \in \mathcal T_d$, vogliamo scriverlo come unione di elementi di $\mathcal T_d$.\\
	Sia $x \in U$, allora esiste $n \in \mathbb N$ tale che:
	\[B_{\frac 1{2^{n-1}}}(x)\subseteq U\]
	Per la densità di $E$, ho che:
	\[\exists e \in E \cap B_{\frac 1{2^n}}(x)\]
	Infatti ho che $d(x,e)<\frac 1{2^n}$, quindi $x \in B_{\frac 1{2^n}}(e)$. Segue quindi che:
	\[ \forall y \in B_{\frac 1{2^n}}(e), d(x,y)\leq d(x,e) + d(e,y) \leq \frac 1{2^n} + \frac 1{2^n} = \frac 1{2^{n-1}} \]
	In questo modo ho mostrato che $\forall x \in U, \exists V_x \in \mathcal B : x \in V_x \subseteq U$. Cioè che:
	\[ U = \bigcup_{x \in U}V_x \]
	Quindi $\mathcal B$ è una base di $\mathcal T_d$ ed è numerabile, quindi $(X, d)$ è $N2$
\end{proof}

\begin{prop}{}{}
	In uno spazio topologico $N2$, ogni ricoprimento aperto ammette un sottoricoprimento al più numerabile
\end{prop}

\begin{proof}
	Sia $\mathcal B$ una base numerabile e sia $A = \{U_i\}_{i \in I}$ ricoprimento aperto. Per ogni punto scegliamo $U_x \in A$ che contenga $x$ e $B_x \in \mathcal B$ tale che $x \in B_x \subseteq U_x$\\
	\textit{È possibile sceglierli in quanto $A$ è un ricoprimento e $\mathcal B$ è una base}\\
	Segue quindi che:
	\[ X = \bigcup_{x \in X}U_x = \bigcup_{x \in X}B_x \]
	Definiamo allora $\mathcal B' \subseteq \mathcal B$ dato da:
	\[ \mathcal B' = \{B_x : x \in X\}\]
	Da cui segue che $\mathcal B'$ è al più numerabile. Inoltre, per ogni $B' \in \mathcal B$, scelto $U'$ in $A$ tale che $B'\subseteq U'$ e ottengo un sottoricoprimento numerabile
\end{proof}

\subsection{Risultati degli assiomi di Separazione}

\begin{prop}{}{}
	$X$ è $T1$ se e solo se $\forall x \in X$, $\{x\}$ è chiuso, cioè se e solo se la sua topologia è più fine della cofinita
\end{prop}

\begin{proof}
	\fbox{$\Rightarrow$} Sia $x \in X$ e sia $y \neq x$. Sappiamo che lo spazio topologico è $T1$, quindi $\exists V_y$ intorno aperto di $y$ si ha che $x \not \in V_y$. In questo modo si ha che:
	\[ X\setminus \{x\} = \bigcup_{y \in X}V_y \]
	Cioè $X\setminus \{x\}$ è un'unione di aperti, quindi è aperto, da cui segue facilmente che $\{x\}$ è chiuso.

	\fbox{$\Leftarrow$} Sia $\{x\}$ chiuso, $\forall x \in X$. Siano $x \neq y$, allora $X\setminus \{x\}$ e $X \setminus \{y\}$ sono aperti, ma si ha che:
	\[ y \in X\setminus\{x\} \not \ni x \qquad x \in X\setminus \{y\} \not \ni y\]
	Da cui segue che $X$ è $T1$
\end{proof}

Uno spazio topologico $T2$ è detto anche di \textbf{Housdorff}.

\begin{prop}{}{}
	$T4 + T1 \Rightarrow T3 + T1 \Rightarrow T2 \Rightarrow T1$
\end{prop}

\begin{proof}
	\fbox{$T4 + T1 \Rightarrow T3 + T1$} Per ipotesi abbiamo che $\forall x \in X, \{x\}$ è chiuso e che $\forall F,G$ chiusi e disgiunti, $\exists U,V$ aperti con $F \subseteq U$ e $G \subseteq V$ tali che $U \cap V = \varnothing$. Allora abbiamo che $\forall F$ chiuso e $\forall x \not \in F$, $\exists U,V$ aperti che separano i due chiusi $F$ e $\{x\}$, cioè $X$ è $T3$

	\fbox{$T3 + T1 \Rightarrow T2$} Dato $F$ chiuso e $x \notin F$, $\exists U,V$ aperti che li separino. Sia $F = \{y\}$ e $y \neq x$. Allora li possiamo separare con due aperti. Quindi $X$ è $T2$

	$\fbox{$T2 \Rightarrow T1$}$ Segue dalla definizione
\end{proof}

\begin{prop}{}{}
	Sia $(X, d)$ uno spazio metrico con $\mathcal T_d$ topologia indotta da $d$. Allora \[ (X, \mathcal T_d)\text{ è }T4,T3,T2,T1 \]
\end{prop}

\begin{proof}
	Sappiamo che $(X, \mathcal T_d)$ è $T2$ dall'esempio \ref{dT2}, mostriamo che è $T4$ (sapendo questo ho automaticamente che è anche $T3$)
	Siano quindi $F, G$ due chiusi disgiunti. Allora $\forall A \subseteq X, \forall x \in X$ definisco:
	\[d_A(x) = \inf_{y \in A}d(x,y)\]
	Quindi vale che $d_A(x) = 0$ se e solo se $x \in \overline A$ \textit{(Da dimostrare)}\\
	Vale anche che $x \mapsto d_A(x)$ è una funzione continua da $(X, d)$ a $(\mathbb R, \mathcal T_\mathcal E)$ \textit{(da dimostrare)}\\
	Consideriamo allora la funzione:
	\[\delta : X \to ([0,1], \mathcal T_\mathcal E)\qquad x \mapsto \frac{d_F(x)}{d_F(x) + d_G(x)}\]
	Ho che $\delta$ è una funzione ben definita e è continua in quanto somma e prodotti di funzioni continue. Inoltre ho che:
	\[ \delta^{-1}(0) = \{x \in X: x \in F\} = F \qquad \delta^{-1}(1) = \{x \in X : x \in G\} = G\]
	Mi basta quindi porre:
	\[U = \delta^{-1}\left(\left[0, \frac 12\right)\right)\qquad V = \delta^{-1}\left(\left(\frac 12, 1\right]\right)\]
	Infatti $U$ e $V$ sono due aperti disgiunti che contengono rispettivamente $F$ e $G$
\end{proof}

\begin{lemma}{}{}
	Sia $X$ spazio topologico $Ti$ con $i \in \{1,2,3,4\}$ e $f:X \to Y$ omeomorfismo. Allora $Y$ è $Ti$
\end{lemma}

\begin{proof}
	\fbox{$T1,T2$} Se ho $U$ aperto con $x \in U$, allora $f(U)$ è aperto con $f(x)$ e $f(x) \in f(U)$. Questo vale anche per gli insiemi che non contengono $x$.\\
	Se ho $x \neq y$ con $x \in U, y \not \in U$ e $x \not \in V, y \in V$, allora ho che:
	\[ f(x) \in f(U) \not \ni f(y)\qquad f(x)\not \in f(V) \ni f(y) \]
	Se avessi che $U \cap V = \varnothing$ allora $f(V)\cap f(U) = \varnothing$

	\fbox{$T3,T4$} Se ho $F$ chiuso e $U$ aperto tale che $F \subseteq U$, allora ho $f(F)$ chiuso e $f(U)$ aperto con $f(F) \subseteq f(U)$.\\
	Questo in particolare ci dice che se ho $F,G$ chiusi di $Y$ disgiunti, allora posso separare $f^{F}$ e $f^{-1}(G)$ con $U,V$ aperti in $X$. Infatti se ho $f^{-1}(F)\subseteq U$ e $f^{-1}(G)\subseteq V$ disgiunti, allora segue che $F \subseteq f(U)$, $G \subseteq f(V)$ e $f(U) \cap f(V) = \varnothing$\\
	Dati invece $F\subseteq Y$ e $y \not \in F$, presi $x \in f^{-1}(y)$ e $f^{-1}(F)$, allora per $T3$ di $X$ ho che esistono $U,V$ aperti con $x \in U$, $f^{-1}(F) \subseteq V$, da cui posso separare $F$ e $y$.
\end{proof}

\begin{prop}{}{}
	Sia $(X, \mathcal T_X)$ e $Y \subseteq X$ con topologia indotta. Se $X$ è $T1, T2$ o $T3$, allora anche $Y$ lo è
\end{prop}

\begin{proof}
	\fbox{$T1$} Visto che $X$ è $T1$ abbiamo che $\forall x \in X, \{x\}$ è chiuso, quindi $\forall y \in Y$ si ha che:
	\[ \{y\}\cap Y = \{y\} \text{che è ancora chiuso per la topologia indotta} \]
	Da cui segue che $Y$ è $T1$.

	\fbox{$T2$} Siano $y_1, y_2 \in Y$ allora visto che $X$ è $T2$:
	\[ \exists U,V \text{aperti di }X : y_1\in U, y_2 \in V, V \cap U = \varnothing \]
	Ma abbiamo che $y_1 \in U \cap Y$ e $y_2 \in V \cap Y$ e questi sono aperti in $Y$ la cui intersezione è l'insieme vuoto.

	\fbox{$T3$} Sia $F \subseteq Y$ chiuso e sia $y \not \in F, y\in Y$. Dal fatto che $F$ sia chiuso in $Y$, abbiamo che $\exists G$ chiuso in $X$ tale che $F = G \cap Y$. Siccome si ha che $y \in Y$ e $y \not \in F$, segue che $y \not \in G$.
	Sappiamo poi che $X$ è $T3$, quindi esistono due aperti $U,V$ in $X$ tali che:
	\[ y \in U, F \subseteq V, V \cap U = \varnothing\]
	Essendo poi $(U \cap Y)$ e $(V \cap Y)$ aperti in $Y$, abbiamo che:
	\[ y \in (U \cap Y)\quad F \subseteq (V \cap Y) \]
	Quindi $Y$ è $T3$
\end{proof}

\textit{Perché non possiamo dire che se $X$ è $T4$, allora $Y$ è $T4$?}
Consideriamo il caso in cui:
\begin{center}
	\begin{tikzpicture}
		\draw (0,0) circle (3cm and 1cm);
		\draw[smooth, green] plot[tension = 1] coordinates{(1,0) (2,1) (0,2) (-1,2) (0.5,1) (1,0)};
		\draw[smooth, x = -1cm, blue] plot[tension = 1] coordinates{(1,0) (2,1) (0,2) (-1,2) (0.5,1) (1,0)};
		\draw (-3, 2) node{$X$} (-3, -0.5)node{$Y$};
		\draw[green] (1, -0.25) node{$F$};
		\draw[blue] (-1,-0.25) node{$G$};
	\end{tikzpicture}
\end{center}
In questo caso abbiamo trovare due chiusi disgiunti in $Y$ ma che non sono in $X$, quindi non esistono due aperti disgiunti che li contengono.

\begin{prop}{}{}
	Siano $(X, \mathcal T_X)$ e $(Y, \mathcal T_Y)$ con $X, T$ spazi $T1,T2,T3$. Sia poi $X \times Y$ il prodotto con topologia prodotto. Allora quest'ultimo è $T1,T2,T3$
\end{prop}

\begin{proof}
	\fbox{$T1$} Considero $\{(x,y)\} = \{x\}\times \{y\} = p_X^{-1}(\{x\}) \cap p_Y^{-1}(\{y\})$, cioè possiamo considerare un singolo punto del prodotto come intersezione delle preimmagini delle proiezioni. Tuttavia abbiamo che $\{x\}$ e $\{y\}$ dei chiusi ed essendo $p$ continua, abbiamo che $\{(x,y)\}$ è chiuso, quindi $X \times Y$ è $T1$

	\fbox{$T2$} Siano $(x_1,y_1)$ e $(x_2,y_2)$ diversi tra loro. Affinché siano diversi, ci basta che solo una variabile sia diversa, quindi possiamo supporre che $x_1\neq x_2$. Siccome poi $X$ è $T2$, quindi esistono $U,V$ aperti di $X$ tali che:
	\[ x_1 \in U\quad x_2 \in V\quad U \cap V = \varnothing \]
	In questo modo abbiamo che $(x_1,y_1) \in p^{-1}_X(U)$ e $(x_2,y_2) \in p_X^{-1}(V)$. Ma questi sono aperti in quanto $p$ è continua, quindi:
	\[ p^{-1}_X(U) \cap p_X^{-1}(V) = \varnothing \]
	Quindi abbiamo che $X \times T$ è $T2$
	\begin{center}
		\begin{tikzpicture}
			\draw (0,0) -- (0,2) -- (3,2) -- node[pos = 0.5, right]{$X \times Y$} (3,0) -- (0,0) (-1,0) -- (-1,2) node[left]{$X$} (0,-1) -- (3,-1) node[right]{$Y$};
			\filldraw[blue] (1,1.5) circle (1pt) (2,1) circle(1pt) (1,-1) circle(1pt) (2,-1) circle(1pt) (-1,1) circle(1pt) (-1,1.5) circle(1pt);
			\draw[dashed, green] (-1,1.375) -- (3,1.375) (-1,1.625) -- (3,1.625);
			\filldraw[green, very nearly transparent] (0,1.375) -- (0,1.625) -- (3,1.625) -- (3,1.375) -- (0,1.375);
			\draw[dashed, green] (-1,0.875) -- (3,0.875) (-1,1.125) -- (3,1.125);
			\filldraw[green, very nearly transparent] (0,0.875) -- (0,1.125) -- (3,1.125) -- (3,0.875);
			\draw[->] (1.5, -0.25) -- node[right, pos = 0.5]{$p_Y$} (1.5, -0.75);
			\draw[->] (-0.25, 0.5) -- node[below, pos = 0.5]{$p_X$}(-0.75,0.5);
		\end{tikzpicture}
	\end{center}

	\fbox{$T3$} \textit{Non possiamo fare come prima perché potremmo avere delle condizioni scomode come quella che segue}
	\begin{center}
		\begin{tikzpicture}
			\draw (0,0) -- (0,2) -- (3,2) -- node[pos = 0.5, right]{$X \times Y$} (3,0) -- (0,0) (-1,0) -- (-1,2) node[left]{$X$} (0,-1) -- (3,-1) node[right]{$Y$};
			\draw (1.5,1.75) to[out = 0, in = 90] (2.5,1.5) to[out = 270,  in = 0] (1.5,1.25) to [out = 180, in = 90] (1,1) to[out = 270, in = 180] (1.5, 0.75) to[out = 0, in = 90] (2.5,0.5) to [out = 270, in = 0] (1.5, 0.25) to[out = 180, in = 270] (0.5, 1) node[left]{$F$} to [out = 90, in = 180] (1.5, 1.75);
			\draw[->] (1.5, -0.25) -- node[right, pos = 0.5]{$p_Y$} (1.5, -0.75);
			\draw[->] (-0.25, 0.5) -- node[below, pos = 0.5]{$p_X$}(-0.75,0.5);
			\filldraw (1.5, 1) circle(1pt) node[right]{$x$};
		\end{tikzpicture}
	\end{center}
	\textit{Quello che faremo qui sostanzialmente è passare per il complementare di $F$, prendere un aperto lì dentro che contenga $x$, farne le proiezioni per poi riportarlo in $X \times Y$}\\
	Sia quindi $F$ chiuso in $X \times Y$ e sia $(x, y) \not \in F$. Sapendo che $(X \times Y)\setminus F$ è aperto, esistono $V_x$ e $V_y$ aperti in $X$ e $Y$ rispettivamente tali che $x \in V_x$ e $y \in V_y$ tali che:
	\[ (V_x\times V_y) \subseteq (X \times Y \setminus F) \]
	Consideriamo i chiusi $X \setminus V_x$ e $Y \setminus V_y$. Allora abbiamo che:
	\[ x \not \in X \setminus V_x\qquad y \not \in Y\setminus V_y \]
	Sappi:amo tuttavia, per ipotesi, che $X$ e $Y$ sono $T3$, abbiamo che $\exists V_x', W_x$ aperti di $X$ e $V'_y$ e $W_y$ aperti di $Y$ tali che:
	\[ x \in V'_x\qquad y \in V'_y \qquad X \setminus V_x \subseteq W_x\qquad Y \setminus V_y \subseteq W_y\]
	In questo modo abbiamo che $(x,y) \in V'_x \times V'_y$, che è un aperto in $X \times Y$ e abbiamo che $F \subseteq p_X^{-1}(W_x) \cup p^{-1}_Y(W_y)$, che è aperto in $X \times Y$.
	Infatti abbiamo che $(X\setminus V_x) \times Y \subseteq p_X^{-1}(W_x)$ e $X\times (Y \setminus V_y)\subseteq p_Y^{-1}(W_y)$. Quindi abbiamo che:
	\[ (X \times Y) \setminus (V_x \times V_y) \subseteq p_X^{-1}(W_x) \cup p_Y^{-1}(W_y) \]
	Infine abbiamo che:
	\[ (V_x' \times V_y') \cap (p^{-1}_X(W_x) \cup p_Y^{-1}(W_y)) = \varnothing\]
	Quindi $X \times Y$ è $T3$

	\textit{In parole povere quello che abbiamo fatto è stato: nel complementare di $F$ abbiamo preso un aperto che contenesse $(x,y)$; abbiamo fatto le proiezioni su $X$ e $Y$. Ma questi sono $T3$ e la proiezione di un chiuso è un chiuso. Quindi esistono due aperti che li separino. Per il punto prendiamo un aperto dentro le rispettive proiezioni, per poi farne il prodotto cartesiano in $X \times Y$, mentre per il chiuso facciamo l'unione delle intersezioni, in questo modo lasciamo esterno l'aperto che conteneva il punto}\\
\end{proof}

Prima di enunciare il prossimo risultato, diamo la seguente definizione:

\begin{defn}{Diagonale}{}
	Sia $(X, \mathcal T)$ uno spazio topologico. Si definisce \textbf{Diagonale} di $X$ l'insieme:
	\[ \Delta_X = \{(x,x) \in X \times X : x \in X\} \]
\end{defn}

\begin{thm}{Caratterizzazione di $T2$}{}
	Uno spazio topologico $(X, \mathcal T_X)$ è $T2$ se e solo se la sua diagonale $\Delta_X$ è chiuso in $X \times X$, con $X \times X$ prodotto con topologia indotta
\end{thm}

\begin{proof}
	\fbox{$\Rightarrow$} Sia $X$ uno spazio topologico $T2$ e sia $(x,y) \in X \times X$. Voglio mostrare che se:
	\[ x \neq y \quad \Rightarrow \quad (x,y)\not \in \Delta_X \]
	Siccome $X$ è $T2$, esistono allora degli aperti $U_x, V_y$ in $X$ tali che:
	\[ x \in U_x \qquad y \in V_y \qquad U_x \cap V_y = \varnothing \]
	Allora necessariamente si ha che:
	\[ (U_x \times V_y) \cap \Delta_X = \varnothing \]
	Infatti, se così non fosse, esisterebbe un elemento nell'intersezione, quindi ci sarebbe una coppia $(z,z) \in U_x \times V_y$, quindi $z \in U_x \cap V_y$, ma ciò è assurdo in quanto avevamo supposto che $U_x$ e $V_y$ erano disgiunti. Sappiamo poi che:
	\[ (X \times Y)\setminus(\Delta_X) = \bigcup_{x \neq y}U_x \times V_y \]
	Quest'ultimo insieme è aperto (in quanto unione di aperti), quindi necessariamente $\Delta_X$ è chiuso.

	\fbox{$\Leftarrow$} Sia $\Delta_X$ chiusa e siano $x \neq y \in X$. Allora segue subito che:
	\[ (x,y) \in (X \times X) \setminus \Delta_X \]
	Allora esistono degli aperti $U$ e $V$ in $X$ tali che:
	\[ U \times V \subseteq (X \times X)\setminus \Delta_X : (x,y) \in U \times V \]
	\textit{Sto usando la base canonica}. Ma allora ho che $U \cap V = \varnothing$, con $x \in U$ e $y\in V$. Quindi $X$ è $T2$
\end{proof}

\begin{cor}{}{}
	Siano $f:X \to Y$ e $g:X \to Y$ funzioni continue con $Y$ $T2$, allora:
	\[ \{x \in X : f(x) = g(x)\}\text{ è chiuso} \]
\end{cor}

\begin{proof}
	Sappiamo che $\Delta_Y$ è chiusa in $Y \times Y$. Consideriamo la funzione:
	\[ X \xrightarrow{(f,g)} Y \times Y\qquad x \mapsto(f(x), g(x)) \]
	Sappiamo che questa funzione è continua per la proprietà universale del prodotto. Quindi vale:
	\begin{center}
		\begin{tikzcd}
			& X \arrow[d, "{(f,g)}"] \arrow[ddl, bend right, "f"'] \arrow[ddr, bend left, "g"]\\
			& Y \times Y \arrow[dr, "p_1"'] \arrow[dl, "p_2"]\\
			Y && Y
		\end{tikzcd}
	\end{center}
	Quindi $\{x \in X : f(x)=g(x)\} = (f,g)^{-1}(\Delta_Y)$, che è chiuso
\end{proof}

Prima di enunciare il prossimo corollario, diamo la seguente definizione:

\begin{defn}{Grafico di una funzione}{}
	Sia $f:X \to Y$ una funzione. Si definisce il \textbf{Grafico di }$f$ l'insieme:
	\[ \Gamma_f = \{(x, f(x)) : x \in X\} \]
\end{defn}

\begin{cor}{}{}
	Sia $f:X \to Y$ continua con $Y$ $T2$, allora il suo grafico $\Gamma_f$ è chiuso in $X \times Y$
\end{cor}

\begin{proof}
	Consideriamo la funzione:
	\[ X\times Y \xrightarrow{(f, id_Y)} Y\times Y \]
	Questa è ancora continua per lo stesso motivo del corollario precedente. Abbiamo allora che:
	\[ (f, id_Y)^{-1}(\Delta_Y) = \{ (x,y) \in X \times Y: f(x) = y \} = \Gamma_f \]
	Quindi è chiuso
\end{proof}

\begin{es}
	$\mathbb R/_\mathbb Q$ non è $T1$, infatti $[\mathbb Q]$ non è chiuso
\end{es}

\begin{es}
	La retta con doppia origine non è $T2$, in quanto le sue origini non sono separabili. Infatti, se prendiamo le funzioni $f$ e $g$ definite come $f(x) = (x,1)$ e $g(x) = (x, -1)$ e poi le collassiamo, non è possibile separare le due origini.
\end{es}

\begin{prop}{}{}
	Sia $X$ spazio topologico e sia $G \curvearrowright X$ con quoziente $\pi: X \to X/_G$. Allora $X/_G$ è $T2$ se e solo se l'insieme
	\[ K = \{(x, g(x)) : x \in X, g \in G\} \text{ è chiuso} \]
\end{prop}

\begin{es}
	Consideriamo $\mathbb Z \curvearrowright \mathbb R$, azione di gruppo tale che:
	\[ (x, n) \mapsto x+n \]
	Allora abbiamo che:
	\[ K = \{ (x, x+n) : x \in \mathbb R, n \in \mathbb Z \} = \bigcup_{n \in \mathbb Z} \{ y = x+n \text{ retta} \}\]
	Questo è chiuso in quanto complementare dell'aperto $\mathbb R \times (\mathbb R \setminus \mathbb Z)$, quindi $\mathbb R/_\mathbb Z$ è chiuso
\end{es}

\begin{proof}
	\fbox{$\Rightarrow$} Sappiamo che $X/_G$ è $T2$, cioè $\Delta_{X/_G}$ è chiuso in $X/_G \times X/_G$
	Consideriamo la funzione:
	\[ X \times X \xrightarrow{(\pi,\pi)} X/_G \times X/_G \qquad (x,y) \mapsto (\pi(x), \pi(y)) \]
	Questa è continua e aperta in quanto prodotto di funzioni continue e aperte. Sottolineiamo che è continua, questo significa che sulla base canonica ho:
	\[ (\pi, \pi)(U \times V) = (\pi(U), \pi(V)) \]
	Che è aperto (per l'esercizio $2.15$). Quindi, visto che $(\pi,\pi)$ è chiusa e $\Delta_{X/_G}$ è chiusa, abbiamo che:
	\[ K = (\pi, \pi)^{-1}(\Delta_{X/_G}) \text{ è chiuso}\]

	\fbox{$\Leftarrow$} Sappiamo che $K$ è chiuso, allora $(X \times X)\setminus K$ è aperto e siccome $(\pi,\pi)$ è aperta, allora si ha che:
	\[ (\pi, \pi)((X\times X) \setminus K) = (X/_G \times X/_G)\setminus \Delta_{X/_G}\]
	Quindi la diagonale è chiusa, quindi $X/_G$ è $T2$
\end{proof}

\begin{cor}{}{}
	Sia $(X, \mathcal T_X)$ $T2$ e sia $G \curvearrowright X$ tale che $|G|< + \infty$, allora $X/_G$ è $T2$
\end{cor}

\begin{proof}
	Sappiamo che:
	\[ K = \bigcup_{g \in G}\Gamma_g \]
	Però l'unione è finita e $\Gamma_g$ è chiuso, quindi $K$ è chiuso, quindi $X/_G$ è $T2$
\end{proof}

\begin{prop}{}{}\label{4.5.14}
	Sia $(X, \mathcal T_X)$ $T2$ e sia $G \curvearrowright X$ e sia $A \subseteq X$ aperto tale che:
	\begin{enumerate}
		\item $\pi(A) = X/_G$
		\item $G(A)=\{g \in G: A \cap g(A) \neq \varnothing\}$ è finito
	\end{enumerate}
	Allora il quoziente è $T2$
\end{prop}

\begin{es}
	Sia $\mathbb Z \curvearrowright \mathbb R$. Consideriamo $A = (-\frac 12, 1)$. Sicuro abbiamo che $\pi(A) = \mathbb R/_\mathbb Z$. Inoltre abbiamo che:
	\[ \{g \in G: A \cap g(A) \neq \varnothing\} = \{0,1,-1\} \]
	Allora abbiamo che $\mathbb R/_\mathbb Z$ è $T2$
\end{es}

\begin{cor}{}{}
	Sia $(X, \mathcal T_X)$ $T2$ e sia $G \curvearrowright X$ tale che $|G|< + \infty$, allora $X/_G$ è $T2$
\end{cor}

\begin{proof}
	Mi basta prendere $A = X$ e applico la proposizione
\end{proof}

\begin{es}
	Consideriamo $\mathbb Z^2 \curvearrowright \mathbb R^2$ tale che:
	\[ g: \mathbb Z^2 \times \mathbb R^2 \to \mathbb R^2 \qquad (n_1,n_2) \times(x_1,x_2) \mapsto (x_1+n_1, x_2+n_2) \]
	Voglio mostrare che $\mathbb R^2/_{\mathbb Z^2}$ sia $T2$.\\
	\textit{Sostanzialmente quello che l'azione di gruppo fa è creare un reticolato per ogni punto}.
	\begin{center}
		\begin{tikzcd}
			&&& \bullet \arrow[r, bend right] & \bullet\\
			&& \bullet \arrow[r, bend right] & \bullet \arrow[u, bend right] \arrow[r, bend right] & \bullet \arrow[u, bend right]\\
			\bullet \arrow[r, bend right] & \bullet \arrow[r, bend right] & \bullet \arrow[u, bend right] \arrow[r, bend right] & \bullet \arrow[u, bend right] \arrow[r, bend right]& \bullet \arrow[u, bend right] \arrow[r, bend right] & \bullet \arrow[r, bend right] & \bullet\\
			&&& \bullet \arrow[u, bend right]
		\end{tikzcd}
	\end{center}
	Abbiamo che $[0,1)^2$ è un insieme di rappresentanti, ma ci serve un insieme aperto per applicare la proposizione \ref{4.5.14}.14, quindi possiamo prendere $(-\frac 12, 1)^2$. In questo modo abbiamo che:
	\[\{ g \in G: A \cap g(A) \neq \varnothing \} = \{ (i,j) : i,j \in \{0,1,-1\} \}\]
	Quindi è $\mathbb R^2/_{\mathbb Z^2}$ è $T2$
\end{es}

\begin{es}
	Sia $S^n \subseteq \mathbb R^n$ e consideriamo l'azione $\mathbb Z/_{2 \mathbb Z} \curvearrowright S^n$ tramite:
	\[ (1,x) \mapsto x \quad (id) \qquad (-1,x) \mapsto -x \quad(-id)\]
	Allora $S^n/_{\mathbb Z_{/_{2\mathbb Z}}}$ è $T2$
\end{es}

Dimostriamo la proposizione che avevamo lasciato in sospeso.

\begin{proof}[Dimostrazione della Proposizione 4.5.14]
	Sia $G(A) = \{g_1,...,g_n\}$, possiamo numerarli in quanto sono finiti.\\
	Siano $\xi_1, \xi_2 \in X/_G$ diversi tra loro, allora esistono $p,q \in A$:
	\[ [p]=\xi_1\qquad [q]=\xi_2 \]
	Sappiamo che $X$ è $T2$, allora $\forall g_i \in G(A), \exists U_i, V_i$ aperto di $X$ disgiunti e $p \in U_i, g_i(q) \in V_i$. Siano
	\[ U = \bigcap_{i=1}^n U_i \cap A\qquad V = \bigcap_{i = 1}^n V_i \cap A\]
	Questi sono entrambi aperti di $X$, il primo è un intorno di $p$, il secondo è un intorno di $q$.\\
	Sappiamo che i due insiemi sono disgiunti in quanto, abbiamo che tra gli elementi di $G(A)$ c'è l'identità $id_G$. Per comodità possiamo supporre che sia $g_1$, allora abbiamo che $g^{-1}(V_1)=V_1$ e abbiamo che $U_1 \cap V = \varnothing$, sapendo poi che $U \subseteq U_1$ e $V \subseteq V_1$, si ottiene che $U \cap V = \varnothing$\\
	Per arrivare al fatto che valga per ogni punto, iniziamo con il dimostrare che $\forall g \in G, U \cap g(V) = \varnothing$\\
	Siccome $U,V \subseteq A$, $\forall g \in G \setminus G(A)$ abbiamo che $U \cap g(V)= \varnothing$. Sia ora $g = g_i \in G(A)$, allora abbiamo che $g_i(V) \subseteq (g_i\circ g_i^{-1})(V) = V_i$. Sappiamo però che:
	\[ U \subseteq U_i \quad \text e \quad U_i \cap V_i = \varnothing \quad \Rightarrow \quad g(V)\cap U = \varnothing \]
	Cioè $U_i$ è disgiunto dalla saturazione di $V$.\\
	Poi sappiamo che:
	\[ (\pi^{-1}\circ \pi)(U) = \bigcup_{g \in G}g(U)\qquad \text e \qquad (\pi^{-1}\circ \pi)(V) = \bigcup_{g \in G}g(V) \]
	Dobbiamo mostrare che questi due insiemi sono disgiunti.\\
	Questo è equivalente al dover mostrare che $\forall g', g'' \in G$
	\[ g'(U)\cap g''(V) = \varnothing \]
	Ma questo è equivalente a:
	\begin{align*}
		g'(U) \cap g''(V) &= g'(U \cap (g')^{-1}(g''(V))) = g'(U \cap \overline g(V)) = g'(\varnothing) = \varnothing
	\end{align*}
	Sappiamo che $U \cap \overline g(V) = \varnothing$ per quanto fatto in precedenza
\end{proof}

Graficamente quello che abbiamo fatto in questa dimostrazione è stato il seguente:

\begin{center}
	\textit{Con il caso $G(A) = \{g_1 = id_G, g_2 g_3\}$}\\
	\begin{tikzpicture}
		\draw (5,0) arc(0:360:5cm and 3cm) node[right]{$A$};
		\filldraw (-3,1) circle (1pt) node[above]{$p$} (2,1) circle(1pt) node[right]{$q$} (0,-2) circle(1pt) node[below]{$g_2(q)$} (4, -3) circle(1pt) node[below]{$g_3(q)$};
		\draw[green] (-3,4) arc(90:450: 1cm and 2cm)node[below]{$U_1$} (2,0) arc(-90:270:1cm) node[below]{$V_1$};
		\draw[blue] (-2,0) arc(-90:270:2 cm and 1cm) node[below]{$U_2$} (1, -2) arc(0:360:1) node[right]{$V_2$};
		\draw[red] (-5.5, 1) arc(-180:180:1.5cm) node[left]{$U_3$} (4,-2.5) arc(90:450:0.5cm) node[above]{$V_3$};
		\begin{scope}
			\clip (-3,4) arc(90:450: 1cm and 2cm);
			\clip(-2,0) arc(-90:270:2 cm and 1cm);
			\fill[red, very nearly transparent] (-5.5, 1) arc(-180:180:1.5cm);
		\end{scope}
		\fill[green, very nearly transparent] (2,1) circle(0.5cm) (0,-2) circle(0.5cm) (4,-3) circle(0.5cm);
	\end{tikzpicture}\\
	\textit{In rosso $U$ mentre in verde $g_1(V), g_2(V), g_3(V)$}
\end{center}

\subsection{Conseguenze sulla Compattezza}

\begin{prop}{}{}
	Sia $f:X \to Y$ continua con $X$ compatto, allora $f(X)$ è compatto con la topologia indotta da $Y$
\end{prop}

\begin{proof}
	Sia $\mathcal A = \{U_i\}_{i \in I}$ un ricoprimento aperto di $f(X)$. Per definizione di topologia indotta:
	\[ \forall i, \exists V_i\text{ aperto di }Y: U_i = V_i \cap f(X) \]
	Chiamiamo $\mathcal B = \{f(V_i)\}_{i \in I} = \{f(U_i)\}_{i \in I}$. Questo è un ricoprimento aperto di $X$, in quanto $f$ è continua, $X$ è un aperto e $U_i$ è un ricoprimento di $f(X)$.\\
	Ma abbiamo che $X$ è compatto, quindi:
	\[ \exists i_1,...,i_n: X = f^{-1}(V_{i_1}) \cup \cdots \cup f^{-1}(V_{i_n}) \]
	Quindi abbiamo che:
	\[ f(X) = f(f^{-1}(V_{i_1}))\cup \cdots \cup f(f^{-1}(V_{i_n})) = U_{i_1}\cup \cdots \cup U_{i_n}\]
	Ma questo è un sottoricoprimento di $\mathcal A$, quindi $f(X)$ è compatto.
\end{proof}

\begin{cor}{}{}
	La compattezza è una proprietà topologica (invariante per omomorfismo)
\end{cor}

\begin{cor}{}{}
	Il quoziente di un compatto è compatto
\end{cor}

\begin{prop}{}{}
	Sia $X$ compatto e sia $F \subseteq X$ chiuso, allora $F$ con la topologia indotta è compatto.
\end{prop}

\begin{proof}
	Sia $\mathcal A = \{U_i\}_{i \in I}$ ricoprimento aperto di $F$, allora:
	\[ \forall i,\exists V_i \text{ aperto di }X : U_i = V_i \cap F \]
	Ho sicuramente che:
	\[ F \subseteq \bigcup_{i \in I}V_i \qquad \Rightarrow \qquad X = \bigcup_{i \in I}V_i \cup (X \setminus F) \]
	Ma tutti questi sono insiemi aperti, quindi per compattezza di $X$:
	\[\exists i_1,...,i_n : X = V_{i_1}\cup \cdots \cup V_{i_n} \cup (X\setminus F)\]
	Da cui segue che:
	\[ F \subseteq V_{i_1}\cup \cdots \cup V_{i_n} \quad \Rightarrow \quad F \subseteq U_{i_1}\cup \cdots \cup U_{i_n} \]
	E questo è un sottoricoprimento aperto di $F$
\end{proof}

\begin{prop}{}{}
	Sia $X$ spazio topologico e sia $K_1,...,K_n\subseteq X$ compatti rispetto alla topologia indotta da $X$. Allora anche la loro unione è compatta rispetto alla topologia indotta.
\end{prop}

\begin{proof}
	Sia $\mathcal A = \{U_i\}_{i \in I}$ ricoprimento aperto di $K_1\cup \cdots \cup K_n$. Allora:
	\[ \forall j \in \{1,...,n\}, U_i \cap \mathcal A_j \text{ è un ricoprimento aperto di }K_j\]
	Quindi in particolare:
	\[ \exists i_{j_1},...,i_{j_{n_j}}: K_j = K_j \cap (U_{i_{j_{1}}} \cup \cdots \cup U_{i_{j_{n_j}}}) \]
	Quindi se prendiamo l'unione abbiamo che:
	\[ K_1\cup \cdots \cup K_n = \bigcup_j \left(U_{i_{j_1}} \cup \cdots \cup U_{i_{j_{n_j}}}\right)\]
	Ma questo è un sottoricoprimento finito, quindi:
	\[ K_1 \cup \cdots \cup K_n \text{ è compatto} \]
\end{proof}

\begin{prop}{}{}
	Sia $X$ $T2$ e $K$ compatto con topologia indotta, allora $K$ è chiuso
\end{prop}

\begin{proof}
	Mostriamo che $X \setminus K$ è aperto. Sia quindi $y \in X \setminus K$ e $x \in K$. Sappiamo tuttavia che $X$ è $T2$, quindi:
	\[ \exists U_{y,x}, V_{y,x}\text{ aperti disgiunti}: y \in U_{y,x}, x \in V_{y,x} \]
	Abbiamo quindi che al variare di $x \in K$ otteniamo che:
	\[ K \subseteq \bigcup_{x \in K}V_{y, x} \]
	Equivalentemente abbiamo che $\{ V_{y,x} \cap K \}_{x \in K}$ è un ricoprimento aperto di $K$. Sappiamo tuttavia che $K$ è compatto, quindi:
	\[\exists x_1,...,x_n \in K: K \subseteq V_{y,x_1}\cup \cdots \cup V_{y,x_n}\]
	In particolare abbiamo quindi che l'insieme $U_y$ intorno di $y$ è:
	\[ U_y = U_{y,x_1} \cap \cdots \cap U_{y,x_n} \subseteq X \setminus K \]
	Ma questo è un numero finito di aperti, quindi:
	\[ X \setminus K = \bigcup_{y \in X \setminus K}U_y \text{ aperti}\Rightarrow X \setminus K \text{ aperto } \Rightarrow K \text{ chiuso} \]
\end{proof}

\begin{thm}{Compatto Housdorff}{}
	Sia $X$ compatto, $Y$ $T2$ e $f:X \to Y$ una funzione continua. Allora $f$ è chiusa.
\end{thm}

\begin{proof}
	Sia $F \subseteq X$ chiuso, allora essendo $X$ compatto si ha che anche $F$ è compatto. Inoltre, essendo $f$ continua, $f(F)$ è compatto e, essendo $Y$ $T2$, $f(F)$ è chiuso.
\end{proof}

\begin{thm}{}{}
	Lo spazio topologico $([0,1], \mathcal T_\mathcal E)$ è compatto.
\end{thm}

\begin{proof}
	Sia $\mathcal A = \{U_i\}_{i \in I}$ ricoprimento di $[0,1]$ e sia $\mathcal B = \{V_i\}_{i \in I}$ la famiglia con $\forall i, V_i = [0,1] \cap U_i$ e $V_i$ aperti di $\mathbb R^+ \cup \{0\}$. Sia ora $X\subseteq\mathbb R^+ \cup \{0\}$ definito come:
	\[ X =\{ t \in \mathbb R^+ : [0, t]\text{ sia ricoperto da un numero finito di elementi di }\mathcal B \} \]
	Abbiamo che $0 \in X$, infatti $\mathcal A$ è un ricoprimento di $[0,1]$, quindi $\exists i_0$ con $0 \in U_{i_0}$, quindi $0 \in V_{i_0}$. Inoltre è aperto in $\mathbb R^+ \cup \{0\}$ in quanto $\exists \vareps$ con $[0,\vareps)\subseteq V_{i_0}$, quindi $\vareps/2 \in X$, da cui segue che $\sup X>0$. Poniamo $b = \sup X$. Ora ci possono essere due casi, se $b >1$ oppure se $b\leq 1$.\\
	Se $b>1$ abbiamo sostanzialmente finito, in quanto riusciamo ad ottenere un sottoricoprimento d finito da $\mathcal A$, quindi $[0,1]$ compatto.\\
	Mostriamo che $b>1$ è l'unica opzione. Supponiamo per assurdo che $b \leq 1$. In questo caso:
	\[\exists U_{i,b} : b \in U_{i,b}\]
	Ma abbiamo che $U_{i,b} = [0,1] \cap V_{i,b}$. Sia $U_{i,b}$ sia $V_{i,b}$ sono aperti di $\mathbb R^+ \cup \{0\}$, quindi $\exists \vareps >0$ tale che $(b-\vareps, b + \vareps) \subseteq V_{i,b}$. Ma avevamo che $b = \sup X$, allora $\forall t$, l'intervallo $[0, b-t]$ è ricoperto da un numero finito di elementi di $\mathcal B$. Siano quindi $V_{i,1_t},...,V_{i,n_t}$ tali elementi.
	Se abbiamo che $t<\vareps$ allora ho che:
	\[ V_{i,1_t} \cup \cdots \cup V_{i,n_t} \cup V_{i,b} \text{ ricopre almeno }\left[0, b + \frac \vareps 2 \right]\]
	Ma questo è assurdo, in quanto avremmo che $b + \frac\vareps 2 \in X$, quindi non sarebbe rispettata la condizione di $b = \sup X$. Quindi necessariamente $b>1$
\end{proof}

\begin{prop}{}{}
	Sia $K\subseteq(\mathbb R, \mathcal T_\mathcal E)$. Questo compatto rispetto alla topologia indotta se e solo se $K$ è chiuso in $\mathbb R$ ed è limitato
\end{prop}

\begin{proof}
	\fbox{$\Leftarrow$} Sia $K$ limitato, allora esiste $N>0$ tale che:
	\[ K \subseteq (-N, N) \subseteq [-N, N] \]
	Sappiamo però che $[-N, N] \cong [0,1]$, quindi sappiamo che $[-N, N]$ è compatto. Sapendo anche che è chiuso abbiamo che:
	\[ K = K \cap [-N, N]\text{ è ancora chiuso}\qquad \Rightarrow \qquad K \text{ compatto} \]

	\fbox{$\Rightarrow$} Sia $K$ compatto. Sappiamo che $\mathbb R$ è $T2$, allora $K$ è compatto. Sia $\mathcal A = \{(-N, N) \cap K\}_{N \in \mathbb N}$. Questo è un ricoprimento aperto di $K$, quindi:
	\[\exists N_1,...,N_n \in \mathbb N : K \subseteq \bigcup_{i = 1}^n(-N_i, N_i)\]
	Poniamo $\overline N = \max N_i$, allora $K \subseteq (-\overline N, \overline N)$, allora $K$ è limitato.
\end{proof}

\begin{thm}{di Wallace o Lemma del Tubo}{}\label{Wallace}
	Siano $X, Y$ spazi topologici e siano $A \subseteq X, B \subseteq Y$ compatti e $W \subseteq X \times Y$ aperto tale che $A \times B \subseteq W$, allora:
	\[ \exists U \text{ aperto di }X, Y \text{ aperto di }Y : U \times V \subseteq W, A \subseteq U, B \subseteq V \]
\end{thm}

\begin{cor}{}{}
	Sia $X$ spazio topologico compatto e $T2$, allora è $T4$
\end{cor}

\begin{proof}
	Prendiamo $F, G$ chiusi tali che $F\cap G = \varnothing$, allora:
	\[ F \times G \subseteq (X \times X)\setminus \Delta_X \]
	Sappiamo però che $X$ è $T2$, quindi la diagonale è chiusa, quindi $(X \times X)\setminus \Delta_X$ è aperto. Sappiamo poi che $F, G$ sono chiusi in un compatto, quindi sono compatti loro stessi. Per il teorema \ref{Wallace}.10 di Wallace abbiamo che $\exists U,V$ aperti di $X$ tali che:
	\[ F \subseteq U, G \subseteq V : F \times G \subseteq (U \times V)\subseteq (X \times X )\setminus \Delta_X \]
	Ma abbiamo che $U$ e $V$ sono disgiunti, quindi $X$ è $T4$
\end{proof}

\begin{cor}{}{}
	Sia $X$ compatto e sia $Y$ spazio topologico qualunque, allora $\pi_Y: X \times Y \to Y$ è chiusa
\end{cor}

\begin{proof}
	Sia $C \subseteq X \times Y$ chiuso. Se $\pi_Y(C)=Y$ allora non c'è niente da dimostrare\\
	Supponiamo quindi $\exists y \not \in \pi_Y(C)$, consideriamo allora $X \times \{y\} \subseteq (X \times Y)\setminus C$. Per ipotesi abbiamo che $X$ è compatto e che $(X \times Y)\setminus C$ è aperto, in quanto $C$ è chiuso. Inoltre abbiamo che per la topologia indotta su $\{y\}$, in cui la topologia grossolana e quella discreta coincidono, $\{y\}$ è compatto. Quindi valgono le ipotesi del teorema \ref{Wallace}.10 di Wallace, quindi esistono due aperti $U, V_p$ aperti con $X \subseteq U, \{y\}\subseteq V_p$ tali che:
	\[ X \times \{y\} \subseteq U \times V_p \subseteq (X \times Y)\setminus C \]
	Per ragioni piuttosto evidenti si ha che $U=X$. Abbiamo inoltre che $V_p \cap \pi_Y(C) = \varnothing$ per la condizione precedente, allora abbiamo che:
	\[ Y\setminus \pi_Y(C) = \bigcup_{p \in Y \setminus \pi_Y(C)} V_p \]
	Essendo tutti questi aperti, abbiamo che $Y \setminus \pi_Y(X)$ è aperto, quindi $\pi(C)$ è chiuso, quindi $\pi$ è chiusa
\end{proof}

Enunciamo un lemma estremamente utile, ma che dimostreremo solo in seguito.

\begin{lemma}{}{}
	Sia $f:X \to Y$ funzione continua e chiusa con $X$ e $Y$ spazi topologici e con $Y$ compatto tale che $\forall y \in Y$, $f^{-1}(y)$ è compatto, allora $X$ è compatto
\end{lemma}

\textit{In maniera del tutto euristica, possiamo considerare $X$ come un salame e $Y$ rappresenta la lunghezza del salame. Se le fette sono compatte, allora era compatto inizialmente il salame}

\begin{prop}{}{}
	Siano $X,Y$ spazi topologici compatti, allora $X \times Y$ è compatto.
\end{prop}

\begin{proof}{}{}
	Consideriamo $\pi_Y: X \times Y \to Y$. Per il corollario precedente, abbiamo che $\pi_Y$ è chiusa. Essendo una proiezione, abbiamo banalmente che è continua. Per ipotesi $Y$ è compatto. Consideriamo poi $y \in Y$, allora:
	\[ \pi^{-1}(Y) = X \times \{y\} \cong X \]
	Ma $X$ è compatto, quindi anche $\pi_Y^{-1}(y)$. Ma allora per il lemma $X \times Y$ è compatto
\end{proof}

\begin{oss}
	Questo vale per il prodotto di finiti compatti.
\end{oss}

\begin{proof}[Dimostrazione del Teorema \ref{Wallace}.10 di Wallace]
	Cominciamo dal caso in cui $A = \{a\}$.\\
	Abbiamo che $\{a\} \times B \subseteq W$ aperto, allora $\forall b \in B, \exists U_b, V_b$ aperti di $X$ e di $Y$ rispettivamente tali che:
	\[ a \in U_b \qquad b \in V_b\qquad U_b \times V_b \subseteq W \]
	Ottengo che $\{V_b \cap B\}_{b \in B}$ è un ricoprimento aperto di $B$, ma $B$ è compatto per ipotesi, quindi esistono $b_1,...,b_n \in B$ tali che:
	\[ V_{b_1}\cup\cdots \cup V_{b_n} \supset B \]
	Poniamo:
	\[ U = \bigcap_{i = 1}^n V_{b_i} \ni a\qquad V = V_{b_1}\cup \cdots \cup V_{b_n} \supset B \]
	Abbiamo che $U$ e $V$ sono aperti e $\{a\}\times B\subseteq U \times V$. Non ci resta da mostrare che $U \times V \subseteq W$. Ma avevamo che:
	\[ U \times V \subseteq \bigcup_{i = 1}^n (U_{b_i}\times V_{b_i}) \subseteq W \]
	Infatti se $x \in \bigcap U_i \Rightarrow x\in U_{b_i}, \forall i$, da cui, se $(x,y) \in U\times V$ esiste $i$ tale che $y \in V_{b_i}$, da cui $(x,y) \in U_{b_i} \times V_{b_i}$\\
	\textit{Graficamente abbiamo che}:
	\begin{center}
		\begin{tikzpicture}
			\draw[thick] (-2,0) -- (2,0) node[right]{$\{a\}\times B$};
			\draw (-2.5,-0.5) rectangle ++ (1,0.75) (-1.75, -1) rectangle ++ (0.75, 1.5) (-1, -0.40) rectangle (1.5, 1) (0.5, -0.75) rectangle ++ (1.5, 1.5);
			\draw[blue, thick] (-2.5, -0.4) -- (2,-0.4) -- (2, 0.25) -- (-2.5, 0.25) node[above]{$U \times V$} -- (-2.5,-0.4);
			\draw (0,0) circle (4cm and 1.5cm);
			\draw (3.5,1.2) node{$W$};
		\end{tikzpicture}
	\end{center}

	Sia ora $A$ qualunque. Sfruttando il passo precedente abbiamo che:
	\[ \forall a \in A, \exists U_a, V_a \text{ aperti }: (a,b) \in U_a \times V_a \subseteq W\]
	Allora abbiamo che $\{ U_a \cap A \}_{a \in A}$ è un ricoprimento aperto di $A$, ma $A$ è compatto, quindi $\exists a_1,...,a_m$ tali che:
	\[ U_{a_1} \cup \cdots \cup  U_{a_m} \supset A\]
	Definiamo poi:
	\[ U = U_{a_1} \cup \cdots \cup A_{a_m} \supset A \qquad V = \bigcap_{i = 1}^m V_{a_i} \supset B \]
	Ho quindi che:
	\[ A \times B \subseteq U \times V \subseteq \bigcup_{i = 1}^m (U_{a_i}\times V_{a_i})\subseteq W\]
	Inoltre abbiamo che $U \times V$ è aperto perché unione di aperti
\end{proof}
\textit{Abbiamo sostanzialmente ripetuto quello fatto nella prima parte per ogni punto di $A$ e poi abbiamo sostanzialmente "unito i rettangoli orizzontali"}

\begin{thm}{}{}
	Sia $X$ compatto e $T2$ e sia $\pi: X \to X/_\sim$. Allora si equivalgono:
	\begin{enumerate}
		\item $X/_\sim$ è $T2$
		\item $\pi$ è chiusa
		\item $K = \{(x_1,x_2) : \pi(x_1) = \pi(x_2)\}$ è chiuso
	\end{enumerate}
\end{thm}

\begin{es}
	Sia $X$ compatto e $T2$ e sia $F \subseteq X$ chiuso, allora $X/_F$ è $T2$ (sfruttando il fatto che $\pi$ è chiusa)
\end{es}

\begin{es}
	Riprendendo l'esempio \ref{sndn}, con questo teorema possiamo dire che:
	\[ D^n/_{S^{n-1}}\text{ è }T2 \]
\end{es}

\begin{proof}
	\fbox{$1 \Rightarrow 3$} Ho che $\Delta_{X/_\sim}\subseteq X/_\sim \times X/_\sim$ chiusa e che $K = (\pi,\pi)^{-1}(\Delta_{X/_\sim})$ chiuso.

	\fbox{$3 \Rightarrow 2$} $\pi$ è chiusa se e solo se $\forall F$ chiuso ho $(\pi^{-1} \circ \pi)^{-1}(\Delta_{X/_\sim})$ è chiuso. Ma abbiamo che:
	\[ \pi^{-1}(\pi(F)) = p_1(K \cap p_2^{-1}(F)) \]
	Infatti, rappresentando con un grafo abbiamo che:
	\begin{center}
		\begin{tikzcd}
			& X \times X \arrow[dl, "p_1"'] \arrow[dr, "p_2"]\\
			X && X
		\end{tikzcd}
	\end{center}
	Abbiamo quindi che $p_2^{-1}(F) = X \times F$ che è chiuso in quanto $p_2$ è continua e $F$ è chiuso. Abbiamo poi che:
	\[ K \cap p_2^{-1}(F) = \{(x,y) : y \in F, x \sim y\} \]
	Ma $K$ è chiuso, quindi anche $K \cap p_2^{-1}(F)$ è chiuso. Abbiamo poi che:
	\[ p_1(K \cap p_2^{-1}(F)) = \{ \in X: \exists y \in F: x \sim y\} = \pi^{-1}(\pi(F)) \]
	Ma abbiamo anche che $p_1$ è chiusa e $X$ è compatto, quindi $p_1(K \cap p_2^{-1}(F))$ è chiuso, quindi $\pi$ è chiusa.

	\fbox{$2 \Rightarrow 1$} Siano $[a]\neq [b] \in X/_\sim$ e siano:
	\[ A = \pi^{-1}([a]) \qquad \text e \qquad B = \pi^{-1}([b]) \]
	Questi sono due chiusi in quanto $A= \pi^{-1} = \pi^{-1}(\pi(\{a\}))$ e $B = \pi^{-1}(\pi(\{b\}))$ e $\{a\}$ e $\{b\}$ sono chiusi in $X$ in quanto $X$ è $T1$ e perché la saturazione di chiusi è ancora chiusa. Ma $A$ e $B$ sono chiusi in un compatto e $\pi$ è chiusa, quindi sono compatti loro stessi e $A \cap B = \varnothing$. Allora abbiamo che:
	\[A \times B \subseteq (X \times X)\setminus \Delta_X\]
	Quest'ultimo insieme è aperto (in quanto $X$ è $T2$), quindi possiamo applicare il teorema di Wallace, per cui $\exists U,V$ aperti di $X$ tali che:
	\[ A \times B \subseteq U \times V \subseteq (X \times X)\setminus \Delta_X \]
	Allora abbiamo che $U \cap V = \varnothing$. Sappiamo inoltre $\pi(X \setminus U)$ e $\pi(X \setminus V)$ sono chiusi in $X/_\sim$, allora:
	\[ X/_\sim \setminus \pi(X \setminus U) \text{ e } X/_\sim \setminus \pi(X \setminus V) \text{ sono aperti}\]
	Inoltre questi rispettivamente contengono $[a]$ e $[b]$. Sappiamo che $X = (X\setminus U) \cup (X \setminus V)$ in quanto $U\cap V = \varnothing$. Mostriamo che questa proprietà è preservata nel quoziente. Supponiamo allora:
	\[ [c] \in (X/_\sim \setminus \pi(X\setminus U)) \cap (X/_\sim \setminus \pi(X \setminus V)) \]
	Tuttavia, per quanto fatto poco prima, abbiamo che $c \not \in X \setminus U$ e contemporaneamente $x \not \in X\setminus V$, quindi tale $c$ non può esistere. Quindi sono disgiunti, quindi $X/_\sim$ è $T2$
\end{proof}

\newpage

\section{Teoria delle Categorie (Funtori)}

\subsection{Funtori Covarianti e Controvarianti}

\begin{defn}{Funtori Covarianti}{}
	Date due categorie $\mathcal C$ e $\mathcal D$, un \textbf{Funtore Covariante} $F: \mathcal C \to \mathcal D$ è il dato di:
	\begin{enumerate}
		\item Una funzione $F: \bm{Ob}(\mathcal C) \to \bm{Ob}(\mathcal D)$
		\item $\forall A, B \in \bm{Ob}(\mathcal C)$ è definita una funzione:
			\[ F: \bm{Mor}_\mathcal C(A, B) \to \bm{Mor}_\mathcal D(F(A), F(B)) \]
			Tale che
			\begin{enumerate}
				\item $\forall A \in \bm{Ob}(\mathcal C)$, si ha $F(id_A) = id_{F(A)}$
				\item $\forall A, B, C \in \bm{Ob}(\mathcal C)$, $\forall f \in \bm{Mor}_\mathcal C(A, B), \forall g \in \bm{Mor}_\mathcal D(B, C)$:
					\[ F(g \circ f) = F(g) \circ F(f) \]
			\end{enumerate}
	\end{enumerate}
\end{defn}

\begin{es}[Funtori Dimenticanti]
	Data una categoria concreta, il funtore dimenticante dimentica una parte della struttura. Per Esempio:
	\begin{align*}
		\bm{Top} & \to \bm{Set}\\
		(X, \mathcal T) & \mapsto X\\
		f:(X, \mathcal T) \to (Y, \mathcal S) &\mapsto f:X \to Y
	\end{align*}
\end{es}

\begin{es}
	Altri esempi di Funtori dimenticanti sono:
	\begin{align*}
		\bm{Top} & \to \bm{Set}\\
		\bm{Field} & \to \bm{Ring}\\
		\bm{Vect}_{\mathbb K} &\to \bm{Grp}
	\end{align*}
\end{es}

\begin{es}
	Se $\mathbb K \subseteq \mathbb K$ è un'estensione di campi, allora anche la seguente funzione è un funtore dimenticante:
	\[ \bm{Vect}_\mathbb K \to \bm{Vect}_\mathbb F \]
\end{es}

\begin{es}
	Siano dati i gradi delle categorie $\mathcal C$ e $\mathcal D$. In base alla struttura possiamo definire il funtore (con le frecce blu) in modo che
	\begin{center}
		\begin{tikzcd}[row sep = tiny]
			& C \ar[loop above] \ar[dl, "f_1"'] \ar[dr, "f_2"] \arrow[dddd, green, bend right = 30, crossing over]\\
			A \ar[loop left] \arrow[green]{dd} && B \ar[loop right] \arrow[green]{dd}\\
			& D \ar[loop below] \ar[ul, "g_1"] \ar[ur, "g_2"'] \arrow[dd, green, bend left = 30]\\
			A' \ar[loop left] && B' \ar[loop right]\\
			& C' \ar[loop below] \ar[ur, "h2"'] \ar[ul, "h1"]\\
		\end{tikzcd}
	\end{center}
	Mettendolo per scritto, abbiamo che $F(A) = A'$, $F(B) = B'$, $F(C) = F(D) = C'$ e poi i morfismi (non segnati) $F(f_1) = F(g_1) = h_1$ e $F(f_2) = F(g_2) = h_2$.\\
	Ovviamente questa non era l'unica scelta possibile, potevamo per esempio mandare tutto in un unico oggetto, in questo modo tutti i morfismi sarebbero andati in $id_{F(A)}$
\end{es}

\begin{es}
	Siano $\mathbb K$ campo e $W$ spazio vettoriale. Definiamo una funzione $\bm{Hom}(W, \cdot)$ tale che:
	\begin{align*}
		\bm{Vect}_\mathbb K & \to \bm{Vect}_\mathbb K\\
		V & \mapsto  Hom(W,V)\\
		f:V_1 \to V_2 &\mapsto Hom(W, V_1) \to Hom(W, V_2)\\
		& \qquad \qquad \qquad \; g \mapsto f \circ g
	\end{align*}
	Vediamo ora se sono rispettate le proprietà della composizione.
	\begin{itemize}
		\item Sia $id_V : V \to V$, allora ho che:
			\begin{align*}
				\bm{Hom}(id_V): Hom(W, V) & \to Hom(W, V)\\
				g & \mapsto g \circ id_V = g
			\end{align*}
		\item Siano $V_1 \xrightarrow{f} V_2 \xrightarrow{g} V_3$. Allora:
			\begin{align*}
				\bm{Hom}(g \circ f) = Hom(W, V_1) & \to Hom(W, V_3)\\
				h & \mapsto (g \circ f) \circ h\\
				\bm{Hom}(g) \circ \bm{Hom}(f): Hom(W, V_1) & \to Hom(W, V_3)\\
				h & \mapsto g\circ (f \circ h)
			\end{align*}
			Ma la composizione è associativa, quindi questi due sono uguali, quindi:
			\[ \bm{Hom}(g \circ f) = \bm{Hom}(g) \circ \bm{Hom}(f) \]
			Quindi è un funtore
	\end{itemize}
\end{es}

\begin{defn}{Funtore Controvariante}{}
	Date due categorie $\mathcal C$ e $\mathcal D$, un \textbf{Funtore Controvariante} $F$ è il dato di:
	\begin{enumerate}
		\item Una funzione $F: \bm{Ob}(\mathcal C) \to \bm{Ob}(\mathcal D)$
		\item $\forall A,B \in \bm{Ob}$ ho una funzione $F$ tra i morfismi:
			\[ F: \bm{Mor}_\mathcal C(A,B) \to \bm{Mor}_\mathcal D (F(B), F(A)) \]
			tale che:
			\begin{enumerate}
				\item $\forall A \in \bm{Ob}(\mathcal C)$ $F(id_A) = id_{F(A)}$
				\item $\forall A, B, C \in \bm{Ob}(\mathcal C)$, $\forall f \in \bm{Mor}_\mathcal C(A, B)$, $\forall g \in \bm{Mor}_\mathcal D(B, C)$ vale:
					\[ F(g \circ f) = F(f) \circ F(g)\]
			\end{enumerate}
	\end{enumerate}
\end{defn}

\begin{es}
	$\bm{Hom}(\cdot, W): \bm{Vect}_\mathbb K \to \bm{Vect}_\mathbb K$ è un funtore controvariante.
	\begin{align*}
		\bm{Vect}_\mathbb K & \to \bm{Vect}_\mathbb K\\
		V & \mapsto Hom(V, W)\\
		f:V_1 \to V_2 & \mapsto Hom(V_2, W)  \to Hom(V_1,W)\\
		&\qquad \qquad \qquad \; \; g \mapsto g \circ f
	\end{align*}
\end{es}

\begin{es}[Dualità]
	Se $W = \mathbb K$, allora $\bm{Hom}(\circ, \mathbb K)$ è il funtore dualità:
	\begin{align*}
		V & \to V^*\\
		V_1 \xrightarrow f V_2 & \mapsto V_2^* \xrightarrow{f^*} V_1^*
	\end{align*}
\end{es}

\begin{es}[Fascio delle Funzioni Continue]
	Consideriamo il funtore "Fascio delle Funzioni Continue" definito come:
	\begin{align*}
		\mathcal C(X, \mathbb R): \bm{Top}(X) & \to \bm{Vect}_\mathbb R\\
		U & \mapsto C(U, \mathbb R) = \{f:U \to (\mathbb R, \mathcal T_\mathcal E) \text{ continue}\}\\
		V \hookrightarrow U & \mapsto C(U, \mathbb R) \mapsto C(V,\mathbb R)\\
		& \qquad \qquad \; f \mapsto f|_V
	\end{align*}
\end{es}

\begin{oss}
	Un funtore controvariante $F: \mathcal C \to \mathcal D$ è un funtore tra $\mathcal C$ e $\mathcal D^{op}$
\end{oss}

\subsection{Proprietà dei Funtori}

\begin{defn}{Proprietà dei Funtori}{}
	Un funtore $F: \mathcal C \to \mathcal D$ è detto:
	\begin{itemize}
		\item \textbf{Fedele}: $\forall A, B \in \bm{Ob}(\mathcal C)$ la funzione:
			\[ F:\bm{Mor}_\mathcal C(A,B) \to \bm{Mor}_\mathcal D(F(A), F(B)) \text{ è iniettiva}\]
		\item \textbf{Pieno}: Come sopra ma suriettiva
		\item \textbf{Pienamente Fedele}: Se è pieno e fedele
		\item \textbf{Essenzialmente Suriettivo}: Se:
			\[ \forall D \in \mathcal D, \exists A \in \mathcal C: D \cong F(A) \]
	\end{itemize}
\end{defn}

\begin{es}
	Il funtore dimenticante $\bm{Top} \to \bm{Set}$ è fedele ma non pieno
\end{es}

\begin{es}
	Il funtore dualità tra Spazi Vettoriali di dimensione finita è pienamente fedele ed essenzialmente suriettivo (infatti $V \cong (\mathbb K^n)^*$ per un opportuno $n$)
\end{es}

\newpage

\section{Connessione per Archi}

\subsection{Connessioni per Archi e Componenti Connessi per Archi}

\begin{defn}{Arco/Cammino e Cammino Chiuso/Laccio}{}
	Dato $(X, \mathcal T)$ spazio topologico, una funzione continua $\gamma:([0,1], \mathcal T_\mathcal E) \to (X, \mathcal T)$ è detta \textbf{Arco} o \textbf{Cammino}. Se inoltre si ha che $\gamma(0) = \gamma(1)$, allora prende il nome di \textbf{Cammino Chiuso}, \textbf{Laccio} oppure \textbf{Cappio}
\end{defn}

\begin{es}
	Sia $X \subseteq (\mathbb R^n, \mathcal T_\mathcal E)$ stellato. Allora è connesso per archi, infatti
	\begin{center}
		\begin{tikzpicture}
			\draw (2,2) to[out = 135, in = 0] (0,1) to[out = 180, in= 45] (-2,2) to[out = 235, in = 90] (-1,0) to[out = 270, in = 135] (-2,-2) to [out = 325, in = 180] (0,-1) to [out = 0, in = 225] (2,-2) to[out = 45, in = 270] (1,0) to [out = 90, in = 315] (2,2);
			\filldraw (-1.5, 1.5) circle(1pt) node[above]{$x$} (-1.5,-1.5) circle(1pt) node[below]{$y$} (0,0) circle(1pt);
			\draw (-1.5,1.5) -- (0,0) -- (-1.5,-1.5);
		\end{tikzpicture}
	\end{center}
	Supponiamo, a meno di traslazioni, che il centro della stella stia nell'origine. Allora possiamo definire $\gamma$ cammino come:
	\[ \gamma = \begin{cases} (1-2t)x & \text{se } t \in [0,\frac 12]\\ (2t-1)y& \text{se }t \in [\frac 12,1] \end{cases}\]
\end{es}

\begin{defn}{Insieme dei Cammini}{}
	Dati $x,y \in (X, \mathcal T)$, definiamo come l'\textbf{Insieme dei Cammini} l'insieme:
	\[ \Omega(X, x, y) = \{\gamma:[0,1] \to X: \gamma(0) = x, \gamma(1) = y\} \]
\end{defn}

\begin{oss}
	$X$ è connesso per archi se e solo se $\forall x, y \in X$, $\Omega(X, x, y) \neq \varnothing$
\end{oss}

\begin{prop}{}{}
	Se $X, Y$ sono connessi per archi, allora $X \times Y$ è connesso per archi
\end{prop}

\begin{proof}
	Siano $(x_1,y_1), (x_2,y_2) \in X \times Y$. Per ipotesi esistono $\gamma_1:[0,1] \to X$ cammino tale che $\gamma_1(0) = x_1$ e $\gamma_1(1) =x_2$ e $\gamma_2:[0,1] \to Y$ tale che $\gamma_2(0) = y_1$ e $\gamma_2(1) = y_2$. Definiamo $\gamma:[0,1] \to X \times Y$ come:
	\[ t \mapsto (\gamma_1(t), \gamma_2(t)) \]
	Questa è continua per la proprietà universale del prodotto e $\gamma(0) = (x_1, y_1)$ e $\gamma(1) = (x_2,y_2)$. Quindi $X \times Y$ è connesso per archi
\end{proof}

\begin{defn}{Giunzione di Cammini}{}
	Siano $x, y, z \in (X, \mathcal T)$ e sia $\alpha \in \Omega(X,x,y)$ e $\beta \in \Omega(X, y,z)$. Definisco la \textbf{Giunzione} di $\alpha$ e $\beta$ come il cammino $\alpha * \beta \in \Omega(X, x, z)$ definito da:
	\[\alpha* \beta = \begin{cases} \alpha(2t) & t \in [0, \frac 12]\\ \beta(2t-1) & t \in [\frac 12 -1 ] \end{cases}\]
\end{defn}

\begin{defn}{Inversione di un cammino}{}
	Definiamo l'\textbf{Inversione di un cammino} come:
	\[ i(\alpha) \in \Omega(X, y, x)\quad i(\alpha)(t) = \alpha(1-t) \]
\end{defn}

\begin{prop}{}{}
	Sia $(X, \mathcal T)$ spazio topologico e siano $A, B \subseteq X$ con topologia indotta tali che siano connessi per archi e disgiunti, allora $A\cup B$ è connesso per archi.
\end{prop}

\begin{proof}
	Sia $x_0 \in A \cap B$ e siano $x_1, x_2 \in A \cup B$.\\
	Se entrambi sono contenuti in $A$ o contenuti in $B$, allora non c'è nulla da dimostrare. Supponiamo $x \in A$ e $x_2 \in B$. Visto che $A$ e $B$ sono connessi per archi, allora esistono $\gamma \in \Omega(X, x_1,x_2)$ e $\alpha \in \Omega(Y, x_0, x_2)$. Allora posso definire:
	\[ \gamma * \alpha \in \Omega(A \cup B, x_1, x_2) \]
	Quindi è connesso per archi.
\end{proof}

\begin{prop}{}{}
	Sia $f: X \to Y$ continua e $X$ connesso per archi. Allora $f(X)$ è connesso per archi
\end{prop}
Graficamente quello che stiamo dicendo è:
\begin{center}
	\begin{tikzpicture}
		\draw (0,0) arc(-90:270:1cm and 2cm) node[below]{$X$} (3,0) arc(-90:270:1cm and 2cm)node[below]{$f(X)$};
		\draw[thick] (-0.5, 1) node(a){} (-0.5, 1) arc(180:90:0.5cm and 2cm) node(a'){} (2.75, 2.5) node(b'){} (2.75,2.5) arc(90:0:0.5cm and 2cm) node(b){};
		\draw[blue, ->] (a) -- (b);
		\draw[->, blue] (a') -- (b');
	\end{tikzpicture}
\end{center}
\begin{proof}
	Siano $y_1, y_2 \in f(X)$, allora esiste $x_1,x_2 \in X: f(x_1) = y_1$ e $f(x_2) = y_2$. Ma $X$ è connesso per archi, quindi:
	\[\exists \gamma \in \Omega(X, x_1, x_2)\qquad \Rightarrow \qquad f \circ \gamma \in \Omega(f(X), y_1, y_2)\]
\end{proof}

\begin{cor}{}{}
	La connessione per archi è invariante per omeomorfismo
\end{cor}

\begin{proof}
	Un omeomorfismo è invertibile come funzione continua
\end{proof}

\begin{es}
	$[0,1) \not \cong (0,1)$ con la topologia euclidea. Infatti se per assurdo esistesse un omeomorfismo
	\[f:[0,1) \to (0,1)\]
	Allora $\forall p \in [0,1)$ ho che:
	\[ f|_{[0,1)\setminus \{p\}}: [0,1)\setminus \{p\} \to (0,1) \setminus \{f(p)\} \text{ è un omeomorfismo}\]
	Ma $[0,1)\setminus \{0\} = (0,1)$ è connesso per archi, mentre $(0,1)\setminus \{f(p)\}$ non è connesso per archi per nessun $p \in (0,1)$
\end{es}

\begin{es}
	$[0,1], [0,1), (0,1)$ non sono omeomorfi tra di loro rispetto alla topologia euclidea. In quanto il primo è compatto, mentre gli altri no
\end{es}

\begin{defn}{$\sim_{CPA}$}{}
	Dati due punti $x,y \in (X, \mathcal T)$ spazio topologico, definiamo $x \sim_{CAP}y$ se e solo se:
	\[ \Omega(X, x, y)\neq \varnothing \]
\end{defn}

\begin{defn}{Componente Connessa per Archi}{}
	Dato $(X, \mathcal T)$ spazio topologico, definiamo $Y \subseteq X$ \textbf{Componente Connessa per Archi} se:
	\begin{enumerate}
		\item $(Y, \mathcal T_Y)$ connesso per archi $\mathcal T_Y$ topologia indotta
		\item $\forall Z$ connesso per archi e $Y \subseteq Z$, $Y = Z$
	\end{enumerate}
\end{defn}

\begin{prop}{}{}
	$\sim_{CPA}$ è una relazione di equivalenza
\end{prop}
\begin{proof}
	Verifichiamo che le tre proprietà siano effettivamente verificate
	\begin{itemize}
		\item[(R)] $x \sim_{CPA} x$ tramite $f:X \to X, f(x) = x$
		\item[(S)] Se $\gamma \in \Omega(X, x, y)$, allora $i(\gamma) \in \Omega(X, y, x)$
		\item[(T)] Se $\gamma \in \Omega(X, x, y)$ e $\alpha \in \Omega(X, y, z)$, allora $\gamma *\alpha \in \Omega(X, x, z)$
	\end{itemize}
\end{proof}

\begin{prop}{}{}
	Sia $Y\subseteq X$ componente connessa per archi e sia $x \in Y$, allora $Y = [x]_{\sim_{CPA}}$
\end{prop}
\begin{proof}
	\fbox{$[x]_{\sim_{CPA}}\subseteq Y$} Siano $x \in Y$ e $Y$ connesso per archi, allora $Y \cup [x]_{\sim_{CPA}}$ è ancora connesso per archi, allora $[x]_{\sim_{CPA}}\cup Y = Y$ per massimalità.

	\fbox{$Y\subseteq [x]_{\sim_{CPA}}$} Sappiamo che $\forall y \in Y$, $\Omega(Y, x, y)\neq \varnothing$, ma allora vale anche che $\Omega(X, x, y)\neq \varnothing$, quindi $x \sim_{CPA} y$
\end{proof}

\begin{defn}{$\pi_0(X)$}{}
	Dato $(X, \mathcal T)$ spazio topologico, definiamo
	\[\pi_0(X) = X/_{\sim_{CPA}}\]
	\underline{solamente come insieme}
\end{defn}

\begin{prop}{}{}
	$\pi_0$ è un funtore $\bm{Top} \to \bm{Set}$
\end{prop}
\begin{proof}
	Vogliamo definire, data $f:X \to Y$, $pi_0$ come:
	\begin{align*}
		\pi_0(f): X/_{\sim_{CPA}} & \to Y/_{\sim_{CPA}}\\
		[x] & \mapsto [f(x)]
	\end{align*}
	Dobbiamo vedere se è ben definita.\\
	Sia $y \in X$ tale che $[x]_{\sim_{CPA}} = [y]_{\sim{CPA}}$, allora $\exists \gamma \in \Omega(X, x, y)$, allora abbiamo direttamente che:
	\[ f \circ \gamma \in \Omega(Y, f(x), f(y))\qquad \Rightarrow \qquad [f(x)]_{\sim_{CPA}} = [f(y)]_{\sim_{CPA}}\]
	Dobbiamo mostrare che valgono le proprietà del funtore.
	\begin{itemize}
		\item Dobbiamo mostrare che $\pi_0(id_X) = id_{\pi_0(X)}$. Sappiamo però che $\pi_0$ manda $[x]_{\sim_{CPA}}$ in $[id_X(x)] $$= [x]$, quindi è verificata.
		\item $\pi_0(f,g)[x] = [g(f(x))] = \pi_0(g)[f(x)]= \pi_0(g) \circ \pi_0(f)[x]$
	\end{itemize}
	Abbiamo dimostrato anche che è un funtore covariante
\end{proof}

\begin{prop}{}{}
	Sia $X\subseteq(\mathbb R, \mathcal T_E)$. $X$ è connesso per archi rispetto alla topologia indotta se e solo se $X$ è un intervallo
\end{prop}
\begin{proof}
	$X$ intervallo, allora $\forall a,b \in X (a \leq b)$ e $\forall x: a \leq c \leq b$ si ha che $c \in X$.

	\fbox{$\Leftarrow$} Un intervallo è convesso, quindi è connesso per archi

	\fbox{$\Rightarrow$} Siano $a,b \in X (a \leq b)$ e $X$ connesso, allora $\exists \gamma \in \Omega(X, a,b)$, quindi $\exists \gamma \in \Omega(\mathbb R, a, b)$ con $Im(\gamma) \subseteq X$. Sia $c$ tale che $a \leq c\leq b$ e applico il teorema dei valori intermedi, allora $x \in Im(\gamma)$, quindi $c \in X$
\end{proof}

\subsection{Omotopie}

\begin{defn}{Funzioni Omotope e Omotopia}{}
	Siano $(X, \mathcal T_X)$ e $(Y, \mathcal T_Y)$ spazi topologici e siano $f_0, f_1 : X \to Y$. $f_0$ e $f_1$ si dicono \textbf{Omotope} se esiste $F:X \times [0,1] \to Y$ continua tale che:
	\[ F(x,0) = f_0(x)\qquad \text e \qquad F(x, 1) =f_1(x) \]
	La funzione $F$ prende il nome di \textbf{Omotopia}
\end{defn}

Graficamente quello che abbiamo è:
\begin{center}
	\begin{tikzpicture}
		\draw (0,0) to[out = 0, in = 180] (3,2) node[below]{$X$};
		\draw(4,-0.5) -- (9,-0.5) -- (11,3) -- (6,3) -- (4,-0.5);
		\draw (5, 0) to[out = 0, in = 180] node(0)[pos = 0]{} node(1)[pos = 0.25]{} node(2)[pos = 0.5]{} node(3)[pos = 0.75]{} node(4)[pos = 1]{} (9,0.5) node[below]{$f_1$};
		\draw (5.5,1) to[out = 0, in = 180] node(a)[pos = 0]{} node(b)[pos = 0.25]{} node(c)[pos = 0.5]{} node(d)[pos = 0.75]{} node(e)[pos = 1]{} (10,2.5) node[right]{$f_0$};
		\draw[blue] (5.25, 0.5) to[out = 0, in = 180] (9.5, 1.5) node[below left]{$F(x, t)$};
		\draw[green, ->] (a) -- (0);
		\draw[green, ->] (b) -- (1);
		\draw[green, ->] (c) -- (2);
		\draw[green, ->] (d) -- (3);
		\draw[green, ->] (e) -- (4);
	\end{tikzpicture}
\end{center}

\begin{oss}
	Per $x$ fissato, possiamo definire $[0,1] \to Y$ come $t \mapsto F(x, t)$.\\
	Per $t$ fissato invece possiamo definire $X \to Y$ come $x \mapsto F(x,t)$.
\end{oss}

\underline{Moralmente} $F$ è un cammino tra $f_0$ e $f_1$ in $\bm{Mor}_{\bm{Top}}(X, Y)$

\begin{defn}{$F*G$ Omotopia}{}
	Date $F$ omotopia tra $f_0$ e $f_1$ e $G$ omotopia tra $f_1$ e $f_2$, definiamo $F*G$ omotopia tra $f_0$ e $f_2$ tramite:
	\[ F*G = \begin{cases}
		F(x, 2t) & t \in [0,\frac 12]\\
		G(x, 2t-1) & t \in [\frac 12, 1]
	\end{cases}\]
	Analogamente $i(F)$ è un'omotopia tra $f_1$ e $f_0$ definita da
	\[ i(F) = (x,t) = F(x, 1-t) \]
\end{defn}

\begin{prop}{}{}
	L'omotopia tra funzioni è una relazione di equivalenza
\end{prop}
\begin{proof}
	Come per la connessione per archi, usando l'omotopia costante, inversione e giunzione
\end{proof}

\begin{es}
	Sia $Y$ convesso di $\mathbb R^n$ oppure stellato, allora ogni coppia di funzioni da $X$ a $Y$ è omotopa. Infatti, siano $f_0, f_1:X \to Y$. Definiamo:
	\[ F(x,t) = tf_1(x) + (1-t)f_0(x) \in Y \]
	Per $Y$ stellato basta mandarlo prima tutto $f(X)$ nel centro e poi portarlo fuori
\end{es}

\begin{es}
	Sia $X = \{p\}$ un punto e sia $Y = \mathbb R\setminus\{0\}$ e siano le funzioni:
	\[ f_+(p)=1\qquad \text e\qquad f_-(p)=-1 \]
	Queste funzioni non sono omotope in quanto una loro omotopia dovrebbe avere un cammino tra $1$ e $-1$ in $\mathbb R\setminus\{0\}$, il che è assurdo
\end{es}

\begin{lemma}{}{}
	Siano $f,g:X \to Y$ due funzioni omotope, allora $\pi_0(f) \equiv \pi_0(g)$
\end{lemma}
\begin{proof}
	Quello che abbiamo è che:
	\begin{center}
		\begin{tikzpicture}
			\draw (2,0) arc(0:360:2cm and 1cm)node[right]{$X$} (4,0) arc(180:540:2 cm and 1.5 cm)node[left]{$Y$};
			\filldraw (1,0) circle(1pt) node(x){} node[left]{$x$} (5,0.5) circle (1pt) node(fy){} node[right]{$f(y)$} (6.5,-0.5) circle(1pt) node(gy){} node[right]{$g(y)$};
			\draw[->] (x) to[out = 45, in = 135] (fy);
			\draw[->] (x) to[out = -45, in = -135] (gy);
			\draw[blue, ->] (fy) to[out = -135, in = 60] (gy);
		\end{tikzpicture}
	\end{center}
	Abbiamo che $\forall x \in X$, devo mostrare che $[f(x)] = [g(x)]$, ma $F(x, t) \in \Omega(Y, f(x), g(x))$, quindi è verificato.
\end{proof}

\begin{defn}{Equivalenza Omotopica}{}
	$f:X \to Y$ funzione continua è detta \textbf{Equivalenza Omotopica} se $\exists g:Y \to X$ continua tale che:
	\[ g \circ f \sim id_X\qquad \text e f \circ g \sim id_Y \]
	In tal caso $X$ e $Y$ si dicono \textbf{Omotopicamente Equivalenti} o \textbf{Omotopi} e si indica con $X \sim Y$
\end{defn}

\begin{es}
	Sia $X \subseteq (\mathbb R, \mathcal T_\mathcal E)$ convesso, questo è omotopo ad un punto $\{p\}$\\
	Infatti se prendiamo $f:X \to \{p\}$ tale che $f(x) = p, \forall x \in X$ e prendiamo $g:\{p\} \to X$ tale che $g(p) = x_0 \in X$. Allora abbiamo che $g \circ f:X \to X$ è omotopa a $id_X$ per il fatto che $X$ è convesso. Ma abbiamo anche che $f \circ g: p \mapsto x_0 \mapsto x$, quindi $ \circ g = id_{\{p\}}$
\end{es}

\begin{oss}
	Due spazi omeomorfi sono anche omotopi, cioè $X \equiv Y \Rightarrow X \sim Y$. Basta infatti prendere $f\circ g = id_Y$ e $g \circ f = id_X$. Però non è vero il contrario: basta infatti prendere un insieme convesso con almeno due punti e il singoletto e la compattezza non è preservata dall'equivalenza omotopica. Cioè, per esempio, $\mathbb R \sim \{p\}$
\end{oss}

\begin{prop}{}{}
	$X \sim Y \Rightarrow \pi_0(X) \cong \pi_0(Y)$
\end{prop}

\begin{proof}
	Ho $f:X \to Y$ e $g:Y \to X$ con $f\circ g = id_Y$ e $g \circ f = id_X$. Quindi:
	\begin{center}
		\begin{tikzcd}
			{\pi_0(X)} \ar[r, bend left = 40, "{\pi_0(f)}"] \ar[r, bend right = 40, "{\pi_0(g)}"'] & \pi_0(Y)
		\end{tikzcd}
	\end{center}
	Quindi $g \circ f = id_X$, allora abbiamo che:
	\[ \pi_0(g) \circ \pi_0(f) = \pi_0(g \circ f) = \pi(id_X) = id_{\pi_0(X)} \]
	In maniera del tutto analoga possiamo fare con $\pi_0(f) \circ \pi_0(g) = id_{\pi_0(Y)}$
	Quindi abbiamo che $\pi_0(f)$ è invertibile, quindi $\pi_0(X)\cong \pi_0(Y)$
\end{proof}

\begin{defn}{Spazio Topologico Contraibile}{}
	Uno spazio topologico si dice \textbf{Contraibile} se è omotopo ad un punto.
\end{defn}

\begin{es}
	$X$ convesso o stellato di $\mathbb R^n$ è contraibile
\end{es}

\begin{defn}{Sottospazio Retratto per Deformazione}{}
	Sia $(X, \mathcal T_X)$ spazio topologico e $Y \subseteq X$ con topologica indotta. $Y$ è detto \textbf{Retratto per Contrazione} se $\exists R:X \times [0,1] \to X$ continua (detta \textbf{Retrazione}) tale che:
	\begin{enumerate}
		\item $R(x, 0) \in Y, \forall x \in Y$
		\item $R(x, 1) = x, \forall x \in X$
		\item $R(y, t) = Y, \forall y \in Y, t \in [0,1]$
	\end{enumerate}
\end{defn}

\begin{es}
	$0$ è retratto per deformazione di $\mathbb R$ tramite:
	\begin{align*}
		R:\mathbb R \times [0,1] & \to \mathbb R\\
		(x, t) &\mapsto (1-t)x
	\end{align*}
\end{es}

\begin{prop}{}{}
	Sia $Y \subseteq X$ retratto per deformazione, allora $Y \sim X$ tramite inclusione
\end{prop}

\begin{proof}
	Sia $R: X \times [0,1] \to X$ una retrazione e sia $r:X \to Y$ data da $r(x)\cong R(x, 0)$ (avremmo che $i(r(x)) = R(x, 0)$ tramite $i$ immesione). Ho allora che $r \circ i = id_Y$ per il punto $3$ della definizione. Sappiamo poi che:
	\[ i \circ r = id_X \]
	Ma il primo termine è una funzione che manda $x \mapsto R(x,0)$ mentre il secondo manda $x$ in $R(x,1)$, quindi $R$ definisce esattamente un'omotopia tra $i\circ r$ e $id_X$
\end{proof}

\begin{es}
	Sia $S^n \subseteq \mathbb R^{n+1}\setminus \{0\}$ con la topologia euclidea. Allora $S^{n}$ è un retratto per deformazione.
	\begin{center}
		\begin{tikzpicture}
			\draw (1,0) arc (0:360:1cm) node(a)[pos = 0.2]{} node(b)[pos = 0.4]{} node(c)[pos = 0.6]{} node(d)[pos = 0.8]{};
			\draw[white] (2,0) arc (0:360:2cm) node(aa)[pos = 0.2]{} node(bb)[pos = 0.4]{} node(cc)[pos = 0.6]{} node(dd)[pos = 0.8]{};
			\draw (0,0) node(x){$\times$} (0,0) -- (3,0);
			\filldraw (1,0) circle (1pt);
			\foreach \x in {a,b,c,d}{
				\draw[->] (x) -- (\x);
			}
			\draw[->] (aa) -- (a);
			\draw[->] (bb) -- (b);
			\draw[->] (cc) -- (c);
			\draw[->] (dd) -- (d);
		\end{tikzpicture}
	\end{center}
	In particolare, è un detratto per deformazione tramite la funzione:
	\[ R(xt) = xt + (1-t) \frac{x}{\|x\|} \]
\end{es}

\begin{es}
	La retta $\{x=0\}$ è un retratto per deformazione di:
	\[ \{x = 0\} \bigcup_{n \in \mathbb Z}{y = n} \]
	Tramite la funzione:
	\[ R\left(\begin{pmatrix}x\\ y\end{pmatrix}, t\right) = \begin{pmatrix} tx\\ y \end{pmatrix}\]

	\begin{center}
		\begin{tikzpicture}
			\draw (0,-1) -- (0,1);
			\foreach \x in {-1,-0.5, 0, 0.5, 1}{
				\draw (-2, \x) -- (2,\x);
			}
			\draw (3,0) node{$\Rightarrow$};
			\draw (4,-1) -- (4,1);
		\end{tikzpicture}
	\end{center}
\end{es}

\newpage

\section{Connessione}

\subsection{Connessione e Sconnessione}

\begin{defn}{Connessione e Sconnessione}{}
	Sia $(X, \mathcal T)\neq \varnothing$ uno spazio topologico. Si dice che è \textbf{Connesso} se dati $A,B$ aperti non vuoti con $A \cup B = X$, allora $A \cap B \neq \varnothing$. Altrimenti si dice che $X$ è \textbf{Sconnesso}
\end{defn}

\begin{lemma}{}{}\label{sconnessione}
	Si equivalgono:
	\begin{enumerate}
		\item $A$ sconnesso
		\item $\exists F, G$ chiusi non vuoti tali che $F \cup G = X$ e $F \cap G = \varnothing$
		\item $\exists A \subseteq X$ non vuoto, $A \neq X$ che sia claperto
		\item $\exists f:X \to (\{0,1\}, \mathcal T_D)$ continua e suriettiva
	\end{enumerate}
\end{lemma}
\begin{proof}
	\fbox{$1 \Leftrightarrow 2$} È banale, basta passare al complementare

	\fbox{$1 \Leftrightarrow 3$} $A$ è claperto $\Leftrightarrow$ $X\setminus A$ è claperto, allora $A \cup (X\setminus A) = X$. Questa è un'unione vuota e disgiunta di aperti.

	\fbox{$1 \Leftrightarrow 4$} Sia $X = A \cup B$ aperti disgiunti non vuoti. Definiamo $f:X \to \{0,1\}$ con $f(x) = 0$ se $x \in A$ e $f(x) = 1$ se $x \in B$, allora $f$ è continua, suriettiva e ben definita. Inoltre $X = f^{-1}(\{0\})\cup f^{-1}(\{1\})$, aperti disgiunti e non vuoti.
\end{proof}

\begin{lemma}{}{}
	Sia $Y \subseteq X$, $Y$ sottospazio connesso e sia $A\subseteq X$ claperto, allora $Y\cap A = \varnothing$ oppure $Y \cap A = Y$
\end{lemma}
\begin{proof}
	Se $Y \cap A$ è claperto in $Y$, allora per il lemma \ref{sconnessione}.1 è $\varnothing$ oppure $Y$
\end{proof}

\begin{prop}{}{}
	Sia $f:X \to Y$ continua con $X$ connesso, allora $f(X)$ è connesso
\end{prop}
\begin{proof}
	Siano $A, B$ aperti di $f(X)$ non vuoti tali che $f(X) = A \cup B$. Allora, per definizione di topologia indotta, $\exists C, D$ di $Y$ tali che $A = C \cap f(X)$ e $B = D \cap f(X)$. Allora si ha che:
	\[ f^{-1}(C) = f^{-1}(A) \qquad \text e \qquad f^{-1}(D) = f^{-1}(B) \]
	ovviamente si ha anche che $X = f^{-1}(A) \cup f^{-1}(B)$ aperti non vuoti di $X$, ma $X$ è connesso, quindi:
	\[ \exists x \in f^{-1}(A)\cap f^{-1}(B) \quad \Rightarrow\quad f(x) \in A \cap B \]
	Quindi $f(X)$ è connesso
\end{proof}

\begin{cor}{}{}
	La connessione è invariante per omeomorfismi
\end{cor}

\begin{cor}{}{}
	Quozienti di connessi sono connessi
\end{cor}

\begin{es}
	$\mathbb R \setminus \{0\}$ è sconnesso. Infatti può essere scritto come:
	\[ \mathbb R \setminus \{0\} = (-\infty,0) \cup (0, + \infty) \]
\end{es}

\begin{es}
	$\mathbb Q$ è sconnesso. Infatti può essere scritto attraverso i tagli di Dedekind come
	\[ \mathbb Q = \{x \in \mathbb Q: x < \alpha\}\cup \{x \in \mathbb Q: x > \alpha\}\qquad \alpha \in \mathbb Q \]
\end{es}

\begin{es}
	$(X, \mathcal T_\mathcal E), X \neq \varnothing$ è connesso
\end{es}

\begin{es}
	$(X, \mathcal T_D), |X|>1$ è sconnesso. Basta infatti prendere un insieme $A \subseteq X, A \neq \varnothing$ o $X$ e questo è claperto
\end{es}

\begin{es}
	$(X, \mathcal T_{Cof})$ e $|X|\geq \infty$ è connesso. Basta riprendere l'osservazione \ref{conngros}
\end{es}

\begin{thm}{}{}\label{01conn}
	$([0,1], \mathcal T_\mathcal E)$ è connesso
\end{thm}

Prima di darne la dimostrazione, diamo il seguente corollario

\begin{cor}{}{}
	Un insieme connesso per archi è connesso
\end{cor}
\begin{proof}
	Sia $X$ connesso per archi e $X = A \cup B$ aperti non vuoti, allora esiste $x \in A$ e $b \in B$.
	Sappiamo che $X$ è connesso per archi, allora:
	\[\exists \gamma \in \Omega(X, x, y)\]
	Ma allora abbiamo che:
	\[ [0,1] = \gamma^{-1}(A) \cup \gamma^{-1}(B) \]
	Questi sono due insiemi aperti con $0 \in \gamma^{-1}(A)$ e $1 \in \gamma^{-1}(B)$. Ma allora abbiamo che:
	\[ \exists t: \gamma^{-1}(A) \cap \gamma^{-1}(B) \quad \Rightarrow \quad \gamma(t) \in A \cap B\]
	Quindi $X$ è connesso.
\end{proof}

Graficamente quello che è stato fatto è:
\begin{center}
	\begin{tikzpicture}
		\draw (0,2) arc(60:300:1cm and 2cm) node[pos = 0.5, left]{$A$} (0,2) arc(480:240:1 cm and 2cm) node[pos = 0.5, right]{$B$};
		\draw (0,2) -- (0,-1.475);
		\draw[blue] (-1,1) to[out = 270, in = 90] (1,-1);
		\filldraw (-1,1) circle(1pt) (1,-1) circle(1pt);
	\end{tikzpicture}
\end{center}

\begin{proof}[Dimostrazione del teorema \ref{01conn}.7]
	Siano $C, D$ chiusi non vuoti tali che $[0,1] = C \cup D$. Sia $0 \in C$ (se così non fosse cambieremmo i nomi). Chiamiamo $d = \min(D)$. Se $d = 0$, allora $C \cap D \neq 0$\\
	Supponiamo quindi $d>0$ e chiamiamo $E = C \cap [0,d]$ chiuso. Allora $[0,d)\subseteq E$. Ma se andassimo a vedere la chiusura avremmo che:
	\[ [0,d] = \overline{[0,d)} \subseteq E \]
	\textit{In realtà abbiamo l'uguaglianza}. Da questo segue che:
	\[ d \in E \subseteq X \qquad \Rightarrow \qquad C \cap D \neq \varnothing \]
	Quindi $[0,1]$ è connesso
\end{proof}

\begin{cor}{}{}
	Sia $X \subseteq (\mathbb R, \mathcal T_\mathcal E)$. Questo è connesso se e solo se è un intervallo
\end{cor}
\begin{proof}
	\fbox{$\Leftarrow$} $X$ intervallo, allora $X$ è connesso per archi, quindi è connesso

	\fbox{$\Rightarrow$} Sia $X$ non un intervallo, allora esistono $a,b,c$ tali che $a \leq b \leq c$, con $a,c \in X$ e $b \not \in X$. Da questo segue che:
	\[ X = (X \cap \{x>b\}) \cup (X\cap \{x<b\}) \]
	Al primo insieme appartiene $a$ e al secondo insieme appartiene $c$. Questi sono due aperti disgiunti, quindi $X$ è sconnesso.
\end{proof}

\begin{cor}{Teorema dei Valori Intermedi}{}
	Sia $(X, \mathcal T)$ spazio topologico connesso e sia $f:X \to \mathbb R$ continua. Allora $f(X)$ è un intervallo
\end{cor}

Enunciamo una proposizione che ci tornerà comoda per il teorema.

\begin{prop}{}{}
	Sia $Y$ connesso e $f:X \to Y$ continua e suriettiva tale che:
	\begin{itemize}
		\item $\forall y \in Y, f^{-1}(y)$ è connesso
		\item $f$ è aperta o chiusa
	\end{itemize}
	Allora $X$ è connesso
\end{prop}

\begin{proof}
	Sia $f$ aperta (il caso $f$ chiusa è sostanzialmente identica). Siano $A_1$ e $A_2$ aperti di $X$ non vuoti tali che $X = A_1 \cup A_2$. Allora si ha che:
	\[ Y = f(A_1) \cup f(A_2) \text{ aperti non vuoti}\]
	Abbiamo utilizzato che $f$ è suriettiva e aperta. Abbiamo però anche che $Y$ è connesso, quindi $\exists y \in f(A_1)\cap f(A_2)$, quindi:
	\[ f^{-1}(y) = (A_1 \cap f^{-1}(y)) \cup (A_2 \cap f^{-1}(y)) \]
	Questi ultimi due insiemi sono sicuramente non vuoti in quanto $y \in f(A_1) \cap f(A_2)$. Inoltre è aperto per la topologia indotta. Ma abbiamo che $f^{-1}(y)$ è connesso, quindi:
	\[ \exists x \in (A_1 \cap f^{-1})\cup (A_2 \cap f^{-1}(y)) \subseteq A_1 \cap A_2 \]
	Quindi $X$ è connesso
\end{proof}

\begin{thm}{}{}
	Siano $X, Y$ connessi, allora $X \times Y$ è connesso.
\end{thm}
\begin{proof}
	Consideriamo $p_Y:X \to Y$ continua e suriettiva. Sappiamo che $Y$ è connesso e $\forall y \in Y, p^{-1}_Y(\{y\}) = X \times \{y\}$. Ma abbiamo anche che $X \times \{y\} \cong X$ connesso e $p_Y$ aperta, quindi $X$ è connesso.
\end{proof}

\subsection{Componenti Connesse}

\begin{defn}{Componente Connessa}{}
	Sia $C \subseteq X$. $C$ è detta \textbf{Componente Connessa} se $C \neq \varnothing$ è connesso e $\forall A \subseteq X$ connesso e $C\subseteq A$, allora $C = A$
\end{defn}

\begin{thm}{}{}\label{ABC}
	Ogni spazio topologico non vuoto è unione disgiunta delle sue componenti connesse, che sono chiuse
\end{thm}

\begin{cor}{}{}
	Il numero delle componenti connesse è invariante per omeomorfismo.
\end{cor}

\begin{es}
	$\mathbb R\setminus \{0\}\not \cong \mathbb R \setminus \{0,1\}$. Il primo insieme può essere scritto come unione di due componenti $\{x<0\} \cup \{x>0\}$ mentre il secondo come unione di tre componenti $\{x<0\}\cup (0,1) \cup \{x>1\}$. Visto che $2 \neq 3$, questi non sono omeomorfi
\end{es}

\begin{lemma}{A}{}
	Sia $Y \subseteq X$ sottospazio connesso e sia $W\subseteq X$ sottospazio con $Y \subseteq W \subseteq \overline Y \subseteq X $. Allora $W$ è connesso
\end{lemma}
\begin{proof}
	Siano $A_1, A_2 \subseteq W$ aperti non vuoti tali che $A_1 \cup A_2 = W$. Allora abbiamo che:
	\[ (A_1 \cap Y)\cup (A_2 \cap Y) =Y \]
	Ma questi due insiemi sono aperti di $Y$ e se non sono entrambi vuoti, allora esiste $y \in (A_1\cap Y)\cap (A_2\cap Y)$. Allora $A_1$ e $A_2$ non sono disgiunti.\\
	Supponiamo per assurdo che $Y \cap A_1 = \varnothing$. Allora abbiamo che $Y \subseteq W\setminus A_1$ chiuso, quindi $\overline Y^W \subseteq W \setminus A_1$. Ma abbiamo che:
	\[ \overline Y^W = \overline Y^X \cap W = W \]
	Ma abbiamo un assurdo, quindi necessariamente abbiamo $Y \cap A_1 \neq \varnothing$ e $W$ connesso
\end{proof}

\begin{es}[Seno del Topologo]
	Sia $\Gamma$ definito come:
	\[ \Gamma = \overline{\left\{ \left(x, \sin\left(\frac 1x \right)\right): x>0 \right\}} \]
	Abbiamo che $\Gamma$ è connesso, infatti $\{(x, \sin (\frac1x)):x>0\}$ è connesso per archi tramite:
	\[ \gamma:[0,1] \to \mathbb R\qquad t \mapsto (1-t)x + ty \]
	e $(\gamma(t), \sin(\frac 1{\gamma(t)}))$ collega i punti $(x, \sin(\frac 1x))$ e $(y, \sin(\frac 1y))$. Quindi $\Gamma$ è connesso per il lemma appena fatto.\\
	Graficamente $\Gamma$ è:
	\begin{center}
		\begin{tikzpicture}
			\draw[domain=0.01:1,samples=1000] plot ({2*\x}, {sin((1/\x)r)});
			\draw[thick, green] (0,1) -- (0,-1);
		\end{tikzpicture}
	\end{center}
	Mostriamo che $\Gamma$ non è connesso per archi. Sia $\gamma$ un cammino tra $(0, t)$ e $(1, \sin 1)$. Sia poi $T = \sup\{ s: p_1(\gamma(t))=0 \}$. Sia inoltre $a = p_2(\gamma(T)) \in [-1,1]$\\
	Per la continuità di $\gamma$ si ha che:
	\[ \forall \vareps>0, \exists \delta >0: p_2(\gamma(s)) \in (a -\vareps, a + \vareps), \forall s \in (T - \delta, T+ \delta) \]
	Scegliamo allora $\vareps>0$ tale che $[-1,1]\subsetneq (a-\vareps, a+\vareps)$. Allora abbiamo che:
	\[ \forall \delta >0, p_1(\gamma(s))>0, \forall s \in (T, T+\delta) \]
	Poniamo adesso $b = \sup\{p_1(\gamma(s)):s \in (T, T+\delta)\}$. Abbiamo quindi che $(0, b) = p_1(\gamma(T, T+\delta))$. Ma allora abbiamo che $p_2(\gamma(T,T+\delta)) = [-1,1]$. Quindi non esiste un $\delta$ tale che:
	\[ p_2(\gamma(T,T+\delta)) \subseteq (a-\vareps, a + \vareps) \]
	Quindi non può esistere tale funzione $\gamma$
\end{es}

\begin{lemma}{B}{}\label{B}
	Sia $x \in X$ e siano $\{Z_i\}_{i \in I}, I \neq \varnothing$ tali che $Z_i\subseteq X$ connesso e con topologia indotta tali che $x \in Z_i \forall i \in I$. Allora:
	\[ \bigcup_{i \in I}Z_i \qquad \text{è ancora connesso} \]
\end{lemma}
\begin{proof}
	Per questioni di comodità di notazioni, poniamo $W$ come l'unione di tutti i $Z_i$ e sia $A \neq \varnothing$ aperto e chiuso in $W$. Supponiamo $x \in A$ (altrimenti lo sostituiamo con $W\setminus A$). Allora $\forall i \in I$ abbiamo che $A \cap Z_i \neq \varnothing$ ed è aperto e chiuso. Ma abbiamo che $Z_i$ è connesso, quindi $A \cap Z_i = Z_i$ da cui segue che $Z_i \subseteq A$. Quindi:
	\[ \bigcup_{i \in I}Z_i \subseteq A\qquad \Rightarrow \qquad W \subseteq A \]
	Quindi $W$ è connesso
\end{proof}

\begin{cor}{}{}
	$A,B$ connessi ma non disgiunti, allora $A \cup B$ è connesso
\end{cor}

\begin{lemma}{C}{}
	Sia $x \in X$ e sia $Conn(x) = \{Y\subseteq X\text{ connessi }:x \in Y\}$. Allora:
	\[ C(x):= \bigcup_{Y \in Conn(x)} Y \text{ è una componente connessa} \]
\end{lemma}
\begin{proof}
	Abbiamo che $\{x\}$ è connesso con la topologia indotta (che coincide con quella grossolana). Allora sicuramente $\{x\}\in Conn(x)$. Quindi sicuramente $Conn(x)\neq \varnothing$, da cui sicuramente $C(x)\neq \varnothing$. Abbiamo inoltre che per il lemma B \ref{B}.5 abbiamo che $C(x)$ è connesso e $x \in C(x)$.\\
	Supponiamo ora $Y$ connesso con $C(x)\subseteq Y$, allora abbiamo che $x \in Y$ quindi $Y \in Conn(x)$ da cui $Y \subseteq C(x)$
\end{proof}

Dimostriamo ora il teorema \ref{ABC}

\begin{proof}[Dimostrazione del teorema \ref{ABC}.2]
	Sia l'insieme $X$ scritto come:
	\[ X = \bigcup_{x \in X}C(x) \]
	Per il lemma $C$ abbiamo che queste sono tutte componenti connesse.\\
	Siano ora $x, y \in X$. Se $C(x)\cap C(y) \neq \varnothing$, allora esiste $z \in C(x) \cap C(y)$. Allora abbiamo che $C(z)\supseteq C(x)\cup C(y)$ per massimalità. Sempre per massimalità segue che $C(x) = C(y) = C(z)$. Eliminando le ripetizioni ad inizio uguaglianza otteniamo che:
	\[ Z \subseteq X: \forall z_1, z_2 \in Z, C(z_1)\neq C(z_2) \text{ e } \bigcup_{z \in Z}C(z) = X \]
	In questo modo abbiamo che:
	\[ X = \coprod_{z \in Z}C(z)\]
	\textit{Dove $\amalg$ rappresenta l'unione disgiunta}\\
	Si conclude poi con il lemma A, secondo cui le componenti connesse sono tutte chiuse
\end{proof}

\begin{es}[Continuo del seno del topologo]
	Avevamo detto dalla prima parte che questo era connesso ma non per archi. Le sue componenti per archi sono:
	\[ \left\{ \left( x, \sin \left(\frac 1x\right) \right): x>0 \right\} \qquad \text e\qquad \{0\}\times [-1,1] \]
	La prima componente è aperta in $\Gamma$ ma non chiusa mentre la seconda è chiusa in $\Gamma$ ma non aperta
\end{es}

\begin{prop}{}{}
	$X$ è localmente connesso per archi. Allora le componenti connesse per archi coincidono con le componenti connesse e sono sia chiusa sia aperte.
\end{prop}
\begin{proof}
	Sia $x \in X$ e sia $A = \{y \in X: \Omega(X, x,y)\neq \varnothing\}$ le sue componenti connessi per archi. Sia $y \in A$ e sia $U$ un intorno di $y$ connesso per archi, allora $U \subseteq A$. Quindi $A$ è un intorno di tutti i suoi punti, ma allora $A$ è aperto. Se facciamo un ragionamento analogo con $B$ definito per tutti gli $x \not \in A$, abbiamo che:
	\[ X \setminus A = \bigcup_{B\neq A, B\text{ Comp CPA}}B \text{ aperto}\qquad \Rightarrow \qquad A \text{ è chiuso} \]
	Considerando la componente connessa $C(x)\supseteq A \neq \varnothing$, quindi $C(x) = A$
\end{proof}

\begin{es}[Pettine del Topologo]
	Sia $X \subseteq (\mathbb R^2, \mathcal T_\mathcal E)$ con:
	\[ X = [0,1]\times \{0\} \cup \left( \bigcup_{y \in \mathbb N^*} \left\{\frac 1n \right\}\times [0,1] \right)\cup \{0\} \times [0,1]\]
	Graficamente è:
	\begin{center}
		\begin{tikzpicture}
			\draw (0,0) -- (0,2);
			\draw (0,0) -- (2,0);
			\foreach \x in {1,...,50}{
				\draw (2/\x, 2) -- (2/\x,0);
			}
			\filldraw (0,0) -- (0,2) -- (0.05,2) -- (0.05,0);
		\end{tikzpicture}
	\end{center}
	$X$ è certamente connesso per archi (infatti per arrivare da un qualsiasi punto ad un altro del pettine mi basta passare dal manico). Ma è localmente connesso per archi? La risposta è no.\\
	Per ogni punto $(x,y)$ con $x>0$ e $y \neq 0$ abbiamo che:
	\[\exists \vareps: B_\vareps(x,y)\cap X\text{ è connesso per archi} \]
	Infatti:
	\[ \exists n: x = \frac 1n \]
	Ci basterà allora prendere:
	\[ \vareps < \min \left\{ y, \left| \frac 1n - \frac 1{n-1}\right|, \left| \frac 1n - \frac 1{n+1} \right| \right\}\]
	In questo modo segue automaticamente che:
	\[ B_\vareps(x,y) \cap X = \left\{ \frac 1n \right\}\times (y-\vareps, y + \vareps) \]
	Per $x=0$ e $y>0$ le cose non funzionano più tanto bene. Cioè non è localmente connesso per archi.\\
	Se infatti esistesse tale $\gamma \in \Omega(U, (0,y), (\frac 1n, y))$ con $U$ che non interseca il manico del pettine, allora, essendo $\gamma$ una funzione continua, avremmo che sicuramente $\gamma$ passa per $(\frac 1{n+1}, y)$, punto che non fa parte dell'intorno.
\end{es}

\newpage

\section{Varietà Topologiche}

\subsection{Definizioni Base}

\begin{defn}{Varietà Topologica}{}
	Una \textbf{Varietà Topologica} $M$ di dimensione $n$ è uno spazio topologico $M$ tale che:
	\begin{enumerate}
		\item $M$ è $T1$
		\item $M$ è localmente euclideo (con aperti di $\mathbb R^n$)
		\item Ogni componente connessa di $M$ è $N2$
	\end{enumerate}
\end{defn}

\begin{es}
	$(\mathbb R^n, \mathcal T_\mathcal E)$ è una varietà topologica di dimensione $n$
\end{es}

\begin{es}
	Il seguente insieme $M$ definito come:
	\[ M = \bigcup_{n \in \mathbb Z}{n,y}, y \in \mathbb R \subseteq (\mathbb R^2, \mathcal T_\mathcal E)\] è una varietà topologica di dimensione $1$
\end{es}

\begin{es}
	$(\mathbb Q, \mathcal T_\mathcal E)$ non è una varietà topologica. Infatti ogni aperto contiene numerabili punti è non può essere omeomorfo ad un aperto di $\mathbb R^n$ che ha continui punti se $n\geq 1$ oppure un punto se $n = 0$
\end{es}

\begin{es}
	$(\mathbb Z, \mathcal T_\mathcal E)$ è una varietà topologica di dimensione $0$
\end{es}

\begin{es}
	$S^n$ è una varietà topologica di dimensione $n$.\\
	Infatti, sappiamo che è $T2$ e $N2$ in quanto $S^n \subseteq \mathbb R^{n+1}$ sottospazio (e $\mathbb R^{n+1}$ è già $N2$ e $T2$) È localmente euclideo tramite le due proiezioni stereografiche:
	\[ S^n \setminus\{(0,...,0,1)\} \xrightarrow{\cong}\mathbb R^n\qquad \qquad S^n\setminus\{(0,...,0,-1)\} \xrightarrow{\cong} \mathbb R^n \]
\end{es}

\begin{es}
	$U\subseteq \mathbb R^n$ aperto è una varietà topologica di dimensione $n$
\end{es}

\begin{es}
	L'insieme $A=\{(x,y)\in \mathbb R^2:xy = 0\}$ non è una varietà topologica.\\
	Se vogliamo rappresentare questo insieme, avremmo che:
	\begin{center}
		\begin{tikzpicture}
			\draw(-2,0) -- (2,0) (0,2) -- (0,-2);
		\end{tikzpicture}
	\end{center}
	Il problema è proprio nell'origine. In tutte le altre parti, è possibile trovare un aperto di $\mathbb A$ tale che sia isomorfo ad un aperto di $\mathbb R$. Con l'origine, quello che stiamo cercando è un insieme aperto in $\mathbb R$ o in $\mathbb R^2$ tale che sia omeomorfo ad una croce. Che è impossibile (basta infatti togliere un punto e avremmo che dalla croce otteniamo $4$ componente connesse, mentre dall'intervallo ne avremmo due o dal piano una soltanto)
\end{es}

\begin{es}[Importante: Teorema delle Funzioni Implicite]
	Il teorema delle funzioni implicite ci permette di definire varietà topologiche, per esempio:
	\[ S^n = f^{-1}(0)\qquad \text{con }f:\mathbb R^{n+1}\to \mathbb R\quad x \mapsto \|x\|-1 \]
\end{es}

\begin{es}
	$\mathbb P^n(\mathbb R)$ e $\mathbb P^n(\mathbb C)$ sono delle varietà topologiche di dimensione rispettivamente $n$ e $2n$ tramite le carte affini.
\end{es}

\begin{defn}{Carta e Atlante}{}
	Una coppia $(U, \varphi)$ con $U\subseteq M$ aperto di una varietà topologia è detta \textbf{Carta} di $M$ se $\varphi:U \to \mathbb R^n$ è aperta e omeomorfismo sulla sua immagine. Una collezione $\{(U_i, \varphi_i)\}_{i \in I}$ è detto \textbf{Atlante} se:
	\[ \bigcup_{i \in I}U_i = M \]
\end{defn}

Graficamente abbiamo che:
\begin{center}
	\begin{tikzpicture}
		\draw (-3,0) arc (-180:180:3cm and -2.25cm) node[left]{$M$};
		\draw[->] (3.5,0) -- (4,0);
		\draw (4,-2) -- node[pos = 0.5, below]{$\mathbb R^n$} (9,-2) -- (11,2) -- (6,2) -- (4,-2);
		\draw[blue] (0, 1.5) arc (90:450:1cm) node[above]{$U_0$} (1.75, -0.25) arc(0:360:1cm) node[right]{$U_2$} (-1.75, -0.25) arc(-180:180:1cm) node[left]{$U_1$};
		\begin{scope}
			\clip (0,0.5) circle(1cm);
			\fill[blue!50, nearly transparent] (-0.75,-0.25) circle(1cm);
		\end{scope}
		\filldraw (-0.5, 0.25) circle(1pt) node[above left]{$P$};
		\draw[green] (7, 1) arc(-180:180:1.5cm and 0.5cm) node[left]{$\varphi_0(U_0)$};
		\draw[green] (8,0) arc(90:450:0.75cm) node[right]{$\quad\varphi_2(U_2)$};
		\draw[green] (6,-01) arc(-90:270:0.5cm and 0.75cm) node[below]{$\varphi_1(U_1)$};
		\begin{scope}
			\clip (8.5, 1) circle(1.5cm and 0.5cm);
			\fill[green, nearly transparent] (7,1) circle (1cm);
		\end{scope}
		\begin{scope}
			\clip (6,-0.25) circle (0.5cm and 0.75cm);
			\fill[green, nearly transparent] (6,0.5) circle (0.5cm and 0.75cm);
		\end{scope}
		\filldraw (6, 0) circle(1pt) node[below]{$\varphi_1(P)$} (7.75,1) circle(1pt) node[right]{$\varphi_0(P)$};
	\end{tikzpicture}
\end{center}
\textit{Le parti evidenziate sono l'intersezione tra $U_0$ e $U_1$ (in blu) e le rispettive immagini rispetto a $\varphi_1$ e $\varphi_0$ (in verde)}

Possiamo definire allora la funzione $\varphi_{i,j}$ come:
\[ \varphi_{ij} := \varphi_i \circ \varphi_j^{-1}:\varphi_j(U_i \cap U_j) \to \varphi_i(U_i \cap U_j) \]
Questa funzione è sicuramente continua, ma è differenziabile?

\begin{defn}{Varietà Differenziabile}{}
	Se $\forall i,j$ le carte dell'atlante di $M$ varietà topologica danno funzioni del tipo $\varphi_{ij}:\varphi_i \circ \varphi_j^{-1}$ di classe $C^\infty$, allora $M$ prende il nome di \textbf{Varietà Differenziabile}
\end{defn}

\newpage

\section{Gruppo Fondamentale}

\subsection{Struttura del Gruppo Fondamentale}

\begin{defn}{Omotopie di Cammini}{}
	Sia $(X, \mathcal T)$ spazio topologico e $a,b \in X$ e $\alpha, \beta \in \Omega(X, a, b)$. Una \textbf{Omotopia di Cammini} $\alpha$ e $\beta$ è una funzione:
	\[ F:[0,1] \times [0,1] \to X \]
	Tale che:
	\[ F(t,0) = \alpha(t)\qquad F(t,1) = \beta(t) \qquad F(0,s) = a\qquad F(1,s) = b \]
\end{defn}

\begin{oss}
	$\forall s \in [0,1]$, la funzione $t \mapsto F(t,s)$ è un cammino tra $a$ e $b$.
\end{oss}

\begin{oss}
	Una omotopia di cammini è un modo "per deformare" un cammino in un altro:
	\begin{center}
		\begin{tikzpicture}
			\draw (-3,0) node(i){$[0,1]$};
			\draw (0,0) arc (180:540:3cm and 2cm) node(x)[left]{$X$};
			\draw[->, green] (i) to[out = 45, in = 135] node[pos = 0.5, above]{$\alpha$} (x);
			\draw[->, blue] (i) to[out = -45, in = -135] node[pos = 0.5, below]{$\beta$} (x);
			\filldraw (2,-1) circle(1pt) node[below]{$a$} (4,1) circle(1pt) node[above]{$b$};
			\draw[->, green] (2,-1) to[out = 135, in = 180] node[above]{$\alpha$} node(1a)[pos = 0.25]{} node(2a)[pos = 0.5]{} node(3a)[pos = 0.75]{} (4,1);
			\draw[->, blue] (2,-1) to[out = 0, in = -90] node[below]{$\beta$} node(1b)[pos = 0.25]{} node(2b)[pos = 0.5]{} node(3b)[pos = 0.75]{} (4,1);
			\draw[->] (2,-1) -- (4,1);
			\draw[->] (1a) -- (1b);
			\draw[->] (2a) -- (2b);
			\draw[->] (3a) -- (3b);
		\end{tikzpicture}
	\end{center}
	Analogamente al caso delle omotopie si definiscono le inversioni e le giunzioni di omotopie:
	\[ i(F)(t,s) = F(t, 1-s)\qquad F \times G(t,s) = \begin{cases}
		F(t,2s) & s \in [0, \frac 12]\\
		G(t, 2s-1) & s \in[\frac 12, 1]
	\end{cases}\]
\end{oss}

\begin{lemma}{}{}
	L'omotopia di cammini è una relazione di equivalenza $\sim$ tra cammini
\end{lemma}

Questo lemma si dimostra esattamente come con le omotopie

\begin{defn}{Cammino Costante}{}
	Il \textbf{Cammino Costante} $1_a$ è il laccio in $a$ $1_a \in \Omega(X, a, a)$ tale che:
	\[ f(t) = a\qquad \forall t \]
\end{defn}

\begin{es}
	Sia $X$ un insieme convesso in $\mathbb R^n$ e sia $a \in \Omega(X, a,a)$ con $a \in X$. Allora $\alpha \sim 1_a$ tramite:
	\[ F(t,s) = \alpha(t) + s(a-\alpha(t)) \]
\end{es}

\begin{defn}{Gruppo Fondamentale}{}
	Sia $X$ spazio topologico e $a \in X$. Si definisce il \textbf{Gruppo Fondamentale} di $X$ in $a$ come:
	\[ \pi_1(X, a) = \Omega(X, a, a)/_\sim \]
\end{defn}

\begin{es}
	Sia $X$ convesso e $a \in X$, allora $\pi_1(X, a) = \{[1_a]\}$
\end{es}

\begin{prop}{}{}
	Siano $\alpha, \alpha' \in \Omega(X, a, b)$ e $\beta, \beta' \in \Omega(X, b, c)$ tali che $\alpha \sim \alpha'$ e $\beta \sim \beta'$. Allora:
	\[ i(\alpha) \sim i(\alpha')\qquad \text e \qquad \alpha*\beta \sim \alpha'*\beta'\]
\end{prop}

\begin{proof}
	Sia $F$ continua tra $\alpha$ e $\alpha'$ e $G$ continua tra $\beta$ e $\beta'$. Allora abbiamo che $F(1-t,s)$ è un'omotopia tra $i(\alpha)$ e $i(\alpha')$. Infatti per $s = 0$ ho $i(\alpha(t))$, per $s = 1$ ho $i(\alpha'(t))$, mentre per $t = 0$ ho $b$ e per $t = 1$ ho $a$.

	Costruiamoci manualmente l'omotopia tra $\alpha *\beta$ e $\alpha'*\beta'$. Definiamo $H(t,s)$ come:
	\[ H(t,s) = \begin{cases}
		F(2t,s) & t \in [0,\frac 12]\\
		G(2t-1,s) & t \in [\frac 12,1]
	\end{cases}\]
	Questa è l'omotopia cercata in quanto abbiamo che:
	\[ H(t,0) = \alpha*\beta(t)\qquad H(t,1) = \alpha'*\beta'(t)\qquad H(0,s) = a\qquad H(1,s) = b \]
\end{proof}

\begin{lemma}{di Riparametrizzazione}{}\label{ripar}
	Sia $\alpha$ un cammino e sia $\varphi:[0,1] \to [0,1]$ continua con $\varphi(0) = 0$ e $\varphi(1) = 1$. Allora $\alpha \sim \alpha \circ \varphi$
\end{lemma}
\begin{proof}
	Basta prendere infatti $F(t,s) = \alpha((1-s)t + s\varphi(t))$. In questo modo abbiamo che:
	\[ F(t,0) = \alpha(t)\qquad F(t,1) = \alpha \circ \varphi(t)\qquad F(0,s) = \alpha(0)\qquad F(1,s) = \alpha(1) \]
\end{proof}

\begin{prop}{}{}\label{prop1}
	La congiunzione $*$ è associativa a meno di giunzione
\end{prop}

\begin{proof}
	Mostriamo che $\alpha*(\beta*\gamma) = (\alpha*\beta)*\gamma$. In particolare abbiamo che:
	\[
		\alpha*(\beta*\gamma)(t) =
		\begin{cases}
			\alpha(2t) & t \in [0,\frac 12]\\
			\beta(4t-2) & t \in [\frac 12, \frac 14]\\
			\gamma(4t-3) & \in[\frac 34 ,0]
		\end{cases} \qquad (\alpha*\beta)+\gamma(t)
		\begin{cases}
			\alpha(4t) & t \in [0,\frac 12]\\
			\beta(4t-1) & t \in [\frac 14, \frac 12]\\
			\gamma(2t-1) & t \in [\frac 12,1]
		\end{cases}
	\]
	Allora tramite la funzione $\varphi:[0,1] \to [0,1]$ con:
	\[
		\varphi(t) =
		\begin{cases}
			2t & t \in[0, \frac 14]\\
			t + \frac 14 & t \in [\frac 14, \frac 12]\\
			\frac{t+1}2 & t \in [\frac 12,1]
		\end{cases}
	\]
	Otteniamo che $(\alpha*\beta)*\gamma(t) = \alpha*(\beta*\gamma)(\varphi(t))$. Dal lemma precedente \ref{ripar}.6 otteniamo l'omotopia che stavamo cercando
\end{proof}

\begin{prop}{}{}\label{prop2}
	Dato $\alpha \in \Omega(X, a,b)$, allora:
	\[ \alpha*1_b \sim \alpha \sim 1_a*\alpha \]
\end{prop}
\begin{proof}
	Utilizziamo il lemma di Riparametrizzazione \ref{ripar}.6. Abbiamo che:
	\[ 1_a * \alpha(t) = \alpha(\varphi_1(t)) \]
	Dove $\varphi_1(t)=0$ se $t \in [0, \frac 12]$ e $\varphi_1(t) = 2t-1$ se $t \in [\frac 12,1]$
	\[ \alpha*1_b(t) = \alpha(\varphi_2(t)) \]
	Dove $\varphi_2(t) = 2t$ se $t \in [0,\frac 12]$ e $\varphi_2(t) = 1$ se $t \in [\frac 12, 1]$
\end{proof}

\begin{lemma}{del Triangolo}{}\label{triang}
	Siano $p_1,p_2,p_3 \in \mathbb R^n$. Sia il triangolo:
	\[ T = \left\{t_1p_1 + t_2p_2 + t_3p_3 : \sum_{i = 1}^3 t_i = 1, t_i\geq0\right\} \]
	Sia poi $f:T \to X, \mathcal T)$ con $f$ funzione continua e siano le funzioni $f_{i,j}:[0,1] \to X$ date da:
	\[ f_{i,j}(t) = f(tp_j + (1-t)p_i) \]
	Allora:
	\[ f_{1,3} \sim f_{1,2} * f_{2,3} \]
\end{lemma}
\begin{proof}
	Graficamente abbiamo che:
	\begin{center}
		\begin{tikzpicture}
			\filldraw (0,0) circle(1pt) node[below]{$p_1$} (2,0) circle(1pt) node[below]{$p_2$} (1,1.5)circle(1pt) node[above]{$p_3$};
			\begin{scope}[decoration={
				markings,
				mark=at position 0.5 with {\arrow{>}}}]
				\draw[thick, postaction={decorate}] (0,0)--(1,1.5);
				\draw[thick, postaction={decorate}] (0,0)--(2,0);
				\draw[thick, postaction={decorate}] (2,0)--(1,1.5);
				\draw[green, postaction={decorate}] (0,0)--(1,0) to[out = 0, in = -60] (1.5,0.75) -- (1,1.5);
				\draw[green, postaction={decorate}] (0,0)--(0.5,0.05) to[out = 5, in = -60] (1.25,1) -- (1,1.5);
				\draw[green, postaction={decorate}] (0,0) to[out = 0, in = -60] (1,1.5);
				\draw[green, postaction={decorate}] (0,0) to[out = 30, in = -90] (1,1.5);
			\end{scope}
			\draw (4,0.75) arc(-180:180:2cm and 1.5cm) node[left]{$X$};
			\filldraw (5,0.5) circle(1pt) node[below]{$f(p_1)$} (6, 1.75) circle(1pt) node[right]{$f(p_3)$} (7,0.5) circle(1pt) node[right]{$f(p_2)$};
			\draw[->] (5,0.5) to[out = 90, in = -90] (6, 1.75);
			\draw[->] (5,0.5) to[out = 45, in = -135] (7, 0.5);
			\draw[->] (7, 0.5) to[out =90, in = -60] (6,1.75);
			\draw[->] (2,1) to[bend left = 30] node[pos = 0.5, above]{$f$} (3.5,0.75);
		\end{tikzpicture}
	\end{center}
	Definiamo $q(t,s)$ come:
	\[ q(t,s)=
		\begin{cases}
			(1-t-ts)p_1 + 2tsp_2 + (t-ts)p_3 & \text{se }t \in[0,\frac 12]\\
			(1-t-s+ts)p_1 + 2(1-t)sp_2 + (t-s+ts)p_3 & \text{se }t \in[\frac 12, 1];
		\end{cases}
	\]
	$q$ ha immagine in $T$ ed è un'omotopia tra $\overline{12}*\overline{23}$ e $\overline{13}$ dentro $T$, quindi possono definire:
	\[ F(t,s) = f(q(t,s)) \qquad \Rightarrow \qquad f_{1,2}*f_{2,3} \sim f_{1,3}\]
\end{proof}

\begin{prop}{}{}\label{prop3}
	$\forall \alpha \in \Omega(X, a, b)$ vale $\alpha*i(\alpha) \sim 1_a$
\end{prop}
\begin{proof}
	Uso un triangolo degenere con $p_1 = p_3 = 0$ e $p_2 = 1$ e poi mi basta applicare il lemma del triangolo \ref{triang}.9. Definiamo poi l'applicazione:
	\[ f:T = [0,1 \to X \qquad t \mapsto \alpha(t)] \]
	In questo modo abbiamo che $f_{1,3}(t) = \alpha(0) = a$, per cui abbiamo che $f_{1,3} = 1_a$. Per le altre abbiamo che:
	\[ f_{1,2}(t)= \alpha(t)\qquad \text e \qquad f_{2,3}(t) = i(\alpha(t)) \]
	Quindi per il lemma del triangolo si ha che $\alpha*i(\alpha) = 1_a$
\end{proof}

Graficamente quello che abbiamo fatto nella dimostrazione è stato:
\begin{center}
	\begin{tikzpicture}
		\begin{scope}[decoration={
			markings,
			mark=at position 0.3 with {\arrow{>}}}]
			\draw[postaction = {decorate}] (-2,0) to[out = -40, in = 130] node[pos = 0.3, above]{$\alpha$} (2,0);
			\draw[postaction = {decorate}] (2,0) to[out = 140, in = -50] node[pos = 0.3, below]{$i(\alpha)$} (-2,0);
		\end{scope}
	\end{tikzpicture}
\end{center}

\begin{thm}{Gruppo Fondamentale}{}{}
	$\pi_1(X, a)$ è un gruppo con l'operazione $*$, elemento neutro $1_a$ e vale:
	\[ \forall[\alpha] \in \pi_1(X, a)\quad [\alpha]^{-1} = [i(\alpha)] \]
\end{thm}

Il gruppo fondamentale prende anche il nome di \textbf{Gruppo di Poincaré} o \textbf{Primo Gruppo di Omotopia}

\begin{proof}
	$*$ è ben definito nel quoziente.\\
	$*$ è associativa per quanto visto nella proposizione \ref{prop1}.7.\\
	$1_a$ è l'elemento neutro per quanto visto nella proposizione \ref{prop2}.8\\
	$[i(\alpha)] = [\alpha]^{-1}$ per quanto visto nella proposizione \ref{prop3}.10
\end{proof}

\subsection{Proprietà del Gruppo Fondamentale}

\begin{oss}
	Se abbiamo che $\Omega(X, a, b) \neq \varnothing$ allora abbiamo che:
	\begin{center}
		\begin{tikzpicture}
			\begin{scope}[decoration={
				markings,
				mark=at position 0.5 with {\arrow{>}}}]
				\draw[postaction = {decorate}] (0,0) to[out = -40, in = 90] node[pos = 0.5, above]{$\gamma$} (2,0);
				\draw[postaction = {decorate}] (0,0) to[out = 60, in = 90] (-1,0) node[left]{$\alpha$} to[out = -90, in = -60] (0,0);
			\end{scope}
			\filldraw (0,0) circle(1pt) node[above right]{$a$} (2,0) circle(1pt) node[right]{$b$};
		\end{tikzpicture}
	\end{center}
	in particolare abbiamo che:
	\[i (\gamma) * \alpha *\gamma\]
\end{oss}

\begin{prop}{}{}
	Dato $\gamma \in \Omega(X, a,b)$ la funzione
	\[ \gamma_{\sharp}:\pi_1(X, b) \to \pi_1(X, a)\]
	definita come:
	\[ \gamma_{\sharp}([\alpha]) = [i(\gamma)*\alpha*\gamma]\]
	è un isomorfismo di gruppi
\end{prop}
\begin{proof}
	Intanto notiamo che $[i(\gamma)*\alpha*\gamma]$ non dipende dalla scelta di $[\alpha]$. Quindi sicuramente $\gamma_\sharp$ è ben definita. Inoltre:
	\[ \gamma_\sharp(\alpha)*\gamma_\sharp(\beta) = [i(\gamma)*\alpha*\gamma]*[i(\gamma)*\beta*\gamma] = [i(\gamma)*\alpha*\beta*\gamma] = \gamma_\sharp(\alpha*\beta) \]
	Quindi $\gamma_\sharp$ è un omomorfismo di gruppi. In realtà possiamo dire che è un isomorfismo in quanto:
	\[ i(\gamma)_\sharp (\gamma_\sharp(\alpha)) = [\gamma*i(\gamma)*\alpha*\gamma*i(\gamma)] = [\alpha]\]
\end{proof}

\begin{defn}{Classe di Isomorfismo}{}
	Se $X$ è connesso per archi è ben definita la \textbf{Classe di Isomorfismo} di $\pi_1(X, a), \forall a \in X$, che viene definita come $\pi_1(X)$
\end{defn}

\begin{defn}{Semplicemente Connessione}{}
	Se $X$ è connesso per archi e $\pi_1(X) = \{0\}$, allora $X$ si dice \textbf{Semplicemente Connesso}
\end{defn}

\begin{es}
	Sia $X \subseteq \mathbb R^n$ convesso, allora $X$ è semplicemente connesso. Quindi anche $(\mathbb R^n, \mathcal T_\mathcal E)$ è semplicemente connesso, così come le sue palle aperte
\end{es}

\begin{prop}{}{}
	Sia $f:X \to Y$ continua e siano $\alpha, \beta \in \Omega(X, a, b)$ e $\gamma \in \Omega(X, b,c)$. Allora:
	\begin{enumerate}
		\item $\alpha \sim \beta \Rightarrow f(\alpha) \sim f(\beta)$
		\item $f(\alpha*\gamma) = f(\alpha) = f(\gamma)$
	\end{enumerate}
\end{prop}
\begin{proof}
	\fbox{2} Questa affermazione è vera per definizione.

	\fbox{1} Sia $F$ omotopia tra $\alpha$ e $\beta$, allora $f(F)$ è omotopia tra $f(\alpha)$ e $f(\beta)$, infatti:
	\[ f(F(t,0)) = f(\alpha, t)\qquad f(F(t,1)) = f(\beta,t)\qquad f(F(0,s)) = f(a)\qquad f(F(1,s)) = f(b) \]
\end{proof}

\begin{thm}{}{}
	$\pi_1$ è un funtore tra $\bm{Top_{pt}}$ e $\bm{Grp}$ dove $\bm{Top_{pt}}$ ha come elementi le coppie $(X, \mathcal T)$ spazio topologico e $x_0 \in X$ e i morfismi:
	\[ f:(X, \mathcal T_X, x_0) \to (Y, \mathcal T_Y, y_0)\qquad \text{tali che }f(x_0) = y_0 \]
\end{thm}
\begin{proof}
	Definiamo $f:(X, \mathcal T_X, x_0) \to (Y, \mathcal T_Y, y_0)$. Allora sappiamo che:
	\[ \pi_1(f):\pi_1(X, x_0) \to \pi_1(Y, y_0)\qquad [x] \mapsto [f(x)] \]
	è un omomorfismo di gruppi e che
	\[[g\circ f \circ \alpha] = \pi_1(g) \circ \pi_1(f)[\alpha]\qquad \text{e}\qquad \pi_1(id_X) = id_{\pi_1(X, x_0)}\]
\end{proof}

Da adesso in avanti iniziamo a chiamare $\pi_1(f)$ come $f_*$ e la chiameremo "$f$ pushforward" oppure $f$ sospinta

\begin{prop}{}{}
	Dati $A$ e $B$ spazi topologici e $a_0 \in A$ e $b_0 \in B$. Allora:
	\[ \pi_1(A \times B, (a_0, b_0)) \cong \pi_1(A,a_0) \times \pi_1(B, b_0)\]
	\textit{Il prodotto in $\bm{Top_{pt}}$} commuta il funtore $\pi_1$
\end{prop}
\begin{proof}
	Tramite la proposizione universale del prodotto abbiamo un isomorfismo:
	\[ \Omega(A \times B, (a_0,b_0), (a_0,b_0)) \cong \Omega(A, a_0, a_0)\times \Omega(B, b_0,b_0) \qquad \gamma \mapsto (p_A \circ \gamma, p_B \circ \gamma)\]
	Lo stesso discorso vale anche per le omotopie. Infatti se $\gamma \sim \beta$ tramite $F$, allora:
	\[ p_1 \circ \gamma \sim p_2 \circ \beta \text{ tramite }p_1 \circ F\qquad p_2 \circ \gamma \sim p_2 \circ \beta\text{ tramite }p_2 \circ F \]
\end{proof}

\begin{prop}{}{}
	Siano $f,g:X \to Y$ continue con $f \sim g$ tramite omotopia $F$ e sia $\gamma:[0,1]\to Y$ definito da $\gamma(s) = F(a_0,s)$, allora:
	\[ \gamma_\sharp \circ f_* = g_* \]
\end{prop}

Graficamente abbiamo che:
\begin{center}
	\begin{tikzpicture}
		\draw (-2,0) -- (0,0) -- (1,1) node[right]{$X$} -- (-1,1) -- (-2,0);
		\filldraw (0,0.5) circle(1pt) node[below]{$a_0$};
		\draw (4,1) to[out = 135, in = 210] node(f)[pos = 0.4]{} (7,1);
		\draw (3,0.5) to[out = 0, in = -120] node(g)[pos = 0.6]{} (6,0);
		\filldraw (f) circle(1pt) node[above]{$f(a_0)$};
		\filldraw (g) circle(1pt) node[below]{$g(a_0)$};
		\draw[->] (2,1) to[out = 45, in = 135] (2.5,1);
		\draw[->, blue] (f) to[out = -90, in = 90] (g);
	\end{tikzpicture}
\end{center}

\begin{proof}
	Dobbiamo mostrare che $\forall \alpha \in \pi_1(X, a_0)$ che:
	\[ [g(\alpha)] = [i(\gamma) * f(\alpha)*\gamma] \qquad \Leftrightarrow \qquad [\gamma\circ g(\alpha)] = [f(\alpha)\circ \gamma]\]
	Consideriamo $G:[0,1]\times[0,1] \to Y$ tale che $(t,s)\mapsto F(\alpha(t), s)$. Abbiamo allora che:
	\begin{center}
		\begin{tikzpicture}
			\begin{scope}[decoration={
				markings,
				mark=at position 0.5 with {\arrow{>}}}]
				\draw[postaction = {decorate}] (0,0) -- node[pos = 0.5, below]{$f(x)$} (2,0);
				\draw[postaction = {decorate}] (0,2) -- node[pos = 0.5, above]{$g(x)$} (2,2);
			\end{scope}
			\draw (0,0) -- node[pos = 0.5, left]{$\gamma$}(0,2);
			\draw (2,0) -- node[pos = 0.5, right]{$\gamma$}(2,2);
			\draw (0,0) -- node[pos = 0.5, above]{$\delta$}(2,2);
		\end{tikzpicture}
	\end{center}
	Quindi abbiamo che:
	\[ G(t,0) = f(\alpha(t))\qquad G(t,1)=g(\alpha(t))\qquad G(0,s) = \gamma(s)\qquad G(1,s) = \gamma(s) \]
	Quindi per il lemma \ref{triang}.9 abbiamo:
	\[ f(\alpha)*\gamma = \delta = \gamma*g(\alpha)\ \]
	Quindi abbiamo sostanzialmente finito
\end{proof}

\begin{cor}{}{}
	Se $f\sim id_X$, allora $f_*$ è un isomorfismo.
\end{cor}

\subsection{Teorema di Van Kampen}

\begin{lemma}{}{}\label{fatto1}
	Siano $A \xrightarrow{f} B \xrightarrow {g}C \xrightarrow{h}D$ in $\bm{Set}$. Se $g \circ f$ è invertibile e $h\circ g$ è iniettiva, allora $f$ è invertibie.
\end{lemma}

\begin{thm}{}{}
	Se $f:X \to Y$ è un'equivalenza omotopica, allora $f_*$ è un isomorfismo
\end{thm}
\begin{proof}
	Sia $g:Y \to X$ tale che $f \circ g \sim id_Y$ e $g \circ f \sim id_X$. Allora abbiamo che $\forall x_0 \in X$:
	\[ \pi_1(X, x_0) \xrightarrow{f_*} \pi_1(Y, f(x_0)) \xrightarrow{g_*} \pi_1(X, g \circ f(x_0)) \xrightarrow{f_*}\pi_1(Y, f\circ g \circ f(x_0)) \]
	Sappiamo che $f \circ g \sim id_Y$, allora $f_* \circ g_* = (f \circ g)_*$ è invertibile. Sappiamo anche che $g \circ f \sim id_X$, quindi $g_*\circ f_* = (f \circ g)_*$ è invertibile. Quindi per il lemma \ref{fatto1}.1, abbiamo che $f_*$ è invertibile, quindi è un isomorfismo.
\end{proof}

\begin{es}
	Se $X$ è invertibile, allora $\forall x_0 \in X$ si ha che:
	\[ \pi_1(X, x_0) = \{1_{x_0}\} \]
\end{es}

\begin{es} $\pi_1(\text{Cilindro})\cong \pi_1(S^1)$. Infatti:
	\begin{center}
		\begin{tikzpicture}
			\draw (0,2) arc (180:540: 1cm and 0.25cm);
			\draw[blue] (2,1) arc (360:180: 1cm and 0.25cm) node[left]{$S^1$};
			\draw[blue, dashed] (0,1) arc (180:0:1cm and 0.25cm);
			\draw (0,0) arc(180:360:1cm and 0.25cm);
			\draw[dashed] (0,0) arc(180:0:1cm and 0.25cm);
			\draw (0,2) -- (0,0) (2,2) -- (2,0);
			\draw[->] (-0.25,0) -- (-0.25,0.75);
			\draw[->] (-0.25,2) -- (-0.25,1.25);
			\draw[->] (2.25,0) -- (2.25,0.75);
			\draw[->] (2.25,2) -- (2.25,1.25);
			\draw[->] (2.5,1) --++ (1,0);
			\draw[blue] (6,1) arc(0:360:1cm and 0.25cm) node[right]{$S^1$};
		\end{tikzpicture}
	\end{center}
	Sostanzialmente quello che facciamo è retratto per deformazione su $S^1$ tramite:
	\[ R((x,y), t) = (x,ty)\qquad \text{con }x \in S^1, y \in I \]
\end{es}

\begin{es}
	$\pi_1(\mathbb R^{n+1} \setminus \{0\}, x_0)\cong \pi_1(S^n, x_0)$, per $n\geq 1$. La scelta di $x_0$ è del tutto ininfluente, in quanto abbiamo che i due insiemi sono connessi per archi. Sappiamo effettivamente che i due gruppi sono isomorfi, ma non sappiamo quale sia tale gruppo
\end{es}

\begin{es}
	Prendiamo la retta dei numeri reali $\mathbb R$, questa può essere scritta come unione di due intervalli aperti non disgiunti del tipo:
	\[ \mathbb R = (-\infty, 1)\cup (-1, + \infty) \]
	Questi due aperti hanno intersezione $(-1,1)$. Sappiamo inoltre che sono convessi, quindi connessi per archi e la stessa cosa vale anche per la loro intersezione. Quindi possiamo dedurre (cosa che già sapevamo) che $\mathbb R$ è convesso, quindi $\pi_1(\mathbb R, x_0)$ è banale
\end{es}

\begin{es}
	Facciamo una cosa simile però con una circonferenza e prendiamo due aperti nel seguente modo:
	\begin{center}
		\begin{tikzpicture}
			\draw[thick] (0,0) circle (1cm);
			\filldraw[green, very nearly transparent] (100:0.9) arc (100:-100:0.9) to[out = 170, in = 170] (-100:1.1) arc (-100:100:1.1) to[out = 190, in = 190] (100:0.9);
			\filldraw[blue, very nearly transparent] (80:0.9) arc (80:280:0.9) to[out = 10, in = 10] (-80:1.1) arc (280:80:1.1) to[out = -10, in = -10] (80:0.9);
		\end{tikzpicture}
	\end{center}
	Gli aperti sono i due insiemi colorati. Notiamo che sono entrambi connessi per archi ma la loro intersezione non lo è. Quindi sicuramente $\pi_1(S^1)$ non è banale
\end{es}

\begin{es}
	Prendiamo invece una sfera di dimensioni superiori (3 nel disegno) e prendiamo gli aperti nel seguente modo:
	\begin{center}
		\begin{tikzpicture}
			\begin{scope}
				\clip (0,0) circle(1cm);
				\draw[dashed] (180:1) arc(180:0:1cm and 0.25cm) (90:1) arc(90:-90:0.25cm and 1cm);
				\fill[green, very nearly transparent] (195:1) arc(180:360:0.97 and 0.23) arc(-15:195:1);
				\draw[green] (195:1) arc (180:360:0.97 and 0.23);
				\draw[green, dashed] (195:1) arc(180:0:0.97 and 0.23);
				\fill[blue, very nearly transparent] (150:1) arc(150:390:1 and 0.25) arc (390:150:1);
				\draw[blue] (150:1) arc(150:390:1 and 0.25) arc (390:150:1);
				\draw[blue, dashed] (165:1) arc(180:0: 0.97 and 0.23);
			\end{scope}
			\draw (0,0) circle(1cm) (180:1) arc (180:360:1cm and 0.25cm) (90:1) arc (90:270:0.25cm and 1cm);
		\end{tikzpicture}
	\end{center}
	Anche qui come prima, prendiamo le due parti colorate. Entrambe sono semplicemente connesse, quindi con gruppo fondamentale banale. Notiamo però che vale:
	\[ \pi_1(\text{Intersezione}) \cong S^{n-1} \]
\end{es}

\begin{lemma}{}{}\label{fatto2}
	Sia $\alpha:[0,1] \to X$ continua e sia $\{A_i\}$ un ricoprimento aperto di $X$. Allora esiste un numero $n \in \mathbb N$ tale che $\forall j \in \{0,...,n-1\}$ si ha che:
	\[ \left.\alpha\right|_{[\frac jn, \frac{j+1}n]} \text{ sia tutto contenuto in uno degli aperti} \]
\end{lemma}

\begin{es}
	$\pi_1((S^1)^n) \cong \pi_1(S^1)^n$. Da questo possiamo vedere subito come un toro non sia isomorfo ad un cilindro, ma:
	\[ \pi_1(\text{Toro}) \cong \pi_1(S^1)^2 \]
\end{es}

\begin{lemma}{}{}\label{fatto3}
	Sia $\alpha$ un cammino e $t_0 \in [0,1]$. Dato $\alpha_1(t) = \alpha(t_0t)$ e $\alpha_2(t) = \alpha(t_0 + (1-t)t_0)$, abbiamo allora che:
	\[ \alpha \sim \alpha_1 * \alpha_2 \]
\end{lemma}

\begin{thm}{di Van Kampen (Prima parte)}{}{}\label{VK}
	Sia $X$ uno spazio topologico e siano $A,B\subseteq X$ aperti tali che:
	\begin{itemize}
		\item $A\cup B = X$
		\item $\{A, B, A\cap B\}$ siano connessi per archi
	\end{itemize}
	Siano $x_0 \in A \cap B$ e siano $i_A:A \to X$ inclusione e  $i_B:B \to X$ inclusione, allora:
	\[ \pi_1(X, x_0) = \langle i_{A*}(\pi_1(A, x_0)), i_{B*}(\pi_1(B, x_0)) \rangle \]
	Cioè:
	\begin{enumerate}
		\item Ogni laccio in $X$ centrato in $x_0$ è omotopo alla giunzione di infiniti lacci o in $A$ o in $B$.
		\item La funzione $\pi_1(A, x_0)*\pi_1(B, x_0) \xrightarrow{(i_{A*}, i_{B*})} \pi_1(X,x_0)$ è una suriezione
	\end{enumerate}
\end{thm}

\begin{cor}{}{}
	Sia $X = A \cup B$ tali che $A, B, A\cap B$ siano connessi per archi e $A$ e $B$ semplicemente connessi. Allora $X$ è semplicemente connesso
\end{cor}

Lo si potrebbe dimostrare osservando che $\pi_1(X)$ è banale perché prodotto libero di $\pi_1(A)$ e di $\pi_1(B)$.

\begin{cor}{}{}
	$S^n$ con $n>1$ e $\mathbb P^n(\mathbb C)$ sono semplicemente connessi, ma non contraibili
\end{cor}

\begin{proof}
	\fbox{$S^n$} Consideriamo $S^n \subseteq \mathbb R^{n+1}$. Questo lo possiamo scrivere come unione di due aperti non disgiunti come:
	\[ A = \left\{x \in S^n: x_{n+1}<\frac 12\right\}\qquad \text e \qquad B = \left\{ x \in S^n: x_{n+1}>-\frac 12 \right\} \]
	Abbiamo che $A$ e $B$ sono connessi per archi (ma anche contraibili). Abbiamo poi che $A\cap B$ è connesso per archi se $n>1$. Quindi per il corollario precedente abbiamo che $A$ e $B$ sono semplicemente connessi, quindi anche $X$ lo è.

	\fbox{$\mathbb P^n(\mathbb C)$}. Quest'insieme può essere scritto come:
	\[ \mathbb P^n(\mathbb C) = U_0\cup \cdots \cup U_n\]
	Dove abbiamo che $U_i = \{[x_0:\cdots:x_n]: x_i \neq 0\}$ sono le carte affini. Consideriamo poi $\varphi_i:U_i \to \mathbb C^n$ omeomorfismi definiti da:
	\[ [x_0:\cdots :x_n] \mapsto \left(\frac {x_0}{x_i}, \cdots, \frac{x_{i-1}}{x_i}, \frac{x_{i+1}}{x_i}, \cdots \frac {x_n}{x_i}\right) \]
	Abbiamo che ogni $U_i$ è semplicemente connesso.\\
	Infatti, consideriamo prima $U_0 \cap U_1$. La sua immagine attraverso $\varphi_0$ è:
	\[ \{(z_1,...,z_n) \in \mathbb C: z_1 \neq 0\} \cong \mathbb C^* \times \mathbb C^{n-1} \]
	\textit{Se $\mathbb K = \mathbb R$, allora avremmo che $U_0\cap U_1$ non è connesso per archi}.\\
	Per il teorema di Van Kampen \ref{VK}.5, abbiamo che $U_0 \cup U_1$ è semplicemente connesso.\\
	Poi continuiamo e consideriamo $U_0\cup U_1$ e $U_2$ aperti e semplicemente connessi. Allora abbiamo che:
	\[ (U_0 \cup U_1) \cap U_2 = (U_0 \cap U_2)\cup (U_1 \cap U_2) \]
	e abbiamo che:
	\[ \varphi_0(U_0\cap U_1 \cap U_2) \cong (\mathbb C^*)^2 \times \mathbb C^{n-2} \]
	Quindi sempre per Van Kampen abbiamo che $U_0 \cap U_1 \cap U_2$ è semplicemente connesso.\\
	Quindi iterando il numero necessario di volte, abbiamo che $\mathbb P^(\mathbb C)$ è semplicemente connesso
\end{proof}

Dimostriamo ora il teorema di Van Vampen

\begin{proof}[Dimostrazione del Teorema \ref{VK}.5]
	Sia $\alpha \in \Omega(X, x_0, x_0)$. Vogliamo mostrare che:
	\[ \alpha \sim \gamma_0 * \gamma_1*\cdots * \gamma_{n-1} \]
	con $\gamma_i \in \Omega(A, x_0,x_0)$ oppure $\gamma_i \in \Omega(B, x_0, x_0)$. Graficamente abbiamo che:
	\begin{center}
		\begin{tikzpicture}
			\draw (0,0) node(0){} to[out = 180, in = 120] (-2,-1)node(1){} to[out = -60, in = 180] (-1,-2)node(2){} to [out = 0, in = 0] (1, -1)node(3){} to[out = 180, in = 0](0,0);
			\draw (-2.5,0.5) -- (-2.5,-2.5) -- (1.5,-2.5) -- (1.5,0.5) -- (-2.5,0.5);
			\filldraw (0) circle(1pt) node[above]{$x_0$} (1) circle(1pt) node[left]{$x_1$} (2) circle(1pt) node[below]{$x_2$} (3)circle(1pt) node[above]{$x_3$};
			\filldraw[green, very nearly transparent] (-2.5,0.5) -- ++ (0,-3) --++ (3,0) --++ (0,3);
			\filldraw[blue, very nearly transparent] (1.5,0.5) --++ (0,-3) --++ (-3,0) --++ (0,3);
		\end{tikzpicture}
	\end{center}
	Per il lemma \ref{fatto2}.3, sappiamo che esiste $n$ tale che $\forall u \in\{0,...,n-1\}$:
	\[Im\left( \left. \alpha \right|_{[\frac in, \frac{i+1}n]} \right)\subseteq A \qquad \text{oppure}\qquad Im\left( \left. \alpha \right|_{[\frac in, \frac{i+1}n]} \right)\subseteq B\]
	Siccome $A, B, A \cap B$ sono connessi per archi, esistono dei cammini $\beta_i$ tra $x_0$ e $x_i$ con $\beta_i$ contenuto in $A$ se $x_i \in A$, in $B$ se $x_i \in B$, oppure in $A \cap B$ se $x_i \in A \cap B$. Pongo:
	\[ \gamma_0 = \alpha_0 * i (\beta_1) \quad \gamma_1 = \beta_1 * \alpha_1 * i(\beta_2)\quad \cdots \quad \gamma_{n-2} = \beta_{n-2}*\alpha_{n-2}*i(\beta_{n-1}) \quad \gamma_{n-1} = \beta_{n-1} * \alpha_{n-1} \]
	Allora segue che:
	\[ \gamma_0 * \gamma_1 * \cdots * \gamma_{n-1} \sim \alpha_0 * \alpha_1 * \cdots * \alpha_{n-1} \sim \alpha \]
	Quindi per il lemma \ref{fatto3}.4 (iterato diverse volte)
\end{proof}

\subsection{Sollevamento}

\begin{thm}{}{}\label{molla}
	$\pi_1(S^1, p) \cong \mathbb Z$ con $\pi_1(S^1, 1) = \{\gamma_n\}$ e $\gamma_n:[0,1]\to S^1$, definita come $t\mapsto e^{2 \pi i nt}$
\end{thm}

Graficamente può essere vista come

\begin{center}
	\begin{tikzpicture}[domain = -3:3]
		\draw[decoration={aspect=0.3, segment length=7.5mm, amplitude=1cm,coil},decorate,opacity=0.9, thin, rotate = -90] (0,0) -- (4,0);
		\draw[thick, rotate = -90] (2,0) circle(0.25cm and 1cm);
	\end{tikzpicture}
\end{center}

Consideriamo in poi consideriamo la funzione $e$ come:
\begin{align*}
	e: \mathbb R \to & S^1 \subseteq \mathbb C\\
	t \mapsto & e^{2 \pi i t}
\end{align*}

\begin{lemma}{}{}
	Sia $U\subseteq S^1$ aperto connesso con $U \neq S^1$. Allora esiste $V \subseteq \mathbb R$ aperto tale che:
	\begin{enumerate}
		\item $e|_V:V \to U$ è un omeomorfismo
		\item Vale l'uguaglianza:
			\[ e^{-1}(U) = \coprod_{n \in \mathbb Z}n + V \]
			Dove $n+V = \{x+n: x \in V\}$
	\end{enumerate}
\end{lemma}

\begin{proof}
	\fbox{1} A meno di una rotazione di $S^1$ (che comunque sono omeomorfismi tra $S^1$), posso supporre che $U\subseteq S^1 \setminus \{1\}$.\\
	Sia adesso $V = e^{-1}(U)\cap [0,1] = e^{-1}(U) \cap (0,1)$ (che è un aperto). Consideriamo adesso $e|_V:V \to U$. Abbiamo che questa è una funzione aperta (in quanto $e$ è aperta e vale $W \subseteq V$ aperto $\Leftrightarrow W\subseteq \mathbb R$ aperto).\\
	Possiamo dire che $e|_V$ è iniettiva in quanto lo è $(e|_{(0,1)})$, è suriettiva in quanto $(0,1]$ è un insieme di rappresentanti dell'azione data da $\mathbb Z$ su $\mathbb R$.\\
	Quindi $e|_V$ è un omeomorfismo.

	\fbox{2} Abbiamo che vale:
	\[ e^{-1}(V) \cap [n,n+1] = e^{-1}(V)\cap (n,n+1) = n+V \]
	Inoltre $e|_{n+V}:n+V \to U$ è ancora un omeomorfismo (lo si dimostra come la funzione precedente lo era).
\end{proof}

\begin{prop}{}{}
	Ogni funzione $f:[0,1] \to S^1$ continua ha un sollevamento:
	\[ \tilde f:[0,1] \to \mathbb R \quad \text{tale che} \quad e\circ \tilde f = f \]
	Inoltre fissato $x_0 \in \mathbb R$ con $e(x_0) = f(0)$, esiste un unico sollevamento di $f$ tale che:
	\[ f(0) = x_0 \]
\end{prop}

Con un diagramma abbiamo che:
\begin{center}
	\begin{tikzcd}
		& \mathbb R \ar[d, "e"] \\
		{[0,1]} \ar[ur, dashed, "{\tilde f}"] \arrow[r, "f"] & S^1
	\end{tikzcd}
\end{center}

\begin{proof}
	Consideriamo $U_0 = S^1 \setminus \{1\}$ e $U_1 = S^1 \setminus \{-1\}$, allora abbiamo che:
	\[ e^{-1}(U_0) = \coprod_{n\ in \mathbb Z}(n,n+1)\qquad \text e \qquad e^{-1}(U_1) = \coprod_{n \in \mathbb Z}\left(n-\frac 12, n+\frac 12\right) \]
	Come nel teorema di Van Kampen, $\exists m \in \mathbb N$ tale che:
	\[f|_{\left[\frac im, \frac{i+1}m\right]} \subseteq U_0 \text{ oppure }U_1\]
	Definiamo quindi:
	\[ y_i = f\left( \frac in \right) \]
	Definiamo allora con $\tilde f$ una funzione tale che:
	\begin{itemize}
		\item $\tilde f(0) = x_0$ con $x_0$ punto tale che $e(x_0) = f(0)$
		\item $e(x_0) = y_0$ e $\exists j \in \{0,1\}$ tale che $y_0 \in U_j$ e
			\[ f\left(\left[ 0, \frac 1m \right]\right) \subseteq U_j\]
	\end{itemize}
	Sia ora $V_0$ la componente connessa di $e^{-1}(U_j)$ contenente $x_0$, allora abbiamo che:
	\[ e|_{V_0}:V_0 \to U_j \text{ è omeomorfismo} \]
	Definiamo allora $\tilde f(t) = e|_{V_0^1} \circ f(t)$ per $t \in [0, \frac 1m]$. Definiamo allora $x_1 = \tilde f(\frac 1m)$, da cui segue in maniera naturale che $e(x_1) = f(\frac 1m) = y_1$.\\
	Iterando otteniamo un sollevamento:
	\[ \tilde f:\left[ 0, \frac 1m \right] \to \mathbb R \qquad \text{di}\qquad f:\left[ 0, \frac 1m \right] \to S^1 \]
	tale che:
	\[ \tilde f(0) = x_0\qquad e(x_0) = y_0 \qquad f\left(\frac 1m\right)=x_i \qquad e(x_i) = y_i \]
	Abbiamo inoltre che per $j \in \{0,1\}$ si ha che:
	\[ f\left( \frac im, \frac{i+1}m \right)\subseteq U_j \]
	Definiamo allora, come fatto in precedenza, $V_i$ come la componente connessa $e^{-1}(U_j)$ contenente $x_i$ e definiamo:
	\[ \tilde f(t)= \left(e|_{V_i}\right)^{-1} \circ f(t)\qquad \text{per }t \in \left[\frac im, \frac{i+1}m\right] \]
	In questo modo otteniamo una $\vec f$ definita a tratti e continua tale che $\tilde f$ sollevi $f$.

	L'unicità di tale sollevamento segue dalla definizione di $\tilde f$ stessa. In particolare segue dal fatto che l'unica scelta fatta è la scelta di $f(0)=x_0$
\end{proof}

Graficamente quello che abbiamo fatto è:
\begin{center}
	\begin{tikzpicture}
		\draw[thick] (-3,0) -- node(x)[pos = 0.5, above]{} (-1,0);
		\draw[thick, smooth, tension = 1] (3,0) to[out = 45, in = 45] (4,0) to[out =-135, in = -135] (5,0);
		\filldraw[green, very nearly transparent] (3,0.5) -- (2,-0.5) -- (4,-0.5) to[out = 30, in = -90] (5,0.5) -- (3,0.5);
		\filldraw[blue, very nearly transparent] (6,0.5) -- (5, -0.5) -- (3,-0.5) to[out = 105, in = -140] (4,0.5) -- (6,0.5);
		\draw (2,-0.5) -- (5,-0.5) -- (6,0.5) -- (3,0.5) -- (2,-0.5);
		\draw[green] (3,0.5) node[above left]{$U_0$};
		\draw[blue] (6,0.5) node[above right]{$U_1$};
		\filldraw (3,0) circle (1pt) node[below left]{$x_0$} (4,0) circle(1pt) node[below left]{$x_1$} (5,0) circle(1pt) node[above]{$x_2$};
		\draw[->] (x) to[out = 30, in = 150] node[pos = 0.5, above]{$f$} (2,0.3);
		\draw[->] (x) to[out = 90, in = 180] node[pos = 0.5, above]{$\tilde f$} (1,3);
		\draw[thick] (3,3) node(1){} arc(180:0:0.5cm) node(2){} arc(-180:0:0.5cm) node(3){};
		\filldraw (1) circle(1pt) node[below]{$x_0$} (2) circle(1pt) node[below left]{$x_1$} (3) circle(1pt) node[above]{$x_2$};
		\filldraw[green, very nearly transparent] (1) circle (1.25cm);
		\draw[green] (2.5,3.5) node{$V_0$};
		\filldraw[blue, very nearly transparent] (3) circle(1.25cm);
		\draw[blue] (5.5,3.5) node{$V_1$};
		\draw[->] (4,1.75) -- node[pos = 0.5, right]{$e$} (4,0.75);
		\draw[->, blue] (5,1) -- node[pos = 0.75, right]{$e|_{V_1}^{-1}$} (5,1.5);
		\draw[->, green] (3,1) -- node[pos = 0.75, left]{$e|_{V_0}^{-1}$} (3,1.5);
	\end{tikzpicture}
\end{center}

\begin{es}
	Solleviamo $\gamma_n$ definita come:
	\begin{align*}
		\gamma_1:[0,1] & \to S^1\\
		t &\mapsto e^{2 \pi int}
	\end{align*}
	Spezziamolo in $4$ pezzi in modo tale che:
	\[ \gamma \left[ \frac{i}{4n}, \frac{i+1}{4n} \right] \subseteq U_j \qquad \text{j} \in \{0,1\} \]
	Fissiamo quindi $0 \in \mathbb R$ con $e(0) = \gamma_n(0)$. Allora abbiamo che:
	\[ \gamma_n\left[0, \frac 1{4n}\right] \subseteq U_1 \qquad \Rightarrow \qquad V_0 \left(-\frac 12, \frac 12\right)\]
	Definiamo allora
	\[\tilde \gamma_n(t) = e|_{V_0}^{-1} \circ \gamma_n(t) = nt\qquad t \in \left[0, \frac 1{4n}\right] \]
	Infatti:
	\[ e(nt) = e^{2\pi int} \]
	Iterando, definiamo il sollevamento $\tilde \gamma_n(t) = nt$
\end{es}

\begin{prop}{}{}
	Ogni $F:[0,1]^2 \to S^1$ ammette un sollevamento $\tilde F:[0,1] \to \mathbb R$ tale che:
	\[e \circ \tilde F = F \]
	Inoltre, dato $x_0 \in \mathbb R$ tale che $e(x_0) = F(0,0)$, esiste un unico sollevamento $\tilde F$ di $F$ tale che:
	\[ \tilde F(0,0) = x_0 \]
\end{prop}

\begin{cor}{}{}
	Siano $\alpha, \beta \in \Omega(S^1,1,1)$ tali che $\alpha \sim \beta$. Siano $\tilde \alpha, \tilde \beta$ due sollevamenti tali che $\tilde \alpha (0) = \tilde \beta(0)$, allora:
	\[ \tilde \alpha(1) = \tilde \beta(1) \]
\end{cor}

\begin{proof}
	Sia $F$ omotopia tra $\alpha$ e $\beta$, quindi $F:[0,1]^2 \to S^1$ tale che:
	\[ F(t,0) = \alpha(0)\qquad F(t,1) = \beta(t)\qquad F(0,s) = 1\qquad F(1,s) = 1 \]
	Per la proposizione precedente abbiamo che esiste un unico sollevamento $\tilde F:[0,1]^2 \to \mathbb R$ di $F$ tale che:
	\[ \tilde F(0,0) = \tilde \alpha(0) \tilde \beta(0) \]
	Abbiamo anche che:
	\[ e \circ \tilde F(t,0) = F(t,0) = \alpha(t) \]
	Da questo segue che $t \mapsto \tilde F(t,0)$ definisce un sollevamento di $\alpha$ tale che valga $\tilde \alpha(0)$ per $t = 0$. Dall'unicità del sollevamento segue anche che:
	\[ \tilde F(t,0) = \tilde \alpha(t) \]
	D'altra parte abbiamo che:
	\[ e \circ \tilde F(0,s) = 1\qquad \Rightarrow \qquad \tilde F(0,s) \subseteq e^{-1}(\{1\}) \mathbb Z \]
	$\tilde F$ definisce una funzione continua $s \mapsto \tilde F(0,s) \in \mathbb Z$ che quindi è costante (in quanto $[0,1]$ connesso). Abbiamo inoltre che:
	\[ \tilde F(0,1) = \tilde \alpha(0) = \tilde \beta(0)\]
	Ma per l'unicità del sollevamento abbiamo che:
	\[ \tilde F(t,1) = \tilde \beta(t) \]
	Quindi, analogamente a quanto fatto prima abbiamo che $\tilde F(1,s)$ è costante, quindi $\tilde F(1,s) = \tilde \alpha(1)$, da questo segue che:
	\[ F(1,1) = \tilde \beta(1) = \tilde \alpha(1) \]
	Per cui $\tilde F$ è un'omotopia tra $\tilde \alpha$ e $\tilde \beta$
\end{proof}

\begin{defn}{Grado di $\alpha$}{}
	Dato $\alpha \in \Omega(S^1,1,1)$ e $\tilde \alpha$ suo sollevamento con $\tilde \alpha(0) = 0$, definiamo il \textbf{Grado di $\alpha$} come:
	\[ \deg(\alpha) = \tilde \alpha(1) \]
\end{defn}

Dimostriamo ora il teorema \ref{molla}.1

\begin{proof}[Dimostrazione del Teorema \ref{molla}.1]
	Definiamo $\varphi$ come:
	\begin{align*}
		\varphi: \pi_1(S^1,1) & \to \mathbb Z\\
		[\alpha] & \mapsto \deg(\alpha)
	\end{align*}
	Dal corollario precedente abbiamo che $\varphi$ è ben definita (in quanto cammini omotopi hanno lo stesso grado). Cerchiamo allora di capire come calcolare:
	\[ \varphi(\alpha * \beta) = \widetilde{\alpha*\beta} \]
	Noi sappiamo già di partenza come calcolare $\varphi(\alpha)$, infatti:
	\[ \varphi(\alpha) = \tilde\alpha(1) = \widetilde{\alpha*\beta}\left(\frac 12\right)\]
	Sappiamo anche che:
	\[ \varphi(\beta) = \tilde \beta(1) = \widetilde{\alpha*\beta}(1) - \widetilde{\alpha*\beta}\left(\frac 12\right) \]
	Da cui segue direttamente che:
	\[ \varphi(\alpha*\beta) = \widetilde{\alpha*\beta}(1) = \widetilde{\alpha*\beta}(1) - \widetilde{\alpha*\beta}\left(\frac 12\right) + \widetilde{\alpha*\beta}\left(\frac 12\right) = \varphi(\beta) + \varphi(\alpha)\]
	Quindi è un omomorfismo di Gruppi.

	$\varphi$ è suriettiva, in quanto $\deg(\gamma_n) = \gamma_n(1) = n$, da cui $\varphi(\gamma_n) = n$.

	$\varphi$ è iniettiva. Infatti, sia $\alpha$ cammino con $\deg(\alpha) = 0$, allora abbiamo che $\tilde \alpha(1) = 0$, per cui abbiamo che $\tilde \alpha \sim 1_0$ tramite un'omotopia $\tilde F$. Ma allora $e \circ \tilde F$ è omotopia tra $\alpha$ e $e \circ 1_0 = 1_1$ ($1_1$ è il cammino costante su $S^1$). Ma allora segue che $\varphi(\alpha) = 0$ se e solo se $[\alpha] = [1_1]$. Quindi è iniettiva.

	Quindi abbiamo dimostrato che è un isomorfismo di gruppi
\end{proof}

Abbiamo quindi che:
\begin{center}
	\begin{tikzcd}
		(A \cap B, x_0) \ar[r, hook, "j_2"] \ar[d, hook, "j_1"] & (B, x_0) \ar[d, hook, "i_2"]\\
		(A, x_0) \ar[r, hook, "i_1"] & (X,x_0)
	\end{tikzcd}
\end{center}
Attraverso il funtore $\pi_1$
\begin{center}
	\begin{tikzcd}
		\pi_1(A \cap B, x_0) \ar[r, "j_{2*}"] \ar[d, "j_{1*}"] & \pi_1(B, x_0) \ar[d] \ar[ddr, "i_{2*}", bend left = 20]\\
		\pi_1(A, x_0) \ar[r] \ar[drr, "i_{1*}", bend right = 20] & \pi_1(A,x_0)\star \pi_1(B, x_0) \ar[dr, "k"]\\
		&& \pi_1(X,x_0)
	\end{tikzcd}
\end{center}
\textit{Dove non ci sono le frecce abbiamo le funzioni canoniche del prodotto libero di gruppi, mentre la funzione $k$ è la stessa della prima forma del teorema di Van Kampen \ref{VK}.5}

Adesso possiamo enunciare la seconda parte del teorema di Van Kampen

\begin{thm}{di Van Kampen (Seconda Parte)}{}\label{VK2}
	Siano $A,B,A\cap B$ aperti e connessi per archi di $X$ tali che $A \cap B = X$. Il diagramma precedente è commutativo e la funzione $k$ è suriettiva. Inoltre:
	\[ Ker(k) = N(j_{1*}(\gamma) * (j_{2*}(\gamma))^{-1})\qquad \gamma \in \pi_1(A \cap B, x_0) \]
	Dove la lettera $N$ sta ad indicare il "sottogruppo normale generato da"
\end{thm}

\begin{oss}
	La relazione $i_1\circ j_1 =i_2 \circ j_2$ ci dice che $i_{1*} \circ j_{1*} = i_{2*} \circ j_{2*}$ e la commutatività del diagramma di dice che:
	\[ k \circ j_{1*}(\gamma) = i_{1*} \circ j_{1*}(\gamma) = i_{2*} \circ j_{2*} (\gamma) = k \circ j_{2*}(\gamma) \]
	Inoltre:
	\[ k \circ j_{1*}(\gamma) = k \circ j_{2*}(\gamma) \qquad \Rightarrow \qquad k(j_{1*}(\gamma) * (j_{2*}(\gamma))^{-1}) = 1_{\pi_1(x)} \]
\end{oss}

\begin{cor}{}{}
	Se ho $A,B,A \cap B$ come nel teorema e $\pi_1(A \cap B, x_0) \sim \{1\}$, allora:
	\[ \pi_1(X, x_0) \cong \pi_1(A, x_0) \cong \pi_1(B, x_0) \]
\end{cor}

\begin{es}[Gruppo Fondamentale non Abeliano]
	Andiamo a studiare il gruppo fondamentale di un bouquet di circonferenze $S^1 \times S^1$. Non possiamo prendere solamente le due circonferenze, in quanto queste sono dei chiusi, quindi possiamo prendere un po' di più delle singole circonferenze.
	\begin{center}
		\begin{tikzpicture}
			\fill[blue, very nearly transparent] (0,0) circle(1.1cm);
			\fill[white] (2,0) circle(1.1cm);
			\begin{scope}
				\clip (2,0) circle(1.1cm);
				\fill[green, very nearly transparent] (2,0) circle(1.1cm);
				\fill[white] (0,0) circle(1.1cm);
				\fill[white] (2,0) circle(0.9cm);
			\end{scope}
			\fill[blue, very nearly transparent](0.9,0) arc(180:135:1.1cm) arc (135:-45:0.1cm) arc(135:225:0.9cm) arc(45:-135:0.1cm) arc(225:180:1.1cm);
			\fill[white] (0,0) circle(0.9 cm);
			\fill[white] (0,0) circle(0.9cm);
			\fill[green, very nearly transparent](1.1,0) arc(0:45:1.1cm) arc (45:225:0.1cm) arc(45:-45:0.9cm) arc(135:315:0.1cm) arc(-45:0:1.1cm);
			\draw (0,0) circle(1cm) (2,0) circle(1cm);
		\end{tikzpicture}
	\end{center}
	Abbiamo che $A\sim S^1$, $B \sim S^1$ e $A \cap B$ è semplicemente connesso. Per cui abbiamo che:
	\[ \pi_1(S^1\times S^1,p) \cong \mathbb Z \star \mathbb Z \]
	Così, in maniera del tutto analoga, abbiamo che:
	\[ \pi_1(\underbrace{S^1\times \cdots S^1}_{n} ) \cong \mathbb Z^{\star n} \]
\end{es}

\begin{es}
	Vediamo un toro con un buco $S^1\times S^1 \setminus\{p\}$. Sfruttando rotazioni, possiamo rappresentarlo come:
	\begin{center}
		\begin{tikzpicture}
			\begin{scope}[decoration={
				markings,
				mark=at position 0.5 with {\arrow{>}}}]
				\draw[postaction = {decorate}] (0,0) -- (2,0);
				\draw[postaction = {decorate}] (0,2) -- (2,2);
			\end{scope}
			\begin{scope}[decoration={
				markings,
				mark=at position 0.55 with {\arrow{>}},
				mark=at position 0.45 with {\arrow{>}}}]
				\draw[postaction = {decorate}] (0,0) -- (0,2);
				\draw[postaction = {decorate}] (2,0) -- (2,2);
			\end{scope}
			\fill[green, very nearly transparent] (0,0) -- (0,2) -- (2,2) -- (2,0);
			\draw (1,1) node{$\times$};
		\end{tikzpicture}
	\end{center}
	Cioè possiamo portare il punto tolto in posizione centrale. Sfruttando poi le retrazioni per deformazioni, possiamo portarlo a:
	\begin{center}
		\begin{tikzpicture}
			\begin{scope}[decoration={
				markings,
				mark=at position 0.5 with {\arrow{>}}}]
				\draw[postaction = {decorate}, blue] (0,0) -- (2,0);
				\draw[postaction = {decorate}, blue] (0,2) -- (2,2);
			\end{scope}
			\begin{scope}[decoration={
				markings,
				mark=at position 0.55 with {\arrow{>}},
				mark=at position 0.45 with {\arrow{>}}}]
				\draw[postaction = {decorate}, green] (0,0) -- (0,2);
				\draw[postaction = {decorate}, green] (2,0) -- (2,2);
			\end{scope}
		\end{tikzpicture}
	\end{center}
	Che può essere rappresentato come un bouquet di circonferenze:
	\begin{center}
		\begin{tikzpicture}
			\begin{scope}[decoration={
				markings,
				mark=at position 0.5 with {\arrow{>}}}]
				\draw[postaction = {decorate}, blue] (0,0)arc(0:-360:1cm);
			\end{scope}
			\begin{scope}[decoration={
				markings,
				mark=at position 0.52 with {\arrow{>}},
				mark=at position 0.48 with {\arrow{>}}}]
				\draw[postaction = {decorate}, green] (0,0)arc(180:540:1cm);
			\end{scope}
		\end{tikzpicture}
	\end{center}
	Da cui segue che:
	\[ \pi_1(S^1 \times S^1 \setminus \{p\}, q) \cong \mathbb Z \star \mathbb Z \]
\end{es}

\begin{es}[$\pi_1$ del Toro con Van Kampen]
	Sia quindi $X$ un toro, cioè:
	\[ X = S^1 \times S^1 \cong [0,1]^2/_\sim \]
	Proprio come nell'esempio precedente possiamo rappresentarlo come
	\begin{center}
		\begin{tikzpicture}
			\begin{scope}[decoration={
				markings,
				mark=at position 0.5 with {\arrow{>}}}]
				\draw[postaction = {decorate}] (0,0) -- (2,0);
				\draw[postaction = {decorate}] (0,2) -- (2,2);
			\end{scope}
			\begin{scope}[decoration={
				markings,
				mark=at position 0.55 with {\arrow{>}},
				mark=at position 0.45 with {\arrow{>}}}]
				\draw[postaction = {decorate}] (0,0) -- (0,2);
				\draw[postaction = {decorate}] (2,0) -- (2,2);
			\end{scope}
			\fill[green, very nearly transparent] (0,0) -- (0,2) -- (2,2) -- (2,0);
		\end{tikzpicture}
	\end{center}
	Cerchiamo due aperti $A$ e $B$ come nel teorema. Allora possiamo prendere:
	\[ A = [0,1]^2/_\sim \setminus \left\{ \left[ \frac 12, \frac 12 \right] \right\} \qquad \text e \qquad B = (0,1)^2/_\sim \cong (0,1)^2\]
	In particolare abbiamo che vale l'isomorfismo per $B$ in quanto può essere vista come un quoziente di un'azione di gruppo (e $\pi$ è aperta).
	In questo modo abbiamo che l'intersezione è:
	\[ A\cap B = (0,1)^2/_\sim \setminus \left\{ \left[ \frac 12, \frac 12 \right] \right\}\cong (0,1)^2/_\sim \setminus \left\{ \left( \frac 12, \frac 12 \right) \right\} \]
	Prendiamo allora un punto appartenente all'intersezione, per esempio $x_0= \{[\frac 14, \frac 14]\}$ e calcoliamo i gruppi fondamentali dei sottospazi topologici. Sappiamo che $B$ è concesso, quindi vale:
	\[ \pi_1(B, x_0) \cong \{1\} \]
	In $A\cap B$ possiamo prendere un laccio $\gamma$ nel seguente modo
	\begin{center}
		\begin{tikzpicture}
			\begin{scope}[decoration={
				markings,
				mark=at position 0.5 with {\arrow{>}}}]
				\draw[postaction = {decorate}] (0,0) -- (2,0);
				\draw[postaction = {decorate}] (0,2) -- (2,2);
				\draw[blue, postaction = {decorate}] (0.5,0.5) -- (0.5,1.5);
			\end{scope}
			\begin{scope}[decoration={
				markings,
				mark=at position 0.55 with {\arrow{>}},
				mark=at position 0.45 with {\arrow{>}}}]
				\draw[postaction = {decorate}] (0,0) -- (0,2);
				\draw[postaction = {decorate}] (2,0) -- (2,2);
			\end{scope}
			\fill[green, very nearly transparent] (0,0) -- (0,2) -- (2,2) -- (2,0);
			\filldraw (0.5,0.5) circle(1pt);
			\draw (1,1) node{$\times$};
			\draw[blue] (0.5,1.5) -- (1.5,1.5) -- (1.5,0.5) -- (0.5,0.5);
		\end{tikzpicture}
	\end{center}
	Tramite delle retrazioni per deformazioni, abbiamo che possiamo portare tutto $A\cap B$ nel laccio $\gamma$, che è omeomorfo ad $S^1$, per cui:
	\[ \pi_1(A\cap B, x_0) = \mathbb Z \gamma \qquad \text{Sono i multipli di }\gamma \]
	Per quanto visto nell'esempio precedente abbiamo che $\pi_1(A, x_0) = \mathbb Z\alpha \star \mathbb Z\beta$ dove $\alpha$ e $\beta$ sono i lacci rappresentati dai "lati del quadrato". Tuttavia questi non passano per $x_0$, quindi non possono essere usati per il gruppo fondamentale. Cerchiamo quindi un cammino $\eta$ che colleghi $[0,0]$ e $x_0$.
	\begin{center}
		\begin{tikzpicture}
			\begin{scope}[decoration={
				markings,
				mark=at position 0.5 with {\arrow{>}}}]
				\draw[postaction = {decorate}] (0,0) -- node[pos = 0.5, below]{$i(\alpha)$} (2,0);
				\draw[postaction = {decorate}] (0,2) -- node[pos = 0.5, above]{$\alpha$} (2,2);
				\draw[blue, postaction = {decorate}] (0.5,0.5) -- (0.5,1.5);
			\end{scope}
			\begin{scope}[decoration={
				markings,
				mark=at position 0.55 with {\arrow{>}},
				mark=at position 0.45 with {\arrow{>}}}]
				\draw[postaction = {decorate}] (0,0) -- node[left]{$\beta$} (0,2);
				\draw[postaction = {decorate}] (2,0) -- node[right]{$i(\beta)$}(2,2);
			\end{scope}
			\fill[green, very nearly transparent] (0,0) -- (0,2) -- (2,2) -- (2,0);
			\filldraw (0.5,0.5) circle(1pt);
			\draw (1,1) node{$\times$};
			\draw[blue] (0.5,1.5) -- node[pos = 0.5,above]{$\gamma$}(1.5,1.5) -- (1.5,0.5) -- (0.5,0.5);
			\draw (0.5,0.5) -- node[above]{$\eta$} (0,0);
		\end{tikzpicture}
	\end{center}
	Possiamo quindi definire:
	\[ \alpha'= \eta * \alpha * i(\eta) = i(\eta)_\sharp (\alpha)\qquad \text e \qquad \beta' = \eta*\beta*i(\eta) = i(\eta)_\sharp(\beta) \]
	Quindi:
	\[ \pi_1(A, x_0) = \mathbb Z \alpha' \star \mathbb Z\beta' \]
	Quindi per il teorema di Van Kampen \ref{VK2}.7 abbiamo che:
	\[ \pi_1(X, x_0) = \frac{\mathbb Z i_{1*}(\alpha') \star \mathbb Z i_{1*}(\beta')}{N(j_{1*}(\gamma^n) * j_{2*}(\gamma^{-n}))_{n \in \mathbb Z}} \]
	Cerchiamo di capire come sono fatti i generatori del nucleo. Ricordiamo che $\gamma$ è un cammino che sta nell'intersezione, quindi, rispetto al diagramma commutativo prima del teorema \ref{VK2} abbiamo che $j_{1*}(\gamma)$ sta in $\pi_1(A, x_0)$ e $j_{2*}(\gamma)$ sta in $\pi_1(B, x_0)$. Necessariamente abbiamo che $j_{2*} = 1_{\pi_1(B, x_0)}$ (è l'unica scelta possibile).
	Per $j_{1*}(\gamma)$ possiamo notare che con un'omotopia possiamo schiacciare $\gamma$ al bordo del quadrato, cioè:
	\[ i(\eta)*j_{1*}(\gamma)*\eta = \beta*\alpha*\beta^{-1} * \alpha^{-1} \]
	Da cui segue che:
	\[ j_{1*} = \beta' * \alpha' * \beta'^{-1} * \alpha'^{-1} \]
	Per cui abbiamo che il gruppo fondamentale di $X$ è diventato:
	\[ \pi_1(X, x_0) = \frac{\mathbb Z i_{1*}(\alpha') \star \mathbb Z i_{1*}(\beta')}{N(\beta'*\alpha'*\beta'^{-1}*\alpha'^{-1})} \]
	Cerchiamo di capire ancora meglio. Sappiamo che $\beta'*\alpha'*\beta'^{-1}*\alpha'^{-1}$ è un elemento di $\mathbb Z\alpha' \star \mathbb Z\beta'$, quindi, stando al diagramma commutativo prima di \ref{VK2}.7, abbiamo che $k(\beta'*\alpha'*\beta'^{-1}*\alpha'^{-1}) \in \pi_1(X,x_0)$.
	Possiamo dire di più. Visto che stiamo quozientando possiamo dire che:
	\[ k(\beta' * \alpha' *\beta'^{-1} * \alpha'^{-1}) = 1_{\pi_1(X, x_0)} \]
	Cioè abbiamo la condizione che:
	\[ k(\beta'*\alpha') = k(\alpha' * \beta') \]
	Questa è la condizione che ci da la commutatività, quindi:
	\[ \pi_1(X,x_0) = \mathbb Z i_{1*}(\alpha') \times \mathbb Z i_{1*}(\beta') \]
\end{es}

\newpage

\section{Prodotti di Spazi Topologici}

\subsection{Prodotti di Spazi Topologici}

\begin{defn}{Prodotto di Insiemi}{}{}
	Dati $\{X_i\}_{i \in I}$ famiglia di insiemi definiamo il \textbf{Prodotto di Insiemi} con:
	\[ \prod_{i \in I}X_i = \{I\text{-uple }x = (x_1,...,x_I: x_i \in X_i, \forall i \in I)\}  \]
	$x_i$ è detta la \textbf{$i$-esima componente} di $x$.
\end{defn}

\begin{es}
	Se esiste un indice $i_0 \in I$ tale che $X_{i_0} = \varnothing$, allora:
	\[ \prod_{i \in I}X_i = \varnothing \]
\end{es}

Se per ogni $i \in I$ abbiamo che $X_i = X$, allora possiamo denotare:
\[ \prod_{i \in I} = X^I \]
Notiamo però che $X^I = \{f:I \to X \text{ funzioni}\} = \bm{Mor_{Set}}(I,X)$

\begin{es}
	Possiamo indicare con $\mathbb R^\mathbb N$ come l'insieme delle successioni in $\mathbb R$
\end{es}

\begin{defn}{Prodotto di Oggetti}{}{}
	Data una categoria $\mathcal C$ e una famiglia di oggetti in $\mathcal C$ $\{X_i\}_{i \in I}$, definiamo il \textbf{Prodotto degli }$X_i, i \in I$ in $\mathcal C$ come un oggetto $P$ con una collezione $\{p_i\}_{i \in I}$ di morfismi $p_i:P \to X_i$ tale che $\forall Z \in \bm{Ob}(\mathcal C)$ e $\forall$ famiglie $\{f_i\}_{i \in I}$ con $f_i:Z \to X_i$
	\[ \exists ! f_p:Z \to P : p_i \circ f_p = f_i, \forall i \in I \]
\end{defn}

Cioè vale:
\begin{center}
	\begin{tikzcd}
		&&& Z \ar[ddlll, bend right = 20] \ar[ddl, bend right = 20] \ar[d, "f_p", blue] \ar[dd, bend right = 20] \ar[ddr, bend left = 20] \ar[ddrrr, bend left = 20]\\
		&&& P \ar[dlll] \ar[dl] \ar[d] \ar[dr] \ar[drrr]\\
		X_1 & \cdots & X_{i-1} & X_i & X_{i+1} & \cdots & X_{I}
	\end{tikzcd}
\end{center}

\begin{es}
	$\prod X_i$ è un prodotto in $\bm{Set}$ e $p_i$ associa una $I$-upla alla $i$-esima coordinata
\end{es}

\begin{defn}{Prebase}{}{}
	Dato $(X, \mathcal T)$ spazio topologico, un insieme $\mathcal P \subseteq \mathcal T$ è detto \textbf{Prebase} se:
	\[ \mathcal B = \{\text{Intersezioni finite di elementi di }\mathcal P\}\text{ è una base} \]
\end{defn}

\begin{defn}{Prebase Canonica}{}{}
	Data $\{(X_i, \mathcal T_i)\}_{i \in I}$ famiglia di spazi topologici, definiamo la \textbf{Prebase Canonica} di $X = \prod X_i$ come:
	\[ \mathcal P_c = \{ p^{-1}_i(U_j) : p_i:X \to X_i, U_j \in \mathcal T_i  \} \]
\end{defn}

\begin{es}
	Se $U = \{1,2\}$, allora abbiamo che la prebase canonica è:
	\[ \mathcal P_c = \{ p_1^{-1}(U), p_2^{-1}(V): U \in \mathcal T_1, V \in \mathcal T_2 \} = \{ U \times X_2:U \in \mathcal T_1 \} \cup \{ X_1 \times V:V \in \mathcal T_2\}\]
	Abbiamo inoltre che:
	\[ \{ \text{Le intersezioni finite in }\mathcal P_c \}= \{U \times V: U \in \mathcal T_1, V \in \mathcal T_2\} = \mathcal B_c \]
\end{es}

\begin{defn}{Base Canonica del Prodotto}{}{}
	Definiamo $\mathcal B_c$ come la \textbf{Base Canonica della Topologia Prodotto} come le intersezioni finite di elementi di $\mathcal P_c$, cioè:
	\[ \mathcal B_c = \left\{ \prod_{i \in I} U_i: U_i \in \mathcal T_i, U_i \neq X_i \text{ per al più un numero finito di }i \right\} \]
\end{defn}

\begin{prop}{}{}
	$\mathcal B_c$ è la base di una topologia di $X = \prod X_i$ detta \textbf{Topologia Prodotto} e vale che $\forall Z \to X$ con $Z$ spazio topologico si ha che:
	\[ f \text{ è continua}\qquad \Leftrightarrow \qquad p_i \circ f \text{ è continua per }\forall i \in I \]
	Inoltre è la topologia meno fine che renda le $p_i$ continue
\end{prop}

\begin{proof}
	Applichiamo il teorema della Base \ref{Base}.2:
	\begin{enumerate}
		\item $\forall i \in I, p_i^{-1}(X_i) \in \mathcal B_c$ e $p_i^{-1}(X_i) = X$, per cui abbiamo che:
		\[ \bigcup_{A \in \mathcal B_c}A = X \]
		\item È immediato in quanto $\forall A,B \in \mathcal B_c$ si ha che $A \cap B \in \mathcal B_c$, quindi $\mathcal B_c$ è una topologia.
	\end{enumerate}
	Inoltre questa è la meno fine che rende le proiezioni $p_i$ continue per la definizione di prebase.\\
	L'ultimo punto è analogo al caso finito
\end{proof}

\begin{es}
	Sia $\{(X_i, \mathcal T_{GR})\}_{i \in I}$, allora abbiamo che $\prod X_i$ ha la topologia grossolana
\end{es}

\begin{es}
	Sia $\{(X_i, \mathcal T_D)\}_{i \in I}$ con $|I = + \infty|$, allora $\prod X_i$ ha una topologia meno fine di quella discreta (infatti posso scegliere insiemi arbitrari solo su un numero finito di indici). Abbiamo inoltre che i singoletti non sono aperti, in quanto dovrei fare intersezioni infinite.
\end{es}

\begin{es}[Topologia Finito Aperta]
	Siano $(Y,d)$ spazio metrico e $(X, \mathcal T)$ spazio topologico e sia $\bm{Mor_{Set}}(X,Y) = Y^X$ con topologia prodotto. Allora questo ha la topologia della \textbf{Convergenza Puntuale} con prebase data da:
	\[ p(s,U) = \{ f:X \to Y: f(s) \in U, s \in X, U \subseteq Y \text{ aperto} \} \]
	Ma abbiamo che $p(s, U) = p^{-1}_s(U)$
\end{es}

Se nell'esempio precedente avessimo avuto un compatto al posto di un singoletto, avremmo avuto che la topologia si sarebbe chiamata \textbf{Compatto Aperta}

\begin{thm}{di Alexander}{}
	Uno spazio topologico è compatto se e solo se da ogni ricoprimento di una prebase, si può estrarre un sottoricoprimento finito.
\end{thm}

\begin{thm}{di Tichonov}{}
	Un prodotto di spazi topologici compatti è compatto.
\end{thm}
\begin{proof}
	Siano $\{(X_i, \mathcal T_i)\}_{i \in I}$ i fattori e sia $X = \prod X_i$ il prodotto. Sia $\mathcal A$ un ricoprimento aperto di elementi di $\mathcal P_c$. Allora lo possiamo suddividere in componenti:
	\[ \mathcal A = \bigcup_{i \in I}\mathcal A_i \]
	Dove $\mathcal A_i$ contiene tutti e i soli elementi di $\mathcal A$ che sono preimmagine di un aperto di $X_i$. Allora possiamo scrivere $\mathcal A_i$ come:
	\[ \mathcal A_i = \{ p_i^{-1}(U_{i,j}) : U_{i,j} \in \mathcal T_i, j \in J_i \} \]
	Definiamo allora $C_i$ come:
	\[ C_i = X_i \setminus \left( \bigcup_{j \in J_i} \right)U_{i,j} \]
	Supponiamo per assurdo che nessun $C_i$ sia vuoto, allora abbiamo che:
	\[ C = \prod_{i \in I} C_i \subset X \text{ è un insieme non vuoto} \]
	Quindi sia $x \in C$, allora abbiamo che $\forall i \in I, \forall j \in J_i$, si ha che:
	\[x \not \in p^{-1}(U_{i,j}) \qquad \Rightarrow \qquad x \not \in \bigcup_{i \in I, j \in J_i} p_i^{-1}(U_{i,j})\]
	Allora abbiamo che $\mathcal A$ non è un ricoprimento, assurdo, in quanto avevamo che lo era per Ipotesi. Esiste quindi necessariamente $i_o \in I$ tale che $C_{i_0} = \varnothing$, per cui $\{U_{i_0, j}\}_{j \in J_{i_0}}$ è un ricoprimento aperto di $X_{i_0}$. Ma abbiamo che $X$ è compatto, quindi esiste un sottoricoprimento finito $U_{1_0,j_1},...,U_{i_0, j_n}$ tale che:
	\[ p_{i_0}^{-1}(U_{i_0,j_1}),...,p_{i_0}^{-1}(U_{i_0,j_n}) \text{ è un sottoricoprimento aperto di }\mathcal A\]
	Quindi per il teorema precedente di Alexander abbiamo che $X$ è compatto
\end{proof}

\begin{thm}{}{}
	Un prodotto di Spazi Topologici connessi è connesso
\end{thm}
\begin{proof}
	Siano $\{( X_i, \mathcal T_i )\}_{i \in I}$ spazi topologici e sia $X = \prod X_i$. Sia $x_0 \in X$ e definiamo:
	\[ F(x_0) = \{y \in X: x_0 \text{ e } y \text{ abbiano un numero finito di coordinate diverse}\} \]
	Vogliamo mostrare che:
	\begin{enumerate}
		\item $F(x_0)$ è connesso
		\item $F(x_0)$ è denso
	\end{enumerate}

	\fbox{1} Sia $J \subseteq I$ finito e consideriamo la funzione:
	\[ h_J: \prod_{j \in J} X_j \to X \qquad h_J(y)_i = \begin{cases}
		y_i & \text{se }i \in J\\
		(x_0)_i & \text{se } i \not \in J
	\end{cases}\]
	In ogni caso abbiamo che $h_J(y) \in F(x_0)$. Abbiamo che $h_J(y)$ è continua e che:
	\[ p_i \circ h_J = \begin{cases}
		\text{Una funzione costante se } i \not \in J\\
		\text{Una proiezione }\displaystyle{\prod_{j \in J}X_j \to X} \text{ se } j \in J
	\end{cases}\]
	Quindi abbiamo che $Im(h_J)$ è un connesso contenuto in $F(x_0)$ ($\prod X_j$ con $j \in J$ è connesso). Abbiamo inoltre che:
	\[ \prod_{J \subseteq I, |J|<+\infty} h_J\left( \prod_{j\in J} X_j \right)  = F(x_0)\]
	Siccome poi $x_0 \in Im(h_J), \forall J$ finito, si ha che $\bigcup Im(h_J)$ è ancora connesso, per cui $F(x_0)$ è connesso.

	\fbox{2} Sia $U \in \mathcal B_c$ non vuoto, allora lo possiamo scrivere come:
	\[ U = \prod_{i \in J} U_i \times \prod_{i \not \in J}X_i \text{ con }J \text{ insieme di indici a meno di riordinarli}\]
	Sapendo che è non vuoto, abbiamo che $U_i \neq \varnothing$. Sia allora $y \in \prod X_i$ per $i \in J$, allora abbiamo che:
	\[ h_J(y) \in F(x_0)\cap U \]
	Da questo abbiamo che $F(X_0)$ è denso in $X$, cioè $X = \overline{F(x_0)}$ è connesso per il lemma B \ref{B}.5 sulla connessione.
\end{proof}

\begin{prop}{}{}
	Un prodotto di Spazi Topologici $T1$ o $T2$ è $T1$ o $T2$
\end{prop}
\begin{proof}
	Siano ${(X_i,\mathcal T_i)}$ spazi topologici $T1$ o $T2$ e sia $X = \prod X_i$. Siano ora $x,y \in X$ punti distinti, allora esiste $i_0: x_{i_0} \neq y_{i_0}$.

	\fbox{$T1$} Siano $U$ e $V$ intorni di $p_{i_0}(x)$ e $p_{i_0}(y)$ rispettivamente, allora, essendo $X_{i_0}$ $T1$:
	\[ p_{i_0}(x) \not \in U \qquad p_{i_0}(y) \not \in U \]
	Cioè:
	\[ x \in p^{-1}_{i_0}(U) \quad y \in p^{-1}_{i_0}(V) \qquad x \not\in p^{-1}_{i_0}(V) \quad  y \not\in p^{-1}_{i_0}(U) \]
	Per cui $X$ è $T1$

	\fbox{2} Si dimostra in maniera analoga scegliendo $U$ e $V$ disgiunti.
\end{proof}

\begin{es}
	Esistono alcuni chiusi nella topologia prodotto della forma:
	\[ \prod_{i \in I}F_i \qquad F_i \subseteq X_i \text{ chiusi} \]
	Infatti $p_i^{-1}(F_i)$ è chiuso per continuità di $p_i$ e
	\[ \prod_{i \in I}F_i = \bigcap_{i \in I} p^{-1}_i (F_i) \]
\end{es}

\newpage

\section{Spazi Metrici}

\subsection{Primi passi}

\begin{defn}{Spazio Chiuso per Successioni}{}
	Dato $(X, d)$ spazio metrico e $C \subseteq X$. $C$ è detto \textbf{Chiuso per Successioni} se ogni successione di elementi in $C$ convergente ha limite in $C$.
\end{defn}

\begin{prop}{}{}
	Data $\mathcal T_d$ topologia associata a $d$, $C$ è chiuso per successioni in $(X,d)$ se e solo se è chiuso per $\mathcal T_d$
\end{prop}

\begin{proof}
	\fbox{$\Rightarrow$} Sia $A = X \setminus C$ e sia $x \in A$. Se non esistesse $\vareps >0:B_\vareps(x)\subseteq A$ avrei che:
	\[ \forall n >0, x_n \in B_{\frac 1n}(x) \cap C \text{ e }x_n \to x \]
	Ma questa è una successione convergente a $C$, quindi è assurdo

	\fbox{$\Leftarrow$} Sia $x_n$ successione in $C$ e sia $\displaystyle{x = \lim_{n \to +\infty}x_n}$. Se $x \not \in C$, allora esiste $\vareps>0$:
	\[ B_\vareps(x)\cap C = \varnothing \]
	Ma $X \setminus C$ è aperto, quindi è assurdo in quanto $x_n$ non può tendere a $C$, quindi $x \in X$
\end{proof}

\begin{oss}
	Abbiamo visto che una funzione tra spazi metrici è continua se e solo se lo è tra le topologie associate. Questo significa che il funtore naturale:
	\begin{align*}
		\bm{Metr} & \to \bm{Top}\\
		(X,d) & \mapsto (X,\mathcal T_d)\\
		f:(X,d)\to(Y,d) & \mapsto f:(X,\mathcal T_d) \to (Y,\mathcal T_d)
	\end{align*}
	È essenzialmente suriettivo
\end{oss}

\subsection{Compattezza e Completezza}

\begin{defn}{Spazio Metrico Completo}{}
	Uno spazio metrico è detto \textbf{Completo} se ogni successione di Cauchy è convergente
\end{defn}

\begin{thm}{}{}
	$(\mathbb R^n, d_\mathcal E)$ e $(\mathbb C^n, d_\mathcal E)$ sono completi
\end{thm}

\begin{es}
	$\{0\}\cup \{\frac 1n\}_{n \in \mathbb N^*} \subseteq (\mathbb R, d_\mathcal E)$ è completo (l'unica successione non banale ha limite nello spazio metrico)
\end{es}

\begin{es}
	$(\mathbb Q, d_\mathcal E)$ non è completo
\end{es}

\begin{oss}
	La completezza è una proprietà dello spazio metrico, \underline{non} topologico
\end{oss}

\begin{es}
	Abbiamo che:
	\[ (\mathbb R, d_\mathcal E) \xrightarrow{\cong} (S^1\setminus\{N\}, d_\mathcal E) \]
	Però il primo è completo mentre il secondo non lo è (in quanto esistono successioni con limite il punto $N$)
\end{es}

\begin{prop}{}{}
	Sia $(X, d_X)$ spazio metrico completo e sia $Y\subseteq X$ con metrica indotta da $d_X$. Allora $Y$ è completo se e solo se $Y$ è chiuso in $(X, \mathcal T_d)$
\end{prop}
\begin{proof}
	\fbox{$\Leftarrow$} Sia $\{y_n\}$ una successione di Cauchy, quindi:
	\[ \exists a = \lim_{n \to +\infty} y_n \in X \]
	Sappiamo tuttavia che $X$ è chiuso, quindi $a \in Y$. Quindi $Y$ è completo

	\fbox{$\Rightarrow$} Una successione convergente è di Cauchy, quindi hanno tutte limite in $Y$
\end{proof}

\begin{defn}{Compattezza per Successioni}{}
	Uno spazio metrico $(X,d)$ è \textbf{Compatto per Successioni} se ogni successione in $X$ ammette una sottosuccessione convergente.
\end{defn}

\begin{defn}{Spazio Metrico Totalmente Limitato}{}
	Uno spazio metrico $(X,d)$ è detto \textbf{Totalmente Limitato} se:
	\[ \forall \vareps>0, \exists x_1,...,x_{n_\vareps}  \in X: X = B_\vareps (x_1)\cup \cdots \cup B_\vareps (x_{n_\vareps})\]
\end{defn}

\begin{es}
	Supponiamo di avere uno spazio metrico $(X,d_D)$ con $|X| = +\infty$, dove $d_D$ è la \textbf{metrica discreta}, cioè $d(x,y) = 0$ se $x = y$, $d(x,y)=1$  altrimenti. Sia $x \in X$, allora abbiamo che $X = B_2(x)$, quindi $X$ è limitato ma non totalmente limitato, in quanto mi basta prendere $\vareps = \frac 12$:
	\[ \forall x \in X, B_{\frac 12}(x) = \{x\} \]
	Avendo poi un'infinità di punti in $X$, non posso riempire $X$ con un numero finito di palle
\end{es}

\textbf{Fatto}: In $\mathbb R^n$ un sottoinsieme è limitato se e solo se è totalmente limitato

\begin{thm}{}{}\label{lungo}
	Sia $(X,d)$ spazio metrico e sia $\mathcal T_d$ la topologia associata a $d$. Allora sono equivalenti:
	\begin{enumerate}
		\item $(X, \mathcal T_d)$ è compatto
		\item $(X, d)$ è compatto per successioni
		\item $(X, d)$ è completo e totalmente limitato
	\end{enumerate}
\end{thm}

Prima di dare la dimostrazione, diamo prima un corollario

\begin{cor}{Teorema di Heine - Borel}{}
	$X \subseteq (\mathbb R^n, \mathcal T_\mathcal E)$ è compatto se e solo se è chiuso in $\mathbb R^n$ ed è limitato
\end{cor}

\begin{proof}
	Segue dal teorema e dal fatto che essendo $X\subseteq \mathbb R^n$ sottospazio compatto di uno spazio completo è anche esso stesso completo e del fatto precedente.
\end{proof}

Diamo ora la dimostrazione del teorema \ref{lungo}.6

\begin{proof}
	\fbox{$1 \Rightarrow 2$} Sia $\{x_n\}_{n \in \mathbb N}$ successione. Se la successione assume finiti valori, allora posso estrarre una sottosuccessione costante (quindi è necessariamente convergente).\\
	Sia ora $\{x_n\}_{n \in \mathbb N}$ successione che assume infiniti valori e distinguiamo $2$ casi.\\
	Supponiamo esista $a \in X$ tale che $\forall U$ intorno di $a$, $U \cap \{x_0\}$ abbia infiniti elementi. Allora posso trovare una sottosuccessione convergente data da $\{y_n\}_{n \in \mathbb N}$ con:
	\[ y_n \in B_{\frac 1n}(a) \cap \{x_n\}_{n\in \mathbb N} \]
	Da questa condizione segue che:
	\[ d(y_n, y_{n+k})<\frac 2n \qquad \text e \qquad d(y_n, a)< \frac 1n\]
	Dalla prima otteniamo che la successione è di Cauchy, mentre dalla seconda otteniamo che $y_n \to a$ per $n\to +\infty$. Quindi converge.\\
	Supponiamo (per assurdo) che non esista un elemento $a$ come quello della prima parte del teorema, cioè:
	\[ \forall x \in X, \exists U_x \text{ intorno aperto tale che }U_x \cap \{x_n\}_{n \in \mathbb N}\text{ abbia finiti elementi} \]
	Da allora segue che:
	\[ X = \bigcup_{x \in X}U_x \text{ è un ricoprimento aperto} \]
	Tuttavia sappiamo che $X$ è compatto per ipotesi, quindi esistono $z_1,...,z_n \in X$ tali che:
	\[ X = U_{z_1}\cup \cdots \cup U_{z_n} \]
	Da cui segue che $\{x_n\}_{n\in \mathbb N}$ assume finiti valori. Ma questo è un assurdo in quanto avevamo assunto che $\{x_n\}_{n \in \mathbb N}$ assumesse infiniti valori.

	\fbox{$2\Rightarrow 3$} Sia $\{x_n\}_{n\in \mathbb N}$ una successione di Cauchy. Allora, sapendo che $X$ è compatto per successioni, abbiamo che esiste una sottosuccessione $\{y_n\}_{n \in \mathbb N}$ convergente ad un valore $a$. Fissiamo allora un valore $\vareps>0$. Per le disuguaglianze triangolari abbiamo che:
	\[ d(x_n,a) \leq d(x_n, y_n) + d(y_n, a) \]
	Tuttavia, dal fatto che $X$ è compatto per successioni abbiamo che:
	\[ \exists \overline n: d(y_n,a) < \frac \vareps 2\qquad \forall n > \overline n \]
	Inoltre, sapendo che $\{x_n\}$ è una successione di Cauchy, possiamo dire che:
	\[ \exists \overline n : d(y_m, y_n) < \frac \vareps 2\qquad \forall n,m > \overline n \]
	Per cui abbiamo che $x_n \to a$. Per cui $X$ è completo.\\
	Sia adesso $\vareps >0$ e sia $x_1 \in X$. Se abbiamo che $X = B_\vareps (x_1)$ abbiamo sostanzialmente finito. Supponiamo quindi $\exists x_2 \in X \setminus B_\vareps (x_1)$. Iteriamo quanto appena fatto: se $X = B_\vareps(x_1)\cup B_\vareps(x_2)$ abbiamo finito, altrimenti esiste un altro valore $x_3 \in X \setminus (B_\vareps(x_1)\cup B_\vareps(x_2))$. Se ad un certo punto terminiamo, abbiamo ricoperto $X$ con un numero finito di pale di raggio $\vareps$. Se (per assurdo) non terminiamo, abbiamo ottenuto una successione $\{z_n\}_{n \in \mathbb N}$. Tuttavia non esistono sottosuccesioni di $\{z_n\}_{n \in \mathbb N}$ che siano convergenti, in quanto abbiamo che:
	\[ d(z_n,z_m) > \vareps\qquad \forall n \neq m \]
	Abbiamo quindi un assurdo e otteniamo che il procedimento deve necessariamente terminare $\forall \vareps$. Per cui $X$ è totalmente limitata.

	\fbox{$3 \Rightarrow 1$} Sia $\mathcal A = \{A_\xi\}_{\xi \in \Xi}$ un ricoprimento aperto di $X$. Vogliamo come prima cosa mostrare che esista un sottoricoprimento numerabile.\\
	Allora $\forall n\in \mathbb N$ ho $x_{1,n},...,x_{k(n),n} \in X$ tali che:
	\[ X = B_{\frac 1n} (x_{1,n}) \cup \cdots \cup B_{\frac 1n}(x_{k(n),n}) \]
	Sia ora $\mathcal B = \{ x_{i,n} : n \in \mathbb N, i \in \{1,...,k(n)\} \}$. Questo è un insieme numerabile. Allora $\forall x_{i,n} \in \mathcal B$ scegliamo, se esiste, un solo elemento $A_{i,n} \in \mathcal A$ tale che:
	\[ B_{\frac 1n} \subseteq A_{i,n} \]
	Sia ora $\mathcal C \subseteq \mathcal B$ il sottoinsieme per cui esiste l'elemento di $\mathcal A$ e questo mi individua numerabili elementi di $\mathcal A$. Sia quindi $x \in X$, allora:
	\[\exists A \in \mathcal A:x \in A \text{ e }\exists \vareps>0 : B_\vareps(x)\subseteq A \]
	Per la totale limitatezza dell'insieme, abbiamo che esistono $n$ e $r \in \{1,...,k(n)\}$ tale che:
	\[x \in \mathcal B_{\frac 1n}(x_{r,n}) \subseteq \mathcal B_\vareps(x)\]
	Da cui segue che $x_{r,n} \in \mathcal C$ e:
	\[ x \in \bigcup_{i,n}A_{i,n} \]
	Per cui abbiamo che $\{A_{i,n}\}_{n\in \mathbb N}$ è un sottoricoprimento numerabile di $X$. Rinumeriamo allora tali elementi e otteniamo un sottoricoprimento $\mathcal A'$ del tipo:
	\[ \mathcal A' = \{A_0,...,A_n,...\} \]
	Supponiamo per assurdo che $\mathcal A'$ non abbia sottoricoprimenti finiti, allora:
	\[ \forall m,\exists y_m \not \in A_0 \cup \cdots \cup A_m \]
	Otteniamo allora una succesione di $\{y_m\}$ che non ha sottosuccessioni convergenti. Infatti, se esistesse una tale sottosuccessione con limite $a$, allora esisterebbe $\overline m$ tale che $a \in A_{\overline m}$.
	Quindi, per la definizione di $\{y_m\}$ che abbiamo dato, esistono infiniti valori di $\{y_m\}$ dentro $A_{\overline m}$. Qui cade un assurdo, in quanto questa cosa va contro la definizione di $\{y_m\}$. Quindi non ci possono essere sottosuccessioni convergenti.\\
	Per la totale limitatezza di $X$, abbiamo che $\exists p \in \{1,...,k(1)\}$ tale che $\{y_n\} \cap B_1(x_{p,n})$ abbia infiniti elementi. Scelgo questi elementi in modo tale da mantenere l'ordine e otteniamo $\{z^{(1)}_n\}$ sottosuccessione di $\{y_1\}$, in modo tale che:
	\[ z^{(1)}_{n} = y_{n'} \qquad \text{con }n\geq n'\]
	Iteriamo il procedimento, allora abbiamo che $\exists p_2 \in \{1,...,k(2)\}$ tale che $\{z_n^{(1)}\} \cap B_{\frac 12} (x_{p_2,2})$ abbia infiniti elementi. Da questo otteniamo una successione $\{ z^{(2)}_n \}$ sottosuccessione di $\{ z_n^{(1)} \}$. Andando avanti in questo modo, otteniamo $z^{(r)}_n$ catene di sottosuccessioni $\forall n \in \mathbb N$, ciascuna contenuta in una sfera di raggio $\frac 1r$. Abbiamo quindi:
	\[\begin{matrix}
		{\color{blue} z_1^{(1)}} & z_2^{(1)} & z_3^{(1)} & z_4^{(1)} & \cdots\\
		z_1^{(2)} & {\color{blue} z_2^{(2)}} & z_3^{(2)} & z_4^{(2)} & \cdots\\
		z_1^{(3)} & z_2^{(3)} & {\color{blue} z_3^{(3)}} & z_4^{(3)} & \cdots\\
		z_1^{(4)} & z_2^{(4)} & z_3^{(4)} & {\color{blue} z_4^{(4)}} & \cdots\\
		\vdots & \vdots & \vdots & \vdots & {\color{blue} \ddots}
	\end{matrix}\]
	Definiamo allora una successione:
	\[ w_s = z^{(s)}_s \]
	Questa è una sottosuccessione di $\{y_m\}$. Ma abbiamo che:
	\[ w_{s+k} = z^{(s+k)}_{s+k} = z^{(s)}_{s+k+\delta} \qquad 	\Rightarrow \qquad d(w_s, w_{s+k})< \frac 2s\]
	Quindi abbiamo che $\{w_s\}$ è una successione di Cauchy. Ma avevamo per ipotesi che $X$ è completo, quindi converge. Qui cade l'assurdo, perché avevamo supposto che $\{y_m\}$ non avesse sottosuccessioni convergenti. Quello che ha causato l'assurdo è proprio l'esistenza di $\{y_m\}$. Quindi non può esistere tale successione, quindi necessariamente $\mathcal A'$ ha un sottoricoprimento finito, quindi anche $\mathcal A$, quindi $X$ è completo
\end{proof}

\end{document}
